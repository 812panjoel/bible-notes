\documentclass[a4paper, 12pt]{article}
\usepackage[a4paper, left=1.25cm, right=1.25cm, top=2cm, bottom=2cm]{geometry}
% \usepackage[UTF8]{ctex}
\usepackage{xeCJK}
%\setCJKmainfont{SimSun} % 设置中文字体
\usepackage[hidelinks]{hyperref}
\usepackage{amsmath, amssymb}
\usepackage{graphicx}
\usepackage{titlesec}
\usepackage{tocloft}

% 调整目录中 section 编号与标题之间的间隔
\setlength{\cftsecnumwidth}{2em} % 设置编号的宽度
% \cftsecnumwidth 是编号占用空间的宽度,增大这个值可以增加编号与标题的间隔

\title{诗篇分享}
\author{潘家}
\date{\today}

\begin{document}

\maketitle

%\newpage

\tableofcontents



%------------------------------------------------------------------------------
\newpage
\section{诗篇第1篇:义人与恶人的道路——如何活出蒙福的人生}
\subsection*{引言:人生的选择——你要走哪条路?}

人生就像一条道路,每个人都必须选择自己的方向。诗篇第1篇清晰地描述了两种截然不同的人生:\textbf{义人的道路和恶人的道路},以及两者最终的结局。今天,我们要深入探讨这篇诗篇,看看如何成为蒙福的人,远离恶人的道路,并活出神所喜悦的生命。

\subsection*{第一部分:蒙福之人的特征(1-3节)}

\textbf{“不从恶人的计谋,不站罪人的道路,不坐亵慢人的座位。”(诗篇 1:1)}

\subsubsection*{远离罪恶的影响}
诗篇1:1说:“不从、不站、不坐”——这是一个\textbf{逐步深入的过程}:
\begin{itemize}
    \item \textbf{“从”恶人的计谋}:听信世界的建议,被错误的价值观影响。
    \item \textbf{“站”罪人的道路}:认同并接受错误的生活方式。
    \item \textbf{“坐”亵慢人的座位}:完全沉浸在罪恶中,甚至嘲讽真理。
\end{itemize}

\subsubsection*{喜爱神的话语,昼夜思想}
\textbf{“惟喜爱耶和华的律法,昼夜思想,这人便为有福。”(诗篇 1:2)}

\begin{itemize}
    \item “喜爱”意味着对神的话语有一种渴慕,如同饥饿时想吃食物一样。
    \item “昼夜思想”表示持续地默想神的话,让神的真理塑造我们的思想和行为。
\end{itemize}

\subsubsection*{生命如同栽在溪水旁的树}
\textbf{“他要像一棵树栽在溪水旁,按时候结果子,叶子也不枯干。”(诗篇 1:3)}

“栽在溪水旁”代表生命有稳定的供应,根基深厚,不会因环境变化而枯干。“按时候结果子”表示属灵生命会有稳定的成长和结果。“叶子不枯干”象征持续的生命力,无论环境如何,都能靠主得胜。

\subsection*{第二部分:恶人的道路与结局(4-5节)}

\textbf{“恶人并不是这样,乃像糠秕,被风吹散。”(诗篇 1:4)}

\subsubsection*{恶人的生命是没有根基的}
恶人不像“栽在溪水旁的树”,而是“像糠秕”,即没有重量、没有价值、毫无稳定性。他们看似繁荣,但终究无法长久。\textbf{没有神的生命,最终会变得空虚和迷失。}

\subsubsection*{他们无法在审判中站立得住}
\textbf{“因此当审判的时候,恶人必站立不住。”(诗篇 1:5)}

神最终要审判世人,恶人无法在神的面前存留。现实中,许多人看似成功,但如果他们的生命没有建立在神的真理上,最终仍会崩溃。

\subsection*{第三部分:两条道路的最终结局(6节)}

\textbf{“因为耶和华知道义人的道路,恶人的道路却必灭亡。”(诗篇 1:6)}

\subsubsection*{神看顾义人的道路}
“知道”不仅仅是指神的知识,更表示神的眷顾、保护和引导。\textbf{跟随神的人,不代表没有挑战,但神必保守他们走在正确的道路上。}

\subsubsection*{恶人的道路最终灭亡}
他们虽然在世上可能暂时繁荣,但最终的结局是灭亡。耶稣也在马太福音7:13-14说过:\textbf{“引到灭亡,那门是宽的,路是大的,进去的人也多。”}世界看似吸引人,但最终的代价是永恒的毁灭。

\subsection*{结论:选择生命之路,成为蒙福之人!}

\begin{itemize}
    \item \textbf{远离恶人的道路,拒绝世界的诱惑}——不要轻易被世界的价值观影响。
    \item \textbf{渴慕神的话语,生命扎根在神的真理中}——建立稳定的属灵生命。
    \item \textbf{坚守义人的道路,最终进入永恒的福分}——走向神所预备的美好结局。
\end{itemize}

诗篇1篇告诉我们,人生只有两条道路,我们必须做出选择。愿你今天就做出正确的决定,走在蒙福的道路上!阿们!

\subsection*{结束祷告}

\textbf{亲爱的天父,}

感谢你赐下你的话语,让我们知道如何成为蒙福的人。求你帮助我们远离恶人的道路,不被世界的诱惑影响,而是单单渴慕你的话语,让我们的生命扎根在你里面。主啊,求你引导我们的道路,使我们一生行在你的旨意中,最终进入你的永恒国度!

奉主耶稣基督的名祷告,阿们!

%-----------------------------------------------------------------------


\newpage
\section{诗篇第2篇:列国为何争闹——基督的主权与我们的回应}

\subsection*{引言:世界的混乱与神的计划}

在当今世界,我们看到国家之间的纷争,社会的不安,以及人心的悖逆。诗篇第2篇描绘了人类如何悖逆神,而神却已设立他的受膏者——耶稣基督为王。今天,我们将从这篇诗篇学习如何面对世上的动荡,并如何回应基督的主权。

\subsection*{第一部分:世界的悖逆(1-3节)}

\textbf{“外邦为何争闹?万民为何谋算虚妄的事?”(诗篇 2:1)}

\subsubsection*{1. 世人的抵挡}
人类天性悖逆神,想要脱离他的管辖。
\begin{itemize}
    \item 世界的价值观往往与神的旨意相悖。
    \item 人们追求权力、金钱,却拒绝神的统治。
\end{itemize}

\subsubsection*{2. 人的计划终究虚妄}
人可以反抗,但神的计划永远不被拦阻。
\begin{itemize}
    \item 纵观历史,许多政权都试图抵挡神,但都归于虚无。
    \item 现实应用:我们要顺服神的旨意,而非按世界的方式行事。
\end{itemize}

\subsection*{第二部分:神的回应(4-9节)}

\textbf{“那坐在天上的必发笑;主必嗤笑他们。”(诗篇 2:4)}

\subsubsection*{1. 神的主权不可动摇}
神不会因人的反抗而受影响,他仍然掌权。

\subsubsection*{2. 基督被立为君王}
\textbf{“我已经立我的君在锡安我的圣山上。”(诗篇 2:6)}

\begin{itemize}
    \item 耶稣基督已得胜,他掌管历史。
    \item 现实应用:在动荡的世界中,我们要信靠基督,而非恐惧。
\end{itemize}

\subsection*{第三部分:智慧的选择(10-12节)}

\textbf{“现在你们君王应当省悟。”(诗篇 2:10)}

\subsubsection*{1. 当存敬畏的心}
敬畏神是智慧的开端,顺服基督才是真正的平安。

\subsubsection*{2. 投靠基督的人有福了}
\textbf{“凡投靠他的,都是有福的。”(诗篇 2:12)}

\begin{itemize}
    \item 世人可以选择悖逆,或者选择信靠基督。
    \item 现实应用:我们要把信心建立在基督之上,而不是世上的短暂权势。
\end{itemize}

\subsection*{结论:投靠基督,得着真平安}

\begin{itemize}
    \item 世界虽动荡,但神的国度永不改变。
    \item 选择顺服基督,才能找到真正的安全感和盼望。
    \item 无论世界如何,我们要坚持信仰,传扬福音。
\end{itemize}

愿我们都能在基督里得着真正的平安和力量!

\subsection*{结束祷告}

\textbf{亲爱的天父,}

感谢你赐下诗篇第2篇,让我们看到世人的悖逆,也看到你的主权。求你帮助我们不随从世界,而是存敬畏的心来顺服基督。愿我们在这动荡的时代,仍然持守信仰,成为光和盐。

奉主耶稣基督的名祷告,阿们!


%-----------------------------------------------------------------------
\newpage
\section{诗篇第3篇:在困境中信靠神——学习信心的功课}

\subsection*{引言:面对人生的挑战}

每个人都会经历困境,有时候我们会感到被人背叛、被环境压迫,甚至心里充满恐惧。大卫在诗篇第3篇中,正是在逃避他儿子押沙龙的追杀时写下的。这篇诗篇教导我们,在困境中如何仰望神、信靠神,并得着他的拯救。

\subsection*{第一部分:困境的现实(1-2节)}

\textbf{“耶和华啊,我的敌人何其加增!有许多人起来攻击我。”(诗篇 3:1)}

\subsubsection*{1. 敌人的增多}
大卫曾经是君王,如今却被自己的儿子追杀,甚至很多百姓也背叛了他。
\begin{itemize}
    \item 人生中也会遇到“敌人”,可能是人际关系的破裂、事业的低谷,甚至是内心的恐惧。
    \item 当我们感到被孤立、被攻击时,是否能像大卫一样转向神?
\end{itemize}

\subsubsection*{2. 外界的嘲讽}
\textbf{“有许多人议论我说:他得不着神的帮助。”(诗篇 3:2)}

\begin{itemize}
    \item 世人常以外在的情况论断一个人的处境,甚至怀疑神的帮助。
    \item 现实应用:我们是否曾因环境的艰难而怀疑神?还是选择坚定信靠他?
\end{itemize}

\subsection*{第二部分:信心的宣告(3-6节)}

\textbf{“但你耶和华是我四围的盾牌,是我的荣耀,又是叫我抬起头来的。”(诗篇 3:3)}

\subsubsection*{1. 神是我们的盾牌}
盾牌代表保护和安全,在攻击来临时,神是我们的避难所。

\subsubsection*{2. 祷告带来平安}
\textbf{“我用我的声音求告耶和华,他就从他的圣山上应允我。”(诗篇 3:4)}

\begin{itemize}
    \item 祷告不是无用的喊话,而是神真实回应的途径。
    \item 现实应用:当我们遇到问题时,第一反应是否是向神祷告?
\end{itemize}

\subsubsection*{3. 安然入睡的信心}
\textbf{“我躺下睡觉,我醒着,耶和华都保佑我。”(诗篇 3:5)}

\begin{itemize}
    \item 大卫虽然身处危险,却能安然入睡,因为他信靠神。
    \item 现实应用:在焦虑和压力中,我们是否能交托给神,安然入睡?
\end{itemize}

\subsection*{第三部分:神的拯救(7-8节)}

\textbf{“耶和华啊,求你起来!我的神啊,求你救我!”(诗篇 3:7)}

\subsubsection*{1. 呼求神的拯救}
大卫知道,真正的拯救不来自军队或策略,而是来自神。

\subsubsection*{2. 最终的得胜在于神}
\textbf{“救恩属乎耶和华,愿你赐福给你的百姓。”(诗篇 3:8)}

\begin{itemize}
    \item 神最终掌权,真正的胜利属于他。
    \item 现实应用:无论环境如何,我们要相信神掌管一切,并最终带来拯救。
\end{itemize}

\subsection*{结论:在困境中选择信靠神}

\begin{itemize}
    \item 认识到困境是人生的一部分,但我们不孤单。
    \item 透过祷告,我们可以经历神的平安和保护。
    \item 最终,真正的拯救和得胜来自神。
\end{itemize}

愿我们在面对困难时,不是被恐惧压倒,而是像大卫一样,坚定信靠神!

\subsection*{结束祷告}

\textbf{亲爱的天父,}

感谢你借着诗篇第3篇提醒我们,在困境中仍然可以信靠你。求你帮助我们,在面对挑战和压力时,不是焦虑害怕,而是转向你,仰望你的拯救。主啊,你是我们的盾牌,是我们随时的帮助,愿我们的一生都坚定依靠你。

奉主耶稣基督的名祷告,阿们!

%-----------------------------------------------------------------------
\newpage
\section{诗篇第4篇:在困境中寻求神的公义与安慰——学习信心的生活}
\subsection*{引言:面对压力,我们如何回应?}

生活中我们都会遇到压力、误解和挑战。当我们受到不公对待或经历困境时,我们该如何回应?诗篇第4篇是大卫在压力下向神的呼求,这篇诗篇教导我们在困难中如何依靠神,寻求他的公义,并得着真正的平安。

\subsection*{第一部分:向神呼求(1节)}

\textbf{“显我为义的神啊,我呼吁的时候,求你应允我!我在困苦中,你曾使我宽广;现在求你怜恤我,听我的祷告。”(诗篇 4:1)}

\subsubsection*{1. 认识神是公义的神}
大卫称神为“显我为义的神”,表明他知道神掌管公义。
\begin{itemize}
    \item 在面对误解和不公时,我们应当首先仰望神,而不是急于自我辩护。
    \item 现实应用:当我们在生活或工作中遇到不公待遇时,是否愿意先向神祷告,而不是陷入愤怒和苦毒?
\end{itemize}

\subsubsection*{2. 祷告带来释放}
大卫说:“我在困苦中,你曾使我宽广。”这表明,他过去的经历让他看到神如何在困境中开路。
\begin{itemize}
    \item 现实应用:回顾我们的人生,有多少次神带我们走出困境?
    \item 祷告不仅是寻求帮助,更是让我们经历神的信实。
\end{itemize}

\subsection*{第二部分:世界的虚妄与信靠神的智慧(2-5节)}

\textbf{“你们这上流人哪,你们将我的荣耀变为羞辱要到几时呢?你们喜爱虚妄,寻找虚假,要到几时呢?”(诗篇 4:2)}

\subsubsection*{1. 世人的虚妄}
世界追求虚假的荣耀和短暂的满足,而忽略了真正的价值。
\begin{itemize}
    \item 大卫看到那些攻击他的人,他们迷失在谎言和虚妄之中。
    \item 现实应用:我们是否常常被世界的名声、金钱、地位所迷惑?
\end{itemize}

\subsubsection*{2. 神所拣选的人必蒙保守}
\textbf{“你们要知道,耶和华已经分别虔诚人归他自己;我求告耶和华,他必听我。”(诗篇 4:3)}

\begin{itemize}
    \item 神眷顾那些专心倚靠他的人。
    \item 现实应用:当世界否定我们时,我们是否仍然相信神看顾我们?
\end{itemize}

\subsubsection*{3. 在怒气中谨慎,不犯罪}
\textbf{“你们应当畏惧,不可犯罪;在床上的时候,要心里思想,并要肃静。”(诗篇 4:4)}

\begin{itemize}
    \item 生气时容易做出错误决定,但神提醒我们要冷静思考。
    \item 现实应用:当我们愤怒或受伤害时,是否愿意安静下来,在神面前省察自己?
\end{itemize}

\subsection*{第三部分:真正的平安与满足(6-8节)}

\textbf{“有许多人说:谁能指示我们什么好处?耶和华啊,求你仰起脸来光照我们。”(诗篇 4:6)}

\subsubsection*{1. 世界的好处 vs. 神的恩典}
世界的人都在寻找“好处”,但真正的福分是神的同在。
\begin{itemize}
    \item 现实应用:我们是否常常向神祈求世俗的成功,而不是渴慕神的同在?
\end{itemize}

\subsubsection*{2. 喜乐胜过物质的满足}
\textbf{“你使我心里快乐,胜过那丰收五谷新酒的人。”(诗篇 4:7)}

\begin{itemize}
    \item 物质的丰盛不能带来真正的喜乐,唯有神能满足人心。
    \item 现实应用:当我们经历缺乏时,是否仍然能靠主喜乐?
\end{itemize}

\subsubsection*{3. 平安入睡的秘诀}
\textbf{“我必安然躺下睡觉,因为独有你耶和华使我安然居住。”(诗篇 4:8)}

\begin{itemize}
    \item 大卫虽在困境中,仍能安然入睡,因为他信靠神。
    \item 现实应用:今天的我们,是否常因焦虑失眠?愿我们学习将重担交托给神,得着真正的平安。
\end{itemize}

\subsection*{结论:活出信心的人生}

\begin{itemize}
    \item 在困境中,首先向神祷告,信靠他的公义。
    \item 不要被世界的虚妄迷惑,而要专心寻求神。
    \item 真实的平安和满足不在于环境,而在于对神的信靠。
\end{itemize}

愿我们都能学习大卫的信心,在压力和挑战中紧紧抓住神,得着从他而来的平安与喜乐。

\subsection*{结束祷告}

\textbf{亲爱的天父,}

感谢你借着诗篇第4篇教导我们,在压力和挑战中如何寻求你。求你帮助我们,不被世界的虚荣和谎言迷惑,而是单单倚靠你。主啊,我们把一切焦虑和重担交托给你,求你赐给我们心灵的平安,使我们能安然入睡,并在你的光中行走。

奉主耶稣基督的名祷告,阿们!



%-----------------------------------------------------------------------
\newpage
\section{诗篇第5篇:向神呼求——在患难中信靠神}

\subsection*{引言:在困难中,你向谁呼求?}

在生活中,我们都会经历不公、攻击、困惑和挑战。在面对这些困境时,我们的第一反应是什么?是愤怒、抱怨,还是寻求神的帮助?诗篇第5篇是大卫在遭遇恶人攻击时的祷告,他向神倾诉痛苦,并坚定信靠神的公义与保护。今天,我们要学习大卫的祷告,如何在困境中信靠神。

\subsection*{第一部分:向神倾诉(1-3节)}

\textbf{“耶和华啊,求你留心听我的言语,顾念我的心思!”(诗篇 5:1)}

\subsubsection*{1. 诚心向神祷告}
大卫在痛苦中,第一件事就是转向神,而不是向人抱怨。他用“言语”“心思”“呼求”表达对神的渴望。
\begin{itemize}
    \item 许多人遇到困难时,会选择向朋友诉苦,但我们应首先转向神。
    \item 祷告不仅是说话,更是向神倾诉我们内心最深的感受。
\end{itemize}

\subsubsection*{2. 早晨寻求神的带领}
\textbf{“耶和华啊,早晨你必听我的声音。”(诗篇 5:3)}

\begin{itemize}
    \item 早晨是新的一天的开始,代表着把一切交托给神。
    \item 在一天的开始,我们应先寻求神的引导,而不是被焦虑充满。
    \item 现实应用:养成每日晨祷的习惯,让神带领我们的一天。
\end{itemize}

\subsection*{第二部分:神憎恶罪恶,保护义人(4-8节)}

\subsubsection*{1. 神不喜悦罪恶}
\textbf{“凡作孽的,都是你所恨恶的。”(诗篇 5:5)}

神是圣洁的,他不能容忍罪恶。大卫深知神的公义,因此他选择远离恶人,不与他们同流合污。

\subsubsection*{2. 神引导敬畏他的人}
\textbf{“至于我,我必凭你丰盛的慈爱进入你的居所。”(诗篇 5:7)}

\begin{itemize}
    \item 大卫知道自己不是因行为,而是因神的慈爱才能来到他面前。
    \item 我们也需要神的恩典才能行在他的道路上。
    \item 现实应用:依靠神的恩典,而不是自己的努力来讨神喜悦。
\end{itemize}

\subsection*{第三部分:神的保护与恶人的结局(9-12节)}

\subsubsection*{1. 恶人的欺骗与最终审判}
\textbf{“他们的喉咙是敞开的坟墓,他们用舌头谄媚人。”(诗篇 5:9)}

恶人满口谎言,最终神必定审判他们。现实中,我们也常遇到谎言、欺骗,但要相信神的公义。

\subsubsection*{2. 义人的喜乐与神的保护}
\textbf{“凡投靠你的,愿他们喜乐。”(诗篇 5:11)}

\begin{itemize}
    \item 投靠神的人,不仅得到保护,更能在困境中喜乐。
    \item 现实应用:面对挑战时,选择信靠神,而不是靠自己。
\end{itemize}

\subsection*{结论:在困境中信靠神}

\begin{itemize}
    \item 遇到困难时,首先转向神,向他倾诉。
    \item 远离罪恶,依靠神的恩典行走正路。
    \item 信靠神的保护,在他里面得喜乐。
\end{itemize}

愿我们都能学习大卫,在困境中仰望神,得着真正的平安和力量!

\subsection*{结束祷告}

\textbf{亲爱的天父,}

感谢你赐下诗篇5篇,让我们学习如何在困境中依靠你。求你帮助我们不被环境影响,而是单单信靠你的公义与慈爱。让我们远离罪恶,寻求你的引导,在你里面得着真正的平安和喜乐。愿你保守我们的道路,使我们一生行在你的旨意中。

奉主耶稣基督的名祷告,阿们!

%-----------------------------------------------------------------------------
\newpage
\section{诗篇第6篇:从痛苦到盼望}

\subsection*{引言:人生的低谷如何面对?}

在生命中,我们都会经历痛苦、困惑甚至绝望的时刻。诗篇第6篇是大卫在极大痛苦中向神的呼求,表达了他内心的挣扎,同时也展现了他对神的信靠。今天,我们要借此篇诗篇探讨:当我们陷入困境时,应当如何向神呼求,并在他的应许中找到盼望。

\subsection*{第一部分:痛苦的呼求(1-3节)}

\textbf{“耶和华啊,求你不要在怒中责备我,也不要在烈怒中惩罚我。耶和华啊,求你可怜我,因我是软弱的;耶和华啊,求你医治我,因为我的骨头战兢。”(诗篇 6:1-2)}

\subsubsection*{1. 罪的重担与神的管教}

大卫在诗篇一开始便向神呼求,求神不要在怒中责备他。这表明他意识到自己的软弱,并渴望神的怜悯。现实生活中,我们常常因罪恶感、失败或苦难而感到沉重,但神的管教是出于爱。

\subsubsection*{2. 身心的极度痛苦}

“骨头战兢”代表着身体与心灵的极度痛苦。很多时候,我们的痛苦不仅是外在的,更是内心的折磨。面对这样的痛苦,我们是否愿意像大卫一样,坦然地向神倾诉?

\subsection*{第二部分:痛苦中的信心(4-7节)}

\textbf{“耶和华啊,求你转回,搭救我!因你的慈爱拯救我。因为在死地无人记念你,在阴间有谁称谢你?”(诗篇 6:4-5)}

\subsubsection*{1. 依靠神的慈爱求拯救}

大卫在痛苦中仍然抓住神的慈爱,知道神的拯救是他唯一的希望。同样,当我们陷入低谷,不知道如何继续时,我们是否仍然相信神的爱不会离开我们?

\subsubsection*{2. 痛苦的现实}

“我因叹息而困乏,每夜流泪,把床榻漂起。”(诗篇 6:6)

大卫的痛苦如此深重,以至于整夜哭泣。现实生活中,我们也会有这样的时刻——无法入眠,心里充满忧伤。然而,神看见我们的眼泪,并且不会忽视。

\subsection*{第三部分:从痛苦到得胜(8-10节)}

\textbf{“你们一切作孽的人,离开我吧!因为耶和华听了我哀哭的声音。耶和华听了我的恳求,耶和华必收纳我的祷告。”(诗篇 6:8-9)}

\subsubsection*{1. 祷告带来的转变}

在前面的经文中,大卫仍处于痛苦之中,但到了第8节,他的语气开始改变。他宣告:“耶和华听了我的祷告。”这表明,他在祷告后经历了从绝望到信心的转变。

\subsubsection*{2. 确信神的回应}

大卫不再沉溺于痛苦,而是确信神听见了他的祷告,并且必定回应。同样,我们的祷告不会徒劳,神必在合适的时间成就他的旨意。

\subsection*{结论:从哀哭到喜乐}

诗篇第6篇向我们展示了一个重要的属灵原则:

\begin{itemize}
    \item 在痛苦中,我们要来到神的面前,不隐藏自己的挣扎。
    \item 无论环境如何,我们要依靠神的慈爱,相信他必拯救。
    \item 祷告能改变我们的心,使我们从绝望转向信心。
\end{itemize}

当我们经历痛苦时,不要远离神,而要像大卫一样,向他呼求。神是信实的,他听见我们的祷告,并必定带领我们走向得胜。

\subsection*{结束祷告}

\textbf{亲爱的天父,}

感谢你赐下诗篇第6篇,让我们看到大卫如何在痛苦中向你呼求,并最终在你里面找到盼望。主啊,我们承认自己的软弱和痛苦,求你怜悯我们,拯救我们脱离困境。愿你坚固我们的信心,使我们在绝望中仍然仰望你,因知道你必听我们的祷告。

奉主耶稣基督的名祷告,阿们!


%-----------------------------------------------------------------------------
\newpage
\section{诗篇第7篇:向神寻求公义}
\subsection*{引言:当我们遭遇不公}

在现实生活中,我们常常会遇到不公正的事情:被误解、被诬告、甚至被恶意攻击。在这样的时刻,我们如何回应?诗篇第7篇是大卫在面对不公和逼迫时的祷告,为我们提供了极大的启示。

\subsection*{第一部分:向神寻求公义(1-5节)}

\textbf{“耶和华我的神啊,我投靠你,求你救我脱离一切追赶我的人,将我救拔出来。”(诗篇7:1)}

\subsubsection*{1. 投靠神,而非靠自己报仇}

大卫在被仇敌追赶时,首先想到的是投靠神,而不是自己寻求报复。在现实中,我们也会遇到人际冲突和不公正的对待,但神要我们把一切交托给他,而不是用人的方式解决。

\subsubsection*{2. 让神鉴察我们的内心}

\textbf{“耶和华我的神啊,我若行了这事,若有罪孽在我手里……”(诗篇7:3)}

大卫在寻求神的公义之前,首先省察自己是否有错。这提醒我们,当遭遇冲突时,首先要检查自己的行为,而不是立即指责他人。

\subsection*{第二部分:神是公义的审判者(6-11节)}

\textbf{“耶和华按公义审判民众,耶和华是义人的盾牌,是拯救心里正直的人。”(诗篇7:10)}

\subsubsection*{1. 神掌权,他是最终的审判者}

世上的法庭和人心都会有偏差,但神是公义的审判者,他的判断是完全公正的。因此,当我们面对不公时,可以安心地把公义交给神。

\subsubsection*{2. 义人的保护者,恶人的惩罚者}

神不仅审判恶人,也保护义人。无论环境如何,我们要相信神必定为持守正直的人伸冤。

\subsection*{第三部分:赞美神的公义(12-17节)}

\textbf{“我要照着耶和华的公义称谢他,歌颂耶和华至高者的名。”(诗篇7:17)}

\subsubsection*{1. 选择信靠,而不是抱怨}

即使大卫的困境还没有完全解决,他仍然选择赞美神。这提醒我们,即使在困境中,也要相信神的公义。

\subsubsection*{2. 公义终将得胜}

虽然恶人可能暂时猖狂,但最终神的公义一定会显明。因此,我们要坚持走在正义的道路上,不因暂时的艰难而灰心。

\subsection*{结论:如何回应现实中的不公?}

\begin{itemize}
    \item \textbf{向神呼求,而不是自己伸冤}——当面对不公时,首先向神祷告,而不是急于报复。
    \item \textbf{省察自己,行在正直中}——在控诉别人之前,先检查自己的行为是否合神心意。
    \item \textbf{相信神的公义,等候他的伸冤}——神的审判不会迟到,终有一天,他会彰显公义。
    \item \textbf{在困境中仍然赞美神}——即使问题未解决,仍然相信并赞美神的公义。
\end{itemize}

诗篇7篇提醒我们,神是公义的审判者,他必保护义人,惩治恶人。让我们把生命交托给神,在任何环境中都信靠他。

\subsection*{结束祷告}

\textbf{亲爱的天父,}

感谢你通过诗篇7篇教导我们,你是公义的神,是我们最终的审判者。当我们遭遇不公和逼迫时,求你帮助我们把一切交托给你,而不是用自己的方式解决。求你鉴察我们的心,使我们行在正直中,免得自己落入罪中。主啊,我们信靠你的公义,相信你会按你的时间伸张正义。在等待的过程中,求你赐给我们信心和忍耐,让我们即使在困境中,也能赞美你的名。

奉主耶稣基督的名祷告,阿们!

%-----------------------------------------------------------------------------
\newpage
\section{诗篇第8篇:神荣耀的显现}
\subsection*{引言:仰望创造的奇妙}

诗篇第8篇是大卫对神创造奇妙作为的赞美。当我们仰望浩瀚的星空、观察大自然的奥秘,我们是否曾思考过自己的渺小?然而,神却眷顾我们,使我们在受造物中拥有尊贵的地位。今天,我们一起来探讨诗篇第8篇,学习如何在生活中回应神的荣耀。

\subsection*{第一部分:神的荣耀在创造中彰显(1-3节)}

\textbf{“耶和华我们的主啊,你的名在全地何其美!你将你的荣耀彰显于天。”(诗篇 8:1)}

\subsubsection*{1. 透过自然看见神的伟大}
诗篇8:1-3提醒我们,自然界的一切都指向神的荣耀。
\begin{itemize}
    \item 宇宙的浩瀚和秩序——展现神的权能与智慧。
    \item 大自然的精妙设计——反映神的完美与恩典。
\end{itemize}

现实生活中,我们可以透过对科学、艺术、音乐的欣赏,更深地认识神的创造之美。

\subsection*{第二部分:人的渺小与尊贵(4-6节)}

\textbf{“人算什么,你竟顾念他?世人算什么,你竟眷顾他?”(诗篇 8:4)}

\subsubsection*{1. 在伟大宇宙中人的渺小}
面对无尽的星辰,人类似乎微不足道。但神却顾念我们,赋予我们特殊的使命和地位。

\subsubsection*{2. 神赋予人的尊贵地位}
\textbf{“你叫他比天使微小一点,并赐他荣耀尊贵为冠冕。”(诗篇 8:5)}

这意味着:
\begin{itemize}
    \item 我们被造是有目的的——活出神的形象。
    \item 神赋予我们管理世界的权柄——要有智慧地治理万物。
    \item 我们的价值不在于世界的标准,而在于神对我们的心意。
\end{itemize}

\subsection*{第三部分:回应神的荣耀(7-9节)}

\textbf{“凡走路的兽、空中的鸟、海里的鱼,都服在他的脚下。”(诗篇 8:7-8)}

\subsubsection*{1. 以敬畏的心回应神}
我们应当以敬拜和顺服回应神,因为他的名在全地何其美!

\subsubsection*{2. 以行动活出神的荣耀}
既然神赋予我们管理世界的责任,我们应该:
\begin{itemize}
    \item 爱护环境,善待受造物。
    \item 尊重他人,活出神形象的荣耀。
    \item 以感恩的心敬拜神,传扬他的名。
\end{itemize}

\subsection*{结论:以敬畏的心度过每一天}

诗篇8篇提醒我们,神的荣耀无处不在,而人类在神的创造中被赋予尊贵的地位。让我们以敬畏的心回应神,在生活中彰显他的荣耀。

\subsection*{结束祷告}

\textbf{亲爱的天父,}

感谢你创造了如此奇妙的世界,也赐给我们尊贵的地位。求你帮助我们在日常生活中活出你的荣耀,珍惜你所赐的一切,并以敬畏的心回应你的呼召。愿你的名在全地何其美!

奉主耶稣基督的名祷告,阿们!

%----------------------------------------------------------------------------
\newpage
\section{诗篇第9篇:在困境中信靠神}
\subsection*{引言}

亲爱的弟兄姐妹,今天我们要一起查考\textbf{诗篇第9篇},这是一首大卫的诗歌,充满了对神的颂赞,也表达了他在困境中的信靠。在现实生活中,我们都会遇到困难、委屈、甚至是恶人的攻击,但神的话语教导我们如何在艰难中仍然坚守信心,赞美他的公义与慈爱。



\subsection*{第一部分:感恩与信靠--无论环境如何,心仍向神(1-2节)}


\textbf{“我要一心称谢耶和华,我要传扬你一切奇妙的作为!我要因你欢喜快乐,至高者啊,我要歌颂你的名!”(诗9:1-2)}

\vspace{0.2cm}

大卫在诗篇的开头就带领我们进入感恩的状态。他用“\textbf{一心}”来表达自己全然的感恩,而不是出于习惯或形式。今天,我们是否愿意在\textbf{任何环境}下都对神感恩?在顺境中,我们容易称谢,但在逆境中,我们是否仍能颂赞神的名?

\vspace{0.2cm}

\textbf{现实应用: } 
有时,我们的工作、学业或人际关系遇到难处,就会开始抱怨,甚至质疑神的良善。但大卫的榜样告诉我们,\textbf{敬拜和感恩不是基于环境,而是基于神的信实}。即便我们眼前的情况看似不如意,神仍然掌权,我们要因他的作为欢喜。


\subsection*{第二部分:公义的神必为受欺压的人伸冤(7-10节)}

\textbf{“惟耶和华坐着为王,直到永远;他已经为审判设摆他的宝座。他要按公义审判世界,按正直判断万民。”(诗9:7-8)“耶和华又要给受欺压的人作高台,在患难的时候作高台。”(诗9:9)}

\vspace{0.2cm}

这几节经文给我们极大的安慰——神不是远离我们痛苦的神,而是\textbf{公义的审判者},也是我们\textbf{在患难中的高台}。当我们感到被不公待遇伤害时,我们可以确信神不会视而不见,他掌权,最终必会施行公义。

\vspace{0.2cm}

\textbf{现实应用: }
在生活中,我们可能会遭遇不公,如被误解、被欺压,甚至遭受恶人的攻击。然而,我们不需要用自己的方式去复仇,而是要\textbf{把冤屈交托给神}。他是公义的,他的时间和方式远超过我们的理解。我们需要做的就是忍耐、祷告,并等候神的公义彰显。


\subsection*{第三部分:恶人终将灭亡,敬畏神的人必得庇护(15-18节)}

\textbf{“外邦人陷在自己所掘的坑中;他们的脚在自己暗设的网罗里缠住了。”(诗9:15)}
\textbf{“困苦人必不永久被忘,困苦人的指望必不永远落空。”(诗9:18)}

\vspace{0.2cm}

大卫在这里提醒我们,恶人虽然暂时猖狂,但他们最终会落入自己设下的网罗里。神的公义必然执行,不会让恶人永远逍遥法外。同时,神的目光也没有离开受苦的人,他会亲自施行拯救。

\vspace{0.2cm}

\textbf{现实应用:}
当我们看到世界上不公的事情时,例如贪污、欺压、诽谤、罪恶猖獗,我们可能会心生愤怒,甚至觉得神为什么不立即施行审判。但大卫告诉我们,\textbf{神的审判不会缺席,只是他的时间与方式不同于人的期待}。我们的责任是信靠神,而不是自己伸冤。

\subsection*{第四部分:祷告--求神介入,坚定信心(19-20节)}

\textbf{“耶和华啊,求你起来,不容人得胜,愿外邦人在你面前受审判!耶和华啊,求你使他们恐惧,愿外邦人知道自己不过是人。”(诗9:19-20)}

\vspace{0.2cm}

在诗篇的最后,大卫向神祷告,呼求神的审判降临。这提醒我们,面对世界的邪恶,我们不能仅仅是叹息,而是要将问题带到神面前,\textbf{求神亲自掌权}。大卫不是依靠自己,而是求神来掌权施行公义。

\vspace{0.2cm}

\textbf{现实应用:}  
当我们看到社会中的不公,或个人生活中的苦难时,我们可以效法大卫,将一切交托给神,而不是凭血气去争战。\textbf{祷告是我们最强大的武器},能帮助我们在黑暗中保持信心,依靠神等候他的工作。

\subsection*{结论}

诗篇第9篇教导我们几个关键的信仰原则:  

\begin{enumerate}
    \item 无论环境如何,都要感恩信靠神。
    \item 神是公义的,必为受欺压的人伸冤。
    \item 恶人终将灭亡,敬畏神的人必蒙保守。
    \item 祷告是信心的表达,我们要依靠神,而非自己的力量。
\end{enumerate}


愿我们在面对生活的挑战时,学习像大卫一样,将目光转向神,依靠他的公义和慈爱,坚定地行在他的道路上。



\subsection*{结束祷告}

\textbf{亲爱的天父,}

感谢你赐下你的话语,让我们看到你的公义、慈爱与信实。无论我们遇到怎样的困境,求你帮助我们学习大卫的榜样,在任何环境下都能称谢你,信靠你。主啊,求你亲自伸张公义,安慰受苦的人,也让恶人知道他们的有限,使他们悔改归向你。愿你的国降临,愿你的旨意成就在我们的生命中。

奉主耶稣基督的名祷告,阿们!

%-----------------------------------------------------------------------------
\newpage
\section{诗篇第10篇:当恶人猖狂时,我们如何依靠神?}
\subsection*{引言}

亲爱的弟兄姐妹,在这个世界上,我们常常看到恶人猖狂、欺压贫穷人、行事诡诈却似乎毫无惩罚。我们或许会疑惑:\textbf{神为何沉默?他为何不立即伸张公义?} 诗篇第10篇正是大卫在这种疑惑中向神的呼求,也给了我们信心的答案。今天,让我们一同学习这篇诗篇,明白当恶人兴盛时,我们应当如何依靠神。



\subsection*{第一部分:当神似乎沉默时,我们如何回应?(1节)}


\textbf{“耶和华啊,你为何站在远处?在患难的时候,为何隐藏?”(诗10:1)  }

\vspace{0.2cm}

\textbf{大卫的困惑也是我们的困惑。} 当我们看到邪恶得胜,恶人富足,而义人受苦时,我们很容易质疑神的公义,甚至怀疑他是否仍在掌权。  

\vspace{0.2cm}

\textbf{现实应用}
当我们在生活中经历不公,比如被人诽谤、职场受欺压、或看到社会中的腐败,我们可能会问:\textbf{神为什么不干预?为什么他不立即施行公义?} 但圣经告诉我们,\textbf{神的沉默并不代表他的缺席,他有自己的时间和方式来施行审判。} 我们要学会忍耐,信靠他的智慧,而不是凭血气行事。  


\subsection*{第二部分:恶人的特点——他们为何猖狂?(2-11节)}

诗人接下来描述了恶人的特征,我们可以总结为\textbf{四个方面}:  

\subsubsection*{1. 骄傲自夸,不敬畏神(2-4节) }
   \textbf{“恶人因骄横,逼迫困苦人……恶人夸耀他心里所愿的;贪财的,背弃耶和华,并且轻慢他。”(诗10:2-3)  }
   
   \vspace{0.2cm}

   \textbf{恶人认为自己是世界的主宰,他们依靠自己的力量,而不是倚靠神。} 他们藐视神,不认为神会干预他们的行为,因此更加放肆。  

\subsubsection*{2. 行事诡诈,欺压贫穷人(5-7节)}
   \textbf{“凡他所作的,时常稳固;你的审判超乎他的眼界,至于他一切的敌人,他都向他们喷气。”(诗10:5)  }
    
    \vspace{0.2cm}
   
   \textbf{恶人表面上似乎很成功,他们的计谋得逞,财富增长,甚至逃避了法律的制裁。} 他们说话充满诡诈,用谎言和威胁来操纵别人。  

\subsubsection*{3. 暗中谋害无辜者(8-10节) }
   \textbf{“他埋伏在村庄里,暗中杀害无辜的人……他埋伏,蹲伏,如狮子蹲在洞中。”(诗10:8-9)}  

   \vspace{0.2cm}

   他们不光明正大地作恶,而是暗中算计,寻找机会攻击软弱者,像狮子捕食一样。  

\subsubsection*{4. 以为神不会管(11节) }
   \textbf{“他心里说:‘神竟忘记了,他掩面,永不观看。’”(诗10:11)  }

   \vspace{0.2cm}

   \textbf{他们的最大错误就是以为神不会审判。} 但圣经清楚告诉我们,神从未离开,他正在观察,并在合适的时间施行公义。  


\subsubsection*{现实应用:}
在现实生活中,我们可能会遇到“恶人”——无论是在职场、校园,还是社会中,有人利用权力压迫他人,做出不公义的事情却不被惩罚。这时,我们需要记住:\textbf{他们的猖狂只是暂时的,神不会永远容忍罪恶。}  


\subsection*{第三部分:神掌权,终将施行审判(12-18节)}

大卫在下半部分诗篇中,不再是疑惑,而是转向神,祈求他施行公义。他的祷告可以分为两个方面:  

\subsubsection*{1. 祈求神介入(12-15节)}
\textbf{“耶和华啊,求你起来!神啊,求你举手,不要忘记困苦人。”(诗10:12) } 

\vspace{0.2cm}

大卫呼求神采取行动,\textbf{不要让恶人继续猖狂。} 他相信神不会真的忘记受苦者,而是会在适当的时候施行拯救。  

\subsubsection*{2. 坚信神的公义(16-18节)}
\textbf{“耶和华永永远远为王,外邦人从他的地已经灭绝了。”(诗10:16) } 
\textbf{“耶和华啊,谦卑人的心愿,你早已知道……使强暴的人不再威吓他们。”(诗10:17-18)  }

\vspace{0.2cm}

大卫最终得出结论:\textbf{神仍然掌权,他不会让恶人得胜到底。} 神关心谦卑的人,听见他们的呼求,终究会使公义得胜。  


\subsubsection*{现实应用:}
当我们看到世界上不公的事情时,我们可以:  
\begin{itemize}
    \item \textbf{像大卫一样祷告,求神伸张公义},而不是让苦毒充满我们的心。
    \item \textbf{相信神的时间和方式},即使我们现在看不到结果,但神的公义不会落空。
    \item \textbf{自己成为公义的见证人},在生活中行公义,帮助那些受欺压的人。 
\end{itemize}


\subsection*{结论:面对恶人,我们要做什么?}

诗篇第10篇给了我们明确的指引:  

\begin{enumerate}
    \item 当恶人猖狂时,我们不要失去信心,而要向神倾诉。
    \item 不要羡慕恶人的短暂成功,他们终究会面对神的审判。
    \item 神的公义不会缺席,他必定拯救受压迫者。
    \item 我们要祷告,并且自己行公义,使神的公义在我们的生命中彰显。
\end{enumerate}

愿我们在面对恶人时,不是靠自己伸冤,而是依靠神,他是永远掌权的王!  

\subsection*{结束祷告}

\textbf{亲爱的天父,}

我们感谢你通过诗篇第10篇教导我们,在恶人猖狂的时候,我们仍然可以信靠你。主啊,虽然有时候我们不明白你的时间和计划,但我们愿意相信你掌权,你不会让恶人永远得胜。求你安慰受欺压的人,坚定我们的信心,让我们不要被恶所胜,反而靠你的力量行公义,爱怜悯,存谦卑的心与你同行。

奉主耶稣基督的名祷告,阿们!

%-----------------------------------------------------------------------------
\newpage
\section{诗篇第11篇:在动荡中坚定信靠神}
\subsection*{引言}

亲爱的弟兄姐妹,我们生活在一个充满挑战的世界,时常面临信仰的考验。社会动荡、人心诡诈、恶人猖狂,有时甚至连我们赖以生存的根基都似乎动摇了。面对这些困境,我们该如何回应?\textbf{是逃避,还是站立得稳?} 今天,我们一同来学习\textbf{诗篇第11篇},看大卫如何在动荡中坚定信靠神,并从中得到属天的智慧。 


\subsection*{第一部分:信靠神,而非逃避现实(1-3节)}


\hspace{0.4cm}\textbf{“我投靠耶和华,你们怎么对我说:‘你当像鸟飞往你的山去’? {} ”(诗11:1)  
}

\vspace{0.2cm}

大卫在这里表达了自己坚定的信心。他的朋友(或顾问)劝他逃跑,像鸟一样飞往山里,远离危险。但大卫的回答很坚定:\textbf{我已经投靠耶和华,为什么还要逃避?}

\vspace{0.2cm}

\textbf{“看哪,恶人弯弓搭箭,要在暗中射那心里正直的人。”(诗11:2)}

\textbf{“根基若毁坏,义人还能做什么呢?”(诗11:3) } 

\vspace{0.2cm}

恶人似乎已经得势,他们暗中施行诡计,攻击正直人。更严重的是,\textbf{“根基若毁坏,义人还能做什么?” }这里的“根基”可以指社会的公义、道德的标准,或者人的安全感。\textbf{当世界的秩序似乎崩溃,我们该如何自处?}

 
\vspace{0.2cm}

\textbf{现实应用: } 
在现实生活中,我们常常遇到类似的处境:  

\begin{itemize}
     \item 当社会道德败坏时,我们是否仍然坚守信仰?  
     \item 当当职场或学业面临挑战时,我们是否选择妥协,还是坚守正直?  
     \item 当当面对不公义的环境时,我们是否逃避,还是勇敢面对?
\end{itemize}

大卫给了我们答案——\textbf{不论外在环境如何,我们的信心不应建立在世界的根基上,而要建立在神的身上。}  


\subsection*{第二部分:神仍然掌权,公义必然得胜(4-6节)}

\hspace{0.4cm} \textbf{“耶和华在他的圣殿里;耶和华的宝座在天上。他的慧眼察看世人。”(诗11:4)  }

\vspace{0.2cm}

\textbf{尽管地上的根基似乎动摇,神的宝座却永不动摇!} 神仍然坐在他的圣殿里,他没有丢弃他的子民,也没有放弃对世界的掌管。  

\vspace{0.2cm}

\textbf{“耶和华试验义人;惟有恶人和喜爱强暴的人,他心里恨恶。”(诗11:5)  }

\vspace{0.2cm}

\textbf{神正在试验我们,看我们是否真的信靠他。} 我们的信仰不是在安逸时才有效,而是在考验和挑战中才显出真实。神恨恶恶人,他不会让他们最终得胜。  
 

\vspace{0.2cm}

\textbf{“他必将硫磺、烈火、热风降给恶人,这是他们杯中的分。”(诗11:6)   }

\vspace{0.2cm}

这节经文预示着神的公义审判,恶人终将面临毁灭,就像所多玛和蛾摩拉所经历的审判一样。  


\vspace{0.2cm}

\textbf{现实应用: }


\begin{itemize}
    \item 当我们看到世界的腐败时,我们要记住:\textbf{神仍然掌权,恶人不会永远得势。} 
    \item  当我们感到软弱无助时,我们要相信:\textbf{神正在考验我们的信心,让我们在试炼中成长。  }
    \item 当我们面对不公时,不要靠自己伸冤,而要交托给神,因为他必施行公义。  
\end{itemize}



\subsection*{第三部分:神必保守义人,最终得胜(7节)}

\textbf{“因为耶和华是公义的,他喜爱公义;正直人必得见他的面。”(诗11:7)  }


\vspace{0.2cm}

\textbf{神的属性是公义的,他也喜爱公义。 }那些持守信仰、行走在正道上的人,最终会“得见他的面”,这意味着他们将在神的同在中得安慰,并在永恒里享受神的奖赏。  


\vspace{0.2cm}

\textbf{现实应用:}

我们在世上可能会经历苦难、挑战,甚至因信仰受到逼迫,但神应许:\textbf{正直人最终必得见神的面。} 这意味着:  

\begin{itemize}
    \item 神必保守我们到底,直到见他的日子。
    \item 我们的信心和行为不会白费,神看见一切,并且会奖赏那些忠心跟随他的人。
    \item 最终的胜利不是属于恶人,而是属于持守信仰的人。
\end{itemize}



\subsection*{结论:面对困境,我们要如何回应?}

\begin{enumerate}
    \item \textbf{不要逃避,而要信靠神。}即使世界的根基动摇,我们的信心仍然要建立在神身上。
    \item \textbf{相信神掌权,他正在试炼我们的信心。} 恶人不会长久得势,神必定施行公义。  
    \item \textbf{坚持行公义,最终必得见神的面。} 我们要活出正直的生命,因为神喜爱公义,他必保守属他的人。  
\end{enumerate}

愿我们在这个充满挑战的世界中,坚定地信靠神,成为公义的见证人! 



\subsection*{结束祷告}

\textbf{亲爱的天父,}

我们感谢你通过诗篇第11篇教导我们,在困境中如何依靠你。主啊,当世界的根基动摇时,求你坚固我们的信心,让我们不逃避,而是紧紧抓住你的应许。你在天上掌权,你察看世人,你必施行公义。求你赐给我们勇气,让我们在生活中行公义、存正直,不被世界的恶势力所动摇。愿你的国降临,愿你的公义彰显,直到我们见你的面。

奉主耶稣基督的名祷告,阿们!

%-----------------------------------------------------------------------------
\newpage
\section{诗篇第12篇:当真理被扭曲时,我们如何倚靠神?}
\subsection*{引言}

亲爱的弟兄姐妹,我们是否曾感到这个世界充满谎言,正直人的声音被压制,而恶人却趾高气扬?在职场、社交圈,甚至在一些国家的政策中,我们都能看到欺骗、诡诈、言语操纵的现象。面对这样的环境,我们该如何自处?\textbf{是随波逐流,还是坚守信仰?} 今天,我们一同学习\textbf{诗篇第12篇},看看大卫如何在谎言横行的时代依靠神,并从中得到力量。  




\subsection*{第一部分:当世界充满谎言时,义人何去何从?(1-2节)}


\hspace{0.4cm}\textbf{“耶和华啊,求你帮助,因为虔诚人断绝了,世人中间的忠信人没有了。”(诗12:1)  }

\textbf{“人人向邻舍说谎;他们说话是嘴唇油滑,心口不一。”(诗12:2)  }

\vspace{0.2cm}

\textbf{大卫在这里向神发出迫切的呼求:他看到敬虔人变少,忠信的人消失,世界充满谎言和虚伪。} 这不只是他的时代的现象,也是我们今天所经历的现实。

\vspace{0.2cm}

\textbf{现实应用: } 
在我们的生活中,是否也常遇到类似的情况?  

\begin{itemize}
     \item 在社交媒体上,谣言和虚假信息泛滥,人们只相信自己想听的,而不是事实。
     \item 在职场上,一些人为了利益不惜撒谎、欺骗、操控人心。
     \item 在社会中,一些人表面上说着公义的话,背后却隐藏着私心和诡诈。
\end{itemize}

当我们发现周围的人都在说谎、心口不一,我们可能会感到孤独,甚至想要妥协。但大卫提醒我们,\textbf{即使世人不忠信,神仍然是信实的,我们可以向他呼求! } 



\subsection*{第二部分:神必审判诡诈之人(3-4节)}

\hspace{0.4cm}\textbf{“凡油滑的嘴唇和夸大的舌头,耶和华必要剪除。”(诗12:3)  }

\textbf{“他们曾说:‘我们必能以舌头得胜;我们的嘴唇是我们的,谁能作我们的主呢?’”(诗12:4) } 

\vspace{0.2cm}

\textbf{恶人倚靠自己的言语,以为可以操纵一切,甚至狂妄地认为没有人能管他们。} 但神明确应许,\textbf{他必剪除诡诈之人的舌头,施行审判。  }

\vspace{0.2cm}

\textbf{现实应用:} 

\begin{itemize}
    \item 在社会中,我们看到许多权势者用言语操控大众,甚至扭曲真理来符合自己的目的。
    \item 在职场或学校里,有些人通过造谣、欺骗和夸张的言辞来获取好处。
    \item 甚至在一些信仰团体中,也有人以花言巧语引导人走向错误的道路。
\end{itemize}

但神不会沉默,他一定会施行公义!这提醒我们,不要因恶人的口才、操纵和诡诈而灰心,而要持守正直。  



\subsection*{第三部分:神的应许——他必保守他的子民(5-6节)}

\hspace{0.4cm}\textbf{“耶和华说:‘因困苦人的哀求,因穷乏人的叹息,我现在要起来,把他安置在他所切慕的稳妥之地。’”(诗12:5)  }

\vspace{0.2cm}

在充满欺骗和压迫的世界里,神没有忘记那些受欺压的人。\textbf{他必亲自起来,为困苦人伸冤,保守他们,使他们安稳。}  
  
\vspace{0.2cm}


\textbf{“耶和华的言语是纯净的言语,如同银子在泥炉中炼过七次。”(诗12:6)  
}

\vspace{0.2cm}

\textbf{与恶人的谎言相对,神的言语是完全真实、毫无掺杂的。} 他的应许永不落空,我们可以完全信靠他的话语。  

  
\vspace{0.2cm}

\textbf{现实应用:}

\begin{itemize}
    \item 当我们被诬陷、被误解时,要记住,神的应许是真实的,他必为我们伸冤。
    \item 当我们被世界的虚假信息迷惑时,要回到神的话语,他的真理才是最纯净的。
    \item 当我们感到信仰孤单时,要相信神看见一切,他必保护属他的人。
\end{itemize}



\subsection*{第四部分:神必保护义人到底(7-8节)}
\hspace{0.4cm}\textbf{“耶和华啊,你必保护他们;你必保佑他们永远脱离这世代的人。”(诗12:7)  }

\textbf{“下流人在世人中升高,就有恶人到处游行。”(诗12:8)  }

\vspace{0.2cm}

大卫最后宣告,\textbf{无论恶人如何猖狂,神必保守他的子民。} 世界的邪恶可能暂时得势,但义人最终会在神的保守中得胜。  

\vspace{0.2cm}

\textbf{现实应用:}

\begin{itemize}
    \item 当社会的道德标准下降时,我们不要随波逐流,而要持守神的标准。
    \item 当恶人暂时得势时,我们要记住,神的审判终究会来到。
    \item 当我们因信仰遭受逼迫时,要坚定地相信,神的保护永不止息。
\end{itemize}


\subsection*{结论:面对谎言和恶行,我们当如何回应?}

\begin{enumerate}
    \item \textbf{向神呼求,不随波逐流}(1-2节)——即使世界充满谎言,我们要做真理的见证人。  
    \item \textbf{相信神的公义,他必审判恶人}(3-4节)——不要被恶人一时的成功迷惑,神终将施行公义。
    \item \textbf{紧紧抓住神的应许}(5-6节)——神的话语纯净可信,我们可以完全倚靠。  
    \item \textbf{相信神必保守到底}(7-8节)——无论世界如何变坏,神必保护那些忠心跟随他的人。  
\end{enumerate}

愿我们在这充满谎言的世代里,持守信仰,倚靠神的真理,成为光和盐,活出荣耀神的生命!  



\subsection*{结束祷告}

\textbf{亲爱的天父,}

我们感谢你通过诗篇第12篇教导我们,在谎言充斥的世界中如何依靠你。主啊,求你帮助我们,不随从恶人的诡诈,而是持守你的真理。求你保护那些因公义受逼迫的人,使他们在你的应许中得安慰。愿你的话语成为我们的力量,让我们在黑暗中发光,成为诚实正直的见证人。

奉主耶稣基督的名祷告,阿们!

%----------------------------------------------------------------------------
\newpage

\section{诗篇第13篇:在等候中仍然信靠}  

\subsection*{引言}
亲爱的弟兄姐妹,\textbf{你是否曾有过向神呼求,却迟迟未得回应的经历?} 你是否在苦难中感到孤独,甚至怀疑神是否真的在听你的祷告?在诗篇第13篇中,大卫向神发出哀求,他的痛苦和挣扎正是许多信徒都会经历的。然而,他并没有停留在绝望中,而是选择了信靠神、赞美神。今天,我们一同学习这篇诗篇,看看当我们觉得神“沉默”时,该如何回应。  



\subsection*{第一部分:痛苦的呼喊--神啊,你在哪里?(1-2节)}
\hspace{0.4cm}\textbf{“耶和华啊,你忘记我要到几时呢?要到永远吗?你掩面不顾我要到几时呢?”(诗13:1)} 

\textbf{“我心里筹算,终日愁苦,要到几时呢?我的仇敌升高压制我要到几时呢?”(诗13:2)  }

\vspace{0.2cm}

\textbf{大卫的痛苦是双重的}:

\begin{enumerate}
    \item \textbf{他觉得神远离了他},好像神忘记了他的存在,不再垂听他的祷告。  
    \item \textbf{他被敌人逼迫},陷入苦难,内心充满忧虑和挣扎。 
\end{enumerate}

在这里,“要到几时呢?”重复了四次,表现出大卫深深的焦虑和困惑。他觉得自己的痛苦没有尽头,神似乎迟迟不回应他的哀求。  

\vspace{0.2cm}

\textbf{现实应用} 我们是否也曾有过这样的经历?

\begin{enumerate}
    \item 当我们祷告许久,却看不到神的回应,我们是否怀疑神是否仍然在乎我们?
    \item 当我们面对疾病、经济困难、家庭问题或学业压力时,我们是否曾感到神的沉默?
    \item 当恶人得势,而我们行公义却受苦,我们是否会质问神:“要到几时呢?”
\end{enumerate}


\textbf{大卫的呼喊并不意味着他不信神,而是表明他对神的信靠,正因为他知道神掌管一切,他才敢向神倾诉自己的痛苦。} 我们也应当如此,不要将痛苦压抑在心里,而要勇敢地向神倾诉。  


\subsection*{第二部分:迫切的祷告--求神施恩怜悯(3-4节)}

\hspace{0.4cm}\textbf{“耶和华—我的神啊,求你看顾我,应允我,使我眼目光明,免得我沉睡至死。”(诗13:3)}  
\textbf{“免得我的仇敌说:‘我胜了他!’免得我的敌人在我摇动的时候喜乐。”(诗13:4)  }

\vspace{0.2cm}

在大卫的痛苦中,他并没有放弃,而是\textbf{向神发出迫切的祷告}。他的祷告包含了三个关键请求:  

\begin{enumerate}
    \item \textbf{求神看顾和回应}——大卫知道,真正能改变他的处境的只有神,所以他直接向神呼求。  
    \item \textbf{求神使他眼目光明}——“使我眼目光明”不仅是求神给他力量,也是求神带给他属灵的亮光,使他能看到希望,而不至于陷入绝望。
    \item \textbf{求神维护公义}——大卫希望神伸手拯救他,使敌人无法夸口说自己战胜了神的子民。  
\end{enumerate}


\textbf{现实应用}  
\textbf{我们该如何在困境中向神祷告?}

\begin{enumerate}
    \item \textbf{要具体}——不要只是泛泛地说“神啊,帮助我”,而是向神表达你的具体需要,就像大卫一样。 
    \item \textbf{要相信神仍然掌权}——即使神还没有立刻回应,我们仍然要相信他在看顾。  
    \item \textbf{要求神赐下属灵的光照}——有时候,我们的问题不只是环境,而是我们的心态需要改变,我们需要神的智慧和眼光。  
\end{enumerate}

\textbf{我们在祷告中需要持守信心,即使感觉神的回应迟迟未到,也要坚持寻求他。}



\subsection*{第三部分:信心的转折--因神的慈爱而欢喜(5-6节)}
 
\hspace{0.4cm}\textbf{“但我倚靠你的慈爱;我的心因你的救恩快乐。”(诗13:5)}  

\textbf{“我要向耶和华歌唱,因他用厚恩待我。”(诗13:6)}  

\vspace{0.2cm}

\textbf{诗篇13篇的最后两节展现了戏剧性的转折:从绝望到信心,从哀求到赞美。}

\begin{itemize}
    \item \textbf{大卫选择倚靠神的慈爱},即使他的环境还没有改变,他仍然相信神不会离弃他。  
    \item \textbf{他因神的救恩而喜乐},尽管他尚未看见拯救的实现,但他已经用信心抓住了神的应许。
    \item \textbf{他开始赞美神},因为他深知,神的恩典比眼前的困难更真实、更长久。 
\end{itemize}


\textbf{现实应用 } 
\textbf{当我们还在苦难中,如何学习大卫的信心?}

\begin{enumerate}
    \item \textbf{选择相信神的慈爱}——不要让环境决定你的信心,而要让神的应许成为你生命的依靠。
    \item \textbf{因神的救恩喜乐}——即使环境没有改变,我们可以因耶稣基督已经成就的救恩而喜乐。  
    \item \textbf{用赞美代替埋怨}——当我们选择敬拜和感恩,我们的眼光会从问题转向神,信心也会被坚固。  
\end{enumerate}

\textbf{大卫的信心不是建立在环境的改变上,而是建立在神永恒不变的慈爱和信实上。我们也要如此,在等候中信靠神,在风暴中赞美他!}



\subsection*{结论:在等待神回应时,我们该如何自处?}

\begin{enumerate}
    \item \textbf{向神倾诉,而不是远离神}(1-2节)——即使感觉神沉默,也要继续寻求他。
    \item \textbf{迫切祷告,求神施恩}(3-4节)——不要放弃祷告,而要大胆向神求帮助。 
    \item \textbf{选择信靠神,因他的慈爱喜乐}(5-6节)——即使环境未变,我们仍然可以凭信心赞美神。  
\end{enumerate}

\textbf{亲爱的弟兄姐妹,神没有忘记你,他的应许仍然真实!即使在等待中,我们仍然可以信靠他、赞美他。愿我们都像大卫一样,在痛苦中仍然紧紧抓住神的慈爱,最终经历他奇妙的拯救!}

\subsection*{结束祷告}  
\textbf{亲爱的天父,}

我们感谢你通过诗篇13篇教导我们如何在等候中仍然信靠你。主啊,当我们觉得孤单、无助,甚至怀疑你的回应时,求你坚固我们的信心,让我们知道你仍然掌权。帮助我们不陷入绝望,而是学习像大卫一样,继续祷告、继续信靠、继续赞美。愿你的爱充满我们的心,让我们在风暴中仍然喜乐,因你的慈爱永不改变。

奉主耶稣基督的名祷告,阿们!

%-----------------------------------------------------------------------------
\newpage

\section{诗篇第14篇:愚顽人心 vs. 智慧人的信仰} 

\subsection*{引言}
亲爱的弟兄姐妹,\textbf{我们生活的世界是否越来越偏离神?} 我们是否常听见有人嘲笑信仰,说:“哪有什么神?” 当我们看到世上的罪恶、腐败、无知和骄傲时,我们可能会问:“神在哪里?公义在哪里?” 在\textbf{诗篇第14篇}中,大卫描述了世界的败坏、人心的愚妄,以及神的拯救。这篇诗篇不仅是对古代世界的描写,也是对当今社会的真实写照。今天,让我们一起学习这篇诗篇,看看作为基督徒,我们该如何在这邪恶世代中持守信仰。  


\subsection*{第一部分:愚顽人心的表现--否认神的存在(1节)}
% \textbf{“愚顽人心里说:‘没有神。’他们都是邪恶,行了可憎恶的事;没有一个人行善。”(诗14:1)}  

\subsubsection*{1. 什么是“愚顽人”?}
 
这里的“愚顽人”\textbf{不是指头脑愚蠢,而是指心灵刚硬、不愿承认神的人}。他们拒绝神的存在,因为他们想按照自己的私欲行事,不愿受神的约束。  

\subsubsection*{2. 否认神的后果}

\begin{itemize}
    \item \textbf{道德败坏}——“他们都是邪恶,行了可憎恶的事。”当人心远离神,道德标准会下降,罪恶泛滥。  
    \item \textbf{无善可言}——“没有一个人行善。” 这里不是说人无法行善事,而是说\textbf{在人真正的意义上,没有人能达到神的圣洁标准。} 
\end{itemize}


\subsubsection*{现实应用} 

\hspace{0.7cm}今天,我们也常听到人们说: 

\begin{itemize}
    \item “没有神,世界是自然发展的。”(否认神的创造)  
    \item “人生苦短,及时行乐。”(追求肉体享乐)  
    \item “道德是相对的,每个人可以有自己的标准。”(拒绝绝对真理)  
\end{itemize}

但当一个社会不承认神的时候,道德底线会不断下滑,犯罪、欺骗、暴力、贪婪就会增加。我们正在目睹这一现实:\textbf{人心刚硬,罪恶增多,道德败坏,这正是愚顽人否认神的后果。}  


\subsection*{第二部分:神察看世人--没有义人,人人都偏离(2-3节)}

% \hspace{0.4cm}\textbf{“耶和华从天上察看世人,要看有明白的没有,有寻求神的没有。”(诗14:2)}  

% \textbf{“他们都偏离正路,一同变为污秽;并没有行善的,连一个也没有。”(诗14:3)  }

\subsubsection*{1. 神的审视}
\textbf{神从天上察看世人,寻找敬畏他的人。} 但他看到的是什么?\textbf{没有义人,连一个都没有!} 这说明\textbf{全人类在神的圣洁标准面前都是罪人。}  

\subsubsection*{2. 人人都有罪}
  
这节经文也被\textbf{使徒保罗在《罗马书》3:10-12中引用,说明所有人都需要救赎。} \textbf{没有人天生是义人,我们都需要神的恩典和救恩。  }


\subsubsection*{现实应用 }

\begin{itemize}
    \item 我们不应该自以为义,以为自己比别人好 我们都曾在罪中,唯有神的恩典使我们成为义人。  
    \item 我们要成为寻求神的人,不要随从世界的败坏,而要追求神的公义。 
    \item 不要对世界的堕落感到惊讶,因为圣经早已揭示:人若离开神,必然走向败坏。
\end{itemize}

\subsection*{第三部分:恶人的愚妄--欺压神的子民(4-6节)}

% \begin{itemize}
%     \item “作恶的都没有知识吗?他们吞吃我的百姓,如同吃饭一样,并不求告耶和华。”(诗14:4)  
%     \item “他们在那里大大地害怕,因为神在义人的族类中。”(诗14:5)  
%     \item “你们叫困苦人的谋算变为羞辱;然而耶和华是他的避难所。”(诗14:6)  
% \end{itemize}

\subsubsection*{1. 恶人欺压义人}

大卫描述恶人\textbf{“吞吃”神的百姓,就像吃饭一样},说明他们毫无怜悯之心,毫无敬畏神的心。今天,我们也看到恶人欺压弱势群体,甚至逼迫信仰耶稣的人。  

\subsubsection*{2. 他们终将惧怕 }
大卫指出,恶人终有一天会“大大地害怕”,因为他们终将面对神的审判。\textbf{今天,世界可能嘲笑基督徒,但最终,神会显明他的公义!  }

\subsubsection*{3. 义人有神作避难所  }
尽管恶人羞辱义人,但\textbf{耶和华是义人的避难所。} 这提醒我们,\textbf{当世界反对我们时,我们可以投靠神,他是我们永远的保护。 }

% \vspace{0.2cm}

\subsubsection*{现实应用 }

\begin{itemize}
    \item 当我们因信仰受到逼迫,不要害怕,神是我们的避难所!  
    \item 当我们看到邪恶得势,不要灰心,神最终必审判恶人!
    \item 我们要倚靠神,而不是倚靠人的权势,因为最终得胜的是神的子民!
\end{itemize}


\subsection*{第四部分:救赎的盼望--愿神的救恩临到!(7节)}

% \hspace{0.4cm}\textbf{“但愿以色列的救恩从锡安而出!耶和华救回他被掳的子民,那时雅各要快乐,以色列要欢喜。”(诗14:7)  }

% \vspace{0.2cm}

在诗篇的最后,大卫发出了一个呼求:\textbf{愿神的救恩降临!} 他知道,唯有神能带来真正的拯救。 这节经文也预表了\textbf{耶稣基督的救恩——真正的拯救最终临到了世人。}  

% \vspace{0.2cm}

\textbf{现实应用 }

\begin{itemize}
    \item 无论世界多么黑暗,神的救恩已经成就,耶稣基督就是我们的盼望!
    \item 我们要持守信仰,耐心等候神的救赎和公义的彰显。
    \item 最终,我们会与神一同喜乐,世界的苦难只是暂时的!
\end{itemize}



\subsection*{结论:面对世界的败坏,我们该如何回应?}

\begin{enumerate}
    \item \textbf{不要效法愚顽人,不要随从世界的败坏}(1节)——我们要敬畏神,遵行他的道。 
    \item \textbf{承认自己的软弱,寻求神的义}(2-3节)——不要自以为义,而要仰赖基督的救恩。 
    \item \textbf{在恶人逼迫中倚靠神}(4-6节)——即使世界反对我们,神仍是我们的避难所。 
    \item \textbf{充满盼望,等候神的救恩}(7节)——最终,神的公义和救恩必定完全成就!  
\end{enumerate}

愿我们都能在这罪恶的世代中,持守信仰,敬畏神,成为智慧人,而不是愚顽人!  



\subsection*{结束祷告}
\textbf{亲爱的天父,}

我们感谢你赐下诗篇14篇,让我们明白世人的愚妄和败坏,但你仍然施行拯救。主啊,帮助我们在这个不敬畏你的世代中,仍然持守信仰,不随从世界的邪恶,而是做敬畏你、追求公义的智慧人。求你赐给我们勇气,在逼迫和挑战中,仍然倚靠你,等候你的救恩。愿你的公义早日彰显,愿你的国降临。

奉主耶稣基督的名祷告,阿们!

%-----------------------------------------------------------------------------
\newpage
\section{诗篇第15篇:谁能住在神的帐幕里——属灵标准}

\subsection*{引言}
亲爱的弟兄姐妹,\textbf{我们是否渴望与神亲近,住在他的同在中?} 在这个道德标准不断下滑的世界,什么样的人才能被神悦纳,进入他的圣所,享受他的同在?\textbf{诗篇第15篇}清楚地回答了这个问题。这篇诗篇是大卫的诗,他向神提出一个问题:\textbf{“耶和华啊,谁能寄居你的帐幕?谁能住在你的圣山?”} 这不仅是大卫的提问,也是我们每个人应该思考的问题:\textbf{我们是否符合神的要求?我们是否活出他所喜悦的生命?}  今天,我们将深入剖析诗篇15篇,看看神所喜悦的人的特质,并将其应用到我们的日常生活中。  


\subsection*{第一部分:问题--谁能住在神的帐幕里?(1节)}


\subsubsection*{1. 住在神的帐幕里意味着什么?  }

\hspace{0.4cm}\textbf{“帐幕”}:指的是摩西时代的会幕,代表神的同在。

\textbf{“圣山”}:通常指锡安山,象征神的居所。  

大卫的这个问题\textbf{不是指物理上的进入圣殿,而是指一个人与神的关系——谁能亲近神,与神同行,成为他所喜悦的人?}  

\subsubsection*{2. 我们今天的“帐幕”在哪里?}

在旧约,神的同在是在\textbf{会幕}和\textbf{圣殿}中。  
在新约,耶稣来了,\textbf{他就是神的帐幕(约1:14),并且信徒的身体成为圣灵的殿(林前6:19)}。  
今天,凡相信耶稣的人,都能住在神的同在中!但神仍然要求我们活出合他心意的生命。  


\subsection*{第二部分:答案--神所喜悦之人的品格(2-5节)}

\textbf{大卫在接下来的经文中,列出了神所喜悦之人的十个特质,可以分为三大方面:品格、言语和行为。
}


\subsubsection*{1. 品格上的正直(2节)}

神看重的是人的内心,而不仅仅是外在的宗教行为。
\textbf{“行为正直”}:生活光明磊落,不虚伪,不欺诈。  
\textbf{“做事公义”}:按照神的标准行事,不随从世俗风气。  
\textbf{“心里说实话”}:内心真实,不虚假,不欺骗。 
\vspace{0.2cm}

\textbf{现实应用} 

\begin{itemize}
    \item \textbf{在工作中},我们是否秉公行事,拒绝不义之财?  

    \item \textbf{在人际关系中},我们是否诚实守信,不说谎话?  

    \item \textbf{在生活中},我们是否表里如一,不做双面人?  

\end{itemize}




\subsubsection*{2. 言语上的谨慎(3节)}

\textbf{一个敬虔的人,不仅行为端正,更重要的是能控制自己的舌头。  }
\textbf{不说毁谤人的话}:不传播谣言,不恶意中伤他人。  
\textbf{不伤害朋友}:即使面对冲突,也不恶言相向。  
\textbf{不随伙毁谤}:不参与八卦,不跟风攻击别人。  


\textbf{现实应用} 

\begin{itemize}
    \item \textbf{社交媒体}:我们是否在网上随意评论、攻击他人?  

    \item \textbf{职场交流}:我们是否参与同事之间的流言蜚语?  

    \item \textbf{家庭生活}:我们是否常用负面的话伤害家人?  

\end{itemize}

\textbf{《雅各书》3:6说:“舌头就是火,在我们百体中,舌头是个罪恶的世界。” 神希望我们用舌头造就人,而不是毁坏人。}



\subsubsection*{3. 行为上的圣洁(4-5节)}
\hspace{0.4cm}\textbf{“他眼中藐视匪类,却尊重那敬畏耶和华的人。”(诗15:4上)}  

\begin{itemize}
    \item \textbf{远离恶人,不随从罪恶}:不以恶人为榜样,不羡慕他们的成功。  

    \item \textbf{尊重敬畏神的人}:重视属灵同伴,珍惜教会的弟兄姐妹。  

\end{itemize}

\textbf{“他发了誓,虽然自己吃亏,也不更改。”(诗15:4下)}

\begin{itemize}
    \item 信守承诺,宁可吃亏,也不说变卦。  

    \item 真实可靠,不因利益而违背承诺。  

\end{itemize}

\textbf{“他不放债取利,不受贿赂以害无辜。”(诗15:5上)}

\begin{itemize}
    \item 不因金钱而行不义之事,宁可亏己,也不亏人。  

    \item 公平公正,不因贿赂而损害他人。  

\end{itemize}

\textbf{现实应用}  
\begin{itemize}
    \item \textbf{商业诚信}:是否愿意牺牲利润,维护诚信?  

    \item \textbf{人际关系}:是否愿意为了正义而站出来,而不是屈服于不义?  

    \item \textbf{理财观念}:是否爱财如命,甚至不择手段地获取财富?  

\end{itemize}



\subsection*{第三部分:结论--这样的人必永不动摇(5节下)} 

神应许,凡是符合这些品格、言语、行为标准的人,必然\textbf{在他的同在中稳固不动摇}。\textbf{他们的信仰不会因外界环境而动摇,他们的生命根基是建立在神的公义之上。} 

\subsection*{第四部分:我们的回应--如何成为这样的人? }
 
\hspace{0.6cm}诗篇15篇的标准看似很高,我们是否能够完全做到呢?答案是:\textbf{单靠自己,我们做不到,但靠着耶稣基督,我们可以达到!  }

\textbf{我们要依靠基督的救赎}:耶稣是完全符合诗篇15篇标准的那一位,他为我们的罪死在十字架上,使我们能因信称义。  
\textbf{我们要被圣灵更新}:靠着圣灵的能力,我们才能活出诗篇15篇的生命。  
\textbf{我们要每天追求圣洁}:靠近神,读经祷告,与敬虔的信徒同行。  


\subsection*{结束祷告}
\textbf{亲爱的天父,}

我们感谢你赐下诗篇15篇,让我们明白你所喜悦的生命是怎样的。主啊,帮助我们活出正直、公义、诚实、敬畏你的生活,让我们的言语、行为都荣耀你。我们承认,我们靠自己无法达到这个标准,但求你用圣灵更新我们,使我们越来越像基督。愿你坚立我们的信仰,使我们在你的同在中,永不动摇!

奉主耶稣基督的名祷告,阿们!
%-----------------------------------------------------------------------------
\newpage
\section{诗篇第16篇:在神里面得着满足与保障}

\subsection*{引言:真正的安全感在哪里? }
亲爱的弟兄姐妹,\textbf{在这个充满不确定性的世界,我们如何才能得到真正的安全感和满足?} 
\begin{itemize}
    \item 经济环境起伏不定,人的财富随时可能失去。
    \item 人际关系复杂,朋友和亲人也可能离开我们。  
    \item 健康无法保证,生命脆弱,我们随时可能面临疾病和死亡。
\end{itemize}
然而,大卫在\textbf{诗篇16篇}中告诉我们,真正的安全感和满足,不是来自世界,而是来自\textbf{神自己}。\textbf{如果我们的生命建立在神里面,就能拥有真正的喜乐、平安和永恒的保障。}今天,我们将深入剖析这篇诗篇,看看大卫如何在神里面找到满足,并如何应用到我们的实际生活。  

\subsection*{第一部分:神是我们的保障(1-4节)——全然信靠神}

\subsubsection*{1. 投靠神,而不是世界 }

\hspace{0.6cm}大卫深知,人的力量有限,唯有\textbf{投靠神,才能得到真正的保护}。  
\textbf{世界的保障是短暂的}:财富、权力、关系都不能真正拯救我们。  
\textbf{神的保障是永恒的}:他是全能者,掌管一切,赐下真正的平安。  
\vspace{0.2cm}

\textbf{现实应用} 

\hspace{0.6cm}当我们面临经济压力时,我们是投靠金钱,还是信靠神的供应?

\hspace{0.6cm}当我们面临人生抉择时,我们是倚靠人的帮助,还是寻求神的引导?  


\subsubsection*{2. 以神为至宝,而不是偶像(2-4节) }  

\hspace{0.6cm}\textbf{“我的好处不在你以外”}:真正的祝福不在于金钱、地位,而是神自己。 

\textbf{远离偶像(诗16:4)}:世界上有许多“偶像”,如金钱、权力、享乐,这些都会使人远离神。  
\vspace{0.2cm}

\textbf{现实应用} 

\hspace{0.6cm}我们是否把\textbf{工作、财富、成功}当作偶像,认为它们比神更重要?  

\hspace{0.6cm}我们是否愿意放下\textbf{世界的诱惑},单单以神为满足?  

\textbf{耶稣说:“人若赚得全世界,赔上自己的生命,有什么益处呢?”(太16:26)}  

\textbf{让我们像大卫一样,把神当作我们生命的至宝! }

\subsection*{第二部分:神是我们的产业(5-8节)——以神为满足 }

\subsubsection*{1. 神是我们的产业  }
\hspace{0.6cm}在旧约中,以色列各支派分得土地作为产业。\textbf{但大卫说,他最宝贵的产业不是土地,而是神自己!}  
今天,我们的“产业”是什么?是房子、存款,还是神的同在?  
\vspace{0.2cm}

\textbf{现实应用} 

\hspace{0.6cm}如果有一天,我们失去工作、财富、健康,我们还会满足吗? 

\hspace{0.6cm}我们的满足,是否建立在神的同在和应许之上?

\subsubsection*{2. 神赐下满足和引导(6-7节) }


\hspace{0.6cm}\textbf{大卫的喜乐不取决于环境,而是神的恩典。} 

\textbf{神的带领是最美好的}(7节)。我们常常担心未来,但神已经为我们量定最美好的道路。  

\vspace{0.2cm}

\textbf{现实应用} 

\hspace{0.6cm}当我们不满意自己的生活时,是否能相信神已经量给我们最好的? 

\hspace{0.6cm}我们是否愿意顺服神的带领,而不是抱怨自己的环境?  

\subsection*{第三部分:神是我们的喜乐与永恒的盼望(9-11节)——永远与神同在 } 

\subsubsection*{1. 以神为喜乐}

\hspace{0.6cm}现代人追求快乐,但\textbf{真正的喜乐不是来自外在,而是来自神的同在}。  

“在你面前有满足的喜乐,在你右手中有永远的福乐。”(诗16:11)  

\vspace{0.2cm}


\textbf{现实应用} 

\hspace{0.6cm}我们的快乐是否建立在神的同在,而不是物质的享受?  

\hspace{0.6cm}在困难中,我们是否仍然能靠主喜乐?  

\subsubsection*{2. 复活的盼望(10节)——耶稣基督的应验 }

\hspace{0.6cm}这节经文在新约中被彼得引用,指向\textbf{耶稣基督的复活}(徒2:27)! 

耶稣从死里复活,使我们有了\textbf{永恒的生命盼望}。  

大卫因着信心,知道神必保守他的生命,使他最终进入永恒的荣耀。

\vspace{0.2cm}

\textbf{现实应用}

\hspace{0.6cm}我们是否相信,死亡不是终点,而是进入神的荣耀? 


\hspace{0.6cm}我们是否真正拥有复活的盼望,而不是惧怕未来?  

\subsection*{第四部分:我们的回应——如何活出诗篇16篇的信仰?  }

\begin{enumerate}
    \item \textbf{选择投靠神,而不是世界}——无论环境如何变化,我们都信靠神的保护。  
    \item \textbf{拒绝偶像,单单以神为满足}——不让金钱、事业、享乐成为我们的主。  

    \item \textbf{相信神的带领,感恩他所赐的一切}——不抱怨,而是顺服神的安排。  

    \item \textbf{活在喜乐和永恒的盼望中}——无论环境如何,靠主喜乐,并盼望基督的再来!  

    
\end{enumerate}

\subsection*{结束祷告}
\textbf{亲爱的天父,}

亲爱的天父,我们感谢你赐下诗篇16篇,让我们看见,在你里面才有真正的满足和保障。主啊,求你帮助我们,不倚靠世界,而是单单投靠你。让我们的心不被金钱、权力和世俗的偶像捆绑,而是以你为至宝。主啊,帮助我们在你的引导中喜乐,即使在困难中,也能靠主得胜。愿你的复活盼望充满我们,使我们每天都活在你的荣耀中。

奉主耶稣基督的名祷告,阿们!

%----------------------------------------------------------------------------
\newpage
\section{诗篇第17篇:祷告中寻求公义与神的同在}

\subsection*{引言:当我们遭受不公时,如何回应? }
亲爱的弟兄姐妹,在我们人生的旅途中,总会遇到误解、攻击,甚至不公义的对待。\textbf{在职场上},或许我们因正直而遭受排挤;\textbf{在人际关系中},或许我们因说真话而被人冷落;\textbf{在信仰上},我们可能因坚持真理而被世人讥讽。当这些事情发生时,我们该如何回应?\textbf{大卫在诗篇17篇中给我们做了一个榜样——他向神祷告,寻求神的公义和同在,而不是靠自己的力量复仇。} 今天,我们将深入剖析这篇诗篇,并思考如何在现实生活中应用,让我们的信仰更坚定,生命更有力量。  



\subsection*{第一部分:向神呼求公义(1-5节)——祷告中的清白之心  }
\textbf{“耶和华啊,求你听闻公义的话,留心听我呼吁的声音;求你侧耳听我这无诡诈嘴唇的祷告。”(诗17:1)}  
\subsubsection*{1. 在困境中,我们首先要向神祷告  }
\hspace{0.6cm}大卫没有先向人申诉,而是向神倾诉自己的冤屈。  他的祷告是公义的(1节),是诚实无诡诈的(2节)。  当我们遇到不公义的事情时,我们是否第一时间求告神?

\vspace{0.2cm}

\textbf{现实应用}

\hspace{0.6cm}当同事恶意诽谤我们时,我们是否向神倾诉,而不是直接反击?  

\hspace{0.6cm}当朋友误解我们时,我们是否选择沉静祷告,而不是与人争论?  

\subsubsection*{2. 让神鉴察我们的心(3-5节)  }
\textbf{“你已经试验我的心;你在夜间鉴察我;你熬炼我,却找不着什么。”(诗17:3)}  

\vspace{0.2cm}

大卫勇敢地让神来察验他的内心,证明自己是清白的。这提醒我们:在求神伸冤之前,我们要先反省自己,看看是否有隐藏的罪。神看重的不仅是外在的行为,更是内心的态度。

\vspace{0.2cm}

\textbf{现实应用}

\hspace{0.6cm}当我们遭遇不公时,是否愿意让神先检查我们的心? 

\hspace{0.6cm}我们是否有隐藏的骄傲、苦毒或不愿饶恕的心?

\vspace{0.2cm}

\textbf{耶稣说:“清心的人有福了,因为他们必得见神。”(太5:8)}  

愿我们在祷告中持守清洁的心,让神的公义为我们申冤!  



\subsection*{第二部分:寻求神的保护(6-12节)——神是我们的避难所  }

\subsubsection*{1. 神是我们随时的帮助(6-7节)  }
\hspace{0.6cm}大卫在危难中不断呼求神,因为他深知神是信实的帮助者。 他称神为“\textbf{拯救投靠之人的神}”(7节),表明神必不会让倚靠他的人失望。

\vspace{0.2cm}


\textbf{现实应用}

\hspace{0.6cm}当我们在工作中受排挤时,我们是否相信神仍然掌权?  

\hspace{0.6cm}当我们在人生低谷时,我们是否愿意寻求神的保护,而不是靠自己?  

\subsubsection*{2. 神保护我们如同保护眼中的瞳人(8节)}
  
\textbf{“求你保护我,如同保护眼中的瞳人;将我隐藏在你翅膀的荫下。”(诗17:8)}

\textbf{“眼中的瞳人”}是人体最敏感、最珍贵的部位,神用这个比喻说明我们对他而言极其宝贵。神不会忽视我们的困境,他会细心地保护我们,不让我们受仇敌的伤害。  
\vspace{0.2cm}

\textbf{现实应用}

\hspace{0.6cm}当我们觉得自己渺小、无助时,我们要记得:在神眼中,我们是极其宝贵的!

\hspace{0.6cm}当我们面对危险或试探时,我们要藏在神的翅膀荫下,而不是依靠自己的智慧或力量。  

\subsubsection*{3. 敌人虽恶毒,但神掌权(9-12节)  }
\hspace{0.6cm}大卫描述敌人如狮子一样凶猛(12节),但他没有惧怕,因为神是他的保护者。今天,我们也可能面临来自世界的压力,但神的保护永远可靠! \textbf{“神若帮助我们,谁能抵挡我们呢?”(罗8:31)}  



\subsection*{第三部分:盼望神的公义与荣耀(13-15节)——满足于神的同在  }
\textbf{“至于我,我必在义中见你的面;我醒了的时候,得见你的形象,就心满意足。”(诗17:15)}  

\subsubsection*{1. 交托仇敌,仰望神的伸冤(13-14节)}
\hspace{0.6cm}大卫没有自己复仇,而是求神起来,按公义惩治恶人。今天,我们是否愿意把冤屈交托给神,还是想靠自己的方式解决?  \textbf{“主说:‘伸冤在我,我必报应。’”(罗12:19)}  

\vspace{0.2cm}

\textbf{现实应用}

\hspace{0.6cm}当有人欺负我们时,我们是否愿意等候神的公义,而不是急于报复?

\hspace{0.6cm}当我们遭受不公时,我们是否能相信神的时间和方式才是最完美的?  

\subsubsection*{2. 在神的同在中得满足(15节) }
\hspace{0.6cm}\textbf{真正的满足,不是来自世界的成功,而是来自神的同在。}“我醒了的时候,得见你的形象,就心满意足。” 这意味着\textbf{大卫在今生和永恒中都以神为喜乐的源头。}  

\vspace{0.2cm}

\textbf{现实应用}
\hspace{0.6cm}我们是否用世界的标准来衡量成功,还是以神的同在为最大的喜乐?  
\hspace{0.6cm}我们每天的生活,是否寻求更多亲近神,而不是被世俗的事物填满?  



\subsection*{结论:如何应用诗篇17篇?}

\begin{enumerate}
    \item 当面对不公时,先向神祷告,而不是向人诉苦或报复。
    \item 让神鉴察我们的心,确保我们行事正直、清洁无愧。
    \item 在危难中信靠神的保护,相信他必不会忽略我们的困境。
    \item 把伸冤交托给神,等候他的时间和公义审判。
    \item 在神的同在中得满足,不倚靠世界的成功来定义自己的人生。
\end{enumerate}

\subsection*{结束祷告}
\textbf{亲爱的天父,}

亲爱的天父,我们感谢你藉着诗篇17篇教导我们,在困境中如何寻求你的公义与保护。主啊,帮助我们在不公中信靠你,而不是倚靠自己的力量。让我们成为清心、正直的人,在你的翅膀荫下得着真正的安全。愿我们一生以你的同在为满足,并活出你的荣耀。

奉主耶稣基督的名祷告,阿们!
%-----------------------------------------------------------------------------
\newpage
\section{诗篇第18篇:依靠神得胜的人生}

\subsection*{引言:当困境来临时,你依靠谁?}

诗篇第18篇是大卫在经历困境后对神的颂赞。他回顾自己如何被仇敌追赶,但神成为他的力量和拯救。这篇诗篇不仅是大卫的见证,更是我们每个人在生活中可以学习的信仰功课。今天,我们来探讨这篇诗篇的核心信息,并应用到实际生活中。

\subsection*{第一部分:神是我们的保障(1-6节)}

\textbf{“耶和华是我的岩石,我的山寨,我的救主。”(诗篇 18:2)}

\subsubsection*{1. 认识神是我们的避难所}
诗篇18:2中,大卫用了多个比喻来描述神:
\begin{itemize}
    \item “岩石”——象征稳固和不可动摇。
    \item “山寨”——代表安全和保护。
    \item “救主”——说明神是施行拯救的那一位。
\end{itemize}

当我们面对挑战时,我们依靠的是什么?是金钱、权力,还是人际关系?大卫告诉我们,唯有神是我们真正的保障。

\subsubsection*{2. 祷告是通往得胜的关键}
\textbf{“我在急难中求告耶和华,向我的神呼求。”(诗篇 18:6)}

当我们遇到困难时,我们的第一反应是什么?是焦虑、抱怨,还是来到神面前祷告?大卫的生命见证提醒我们,祷告是经历神拯救的重要途径。

\subsection*{第二部分:神是我们得胜的力量(17-29节)}

\textbf{“你必点着我的灯;耶和华我的神必照明我的黑暗。”(诗篇 18:28)}

\subsubsection*{1. 神使软弱变为刚强}
\textbf{“你曾以力量束我的腰,使我能争战。”(诗篇 18:39)}

大卫不是靠自己得胜,而是靠着神所赐的力量。现实生活中,我们也会遇到困难,但神应许赐给我们足够的力量去面对挑战。

\subsubsection*{2. 神带领我们走向宽阔之地}
\textbf{“你使我的脚快如母鹿的蹄,又使我在高处安稳。”(诗篇 18:33)}

母鹿的蹄能在陡峭的山岩上稳行无阻,象征着神带领我们在困难中仍能稳行。这提醒我们,即便环境险恶,只要依靠神,我们就能稳健前行。

\subsection*{第三部分:神的信实与得胜的应许(30-50节)}

\textbf{“至于神,他的道是完全的,耶和华的话是炼净的。”(诗篇 18:30)}

\subsubsection*{1. 神的信实可靠}
大卫在经历危险后,回顾神的作为,并宣告神的话语是绝对可靠的。现实生活中,我们是否相信神的话语能够引导我们的道路?

\subsubsection*{2. 在胜利中归荣耀给神}
\textbf{“因此我要在外邦中称谢你,歌颂你的名。”(诗篇 18:49)}

当我们经历神的拯救后,我们是否愿意向人见证神的作为?一个真正敬畏神的人,不仅自己经历神的恩典,还愿意向他人分享神的信实。

\subsection*{结论:信靠神,奔跑人生的道路}

诗篇18篇教导我们:
\begin{itemize}
    \item 神是我们的避难所,在困难中我们可以依靠他。
    \item 神赐我们得胜的力量,使我们能勇敢面对人生挑战。
    \item 神是信实的主,我们应当信靠他,并见证他的作为。
\end{itemize}

愿我们都能像大卫一样,在人生的旅程中坚定依靠神,经历他的大能!

\subsection*{结束祷告}

\textbf{亲爱的天父,}

感谢你在我们生命中的带领。你是我们的避难所,是我们随时的帮助。求你帮助我们在困难时不失去信心,而是坚定依靠你。愿你加添我们的力量,使我们可以奔跑人生的道路,并在每一次得胜后归荣耀给你。

奉主耶稣基督的名祷告,阿们!
%-----------------------------------------------------------------------------
\newpage
\section{诗篇第19篇:神的启示与生命的智慧}

\subsection*{引言}

亲爱的弟兄姐妹,今天我们一起来思想《诗篇》第19篇,这是大卫的一首伟大诗歌,它向我们揭示了神的两大启示方式——自然启示与特殊启示,并引导我们进入敬畏神、顺服神的智慧之道。让我们带着渴慕的心,一同进入神的话语。

\subsection*{第一部分:自然启示——从创造看见神(诗19:1-6)}

大卫在诗篇的开篇写道:

\begin{quote}
诸天述说神的荣耀,穹苍传扬他的手段。(诗19:1)
\end{quote}

天地万物本身就是神荣耀的见证者。宇宙的浩瀚、太阳的运转、四季的更替,无不彰显造物主的奇妙智慧。今天,科学愈发发达,我们看到DNA的精妙编码、宇宙的精细调节常数,这些都指向一位智慧且有秩序的创造主。

在生活中,我们是否曾因忙碌而忽略了神在大自然中的启示?每天清晨,当我们看见日出,是否想到神的信实?当我们呼吸新鲜空气,是否想到这是神所赐的恩典?让我们学会在生活的细节中发现神的荣耀。

\subsection*{第二部分:特殊启示——神的话语(诗19:7-11)}

大自然向我们启示神的存在,但要认识他的心意,我们需要更清晰的指引,那就是神的话语。诗篇19:7-9说:

\begin{quote}
耶和华的律法全备,能苏醒人心;耶和华的法度确定,能使愚人有智慧……(诗19:7-9)
\end{quote}

这里,大卫用六个不同的词汇(律法、法度、训词、命令、道理、典章)形容神的话语,并指出它的功效:苏醒人心、使人有智慧、令人喜乐、明亮眼目。这告诉我们,圣经不仅仅是一本书,它是生命的指引,是心灵的良药。

在现实生活中,我们常常面对各种价值观的冲击,甚至会迷失方向。比如,现代社会强调成功、金钱、名望,但神的话语告诉我们:“敬畏耶和华是智慧的开端”(箴9:10)。如果我们每天都花时间阅读并遵行圣经,我们的生命必然会被更新,拥有真正的智慧和喜乐。

\subsection*{第三部分:人心的反思与祷告(诗19:12-14)}

面对神的启示,大卫接着反思自己的内心:

\begin{quote}
谁能知道自己的错失呢?愿你赦免我隐而未现的过错。(诗19:12)
\end{quote}

人最大的盲点就是无法看清自己的罪,因此需要神的光照。在忙碌的生活中,我们是否也有隐而未现的骄傲、自私、冷漠?当我们亲近神,求神鉴察,我们才能被更新。

最后,大卫作出了美好的祷告:

\begin{quote}
耶和华我的磐石,我的救赎主啊,愿我口中的言语,心里的意念,在你面前蒙悦纳。(诗19:14)
\end{quote}

愿这也成为我们的祷告!

\subsection*{结论与应用}

弟兄姐妹,今天我们学习了《诗篇》第19篇,明白神透过自然和圣经向我们启示自己,并提醒我们反思生命。让我们:

\begin{enumerate}
    \item 在生活中更多留意神的创造,培养敬畏神的心。
    \item 坚持每天读经,让神的话语成为我们脚前的灯。
    \item 祷告求神鉴察我们的内心,使我们活出讨他喜悦的生命。
\end{enumerate}

愿神的话语成为我们生命的指南,使我们更加亲近他。阿们!

\subsection*{结束祷告}
\textbf{亲爱的天父,}

感谢你借着大自然向我们彰显你的荣耀,又借着圣经向我们启示你的心意。求你使我们有敏锐的心,看见你的作为;赐给我们渴慕你话语的心,活出智慧的生命。主啊,也求你鉴察我们的内心,洁净我们的罪,使我们的言语和心思都蒙你悦纳。愿我们的一生荣耀你。

奉主耶稣基督的名祷告,阿们!

%---------------------------------------------------------------------------
\newpage
\section{诗篇第20篇:信靠神的得胜之路}

\subsection*{引言:信靠神是得胜的关键}

诗篇第20篇是一首为君王在战争前的祷告诗,表达了对神的信靠和依赖。它不仅适用于大卫时代的战争场景,也适用于我们今天面临的各样挑战。人生充满战斗——无论是事业、家庭、健康,还是属灵的争战,胜败的关键在于我们是否真正信靠神。

\subsection*{第一部分:神是我们在患难中的依靠(1-3节)}

\textbf{“愿耶和华在你遭难的日子应允你;愿名为雅各神的高举你。”(诗篇20:1)}

\subsubsection*{1. 神在患难中应允我们}
人生中不可避免地会遇到困难,但诗篇20:1告诉我们,神在我们遭难的日子应允我们。这意味着:
\begin{itemize}
    \item 神听我们的祷告,他不会忽略我们的呼求。
    \item 他是我们的避难所和力量,在困境中扶持我们。
    \item 依靠神,我们能在风暴中保持平安。
\end{itemize}

\subsubsection*{2. 神是我们的保障和帮助}
\textbf{“愿他从圣所救助你,从锡安坚固你。”(诗篇20:2)}

圣所和锡安象征着神的同在和能力。无论我们面对什么困境,神的帮助是我们真正的保障,而不是世界上的财富、权力或人际关系。

\subsubsection*{3. 神纪念我们的献祭和信心}
\textbf{“愿他记念你的一切供献,悦纳你的燔祭。”(诗篇20:3)}

这节经文提醒我们,敬拜和献祭(包括祷告、顺服、奉献)在神面前是蒙悦纳的。当我们忠心地跟随神,他必顾念我们的需要。

\subsection*{第二部分:信靠神而不是世上的能力(4-8节)}

\subsubsection*{1. 人的计划 vs. 神的旨意}
\textbf{“愿他照你的心愿赐给你,成就你的一切筹算。”(诗篇20:4)}

我们常常有自己的计划,但真正的得胜来自神的旨意。信靠神意味着愿意顺服他的带领,即使他的道路与我们的想法不同。

\subsubsection*{2. 胜利的喜乐来自神}
\textbf{“我们要因你的得救夸胜,要奉我们神的名竖立旌旗。”(诗篇20:5)}

当我们依靠神,得胜就成为见证,我们会因他的救恩而喜乐,不是夸耀自己,而是荣耀神。

\subsubsection*{3. 依靠神的名而非人力}
\textbf{“有人靠车,有人靠马,但我们要提到耶和华我们神的名。”(诗篇20:7)}

在古代,车马代表军事实力,今天它可以代表金钱、科技、权势等。然而,这些都不能带来真正的保障,只有神的名才能使我们站立得稳。

\subsubsection*{4. 结局的对比:信靠神 vs. 依靠世界}
\textbf{“他们都屈身仆倒,我们却起来,立得正直。”(诗篇20:8)}

依靠自己的人最终会失败,但信靠神的人会被扶持,站立得稳。

\subsection*{第三部分:向神祷告并相信他掌权(9节)}

\textbf{“耶和华啊,求你救护;我们呼求的时候,愿王应允我们。”(诗篇20:9)}

本节总结了整篇诗篇:
\begin{itemize}
    \item 祷告是得胜的关键——我们要持续向神呼求。
    \item 神是我们的君王,他掌管一切。
    \item 只要我们信靠他,他必带领我们进入得胜。
\end{itemize}

\subsection*{结论:信靠神,得胜之路}

诗篇20篇提醒我们,真正的得胜不在于人的力量,而在于神的能力:
\begin{itemize}
    \item 在患难中,神是我们的依靠。
    \item 胜利的秘诀在于信靠神而不是世界的资源。
    \item 通过祷告,我们可以经历神的掌权和救赎。
\end{itemize}

愿我们都能做一个真正依靠神的人,走上得胜的道路!

\subsection*{结束祷告}

\textbf{亲爱的天父,}

感谢你赐下你的话语,教导我们信靠你是得胜的关键。求你帮助我们在困境中仰望你,不倚靠自己的聪明和世界的资源,而是单单信靠你的名。愿你在我们的生命中掌权,使我们刚强站立,经历你的得胜。

奉主耶稣基督的名祷告,阿们!
%---------------------------------------------------------------------------
\newpage
\section{诗篇第21篇:君王的喜乐与信靠}

\subsection*{引言:喜乐的根源}

诗篇第21篇是大卫对神的感恩之歌,表达了他在神面前的信靠与喜乐。这首诗歌不仅适用于大卫,也对所有信靠神的人提供了深刻的启示。我们如何在生活中经历神的保守与祝福,并从中得到真正的喜乐?今天,我们一同来探讨这篇诗篇。

\subsection*{第一部分:君王因神的力量而欢喜(1-6节)}

\textbf{“耶和华啊,王必因你的能力欢喜;因你的救恩,他的快乐何其大!”(诗篇 21:1)}

\subsubsection*{1. 喜乐的源头是神的能力}

大卫王的喜乐不是来自他的军队、财富或地位,而是来自神的能力。现实生活中,我们常常以成功、物质或他人的认可作为喜乐的来源,但这些都是短暂的。唯有神的能力才能带给我们真正的满足。

\subsubsection*{2. 神垂听祷告,赐下所求所想}

\textbf{“他心里所愿的,你已经赐给他;他嘴唇所求的,你未尝不应允。”(诗篇 21:2)}

这节经文提醒我们,神乐意回应敬畏他之人的祷告。当我们按照神的旨意祷告,他就会赐下最好的给我们。因此,我们要养成祷告的习惯,并信靠神的供应。

\subsubsection*{3. 神赐下永恒的福分}

\textbf{“你以美福迎接他,把精金的冠冕戴在他头上。”(诗篇 21:3)}

神不仅满足大卫的短期需求,还赐给他永恒的福分。今天,神也为每一个信靠他的人预备了天上的赏赐和荣耀。我们的信仰不仅关乎今生,更关乎永恒。

\subsection*{第二部分:恶人的结局(7-12节)}

\textbf{“王倚靠耶和华,因至高者的慈爱必不摇动。”(诗篇 21:7)}

\subsubsection*{1. 义人坚立,恶人灭亡}

大卫因倚靠神而稳固,而敌人却因悖逆神而毁灭。今天,我们在世界上可能面临挑战和试探,但只要我们紧紧依靠神,就不会被动摇。

\subsubsection*{2. 神是公义的审判者}

\textbf{“你要叫他们如在烈火的炉中;耶和华发怒的时候,要吞灭他们。”(诗篇 21:9)}

神的公义必然临到恶人。虽然我们看到世上仍有不公,但神的审判不会缺席。我们的责任是持守信仰,等待神的公义显明。

\subsection*{第三部分:信靠神,得胜有余(13节)}

\textbf{“耶和华啊,愿你因自己的能力显为至高!这样,我们就唱诗,歌颂你的大能。”(诗篇 21:13)}

诗篇以赞美神的能力作为结束。无论是得胜还是挑战,敬畏神的人最终都能经历神的大能。因此,让我们在生活中常常赞美神,信靠他的带领。

\subsection*{结论:真实的喜乐来自神}

\begin{itemize}
    \item \textbf{真正的喜乐来自神的能力,而非世界的短暂满足。}
    \item \textbf{信靠神的人必然经历他的保守和祝福。}
    \item \textbf{恶人终将面对神的审判,而义人要持守信仰。}
\end{itemize}

让我们效法大卫,学习在神的能力中喜乐,持守信心,最终迎接神的荣耀。

\subsection*{结束祷告}

\textbf{亲爱的天父,}

感谢你赐下诗篇21篇,让我们看到真正的喜乐源自你,而非世界的事物。求你帮助我们在生活中依靠你的力量,不被环境左右,始终持守信仰。主啊,我们愿意信靠你的带领,远离恶人的道路,最终得着你的祝福。

奉主耶稣基督的名祷告,阿们!
%-----------------------------------------------------------------------------
\newpage
\section{诗篇第22篇:从绝望到得胜}

\subsection*{引言:从苦难到荣耀}

诗篇第22篇是一首弥赛亚诗篇,表达了大卫在极度痛苦中的呼求,同时也预表了耶稣基督在十字架上的经历。这篇诗篇不仅让我们看到人在痛苦中的挣扎,也启示了神的信实与救赎。今天,我们将通过这篇诗篇来探讨如何在苦难中信靠神,并最终经历他的拯救。

\subsection*{第一部分:痛苦中的呼求(1-21节)}

\textbf{“我的神,我的神!为什么离弃我?”(诗篇 22:1)}

\subsubsection*{1. 深切的痛苦与孤独}
大卫在极度的痛苦中呼求神,他感到自己被神遗弃。这节经文也正是耶稣在十字架上所呼喊的话(马太福音27:46),预示着基督的受难。许多时候,我们在困境中也会有类似的感受,觉得神远离了我们。

\textbf{实际应用:}当我们面临人生的低谷时,我们要学习像大卫一样向神倾诉,而不是远离神。神允许我们表达痛苦,但他的同在不会真正离开我们。

\subsubsection*{2. 人的嘲笑与攻击}
\textbf{“凡看见我的都嗤笑我。”(诗篇 22:7)}

大卫描述了自己被人讥笑、羞辱和逼迫,而耶稣在被钉十字架时也经历了同样的羞辱(马可福音15:29-32)。这提醒我们,跟随神并不意味着一帆风顺,反而可能遭遇世人的误解和攻击。

\textbf{实际应用:}当我们因信仰而被嘲笑时,不要灰心。要坚定信靠神,知道基督也曾经历同样的苦难。

\subsubsection*{3. 祷告与信靠}
\textbf{“求你快快帮助我!”(诗篇 22:19)}

虽然大卫感到痛苦,他仍然向神祷告,表达对神的信靠。这提醒我们,在最黑暗的时刻,我们仍要坚持祷告,因为神的拯救是信实的。

\textbf{实际应用:}当困境临到,我们要学习坚持祷告,不要让环境影响我们的信心。

\subsection*{第二部分:神的拯救与得胜(22-31节)}

\textbf{“我要将你的名传于我的弟兄,在会中我要赞美你。”(诗篇 22:22)}

\subsubsection*{1. 从苦难到荣耀}
大卫的呼求在后半部分转变为赞美,显示出神的拯救临到。同样,耶稣的十字架苦难最终带来了复活和得胜。

\textbf{实际应用:}我们可能会经历苦难,但最终神的拯救一定会显现。信心的坚持会带来属灵的胜利。

\subsubsection*{2. 神的国度扩展到万民}
\textbf{“地的四极都要想念耶和华,并且归顺他。”(诗篇 22:27)}

大卫预言了福音将传到世界各地,这正是耶稣基督复活后带来的影响。神的国度超越以色列,临到万国万民。

\textbf{实际应用:}作为基督徒,我们要积极参与福音的传扬,让更多人认识神的救恩。

\subsubsection*{3. 义人的敬拜}
\textbf{“世世代代的人都要事奉他。”(诗篇 22:30)}

神的得胜不仅是个人的,也是世世代代的。凡敬畏神的人都要因他的信实而赞美他。

\textbf{实际应用:}我们要养成敬拜神的习惯,无论顺境逆境,都要存感恩的心,赞美神。

\subsection*{结论:从苦难到得胜的信仰之路}

\begin{itemize}
    \item \textbf{即便在痛苦中,也要向神呼求,他没有真正离弃我们。}
    \item \textbf{世界可能会嘲笑我们,但我们要坚持信靠神。}
    \item \textbf{神的拯救最终会显明,我们的信仰之旅将以得胜为终点。}
    \item \textbf{让我们积极传扬福音,使更多人认识神的救恩。}
\end{itemize}

\subsection*{结束祷告}

\textbf{亲爱的天父,}

感谢你赐下诗篇22篇,让我们看到苦难并不是终点,而是得胜的起点。求你帮助我们在困境中仍然信靠你,不被环境左右。主啊,我们愿意跟随你的脚步,在痛苦中持守信仰,在得胜中宣扬你的名。愿你的救恩临到更多的人,使世世代代都敬拜你。

奉主耶稣基督的名祷告,阿们!
%----------------------------------------------------------------------------
\newpage
\section{诗篇第23篇:神是我的牧者}
\subsection*{引言}
\hspace{0.6cm}诗篇二十三篇,是大卫所作的诗篇,也是全书最为人熟知和喜爱的篇章之一。大卫将神比作他的牧者,这不仅是他个人经历的见证,也是所有信靠神的人的共鸣。在这篇短短的诗篇中,神的关怀、引导与保护被展现得淋漓尽致。今天,我们一同从这篇诗篇中汲取智慧,思考神如何在我们日常生活中担当我们的牧者,带领我们走向真正的平安与满足。
\subsection*{第一部分:神是我们的牧者,供应我们一切所需(诗篇 23:1-2)}
\hspace{0.6cm}诗篇开始时,大卫说:“耶和华是我的牧者,我必不至缺乏。”在这句话中,“大卫说‘我的’”,这意味着他个人的经历,也暗示了他与神之间亲密的关系。作为牧者,神的责任不仅仅是照顾羊群,而是满足羊群的所有需求。这里的大卫并不是指“物质的缺乏”,更是指“心灵的饥渴”,神在每个层面上都供应我们的需要。

在现实生活中,我们常常因为种种原因感到焦虑或不安。可能是因为事业的不顺,或是人际关系的挑战,甚至是身体的健康问题。当我们觉得困惑或无助时,诗篇23篇提醒我们,神已经是我们的牧者,他会按照我们的需要供应我们,甚至在我们感到最脆弱时,神的力量和爱会加倍显现。
\subsection*{第二部分:神的引导与安慰(诗篇 23:3-4)}
\hspace{0.6cm}大卫写道:“他使我躺卧在青草地上,领我在可安歇的水边。”这段经文用牧羊人的细致工作来描绘神的关怀。青草地象征着丰盛和滋养,水边象征着平安与恢复。神并不是只是将我们放在平坦的道路上,而是带领我们进入一个适合我们休息和复原的环境。接着,大卫继续说:“我虽然行过死荫的幽谷,也不怕遭害,因为你与我同在。”这句话显然是从大卫多次经历危险的情况中写下的,但他知道,无论身处什么困境,神的同在与安慰是他最大的保障。

在现代社会中,我们每个人都会遇到挑战与困境,甚至是一些深不见底的“死荫幽谷”——例如,重大疾病的困扰、失业、亲人的离世、失去方向的迷茫等等。在这些时刻,神的引导和同在显得尤为重要。诗篇告诉我们,虽然环境不一定改变,但神的同在足以带给我们安慰和力量。真正的安慰不仅是外在的安宁,而是内心深处因神同在而产生的平安。
\subsection*{第三部分:神的保护与复兴(诗篇 23:5-6)}
\hspace{0.6cm}“大卫说:‘你为我摆设宴席,在我敌人面前;你膏了我的头,使我的杯满溢。’”这节经文描绘了神如何在敌人面前赐下丰盛的恩典和保护。就像牧羊人会为羊群提供一个安全的环境,让他们在敌人的威胁中安然无恙,神也在我们生活中提供了这份精神的保护。

神不仅保护我们免受外界的攻击,更通过他的恩典,使我们在困境中得到复兴和满溢的祝福。神的爱和恩典远远超过我们所能想象的,他使我们的生命充满了喜乐和满足。大卫提到,“我的杯满溢”象征着神给予我们的祝福和满足,不仅仅是足够的,而是溢出的祝福,超乎我们的需要和理解。

生活中,虽然我们时常面对挑战,但我们可以相信,在任何情况下,神都会保护我们,并且通过他的恩典让我们得到复原,经历真正的富足。无论是在人际关系的复和,还是在事业中的突破,我们都可以从神那里得到意想不到的恩惠和帮助。
\subsection*{第四部分:信靠神的承诺(诗篇 23:6)}
\hspace{0.6cm}诗篇最后一节说:“只要一生一世,有恩惠和慈爱跟随我,我且要住在耶和华的殿中,直到永远。”这节经文不仅是大卫对神信实与恩典的宣告,也是对我们每一个信靠神的人发出的鼓励。无论人生的道路如何变化,我们都可以坚定地信靠神,因他始终会与我们同在,给我们恩惠与慈爱,直到永远。

现代生活常常充满不确定性,外界的动荡与挑战时刻威胁着我们的平安。可是诗篇二十三篇提醒我们,神的恩惠和慈爱是我们永远的保障。这种恩惠和慈爱,不是短暂的或偶然的,而是神与我们同行的承诺,是他在我们一生中不断提供的支持和帮助。
\subsection*{结论}
\hspace{0.6cm}诗篇二十三篇向我们清楚地展现了神作为牧者的角色——他是我们的供应者、引导者、保护者,并且他的恩惠和慈爱永远伴随着我们。在我们面对生活中的压力与困境时,这篇诗篇提醒我们,要信靠神,依靠他给我们带来的安慰、保护和复兴。无论我们身处何种环境,神的同在和他的引导都是我们最大的依靠。

\subsection*{结束祷告}

\textbf{亲爱的天父,}

我们感谢你在诗篇二十三篇中向我们显明你是我们的牧者,你时刻供应我们所需,带领我们走在平安的道路上。求你帮助我们在日常生活中经历你的同在,无论在顺境或逆境中都能坚定信靠你。愿你的恩惠与慈爱一生跟随我们,直到永远。

奉主耶稣基督的名祷告,阿们!
%---------------------------------------------------------------------------
\newpage
\section{诗篇第24篇:荣耀的王}

\subsection*{引言}
诗篇第24篇是一首宏伟的诗歌,展现了神的主权、圣洁以及他荣耀的降临。这首诗篇通常被认为是大卫在迎接约柜进耶路撒冷时所写,然而它的意义远远超越那历史性的一刻,它预示着基督的降临,也对我们的信仰生活具有极大的启发。今天,我们将一起学习这篇诗篇,并思考如何在实际生活中回应这篇诗篇的呼召。

\subsection*{第一部分:神的主权(1-2节)}
\subsubsection*{经文解析}
“地和其中所充满的,世界和住在其间的,都属耶和华。”(诗篇24:1)这节经文清楚地宣告,世界和其中的一切都属于神。神不仅是创造主,也是万物的拥有者。接下来的第二节进一步解释了神如何立定大地,使之稳固。

\subsubsection*{实际应用}
在生活中,我们是否真正认识到一切都属于神?我们是否在财务、工作、家庭等各个方面承认神的主权?当我们面对经济压力或前途不确定时,我们是否仍然相信神掌管一切?

\textbf{挑战:}今天,我们可以用一个实际行动来回应神的主权——感恩。写下五件你拥有的东西,并向神献上感谢,同时思想如何为神使用这些资源。

\subsection*{第二部分:谁能登耶和华的山?(3-6节)}
\subsubsection*{经文解析}
“谁能登耶和华的山?谁能站在他的圣所?”(诗篇24:3)这节经文提出了一个极为重要的问题:谁能在神的面前站立?答案在第4节给出——“手洁心清,不向虚妄,起誓不怀诡诈的人。”

\subsubsection*{实际应用}
这提醒我们,敬拜神不仅仅是外在的行为,更是内在的品格。我们是否在工作、学习和人际关系中保持清洁的手和纯洁的心?我们是否在社交媒体、言语和承诺上真实无伪?

\textbf{挑战:}今天,我们可以在神面前省察自己的心,认罪悔改,并在具体行动上追求圣洁,比如在待人接物上更加诚实守信。

\subsection*{第三部分:荣耀的王降临(7-10节)}
\subsubsection*{经文解析}
“众城门哪,你们要抬起头来!永久的门户,你们要被举起!那荣耀的王将要进来!”(诗篇24:7)这段话以一种充满敬畏和期待的语气,描述神的荣耀进入他的城。谁是这荣耀的王?就是万军之耶和华。

\subsubsection*{实际应用}
我们是否愿意打开自己生命的“城门”,让荣耀的王进入?在我们的生活中,是否有一些封闭的地方,我们还不愿意让基督掌权?

\textbf{挑战:}今天,我们可以在祷告中邀请耶稣进入我们生命的每一个角落,不仅仅是敬拜的时刻,更是在日常的决策和态度中顺服他的带领。

\subsection*{结论与祷告}
\hspace{0.6cm}诗篇24篇向我们展现了神的主权、人的圣洁要求以及荣耀之王的降临。愿我们在日常生活中,敬畏神、追求圣洁,并敞开心门迎接荣耀的王。

\textbf{祷告:}

亲爱的天父,感谢你借着诗篇24篇提醒我们,你是天地的主,配得我们全心的敬拜。求你洁净我们的心,使我们能坦然无惧地来到你的面前。帮助我们在生活中真正实践信仰,让你的荣耀充满我们的生命。奉主耶稣基督的名祷告,阿们。
%-----------------------------------------------------------------------------
\newpage


\section{诗篇第25篇:仰望神的道路的属灵教导}

\subsection*{引言}

\hspace{0.6cm}亲爱的弟兄姐妹,今天我们要一同学习诗篇第25篇。诗篇25篇是一篇充满信心的祷告诗,诗人向神倾诉自己的困境,同时寻求神的引导和怜悯。它不仅展现了大卫王在困境中的信心,也为我们今天的信徒提供了属灵的指引。

我们常常在生活中遭遇挑战:学业、工作、家庭、健康、人际关系等等。面对这些挑战,我们是否能够像大卫一样,把自己完全交托给神,相信他的道路比我们的更高?今天,让我们透过诗篇25篇,一起学习如何在生活中信靠神,寻求他的带领。

\subsection*{信靠与等候:仰望神的指引(1-5节)}

诗篇25:1-5:
\begin{quote}
耶和华啊,我的心仰望你!
我的神啊,我素来倚靠你。
求你不要叫我羞愧,
不要叫我的仇敌向我夸胜。
凡等候你的必不羞愧,
惟有那无故行奸诈的必要羞愧。
耶和华啊,求你将你的道指示我,
将你的路教训我!
求你以你的真理引导我,教训我,
因为你是救我的神。
我终日等候你。
\end{quote}

大卫以“我的心仰望你”开始他的祷告。这是一种完全的交托和信靠。我们是否也能够如此,把自己的一切交给神,而不是依靠自己的聪明和判断?在现代社会,我们常常习惯于自己掌控生活,而忽略了神的主权。

大卫强调“等候神”,这不仅是耐心地等待,更是一种带着信心的期待。我们在面对人生的选择时,是否愿意耐心等候神的引导,而不是急于做决定?例如,在职业选择、婚姻、重大决定上,我们是否花时间祷告,寻求神的旨意?

\subsection*{悔改与怜悯:承认自己的软弱(6-11节)}

诗篇25:6-7:
\begin{quote}
耶和华啊,求你记念你的怜悯和慈爱,
因为这是亘古以来所常有的。
求你不要记念我幼年的罪愆和我的过犯,
耶和华啊,求你因你的恩惠,
按你的慈爱记念我。
\end{quote}

这里,大卫承认自己的罪,并求神按照他的慈爱待他。这提醒我们,我们都需要神的怜悯。我们是否常常为自己的罪悔改,而不是遮掩?我们是否愿意承认自己的过犯,寻求神的赦免?

在现实生活中,我们可能会因过去的错误而感到羞愧,但神的恩典大过我们的过犯。无论过去如何,只要我们真心悔改,神都会赦免我们。因此,我们要学会放下过去的包袱,活在神的恩典中。

\subsection*{顺服与敬畏:走在神的正道上(12-15节)}

诗篇25:12-15:
\begin{quote}
谁敬畏耶和华,
耶和华必指示他当选择的道路。
他必安然居住,
他的后裔必承受地土。
耶和华与敬畏他的人亲密,
他必将自己的约指示他们。
我的眼目时常仰望耶和华,
因为他必将我的脚从网罗里拉出来。
\end{quote}

敬畏神的人会得到他的指引。今天的世界充满了各种诱惑,我们是否愿意顺服神的旨意,而不是随从世界的潮流?

顺服神可能意味着做出不合世俗期待的选择。例如,在职场上,别人可能选择撒谎、弄虚作假来获取利益,而我们愿意坚持诚信吗?在面对道德抉择时,我们是否站在真理的一边,而不是随波逐流?

\subsection*{祷告与盼望:在苦难中持守信心(16-22节)}

诗篇25:16-18:
\begin{quote}
求你转向我,怜恤我,
因为我是孤独困苦。
我心里的愁苦甚多,
求你救我脱离我的祸患。
求你看顾我的困苦,
我的艰难,赦免我一切的罪。
\end{quote}

大卫在困境中呼求神,我们也可以效法他。当我们感到孤独、痛苦、无助时,是否首先想到的是向神祷告,而不是依赖人的帮助?

面对困难时,我们可以这样祷告:“主啊,我的心甚是忧愁,但我知道你是信实的,求你帮助我度过难关,赐我智慧和勇气。”

\subsection*{结论:实践信仰,依靠神的应许}

\hspace{0.6cm}诗篇25篇教导我们:
\begin{enumerate}
    \item 在困境中信靠神,耐心等候他的指引。
    \item 真实悔改,依靠神的怜悯。
    \item 敬畏神,遵行他的旨意。
    \item 在苦难中持续祷告,持守信心。
\end{enumerate}


弟兄姐妹,让我们今天就开始行动,把生活的每一件事交托给神,信靠他的带领。愿神帮助我们在信仰的道路上越来越坚固。

\subsection*{结束祷告}

天父,我们感谢你赐下诗篇25篇,让我们学习如何信靠你、等候你、顺服你。求你帮助我们在困境中不失去信心,在迷茫中寻求你的道路。愿你保守我们的心,使我们敬畏你、跟随你,直到永远。奉主耶稣基督的名祷告,阿们!
%-----------------------------------------------------------------------------
\newpage
\section{诗篇第26篇:持守正直,行走在主的真理中}


\subsection*{引言}

\hspace{0.6cm}弟兄姐妹们,愿主的平安与我们同在!今天我们要分享的经文是诗篇第26篇,这篇诗篇是大卫向神的祷告,他求神鉴察他的内心,并求神保守他远离恶行,使他能坦然无惧地行走在神的真理中。

\subsection*{一、信靠神,行在正直中(1-3节)}

\hspace{0.6cm}大卫在诗篇26:1中祷告说:“耶和华啊,求你为我伸冤,因我向来行事纯全,我又倚靠耶和华并不摇动。”这里的“行事纯全”并不是指大卫从未犯罪,而是指他在神面前持守正直的心。他信靠神,依赖神的引导,不让罪恶左右自己。

在现实生活中,我们是否能够像大卫一样,在纷繁复杂的世界中持守正直?在面对工作中的试探、生活中的困难、社交中的压力时,我们是否依然坚定信靠神?让我们思考:我们的行为是否反映出对神的忠诚?

\subsection*{二、远离恶行,分别为圣(4-8节)}

\hspace{0.6cm}大卫继续说:“我没有和虚谎人同坐,也不与瞒哄人的同群。我恨恶恶人的会,不与恶人同坐。”(诗26:4-5)

这提醒我们要有分别为圣的心态。在日常生活中,我们或许会遇到腐败、不公、不诚实的环境。作为基督徒,我们要有勇气拒绝妥协,坚守神的标准。例如,在职场中,我们是否愿意为了利益而撒谎?在朋友之间,我们是否随波逐流,而不愿意坚持真理?

这里的分别为圣,并不是说我们完全不接触世人,而是要有清晰的属灵界限,不让恶行影响我们,同时以基督的爱去影响他人。

\subsection*{三、寻求神的引导,坚定依靠主(9-12节)}

\hspace{0.6cm}大卫在最后的祷告中说:“至于我,却要行事纯全,救赎我,怜恤我。”(诗26:11)这表明他知道自己仍然需要神的怜悯和救赎。他愿意持守正直,并依靠神的带领。

在我们的信仰道路上,我们可能会遇到挫折,会有软弱的时候,但大卫的祷告提醒我们,不要靠自己的力量,而要倚靠神。神是公义的审判者,也是慈爱的救主,他必定带领我们走义路。

\subsection*{结语}

诗篇第26篇如同明镜,映照出大卫的心灵深处:
\begin{itemize}
    \item 他对神的全然信靠,如同磐石般坚定不移;
    \item 他对正直生活的执着追求,如同精金般纯粹无瑕;
    \item 他对神的恳切祷告,如同馨香之祭,昼夜献上。
\end{itemize}

这篇诗歌不仅是大卫个人的心声吐露,更是对我们每一个人的深刻启示。

让我们效法大卫的榜样,在人生的道路上,紧紧抓住对神的信靠,无论顺境逆境,都坚信神是我们的避风港,是我们的力量源泉。

同时,矢志不渝地追求正直的生活,远离一切的污秽和罪恶,让我们的言行举止都符合神的旨意,活出圣洁的生命。

并且,坚持不懈地向神祷告,将我们心中的意念、渴望、忧虑都倾诉在神的面前,与神建立亲密无间的关系。

唯有如此,我们才能在纷繁复杂的世界中站稳脚跟,在人生的风浪中乘风破浪,最终抵达神为我们预备的永恒家园。

让我们将这篇诗篇的教导融入到日常生活中,以正直为盾牌,以信靠为铠甲,勇敢地走在神所指引的光明大道上。

\subsection*{结束祷告}

\textbf{亲爱的天父,}

我们感谢你赐下你的话语,使我们明白如何在世上行走。求你鉴察我们的内心,使我们能持守正直,远离罪恶的道路。求你在我们面对试探时给我们力量,使我们不随波逐流,而是坚守你的真理。求你怜悯我们,带领我们走在你的公义之中。

奉主耶稣基督的名祷告,阿们!
%----------------------------------------------------------------------------
\newpage
\section{诗篇第27篇:在主里刚强壮胆——信心之路}
\subsection*{引言}
弟兄姐妹,平安!在这个充满挑战的世界,我们常常面对恐惧、不安、试炼和等待的过程。但诗篇第27篇给了我们极大的安慰,因为它向我们展现了一条信心的道路,使我们在困境中仍能坚定仰望神。今天,我们就从大卫的诗篇27篇来探讨,如何在现实生活中持守信心、刚强壮胆,并经历神的同在。

\subsection*{一、信心的根基——神是我们的亮光和拯救(1-3节)}
\subsubsection*{1. 亮光驱散黑暗}
\hspace{0.6cm}诗篇27:1 说:“耶和华是我的亮光,是我的拯救,我还怕谁呢?” 在黑暗的世界里,我们的信心若是建立在环境、人的承诺或自己的能力之上,终究会动摇。唯有神是我们真正的亮光,使我们能看清前方的道路。

\vspace{0.2cm}

\textbf{生活应用:}

\hspace{0.6cm}面对前途的不确定,我们要相信神会引导我们的脚步。

\hspace{0.6cm}遭遇困难时,不是靠自己的智慧,而是先来到神面前寻求亮光。

\hspace{0.6cm}不被世界的恐慌影响,而是选择信靠神的应许。


\subsubsection*{2. 依靠神的拯救,不惧怕仇敌}
\hspace{0.6cm}大卫曾面对强大的仇敌、扫罗王的逼迫,但他宣告:“虽然有军兵安营攻击我,我的心也不害怕。”(诗篇27:3) 这是一种超越现实环境的信心!

\vspace{0.2cm}

\textbf{生活应用:}

\hspace{0.6cm}当我们在职场或学校里遭遇压力,面对竞争时,我们可以像大卫一样仰望神,而不是被环境压垮。

\hspace{0.6cm}家庭问题、健康危机、经济困难……这些挑战虽大,但神比这一切更大!


\subsection*{二、单单寻求神——亲近神的渴望(4-6节)}
\subsubsection*{1. 生命的优先次序——住在神的殿中}
\hspace{0.6cm}大卫说:“有一件事,我曾求耶和华,我仍要寻求,就是一生一世住在耶和华的殿中。”(诗篇27:4) 这说明他对神的渴慕胜过一切,他不只是求神的帮助,而是要神自己。

\vspace{0.2cm}

\textbf{生活应用:}

\hspace{0.6cm}现代人常追求事业、财富、名声,但我们是否像大卫一样,渴慕神胜过这些?

\hspace{0.6cm}我们是否愿意每天预留时间,安静在神面前祷告、读经、敬拜?

\hspace{0.6cm}我们的信仰不是仅限于主日,而是每一天都活在神的同在中。


\subsubsection*{2. 神是我们患难中的避难所}
\hspace{0.6cm}大卫深知,在困难时,神会把他隐藏在他的帐幕里,使他在磐石上高举(诗篇27:5)。

\vspace{0.2cm}

\textbf{生活应用:}

\hspace{0.6cm}遇到苦难时,我们不是慌张无助,而是安静祷告,等候神的作为。

\hspace{0.6cm}当世界给我们压力时,我们要学习将焦点放在神身上,他是我们的避难所。




\subsection*{三、在等待中持守信心(7-14节)}
\subsubsection*{1. 呼求神,不要灰心(7-10节)}
\hspace{0.6cm}大卫在这里向神祷告:“耶和华啊,我用声音呼吁的时候,求你垂听我。”(诗篇27:7) 这表明他虽有信心,但仍然经历等候的过程。他也说:“我父母离弃我,耶和华必收留我。”(诗篇27:10)

\textbf{生活应用:}

\vspace{0.2cm}

\hspace{0.6cm}当神的回应似乎延迟时,我们仍要持续祷告,而不是怀疑神的信实。

\hspace{0.6cm}有时候,人会让我们失望,但神永远不会离弃我们。

\hspace{0.6cm}现实生活中,无论是家庭的伤害、人际关系的破裂,还是事业的低谷,我们都要仰望神的信实。

\subsubsection*{2. 他的道路高过我们的道路(11-12节)}
\hspace{0.6cm}大卫祈求神:“耶和华啊,求你将你的道指教我,引导我走平坦的路。”(诗篇27:11) 这表明我们必须信任神的带领,即使他的道路不符合我们的期待。

\vspace{0.2cm}

\textbf{生活应用:}

\hspace{0.6cm}当事情没有按照我们的计划发展时,我们要学习顺服神的时间和安排。

\hspace{0.6cm}无论是婚姻、事业还是学业,神的计划是最好的,我们要等候他的引导。
\subsubsection*{3. 刚强壮胆,等候耶和华(13-14节)}
\hspace{0.6cm}诗篇27:14 说:“要等候耶和华,当壮胆,坚固你的心。” 这提醒我们,在等待的过程中,我们必须刚强,不被外界的声音影响。

\vspace{0.2cm}

\textbf{生活应用:}

\hspace{0.6cm}在等候神的医治、供应或突破时,不要失去信心,而是坚定依靠神。

\hspace{0.6cm}世界告诉我们要靠自己解决问题,但神要我们安静等候,直到他动工。

\hspace{0.6cm}许多人在等待中失去耐心,最终选择了错误的道路。但真正的信心是,在未见成就之前,仍然相信神的良善。

\subsection*{结论:信心的抉择}
诗篇27篇带给我们三方面的提醒:
\begin{enumerate}
    \item 神是我们的亮光和拯救,我们无需惧怕。

    \item 在困境中,我们要渴慕神、寻求他的同在,而不仅仅是他的帮助。

    \item 等待神的作为时,我们要刚强壮胆,信靠他的时间和计划。

    
\end{enumerate}

愿我们在面对人生挑战时,都能像大卫一样,选择信靠神,持守信心,刚强壮胆!

\subsection*{结束祷告}
\textbf{亲爱的天父,}

感谢你借着诗篇27篇教导我们,让我们在困境中学会信靠你。主啊,你是我们的亮光,我们不再惧怕,你是我们的避难所,我们愿意全然投靠你。求你赐给我们一颗渴慕你的心,让我们每天亲近你,而不是被世界的事物分心。当我们经历等待的过程,求你坚固我们的信心,使我们刚强壮胆,不被环境动摇。感谢你是信实的神,愿你在我们生命中成就你的美好旨意。

奉主耶稣基督的名祷告,阿们!
%-----------------------------------------------------------------------------
\newpage
\section{诗篇第28篇:向耶和华呼求,经历他的拯救}
\subsection*{引言}
\hspace{0.6cm}弟兄姐妹,平安!在我们的人生旅途中,常会遭遇困难、疑惑和不公正的对待。这些经历可能让我们感到孤立无助,甚至怀疑神是否听见我们的呼求。但诗篇28篇是大卫在困境中的祷告,也是信心的宣告,提醒我们:无论环境如何,神仍掌权,他是我们坚固的保障和得胜的力量。今天,我们就借着诗篇28篇,一同学习如何在困境中倚靠神,经历他的拯救。

\subsection*{一、向耶和华迫切祷告,不要沉默(1-2节)}
\subsubsection*{1. 祷告的重要性}
\hspace{0.6cm}大卫在诗篇28:1-2 迫切地向神祷告:“耶和华啊,我要求告你;我的磐石啊,不要向我缄默!倘若你向我闭口,我就如将死人下到坑中。”
\begin{itemize}
    \item 这表明他正经历极大的困境,甚至感到濒临绝望。
    \item 他知道,若神不回应,他就会像死人一样毫无盼望。
\end{itemize}


\subsubsection*{2. 生活应用:如何在困境中祷告?}

\begin{itemize}
    \item \textbf{持续呼求神}:当我们感到神沉默时,不要停止祷告,而要更加迫切地寻求他。

    \item \textbf{带着信心祷告}:大卫称神为“磐石”,表明他相信神的稳固和信实。我们也要如此,相信神仍然掌权。

    \item \textbf{避免依靠自己的聪明}:在困难中,我们容易依赖自己的方法或人的帮助,而不是首先向神呼求。

\end{itemize}

\textbf{现实案例}



许多信徒在遭遇家庭危机或事业低谷时,第一反应是抱怨或找人诉苦,而不是先来到神面前祷告。我们要学习大卫,第一时间向神呼求,而不是最后才想起神。
\subsection*{二、分别义人与恶人,行事要合神心意(3-5节)}
\subsubsection*{1. 远离恶人的道路}
\hspace{0.6cm}大卫在诗篇28:3-5 祷告:“不要把我和恶人并作奸诈人的一同除掉;他们与邻舍说和平话,心里却是奸恶。”
\begin{itemize}
    \item \textbf{大卫知道神是公义的,他不愿意与恶人为伍。}

    \item \textbf{恶人的特征}: 他们口里说和平的话,心里却充满诡诈。

    \item \textbf{神的公义}: 大卫相信神必按恶人的作为报应他们(28:4)。

\end{itemize}
\subsubsection*{2. 生活应用:如何在现实中分辨?}
\begin{itemize}
    \item 工作和人际关系中,我们是否与不诚实的人妥协?

    \item 社交圈中,我们是否与行为不正之人同流合污?

    \item 面对利益的诱惑,我们是否愿意坚持公义?

\end{itemize}

\textbf{现实案例}
\begin{itemize}
    \item 在职场中,有些人为了升迁而不择手段,但一个真正敬畏神的人,宁愿放弃短暂的利益,也不愿行诡诈。

    \item 在社交媒体时代,许多言论表面上是和平的,但背后却隐藏着谎言和操控。我们需要智慧,分辨真理,不随波逐流。

\end{itemize}
\subsection*{三、信靠神,得着力量和拯救(6-9节)}
\subsubsection*{1. 神是我们得胜的力量}
\hspace{0.6cm}诗篇28:6-7 说:“耶和华是我的力量,是我的盾牌;我心里倚靠他,就得帮助。”

大卫的信心从祷告转向感恩,他确信神已垂听他的祷告。

他将神比作盾牌,说明神不仅保护我们,也帮助我们战胜仇敌。
\subsubsection*{2. 生活应用:如何在挑战中经历神的力量?}
\begin{itemize}
    \item 当我们软弱时,宣告神是我们的力量,不要靠自己的能力硬撑。
    \item 信心不仅仅是等候神,也包括实际的行动——凭信心去做正确的决定。

    \item 神不只是听我们的祷告,他也赐下力量让我们去面对挑战。

\end{itemize}

\textbf{现实案例}
\begin{itemize}
    \item 许多基督徒在面对健康问题或经济压力时,常感到疲惫不堪。但当我们依靠神,不仅在心灵上得着力量,也能在行动上刚强。

    \item 使徒保罗在监狱中依然喜乐,就是因为他相信神是他的力量。今天,我们也可以在任何环境中经历神的供应。

\end{itemize}
\subsection*{结论:在困境中,倚靠神,刚强壮胆!}
诗篇28篇提醒我们:
\begin{enumerate}
    \item 无论处境如何,都要向神迫切祷告,相信他必垂听。
    \item 在这个充满诡诈的世界,我们要远离恶行,持守公义。
    \item 即使环境不变,我们的心可以因神的同在而刚强。
    \item 无论你今天面临的是何种挑战,愿你像大卫一样,坚定地呼求神、持守公义,并经历他的大能!
\end{enumerate}



\subsection*{结束祷告}

\textbf{亲爱的天父,}

我们感谢你借着诗篇28篇教导我们,让我们在困境中学会依靠你。主啊,我们常感到软弱,求你使我们刚强;我们常被环境动摇,求你使我们站立得稳。帮助我们不随从恶人,而是持守你的公义;让我们的心因你的同在而充满平安和喜乐。主啊,你是我们的盾牌和力量,我们愿意一生倚靠你。感谢你垂听我们的祷告。

奉主耶稣基督的名,阿们!
%-----------------------------------------------------------------------------
\newpage
\section{诗篇第29篇:在神荣耀的声音中得力量}
\subsection*{引言}
\hspace{0.6cm}弟兄姐妹,平安!在这个充满不确定性的世界,我们时常被各种声音包围——新闻的声音、人的意见、内心的焦虑……这些声音影响着我们的情绪和信心。但今天,我们要从诗篇29篇来学习,如何在神的声音中得着真正的力量与平安。

诗篇29篇是一首赞美诗,强调神的声音何等威严和大能。大卫在这里不仅呼吁万物敬拜神,也描绘了神的声音如何影响天地、改变环境、坚固属他的人。今天,我们要从三方面来看这篇诗篇的教导,并结合我们的生活应用。

\subsection*{一、荣耀归于神——敬拜的呼召(1-2节)}
\subsubsection*{1. 敬拜的核心是归荣耀给神}
\hspace{0.6cm}诗篇29:1-2 说:
“神的众子啊,你们要将荣耀能力归给耶和华,要将耶和华的名所当得的荣耀归给他,以圣洁的装饰敬拜耶和华。”


这里的大卫不是在向人讲话,而是在向天上的天使和一切受造之物发出敬拜的呼召。

他说,敬拜神不仅仅是我们的责任,更是我们的特权!
\subsubsection*{2. 生活应用:我们的敬拜是否真的归荣耀给神?}

\hspace{0.6cm}许多时候,我们敬拜是出于习惯,而非真心渴慕神。

我们可能在唱诗的时候,嘴唇动着,但心思却在想工作、学业或琐事。

真正的敬拜是把注意力从自己转向神,让神在我们生命中掌权。

\vspace{0.2cm}

\textbf{现实案例}

\hspace{0.6cm}有些人只在顺境中感谢神,在逆境中却质疑神。但真正的敬拜是无论环境如何,都承认神是配得尊荣的。

\hspace{0.6cm}想象你正在经历一场风暴——经济的压力、健康的挑战、家庭的冲突——这时候,神仍然配得你敬拜吗?答案是肯定的!

\vspace{0.2cm}


\textbf{挑战与反思}

\hspace{0.6cm}你最近的敬拜是专注在神身上,还是更多关注自己的感受?

\hspace{0.6cm}你是否愿意在困难中仍然敬拜,将荣耀归给神?



\subsection*{二、神的声音充满大能(3-9节)}
\subsubsection*{1. 神的声音胜过天地万物}
\hspace{0.6cm}诗篇29:3-9 提到神的声音七次,每次都带出他的权柄和大能:
\begin{itemize}
    \item \textbf{神的声音如雷}(3节)——显明他是超乎一切的主宰。

    \item \textbf{神的声音震动大地}(5-6节)——巴嫩的香柏树是古代最坚固的树木,但神的声音能将其震裂!

    \item \textbf{神的声音震动旷野}(8节)——神的话语甚至能影响最荒凉的地方。

\end{itemize}
\subsubsection*{2. 生活应用:神的话语在我们的生命中是否掌权?}

\hspace{0.6cm}在世界各种声音中,我们是否选择听神的声音?

人的意见、新闻的恐慌、社交媒体的焦虑,常常影响我们的决定和信心。

但神的话语能震动一切,带下真正的力量和改变!

\vspace{0.2cm}

\textbf{现实案例}

\hspace{0.6cm}有人因工作或学业的压力,内心充满焦虑。但当他们开始用神的话语取代负面思想,焦虑便渐渐消散。

\hspace{0.6cm}当一对夫妻面临婚姻危机,世界的声音可能会告诉他们“离婚吧”,但神的声音会引导他们寻求修复与和好。

\vspace{0.2cm}

\textbf{挑战与反思}

\hspace{0.6cm}你是否让神的话语成为你每天的指引?

\hspace{0.6cm}你是否更多听人的声音,而不是安静聆听神?



\subsection*{三、神的声音带来平安与力量(10-11节)}
\subsubsection*{1. 神坐着为王,赐下力量和平安}
\hspace{0.6cm}诗篇29:10-11 说:
“洪水泛滥之时,耶和华坐着为王;耶和华坐着为王,直到永远。耶和华必赐力量给他的百姓,耶和华必赐平安的福给他的百姓。”

洪水象征混乱和灾难,但神仍然坐在宝座上,掌管一切。

我们可能看见世界在动荡,国家在变迁,个人生活在起伏,但神没有改变,他仍然在掌权!
\subsubsection*{2. 生活应用:如何经历神的力量和平安?}

\hspace{0.6cm}不要让环境决定你的信心,而要让神的话语决定你的信心!

每天宣告神掌权,无论你面临什么挑战。

\vspace{0.2cm}

\textbf{现实案例}

\hspace{0.6cm}使徒彼得在风暴中行走,因仰望耶稣而站立得稳,但当他转眼看风浪时,便开始下沉(马太福音14:29-30)。

\hspace{0.6cm}今天,我们的环境也可能像风暴一样动摇,但如果我们定睛在神身上,就能经历他的平安。

\vspace{0.2cm}

\textbf{挑战与反思}

\hspace{0.6cm}你是否常常因环境而焦虑,而忘了神仍然掌权?

\hspace{0.6cm}你愿意每天宣告神的声音比世界的声音更大,凭信心前行吗?


\subsection*{结论:让神的声音成为你生命的引导}
诗篇29篇提醒我们:
\begin{enumerate}
    \item 敬拜神,是我们生命的优先次序,无论环境如何,我们都要将荣耀归给神。
    \item 神的声音大有能力,可以震动天地,也可以改变我们的生命。

    \item 在动荡中,神仍然坐着为王,他赐下力量,使我们得着真正的平安。

    \item 今天,你愿意更多聆听神的声音,让他的话语成为你生命的根基吗?

\end{enumerate}


\subsection*{结束祷告}

\textbf{亲爱的天父,}

我们感谢你,因你坐着为王,掌管万有。主啊,在世界充满噪音、焦虑和恐惧的时候,我们愿意聆听你的声音,单单仰望你。帮助我们在敬拜中专注于你,而不是自己的感受;在决策中寻求你的话语,而不是人的意见;在风暴中信靠你,而不是被环境动摇。愿你的声音在我们生命中成为引导,愿你的力量和平安临到我们。

奉主耶稣基督的名祷告,阿们!
%----------------------------------------------------------------------------
\newpage
\section{诗篇第30篇:从哀哭到欢呼——经历神的恩典与拯救}
\subsection*{引言}

\hspace{0.6cm}弟兄姐妹,平安!人生就像四季,有时候阳光明媚,万物生长;但有时候狂风暴雨,让人感到灰心和痛苦。你是否曾经历低谷?是否曾在夜晚流泪、在苦难中挣扎?然而,诗篇30篇给我们带来极大的盼望,它见证了神如何使人的生命从哀哭变为跳舞,从黑暗进入光明。

这篇诗篇是大卫在经历艰难后,对神的感恩和颂赞。今天,我们要从诗篇30篇学习三方面的真理,并思考如何在现实生活中应用,让我们的生命经历从悲哀到喜乐的转变。

\subsection*{一、神是我们的拯救——从低谷到高处(1-3节)}
\subsubsection*{1. 大卫的见证:神将他从深坑中提拔}
\hspace{0.6cm}诗篇30:1-3 说:

“耶和华啊,我要尊崇你,因为你曾提拔我,不叫仇敌向我夸耀。耶和华我的神啊,我曾呼求你,你医治了我。耶和华啊,你曾把我的灵魂从阴间救上来,使我存活,不致下坑。”

% \vspace{0.2cm}


大卫曾处于极大的困境,甚至面临死亡(“阴间”代表死亡的深渊)。

但神垂听了他的祷告,拯救了他,使他从危难中存活。

\subsubsection*{2. 生活应用:当我们陷入低谷时,如何依靠神?}
\begin{itemize}
    \item 当遭遇疾病时:要像大卫一样呼求神,依靠神的医治和恩典。

    \item 当面临失败时:不要被失败定义,而要相信神能重新兴起我们。

    \item 当遇到仇敌攻击时:不要害怕,而要倚靠神的保护。

\end{itemize}

\textbf{现实案例}

\hspace{0.6cm}许多人在生病或遭遇事业危机时,会感到绝望。但当他们转向神,祷告寻求神的帮助,就能经历神的拯救与医治。例如,有一位弟兄失去了工作,起初他陷入忧愁,但当他开始祷告,信靠神,神为他开了一条更好的道路,使他的信心更加坚定。

\vspace{0.2cm}

\textbf{挑战与反思}

\hspace{0.6cm}你是否愿意在困境中相信神,而不是陷入绝望?

\hspace{0.6cm}你是否曾因神的拯救而感恩,并见证他的作为?
\subsection*{二、神的怒气是短暂的,恩典却是永远的(4-5节)}
\subsubsection*{1. 大卫的感恩:神的恩典超过他的管教}
\hspace{0.6cm}诗篇30:4-5 说:

“耶和华的圣民哪,你们要歌颂他,称赞他可纪念的圣名。因为他的怒气不过是转眼之间,他的恩典乃是一生之久。一宿虽有哭泣,早晨便必欢呼。”

% \vspace{0.2cm}

神会管教他的子民,但他的管教是短暂的,恩典却是长久的。

夜晚可能有眼泪,但黎明终会带来欢喜。
\subsubsection*{2. 生活应用:如何在困难中持守盼望?}

\hspace{0.6cm}当我们因罪被神管教时,要谦卑悔改,接受他的恩典。

当我们在困境中哭泣时,要相信神会带来新的开始。

不要被短暂的痛苦捆绑,而要仰望神的应许。

\vspace{0.2cm}

\textbf{现实案例}

\hspace{0.6cm}约伯失去了一切,但他没有放弃信仰,最终神加倍赐福给他。

\hspace{0.6cm}许多夫妻在婚姻中经历困难,但当他们愿意回到神面前,寻求他的引导,就能恢复关系,重新欢喜快乐。

\vspace{0.2cm}

\textbf{挑战与反思}

\hspace{0.6cm}你是否曾因神的管教而远离他,还是愿意顺服,经历他更大的恩典?

\hspace{0.6cm}你是否相信“哭泣是短暂的,喜乐终会到来”?
\subsection*{三、当顺境变为危机时,仍要倚靠神(6-12节)}
\subsubsection*{1. 大卫的悔改与神的拯救}
\hspace{0.6cm}诗篇30:6-7 说:

“至于我,我凡事平顺,便说:‘我永不动摇。’耶和华啊,你曾施恩,叫我的江山稳固;你掩面的时候,我就惊惶。”

% \vspace{0.2cm}

大卫曾因顺境而自满,以为自己不会动摇,但当神收回他的恩典,他才意识到自己的软弱。

这提醒我们,不要在顺利的时候倚靠自己,而要持续仰望神。
\subsubsection*{2. 生活应用:如何在顺境和危机中都倚靠神?}

\hspace{0.6cm}在顺境中,不要骄傲,而要感恩。

在危机中,不要惊惶,而要回转向神。

在任何环境下,都要认定神是我们的依靠。

\vspace{0.2cm}

\textbf{现实案例}

\hspace{0.6cm}很多人在事业顺利时,容易忽略神,但当危机来临,他们才意识到自己的不足。

\hspace{0.6cm}有一位企业家曾依靠自己的智慧做生意,后来公司面临倒闭,他才开始真正寻求神,最终经历神的带领,重建事业。

\vspace{0.2cm}

\textbf{挑战与反思}

\hspace{0.6cm}你是否在顺利的时候仍然仰望神,还是容易倚靠自己?

\hspace{0.6cm}你是否愿意在危机中向神悔改,并相信他仍然掌权?
\subsection*{结论:从哀哭到欢呼,经历神的信实}

\hspace{0.6cm}\textbf{诗篇30篇的核心信息是:}
\begin{enumerate}
    \leftskip=0.7cm
    \item 神是我们的拯救,他能将我们从低谷提升到高处(1-3节)。

    \item 神的恩典大过他的管教,黑夜虽然有哭泣,清晨却带来喜乐(4-5节)。

    \item 无论顺境还是危机,都要倚靠神,因为他是我们真正的保障(6-12节)。

\end{enumerate}

\textbf{你今天是否正处在黑暗的夜晚?}

\hspace{0.6cm}不要害怕,神的恩典比你的困难更大。

\hspace{0.6cm}一宿虽然有哭泣,早晨便必欢呼!

让我们选择信靠神,无论环境如何,都持守盼望,经历从哀哭到欢呼的生命转变!

\subsection*{结束祷告}
\textbf{亲爱的天父,}

我们感谢你,因为你是我们的拯救者,你将我们的哀哭变为跳舞,将我们的忧伤变为喜乐。主啊,在低谷中,我们要仰望你;在顺境中,我们要感恩你。求你赐给我们信心,让我们在等待的夜晚持守盼望,相信清晨的喜乐必然来到。愿我们的生命荣耀你的名,愿我们的一切赞美归给你!

奉主耶稣基督的名祷告,阿们!
%-----------------------------------------------------------------------------
\newpage
\section{诗篇第31篇:在困境中投靠神}
\subsection*{引言}
\hspace{0.6cm}弟兄姐妹,平安!你是否曾经历过痛苦、逼迫、误解,甚至觉得四面受敌,无处可逃?当困难临到时,我们是选择倚靠自己的聪明,还是全然投靠神?诗篇31篇是大卫在极大困境中的祷告和信靠,向我们展示了如何在患难中仍然坚定地依靠神。今天,我们从三方面来看这篇诗篇,并结合现实生活的应用,让我们在任何环境中都能像大卫一样,在神里面找到避难所。

\subsection*{一、神是我们的避难所——全然投靠神(1-8节)}
\subsubsection*{1. 大卫的信心:神是他的磐石和保障}
\hspace{0.6cm}诗篇31:1-2 说:

“耶和华啊,我投靠你,求你使我永不羞愧,凭你的公义搭救我!求你侧耳而听,快快救我!作我坚固的磐石,拯救我的保障。”

大卫没有先求神除去困难,而是先承认神是他的避难所。

他知道,真正的安全感不在环境,而在神里面。
\subsubsection*{2. 生活应用:我们如何在困难中投靠神?}

\hspace{0.6cm}当遭遇职场压力或学业挑战时,我们是否先寻求神?

当遇到人际矛盾或家庭问题时,我们是否将重担交托给神,而不是自己硬扛?

我们是否愿意把自己完全交在神的手中,相信他必保守?

\vspace{0.2cm}

\textbf{现实案例}

\hspace{0.6cm}有一位姐妹在事业低谷时,她的第一反应不是抱怨或自怜,而是每天安静在神面前祷告,求神引导。最终,神不仅帮助她度过难关,还让她在新环境中成为别人的祝福。

\hspace{0.6cm}相反,有些人在压力中选择逃避、抱怨,甚至远离神,结果让自己陷入更深的绝望。

\vspace{0.2cm}

\textbf{挑战与反思}

\hspace{0.6cm}你是否真正相信神是你的避难所,而不是仅在口头上说说?

\hspace{0.6cm}在你面临问题时,第一反应是依靠神,还是靠自己?
\subsection*{二、在困境中倚靠神的手(9-18节)}
\subsubsection*{1. 大卫的挣扎与呼求}
\hspace{0.6cm}诗篇31:9-10 说:

“耶和华啊,求你怜恤我,因为我在急难之中;我的眼睛因忧愁而干瘪,我的身心也不安舒。我的生命为愁苦所消耗,我的年岁为叹息所废去;我的力量因我的罪孽衰败,我的骨头也枯干。”

大卫真实地表达了自己的痛苦:他身心疲惫,内心忧伤。

他没有假装刚强,而是向神倾诉自己的软弱。
\subsubsection*{2. 生活应用:我们如何在痛苦中倚靠神?}

\hspace{0.6cm}不要压抑自己的情绪,而是学会向神倾诉。

即使看不见出路,仍要相信神的信实。

不要让恐惧掌控你的心,而要让神掌管一切。

\vspace{0.2cm}

\textbf{现实案例}

\hspace{0.6cm}有些人遇到问题时,会选择封闭自己,不愿向神或他人倾诉,结果陷入更深的痛苦。

\hspace{0.6cm}但当我们愿意在神面前倾心吐意,他就能用他的话语安慰我们,带领我们走出困境。

\vspace{0.2cm}

\textbf{挑战与反思}

\hspace{0.6cm}你是否愿意在痛苦中来到神面前,而不是远离他?

\hspace{0.6cm}你是否相信,即使环境没有立刻改变,神仍然掌权?
\subsection*{三、交托未来,坚定信靠(19-24节)}
\subsubsection*{1. 大卫的宣告:神是信实的}
\hspace{0.6cm}诗篇31:19-20 说:

“敬畏你、投靠你的人,你为他们所积存的何等丰盛!你当着世人的面,为投靠你的人施行慈爱。你必把他们藏在你面前的隐密处,免得遇见人的计谋;你必暗暗保守他们在亭子里,免受口舌的争闹。”

大卫经历痛苦,但他仍然相信神的美好计划。

他相信神已经为投靠他的人预备了丰盛的恩典。
\subsubsection*{2. 生活应用:如何在不确定的未来中信靠神?}

\hspace{0.6cm}不要只看环境,要看神的应许。

无论环境如何,都要持守信心,耐心等待神的带领。

向神祷告,把一切交托给他,而不是自己忧虑。

\vspace{0.2cm}

\textbf{现实案例}

\hspace{0.6cm}有些人因害怕未来而焦虑,甚至影响身体健康。但当他们选择信靠神,并且把未来交托给神时,就能从焦虑转向平安。

\hspace{0.6cm}
许多人经历了人生的起伏后,回头看才发现,神的计划远比他们自己所想的更好。

\vspace{0.2cm}

\textbf{挑战与反思}

\hspace{0.6cm}你是否愿意交托未来,相信神的信实?

\hspace{0.6cm}你是否愿意停止担忧,而是用祷告来交托一切?
\subsection*{结论:在神里面得安稳}
\hspace{0.6cm}诗篇31篇提醒我们:
\begin{enumerate}
    \leftskip=0.7cm
    \item 神是我们的避难所,在任何困境中,我们都可以投靠他(1-8节)。
    \item 在痛苦中,我们要真实向神倾诉,并坚定依靠他(9-18节)。

    \item 无论未来如何,我们要把自己交托给神,相信他的信实(19-24节)。

\end{enumerate}

今天,不论你正经历什么挑战,都让我们像大卫一样,学会全然投靠神,在他里面得着真正的安稳与力量!

\subsection*{结束祷告}

\textbf{亲爱的天父,}

我们感谢你,因为你是我们的避难所,我们的磐石,我们的拯救。主啊,我们承认,在困境中,我们常常软弱,常常焦虑,甚至怀疑你的带领。但今天,我们愿意选择信靠你。帮助我们在风暴中依靠你的手,在痛苦中仰望你的恩典,在未知的未来中交托自己给你。愿你的平安充满我们的心,让我们在任何环境中都能见证你的信实。

奉主耶稣基督的名祷告,阿们!
%----------------------------------------------------------------------------
\newpage
\section{诗篇第32篇:蒙福之人——罪的赦免与生命的更新}
\subsection*{引言}
各位弟兄姐妹,今天我们要一起思想《诗篇》第32篇。这篇诗是大卫所写的,他经历了罪的捆绑、认罪的痛苦,最终在神的赦免中得着释放和喜乐。今天我们也会面临类似的挣扎,我们都会犯错,会有软弱,但神的恩典比我们的罪更大。当我们真正悔改,神就赦免我们,使我们成为蒙福之人。

\subsection*{一、蒙福之人的特征(1-2节)}

\subsubsection*{1. 罪得赦免是最大的福分}
世界上有许多种“福”——财富、健康、事业成功、家庭和睦,但这些都比不上罪得赦免的福分。因为如果一个人罪不得赦免,他内心就没有真正的平安,灵魂也无法得到永恒的安息。
在现实生活中,我们可能会被过去的错误纠缠,被良心的责备压垮,甚至因为隐藏的罪行而活在恐惧之中。但神的恩典使我们得自由,他愿意赦免一切真心悔改的人。

\subsubsection*{2. 诚实的心才能得着真正的赦免}
第二节特别强调了“凡心里没有诡诈的”才是有福的。也就是说,我们不能在神面前虚假、掩盖自己的罪。很多时候,人习惯用各种理由掩饰自己的过犯,甚至欺骗自己。但神鉴察人心,他看重的是我们真实的悔改。

\subsubsection*{现实应用:}
你是否有隐藏的罪,没有向神坦承?
你是否愿意放下骄傲,在神面前认罪悔改?
\subsection*{二、隐瞒罪恶带来的痛苦(3-4节)}


\subsubsection*{1. 罪会使人内心煎熬}
大卫在犯罪后没有立即认罪,而是尝试隐瞒,这导致了他极大的痛苦。他的痛苦不仅是心理上的,也是身体上的——他“骨头枯干”,甚至感觉自己快要被耗尽。
这也是很多人的真实写照。当我们犯罪却不愿意悔改,良心会责备,内心无法平静,甚至影响身体健康。有些人因罪而焦虑、失眠,甚至患上抑郁症,因为罪的重担实在难以承受。

\subsubsection*{2. 神的管教是出于爱}
神没有让大卫一直沉浸在罪的麻木中,而是用“沉重的手”管教他。这表明神对我们的爱,他不愿意我们沉沦,而是希望我们回转。

\subsubsection*{现实应用:}

你是否曾因某个隐藏的罪而痛苦不安?
你是否经历过神的管教,使你回到正路?
\subsection*{三、认罪带来的释放(5节)}

\subsubsection*{1. 认罪是得赦免的关键}
大卫最终选择向神承认自己的罪,而不是继续掩盖。神的回应是——“你就赦免我的罪孽。”
这给了我们极大的盼望!无论我们的罪有多大,只要我们真心悔改,神都愿意赦免。

\subsubsection*{2. 认罪不是口头上的,而是心灵深处的悔改}
认罪不仅仅是嘴上说说,而是从心里真实地悔改,并愿意改变自己的行为。真心的悔改会带来生命的更新。

\subsubsection*{现实应用:}

你是否愿意向神敞开你的心,毫无保留地承认你的罪?
你是否愿意接受神的赦免,并活出新的生命?
\subsection*{四、投靠神的人必得保守(6-7节)}

\subsubsection*{1. 及时悔改,免得陷入更深的痛苦}
这里提醒我们,要趁神“可寻找的时候”回转,不要等到罪的代价变得更大,或者心灵更加刚硬才悔改。

\subsubsection*{2. 神是我们的避难所}
当我们真正投靠神,他就成为我们的藏身之处,保护我们脱离罪恶和试探。很多时候,我们在试探来临时,靠自己的力量挣扎,结果往往失败。唯有靠神,我们才能真正得胜。

\subsubsection*{现实应用:}

你是否把神当作你的避难所,还是依靠自己的聪明?
你是否在试探和困难中学会寻求神的帮助?
\subsection*{五、神的引导与应许(8-10节)}

\subsubsection*{1. 神愿意亲自指引我们的道路}
神不只是赦免我们的罪,还要引导我们走正路。他的带领比世上的任何智慧都可靠。

\subsubsection*{2. 不要刚硬自己的心}
第9节的骡马代表那些顽梗、不愿顺服神的人。如果我们心里刚硬,神就不得不用更大的管教来提醒我们。

\subsubsection*{现实应用:}

你是否愿意顺服神的引导,而不是固执己见?
你是否在生活中经历过神的带领和劝诫?
\subsection*{结论:喜乐的蒙福人生(11节)}

当我们真正经历神的赦免,我们的生命就会充满喜乐,不再被罪的重担压迫。真正的幸福不是来自外在的成就,而是来自与神和好的关系。

\subsection*{结束祷告}
\textbf{亲爱的天父,}

我们感谢你,因你的慈爱,我们这些本不配的人可以在你面前得赦免。求你鉴察我们的心,让我们勇敢地承认自己的过犯,放下骄傲和隐藏的罪。求你引导我们的脚步,使我们走在正直的路上。愿我们成为真正蒙福的人,靠你喜乐,向世人见证你的恩典。

奉主耶稣基督的名,阿们!
%-----------------------------------------------------------------------------
\newpage
\section{诗篇第33篇:赞美耶和华——信靠神的力量}
\subsection*{引言}
弟兄姐妹,今天我们要分享的是《诗篇》第33篇。这篇诗篇是一首充满敬拜和信靠的诗歌,呼吁我们以喜乐的心来赞美神,因为他是创造的主、全能的掌权者、信实的拯救者。在我们现实的生活中,我们时常会感到焦虑、不安,甚至对未来充满疑惑。但今天,我们要学习如何将我们的信心建立在神永不改变的信实之上,并以赞美回应他的恩典。

\subsection*{一、赞美神的伟大(1-5节)}

\subsubsection*{1. 义人的生命应当充满赞美}
诗篇一开始就告诉我们,义人(就是敬畏神、顺服神的人)应该靠耶和华欢乐。为什么?因为神是信实的,他的作为值得我们不断颂赞。
现实生活中,我们常常把喜乐建立在外在环境上,比如工作顺利、家庭和睦、身体健康。但圣经提醒我们,真正的喜乐来自对神的信靠,而不是外在的条件。

\subsubsection*{2. 赞美是一种主动的回应}
我们不仅仅是在礼拜时唱诗才算是敬拜,生活中的每一天,我们都可以用感恩的心来颂扬神,比如在困境中仍然相信神的美意,在挑战中仍然选择信靠。

\subsubsection*{现实应用:}
你是否因环境变化而失去喜乐?
你是否每天用感恩的心来敬拜神?
\subsection*{二、神的创造彰显他的权能(6-9节)}

\subsubsection*{1. 神是创造的主}
这些经文提醒我们,神用他的话语创造了天地,这表明神的话语带有无限的能力。
在科学如此发达的今天,人类探索宇宙、研究基因工程,但仍然无法完全理解神创造的奥秘。这让我们更加敬畏神,因他是宇宙的主宰。

\subsubsection*{2. 神的命令成就万事}
第9节说:“因为他说有,就有;命立,就立。” 这不仅适用于创造万物,也适用于我们的生命。神的话语是可信赖的,当他应许我们供应、保护、引导,我们就可以完全倚靠。

\subsubsection*{现实应用:}
你是否相信神掌管你人生的每一个细节?
你是否愿意顺服神的计划,而不是依靠自己的聪明?
\subsection*{三、神掌权超越人的计划(10-12节)}

\subsubsection*{1. 人的计划无法超越神的计划}
世界上的人都在谋算自己的未来,各国政府都在制定经济、军事、外交政策,但圣经告诉我们,最终掌权的是神。
在历史上,我们看到许多帝国兴起又衰落,人的计划可能短暂成功,但唯有神的旨意永远长存。因此,我们应当将自己的人生计划交托给神,而不是单靠自己的能力。

\subsubsection*{2. 以神为主的国家和个人才是真正蒙福的}
真正的福气不是物质上的,而是神的同在和引导。当一个国家、一个家庭、一个人以神为中心,他们就会在神的祝福中成长。

\subsubsection*{现实应用:}
你是否在面对未来时,更多依靠神的带领,而不是自己的计划?
你是否愿意在工作、家庭、学业中,把神放在首位?
\subsection*{四、神察看人的心(13-19节)}

\subsubsection*{1. 神鉴察我们的内心}
我们可以在人前隐藏自己的真实想法,但神察看我们的心,知道我们真正的信仰和依靠是什么。

\subsubsection*{2. 人的力量不能带来真正的安全}
君王不能靠军队得胜,勇士不能靠体力得救,财富不能保障人生的幸福。真正的安全感来自信靠神,而不是世上的资源。

\subsubsection*{现实应用:}
你是否把你的安全感建立在金钱、地位、人际关系上?
你是否愿意完全信靠神,相信他会供应你的一切需要?
\subsection*{五、等候神的拯救(20-22节)}

\subsubsection*{1. 信靠神带来真正的平安}
等候神意味着信任他的时间和计划,不焦急、不抱怨,而是耐心等候。

\subsubsection*{2. 赞美是信心的表达}
诗篇最后强调,我们的心“必靠他欢喜”,因为信靠神的人不会被环境打倒,而是能够在等待的过程中仍然喜乐。

\subsubsection*{现实应用:}
你是否愿意耐心等候神的时间,而不是急于求成?
你是否能在困难中依然喜乐,相信神的拯救必然来到?
\subsection*{结论:信靠神,满有喜乐}
《诗篇》第33篇提醒我们,神是全能的创造主、信实的掌权者、慈爱的拯救者。我们的安全感、喜乐和未来都在他的手中,因此我们应当以赞美回应他的恩典,并全然信靠他的带领。

\subsection*{结束祷告}
\textbf{亲爱的天父,}

我们感谢你,因为你是全地的主宰,你的话语坚定,你的慈爱长存。帮助我们在生活中不依靠自己的聪明,而是全然信靠你。无论顺境或逆境,我们都愿意以喜乐的心赞美你。

奉主耶稣基督的名,阿们!
%-----------------------------------------------------------------------------
\newpage
\section{诗篇第34篇:在困境中依靠主}
\subsection*{引言}
\hspace{0.6cm}弟兄姐妹,今天我们要一起学习诗篇第34篇。这篇诗是大卫在逃避亚比米勒王时写的(撒上21:10-15),他经历了极大的艰难,但在困境中,他没有绝望,而是选择了依靠神,并鼓励我们也这样行。

诗篇34篇不仅是一首感恩的诗,也是一篇充满信心的宣告。它向我们揭示了一个伟大的属灵原则——\textbf{无论身处顺境还是逆境,我们都当寻求主,信靠主,他必拯救我们,供应我们,引导我们}。

今天,我们就从三个方面来思考这篇诗篇如何指导我们在现实生活中依靠神。
\subsection*{一、在困境中仍要赞美神(诗34:1-3)}
\hspace{0.6cm}“我要时时称颂耶和华,赞美他的话必常在我口中。”(诗34:1)

大卫当时正在逃亡,处境极度危险,但他的第一反应不是抱怨,而是\textbf{赞美神}。这对我们来说,是一个极大的提醒。在我们遭遇失败、困难、失望甚至逼迫的时候,我们的反应是什么?是埋怨,还是选择向神献上赞美?

\vspace{0.2cm}

\textbf{实际应用:}

在生活中,我们难免会遇到挫折,比如学业上的压力、职场的挑战、人际关系的矛盾等。但神教导我们,即使在痛苦中,也要选择仰望他。\textbf{当我们愿意赞美神,我们的眼光就不再定睛于问题,而是定睛于那位掌管万有的主。}

\vspace{0.2cm}

\textbf{挑战:}

下次当你面对困境时,不妨尝试先向神献上感恩,哪怕只是一句简单的:“主啊,我相信你仍然掌权,我愿意依靠你。”你会发现,神的平安会临到你的心中。

\subsection*{二、寻求神,他必回应(诗34:4-10)}
\hspace{0.6cm}“我曾寻求耶和华,他就应允我,救我脱离了一切的恐惧。”(诗34:4)

大卫在危难中寻求神,而神真的回应了他。这给了我们一个极大的应许:\textbf{无论我们面对什么,只要寻求神,他必垂听。}

\vspace{0.2cm}

\textbf{实际应用:}

在现实生活中,我们常常因为未来的不确定性而感到焦虑,比如害怕失败、担心经济问题、害怕被人误解等。但诗篇34:4告诉我们,\textbf{当我们寻求神,他不仅会赐给我们答案,更重要的是,他会除去我们的恐惧。}

\vspace{0.2cm}

\textbf{见证分享:}

有一位基督徒姐妹,她面临职场的巨大压力,甚至有了焦虑症。但她开始每天花时间读经祷告,把焦虑交托给神。渐渐地,她发现自己的内心变得安稳,神也为她开了新的出路。

\vspace{0.2cm}

\textbf{挑战:}

当你面对恐惧时,不要选择逃避,而是勇敢地来到神面前,把你的重担交给他。你会发现,他真的会帮助你。

\subsection*{三、敬畏神,得蒙祝福(诗34:11-22)}
\hspace{0.6cm}“你们要尝尝主恩的滋味,便知道他是美善,投靠他的人有福了!”(诗34:8)

大卫不仅自己经历了神的恩典,也呼召其他人一同来信靠神。他特别强调,\textbf{敬畏神的人必蒙福。}

\vspace{0.2cm}

\textbf{实际应用:}

在当今社会,很多人追求成功、财富和名声,但却忽略了敬畏神的重要性。真正的祝福不在于外在的财富,而在于内心的满足和平安。\textbf{当我们愿意按着神的方式生活,远离罪恶,行公义,神就会赐福我们。}

\vspace{0.2cm}

\textbf{挑战:}
\begin{itemize}
    \item 在你的生活中,有没有哪些地方需要调整,使自己更符合神的心意?

    \item 你是否愿意选择敬畏神,而不是随从世界的潮流?

\end{itemize}

% \vspace{0.2cm}

\textbf{行动建议:}
\begin{itemize}
    \item 每天花时间读神的话,认识他的心意。

    \item 远离罪恶,选择顺服神的带领。

    \item 在工作、学习和人际关系中,以基督的心为心,做正直的人。

\end{itemize}

\subsection*{结论:在主里得胜}
\hspace{0.6cm}诗篇34篇提醒我们,\textbf{无论何时何地,我们都可以依靠神。}他应许:
\begin{enumerate}
    \leftskip=0.7cm
    \item 凡是寻求他的,他必垂听(34:4)。
    \item 敬畏他的人,必得他的保护(34:7)。

    \item 依靠他的人,必得他的供应(34:9-10)。

\end{enumerate}

愿我们都能在生活中真实地经历神的信实,无论是在顺境还是逆境,都坚定地信靠他。

\subsection*{结束祷告}
\textbf{亲爱的天父,}

感谢祢的话语,让我们看见,在困境中仍然可以赞美祢,寻求祢,并且敬畏祢。主啊,我们承认,许多时候我们面对困难时,会感到害怕,会选择依靠自己的力量,但今天,我们愿意学习大卫的榜样,来寻求祢。求祢帮助我们,让我们在任何情况下,都能经历祢的信实和恩典。谢谢祢的带领,我们愿意一生跟随祢。

祷告奉主耶稣基督的名,阿们!
%-----------------------------------------------------------------------------
\newpage
\section{诗篇第35篇:当遭遇不公时如何依靠神}
\subsection*{引言}
\hspace{0.6cm}弟兄姐妹,今天我们要一起学习诗篇第35篇。这是一篇呼求神伸张公义的诗篇,由大卫在面对冤屈、逼迫和攻击时所写。大卫并没有选择靠自己的力量报复,而是把一切交托给神,求神为他伸冤。

在现实生活中,我们或许也会遭遇不公平的对待:


\begin{itemize}

    \item 在职场上被小人陷害、被误解或遭受不公待遇;
    \item 在人际关系中,被朋友背叛、被人造谣中伤;

    \item 在社会中,看见恶人昌盛,而行义的人却遭受苦难。

\end{itemize}

那么,当我们面对这样的困境时,我们该如何回应?诗篇35篇给了我们三个关键的属灵原则:

\subsection*{一、把冤屈交给神,他必伸张公义(诗35:1-10)}

大卫在诗篇的开头,直接向神呼求帮助,求神为他争战。他没有选择靠自己的方式去报复,而是把一切交托给神。

\subsubsection*{现实应用:当我们被人误解时,我们该如何反应?}
当别人诬陷我们,甚至恶意攻击我们时,我们的第一反应可能是愤怒、争辩,甚至想要反击。但大卫的做法提醒我们:真正的信心是让神来伸张公义,而不是靠自己解决问题。

\subsubsection*{圣经应许:}

\hspace{0.6cm}申命记32:35——“伸冤在我,我必报应。”

罗马书12:19——“不要自己伸冤,宁可让步,听凭主怒。”

\subsubsection*{挑战:}
你最近是否在某件事上受了委屈?
你是否愿意放下自己的怒气,把这件事交给神?
\subsubsection*{行动建议:}

选择祷告,而不是争论:当遭遇误解时,不要急着辩解,而是先到神面前祷告。
依靠神的公义,而不是人的方法:即使短时间内看不到神的伸冤,也要相信神的时间是最完美的。
\subsection*{二、面对恶意攻击时,我们仍要持守正直(诗35:11-18)}

大卫在这里描述了一个令人痛心的现实:他曾善待那些人,但他们却回报他以恶(诗35:12)。

\subsubsection*{现实应用:当你的善意被利用时,你会怎么做?}
我们可能都有类似的经历:
\begin{itemize}
    \item 曾经帮助过的朋友,在关键时刻背叛了我们;

    \item 认真工作,却被上司或同事排挤;

    \item 竭尽所能去爱家人,却遭遇冷漠和误解。

\end{itemize}

人的本能反应是:以后我再也不会帮助别人了! 但大卫没有这样做。他仍然选择正直地行事,并把所有的痛苦向神倾诉,而不是选择苦毒和仇恨。

\subsubsection*{挑战:}
你是否因为过去的伤害,而不愿再信任人?
你是否愿意像大卫一样,即使受伤,仍然保持正直?
\subsubsection*{行动建议:}

不要因别人的恶,而改变自己行善的心(加拉太书6:9)。
把你的伤害交给神,而不是让苦毒占据你的心(以弗所书4:31)。
\subsection*{三、在等待神伸冤时,仍然选择赞美(诗35:19-28)}

大卫并没有等到问题解决后才开始赞美神,他在等待的过程中,就已经选择敬拜神。这是一种极大的信心——相信神的公义,即使暂时还没有看到答案。

\subsubsection*{现实应用:当你看不到神的作为时,你是否仍然愿意赞美他?}
有时候,我们可能会想:“神啊,为什么祢迟迟不行动?为什么恶人还在得意?为什么我仍然被误解?” 但大卫提醒我们,神的时间不等于我们的时间,而真正的信心,是在等待的过程中,仍然选择相信神的良善。

\subsubsection*{挑战:}
你是否曾因神“没有马上伸冤”而心生埋怨?
你是否愿意学习在等待的季节里,仍然赞美神?
\subsubsection*{行动建议:}

写下你仍然可以感恩的事情,即使你现在仍然处于困境。
用诗歌和祷告来宣告神的公义,即使现在看不到答案。
\subsection*{结论:交托给神,得享平安}
诗篇35篇告诉我们,当遭遇不公时,我们可以:
\begin{itemize}
    \item 把冤屈交给神,他必伸张公义(诗35:1-10)。

    \item 持守正直,不让苦毒掌控我们的心(诗35:11-18)。

    \item 在等待的过程中,仍然选择赞美神(诗35:19-28)。

\end{itemize}

神是公义的神,他必不丢弃他的子民。愿我们都能学习大卫的信心,把一切交托给神,在他的时间里,必定看见他的拯救!

\subsection*{结束祷告}
\textbf{亲爱的天父,}

我们感谢祢,因为祢是公义的神。当我们面对不公、被误解、甚至被攻击时,求祢帮助我们,把一切交托给祢,而不是靠自己的方法解决。主啊,求祢赐给我们一颗正直的心,让我们即使受伤,仍然愿意行善,不被苦毒所掌控。并且,在等待祢伸冤的日子里,让我们的口仍然充满赞美,因为我们知道,祢的时间是最完美的!

奉主耶稣基督的名祷告,阿们!
%----------------------------------------------------------------------------
\newpage
\section{诗篇第36篇:在黑暗世界中仰望神的慈爱}
\subsection*{引言}
\hspace{0.6cm}弟兄姐妹,今天我们要一起学习诗篇第36篇。这篇诗篇分为两个鲜明的对比:前半部分(1-4节)描述了恶人的黑暗和败坏,后半部分(5-12节)则展现了神的慈爱和公义。

在当今社会,我们常常会看到邪恶猖獗、道德沦丧,似乎恶人亨通,义人受苦。面对这样的世界,我们应当如何生活?我们是否会灰心、妥协,甚至随波逐流?诗篇36篇告诉我们,在黑暗的世代,我们要选择仰望神的慈爱,活出圣洁的生命!

今天,我们将从三方面来思考这篇诗篇的信息,并将其应用到我们的日常生活中。

\subsection*{一、恶人的本质:远离神,沉溺罪中(诗36:1-4)}

大卫在这里描述了恶人的心态,他们的特点包括:
\begin{itemize}
    \item 心中没有敬畏神(1节):他们行事为恶,却不觉得有错,因为他们不相信神会审判。

    \item 自我欺骗(2节):他们以为自己的邪恶不会被发现,甚至为自己的罪找借口。

    \item 嘴里满是诡诈和谎言(3节):他们的话语没有真实,而是用谎言操纵别人。

    \item 昼夜筹划恶事,不行善道(4节):罪恶已成为他们的生活方式,他们沉溺其中,无法自拔。

\end{itemize}
\subsubsection*{现实应用:这个世界的黑暗影响了我们吗?}
\hspace{0.6cm}在职场中,有些人为了升职,选择欺骗和拉帮结派,我们是否会妥协?

在学校里,有些人嘲笑信仰和道德,我们是否会随波逐流?

在社交媒体上,很多人传播谎言、攻击他人,我们是否也参与其中?
\subsubsection*{挑战:}
面对世上的邪恶,我们是否仍然愿意持守信仰,不随从恶人的脚步?

\subsubsection*{行动建议:}

时刻警醒自己是否正在接受世界的价值观,而不是神的价值观(罗12:2)。
在生活中坚持公义,即使这意味着会受损失,也不要随波逐流(诗1:1-2)。
\subsection*{二、神的属性:他的慈爱、信实、公义和保护(诗36:5-9)}

大卫在这里描述了神的伟大属性,与恶人的黑暗形成了鲜明对比。

1. 神的慈爱(5节)
神的慈爱是无尽的,像高耸的天空,远远超越我们的理解。无论我们遭遇何种挑战,神的爱始终不变。

2. 神的信实(5节)
在人世间,我们常常会遭遇背叛,但神却永远信实,他不会欺骗或遗忘他的子民。

3. 神的公义(6节)
即使这个世界充满不公,神仍然掌权,他最终必审判恶人,伸张公义。

4. 神的保护(7节)
“人之子都投靠在你翅膀的荫下。”(7节)——无论外界如何动荡,投靠神的人都能得着真正的平安。

5. 神的丰富供应(8-9节)
在神那里,我们可以得着生命的泉源(9节),他供应我们一切所需,不仅是物质上的,更是属灵上的滋养。
他的光照亮我们前行的道路(9节),使我们不会陷入黑暗。

\subsubsection*{现实应用:在困难中,我们是否真的依靠神?}
当经济紧张时,我们是否相信神是供应的主?
当遭遇朋友或家人的伤害时,我们是否愿意投靠神的信实,而不是陷入苦毒?
当面对未来的不确定性时,我们是否仍然相信神的公义必然显明?
\subsubsection*{挑战:}
你是否真正经历过神的慈爱、信实和保护?还是只是“知道”这些道理,但在实际生活中仍然靠自己?

\subsubsection*{行动建议:}

每天数算神的恩典,提醒自己他的信实(诗103:2)。
遇到挑战时,选择向神祷告,而不是单靠自己解决问题(腓4:6-7)。
\subsection*{三、我们的回应:远离恶行,选择公义(诗36:10-12)}

大卫在诗的结尾,为义人祷告,求神让他们在恶人的逼迫中仍然坚立,不被恶者践踏(11节)。他深知,恶人终将灭亡,而义人将被神扶持(12节)。

\subsubsection*{现实应用:如何在黑暗世界中持守公义?}
\begin{itemize}
    \item 敬畏神,不效法恶人的行为(诗36:1-4)。

    \item 不断亲近神,经历他的慈爱和信实(诗36:5-9)。

    \item 为世界的公义祷告,并活出神的光(诗36:10-12)。

\end{itemize}
\subsubsection*{挑战:}
你是否愿意在黑暗的世代做光,而不是随从世界的价值观?

\subsubsection*{行动建议:}

\hspace{0.6cm}每天默想神的话,建立属灵的分辨力(诗119:105)。

在人际关系中,以爱回应恶行,不被苦毒控制(罗12:21)。

在工作和学习中,持守诚信,做正直的人(箴言11:3)。

\subsection*{结论:活出与世界不同的生命}
诗篇36篇提醒我们:
\begin{enumerate}
    \item 世界的邪恶无处不在,但我们不属于这世界。

    \item 神的慈爱、公义和信实永不改变,他必保护敬畏他的人。

    \item 我们应当活出圣洁的生命,远离恶行,成为光和盐。

    \item 无论世界如何,我们要坚定地信靠神,仰望他的慈爱,并在他的光中前行。

\end{enumerate}

\subsection*{结束祷告}
\textbf{亲爱的天父,}
我们感谢祢的话语,提醒我们在黑暗的世界中,不要迷失方向。主啊,我们承认,很多时候,我们也会被世界的罪恶影响,甚至想要妥协。求祢赐给我们一颗敬畏祢的心,使我们远离恶行,持守正直。求祢用祢的慈爱环绕我们,让我们经历祢的信实,并活出祢的公义。愿我们的一生都行在祢的光中,荣耀祢的名!奉主耶稣基督的名祷告,阿们!
%------------------------------------------------------------------------------
\newpage
\section{诗篇第37篇:在恶人昌盛的世界中持守信靠}
\subsection*{引言}
\hspace{0.6cm}弟兄姐妹,你是否曾因恶人昌盛而困惑?是否曾因自己行义却受苦而感到不公?诗篇37篇正是大卫在面对类似问题时,从神领受的智慧之言。

在这篇诗篇中,大卫向我们展示了两个对比:恶人的短暂兴旺与义人的最终福分。他鼓励我们不要因恶人得势而焦虑,而是专心信靠神,持守公义,耐心等候神的作为。

今天,我们要透过诗篇37篇的教导,学习三项属灵原则,并将其应用到我们的日常生活中。

\subsection*{一、不要因恶人昌盛而焦虑,乃要信靠神(诗37:1-9)}
“不要为作恶的心怀不平,也不要向行不义的生出嫉妒。”(诗37:1)

1. 恶人终将衰微(2节)
大卫告诉我们,恶人虽然暂时兴旺,但最终必如草被割下,如青菜枯干(2节)。他们的成功是短暂的,而义人的福分是永恒的。

2. 义人当以耶和华为乐(3-4节)
“当倚靠耶和华而行善……又要以耶和华为乐,他就将你心里所求的赐给你。”(诗37:3-4)
\textbf{不要焦虑,而要信靠神}:世人的成功或失败不能决定我们的命运,唯有神掌管一切。
\textbf{在神里面寻求满足}:如果我们以世界的成功为乐,我们的心就会动摇;但如果我们以神为乐,我们的心就得稳固。

3. 将你的道路交托神,耐心等候(5-7节)
“当将你的事交托耶和华,并倚靠他,他就必成全。”(诗37:5)

许多人在等待神的应许时变得焦急,甚至想要靠自己的方式“帮神一把”。
但大卫教导我们,神的时间是最完美的,他必按他的旨意成全我们(7节)。
\subsubsection*{现实应用:我们如何面对现实中的不公?}
\hspace{0.6cm}当看到不法之徒发财,而诚实的人反而吃亏时,我们会不会埋怨神?

当别人用欺骗和权术获得成功,而我们因坚守正道而受苦时,我们是否仍然信靠神?
\subsubsection*{挑战:}

你是否因恶人昌盛而焦虑?
你是否真正相信神的计划比人的计谋更稳固?
\subsubsection*{行动建议:}

选择信靠,而不是抱怨:每天将你的忧虑交托神,提醒自己神是掌权者。
调整你的满足感:不要让财富和地位决定你的喜乐,而是让神成为你最大的满足。
\subsection*{二、恶人的命运 vs. 义人的盼望(诗37:10-22)}
“还有片时,恶人要归于无有;你就是细察他的住处,也要归于无有。”(诗37:10)
\begin{itemize}
    \item 1. 恶人短暂的得意 vs. 义人永恒的基业(10-11节)恶人的势力是短暂的,他们最终要消失(10节)。义人必承受地土,并以丰盛的平安为乐(11节)。这不仅是对以色列的应许,也是对所有敬畏神之人的祝福——他们将在神国度中得享基业。
    \item 2. 恶人的计谋无法胜过义人(12-15节)恶人可能用计谋谋害义人,但最终,他们设下的网罗会落到自己身上(15节)。神必保护敬畏他的人,不让恶人得逞。

    \item 3. 义人蒙福,恶人灭亡(16-22节)“义人的一点点,胜过许多恶人的富余。”(诗37:16)即使义人暂时贫乏,他们因神的同在而富足;而恶人的财富最终必归于无有。

\end{itemize}



\subsubsection*{现实应用:如何面对邪恶势力?}
\hspace{0.6cm}当看到社会上的腐败、不公和恶行猖獗时,我们是否仍然相信神掌权?

当有人用不正当的手段获取利益,而我们因行义受苦时,我们是否会灰心?
\subsubsection*{挑战:}

你是否相信神最终会审判恶人?
你是否愿意等候神的赏赐,而不是急于追求世界的成功?
\subsubsection*{行动建议:}

\hspace{0.6cm}用神的眼光看待财富和成功:衡量人生价值,不是看金钱,而是看与神的关系。

为公义祷告,不因恶人得势而动摇。
\subsection*{三、持守公义,神必引导(诗37:23-40)}
“义人的脚步被耶和华立定,他的道路,耶和华也喜爱。”(诗37:23)

1. 神必扶持义人(23-26节)
义人的脚步是神所引导的,即使跌倒,也不会全然仆倒,因为神的手托住他们(24节)。
义人不必担忧未来,因为神必供应他们所需(25-26节)。

2. 远离恶行,坚守正道(27-31节)
“当离恶行善,就可永远安居。”(诗37:27)
义人的嘴里充满智慧,他们的脚步稳行不移,因为他们遵行神的话(30-31节)。

3. 等候神,他必拯救(32-40节)
恶人虽然谋害义人,但神必拯救他的子民(32-33节)。
义人等候神,神必使他们最终得胜(34节)。
\subsubsection*{现实应用:如何在困境中坚守信仰?}
\hspace{0.6cm}当面对不公时,我们是否愿意忍耐,等候神的作为?

当身处黑暗的环境时,我们是否仍然相信神是我们的盼望?
\subsubsection*{挑战:}

你是否愿意坚守公义,而不是被环境影响?
你是否相信神的赏赐远胜过世界的诱惑?
\subsubsection*{行动建议:}

每日读经,让神的话引导你的脚步(诗119:105)。
在困难中,坚持依靠神,而不是急于求成(箴言3:5-6)。
\subsection*{结论:恶人短暂得势,义人最终蒙福}
诗篇37篇提醒我们:
\begin{enumerate}
    \item 恶人的昌盛是短暂的,最终他们要归于无有。

    \item 义人要信靠神,行善,耐心等候神的审判与赏赐。

    \item 神必扶持敬畏他的人,指引他们的脚步,使他们得享永恒的基业。

    \item 无论这个世界如何变化,让我们坚守信仰,倚靠神的应许,忠心跟随他!

\end{enumerate}

\subsection*{结束祷告}
\textbf{亲爱的天父,}
我们感谢祢的话语,提醒我们不要为恶人得势而忧虑,而要专心信靠祢。求祢帮助我们,在面对不公和困难时,仍然持守公义,耐心等候祢的作为。愿我们的一生都行在祢的旨意中,成为光和盐,见证祢的信实!
奉主耶稣基督的名祷告,阿们!
%-----------------------------------------------------------------------------
\newpage
\section{诗篇第38篇:罪的痛苦与神的怜悯}
\subsection*{引言}
弟兄姐妹,你是否曾因罪恶的重担而感到痛苦? 你是否经历过因犯罪而带来的内疚、羞愧,甚至身体上的痛苦?诗篇38篇是大卫在极度痛苦中向神的哀求,这篇诗篇被称为“认罪诗篇”,它向我们展示了罪的沉重后果、悔改的必要性,以及神的怜悯与拯救。今天,我们将透过诗篇38篇,学习罪的影响、悔改的重要性,以及如何在痛苦中转向神,寻求他的医治与怜悯。

\subsection*{一、罪带来的痛苦与重担(诗38:1-8)}
“耶和华啊,求你不要在怒中责备我,也不要在烈怒中惩罚我。”(诗38:1)

1. 罪使人与神的关系受损(1-2节)
大卫感受到神的责备,他知道自己的罪带来了神的惩罚(1节)。
神的怒气如利箭射入他的内心,使他无法逃避罪的后果(2节)。
罪的可怕之处在于,它不只是让人做错事,而是让人远离神、失去与神的亲密关系。

2. 罪带来的身体和心灵的痛苦(3-8节)
“因你的恼怒,我的肉无一完全;因我的罪过,我的骨头也不安宁。”(诗38:3)

大卫形容自己因罪而身心俱疲:
\begin{itemize}
    \item 肉体受损(3节)——有时,罪的压力会导致实际的健康问题,如焦虑、失眠、甚至疾病。

    \item 罪恶的重压让人不堪重负(4节)——罪使人内疚、自责,如同重担压得喘不过气。

    \item 内心焦虑、精神崩溃(6节)——罪会带来精神上的折磨,使人感到沮丧、忧伤。

\end{itemize}



\subsubsection*{现实应用:罪在我们生命中的影响}
\hspace{0.6cm}你是否曾因犯罪而感到良心不安,甚至影响到你的身体健康?

你是否经历过罪的重压,让你失去平安,甚至与你所爱的人关系破裂?
\subsubsection*{挑战:}

\hspace{0.6cm}我们是否认真看待罪,还是常常轻忽自己的过犯?

当罪带来痛苦时,我们是否愿意向神承认,并寻求他的怜悯?
\subsection*{二、悔改的必要性(诗38:9-14)}
“主啊,我的心愿都在你面前;我的叹息不向你隐瞒。”(诗38:9)

1. 罪使人失去力量和喜乐(9-10节)
大卫承认自己的软弱:“我的心跳动,我的力量衰微,连我眼中的光也没有了。”(诗38:10)
罪不仅让人失去身体的健康,也会带走内心的平安和灵性的活力。

2. 罪使人与他人的关系破裂(11-12节)
“我的良朋密友因我的灾病都躲在旁边站着;我的亲戚也远远地站立。”(诗38:11)
罪的影响不仅是个人的,也会波及我们的家庭、朋友和人际关系。

3. 悔改需要真诚地寻求神(13-14节)
大卫选择沉默,不与人争辩,而是转向神。
真正的悔改是面对神,而不是试图为自己辩护。
\subsubsection*{现实应用:悔改带来的改变}
\hspace{0.6cm}你是否曾经历罪的影响,使你与人疏远,甚至被误解?

你是否曾因罪的后果而感到无助,不知道如何修复破裂的关系?
\subsubsection*{挑战:}

\hspace{0.6cm}我们是否愿意像大卫一样,真实地向神倾诉,而不是试图掩盖罪?

我们是否愿意让神来修复破裂的关系,而不是靠自己的努力去弥补?
\subsection*{三、神的怜悯与拯救(诗38:15-22)}
“主啊,我仰望你;主-我的神啊,你必应允我。”(诗38:15)

1. 依靠神,而不是自己(15节)
真正的悔改不只是承认罪,也包含信靠神,相信他会赦免我们。
大卫知道,自己无法靠行为赎罪,唯有神的怜悯能使他得释放。

2. 承认自己的罪,并寻求神的帮助(17-18节)
“我要承认我的罪孽;我要因我的罪忧愁。”(诗38:18)
真正的悔改带来忧伤,但不是绝望的忧伤,而是转向神的痛悔之心(林后7:10)。

3. 神的救赎是我们唯一的盼望(21-22节)
“耶和华啊,求你不要撇弃我;我的神啊,求你不要远离我。”(诗38:21)
“拯救我的主啊,求你快快帮助我。”(诗38:22)
大卫呼求神的救赎,因为他知道,神的怜悯比人的罪更大,神的爱比人的失败更深。
\subsubsection*{现实应用:在罪的痛苦中如何寻求神?}
\hspace{0.6cm}你是否愿意放下自己的骄傲,真实地向神承认罪,并寻求他的赦免?

你是否相信,无论你的罪有多深,神的恩典总是够你用的?
\subsubsection*{挑战:}

\hspace{0.6cm}我们是否常常在犯罪后逃避神,而不是转向神?

我们是否愿意真心悔改,而不是只在困难时才向神求帮助?
\subsubsection*{行动建议:}

\hspace{0.6cm}每天花时间在神面前省察自己,向神承认任何隐藏的罪。

若因罪疏远了某个关系,求神给你勇气去修复它。

牢记神的应许:“我们若认自己的罪,神是信实的,是公义的,必要赦免我们的罪,洗净我们一切的不义。”(约壹1:9)
\subsection*{结论:罪的痛苦 vs. 神的怜悯}
诗篇38篇提醒我们:

\begin{enumerate}
    \item 罪带来痛苦,使人与神和人隔绝。

    \item 悔改是恢复关系的唯一途径。

    \item 神的怜悯大过我们的罪,只要我们真实悔改,他必赦免。

    
\end{enumerate}

无论你过去的罪有多深,今天都是一个回转的机会。神在等待你,他的怜悯永不止息!

\subsection*{结束祷告}
\textbf{亲爱的天父,}

我们来到你面前,承认我们是软弱、有罪的人。求你赦免我们的过犯,洗净我们,使我们重新得力。求你医治因罪破裂的关系,使我们回到你的怀抱中。愿我们的生命不再被罪捆绑,而是因你的怜悯而自由。

奉主耶稣基督的名祷告,阿们!
%----------------------------------------------------------------------------
\newpage
\section{诗篇第39篇:生命的短暂与智慧的数算}
\subsection*{引言}
\hspace{0.6cm}弟兄姐妹,你是否曾思考过生命的短暂?在忙碌的生活中,我们常常被琐事缠绕,却忽略了最重要的问题:我们的生命到底是为了什么? 诗篇39篇是大卫在深思生命时的祷告,他反思了人生的短暂、世上的虚空,以及如何在有限的生命中活出神的旨意。

今天,我们将透过诗篇39篇,思考三个核心问题:
\begin{itemize}
    \item 我们如何在言语和行为上谨慎行事?

    \item 如何正确看待人生的短暂?

    \item 如何在短暂的人生中寻求神的智慧和怜悯?

\end{itemize}

\subsection*{一、谨慎言行,控制舌头(诗39:1-3)}
“我要谨慎我的言行,免得我舌头犯罪。”(诗39:1)

1. 言语的影响力
大卫意识到自己需要谨慎,不在恶人面前说出抱怨的话(1节)。
他说:“我要用嚼环勒住我的口。”(1节),表明控制舌头的重要性。
言语可以带来生命,也可以带来毁灭(箴18:21),我们要学习控制自己的舌头,不随意发怨言或说出伤害人的话。

2. 内心的挣扎
大卫在痛苦中选择沉默,但他的心却因困苦而焦急(2-3节)。
他最终无法压抑自己的情绪,转而向神倾诉,而不是向人抱怨。

\subsubsection*{现实应用}

\hspace{0.6cm}我们是否曾因一时冲动的话语而伤害他人?

我们是否常在困难中埋怨,而不是寻求神?

面对误解和挑战时,我们是否选择智慧的沉默,而不是情绪化的反应?

\subsubsection*{挑战:}

\hspace{0.6cm}在愤怒或痛苦时,先祷告,不轻易出口伤人。

以正面的言语造就人,而不是用抱怨或恶语拆毁关系。
\subsection*{二、认识人生的短暂(诗39:4-6)}
“耶和华啊,求你叫我晓得我身之终,我的寿数几何。”(诗39:4)

1. 生命如影,短暂无定
大卫祷告求神让他明白人生的有限,以免虚度光阴(4节)。
“看哪,你使我的年日窄如手掌。”(5节)——生命短暂,犹如手掌宽度。
“世人行动实系幻影。”(6节)——世人的劳碌若没有神的旨意,就像影子一样虚空。

2. 现实的反思
我们是否曾被生活的忙碌吞噬,却忘了生命的真正意义?
我们是否以为自己有很多时间,却不知生命的终点何时来临?

\subsubsection*{挑战:}

\hspace{0.6cm}重新调整人生的优先次序,把追求神的国度放在第一位(太6:33)。

不要把全部精力都投在短暂的物质成就上,而是投资在永恒的事物(如灵命成长、家庭关系、服事)。
\subsection*{三、在短暂人生中仰望神(诗39:7-13)}
“主啊,如今我等什么呢?我的指望在乎你。”(诗39:7)

1. 唯有神是我们的盼望
世上的财富、权力、名声都不能满足人心,唯有神是人真正的盼望(7节)。
人若无神,就像“寄居的”与“客旅”(12节),在世上漂泊,没有真正的归属感。

2. 罪的管教与神的怜悯
大卫承认自己的罪,接受神的管教(8-11节)。
他向神求怜悯,盼望神挪去他的愁苦,而不是单靠自己去解决问题(13节)。
\subsubsection*{现实应用}
\hspace{0.6cm}当你面对人生的不确定性时,你是依靠自己,还是转向神?

当你在痛苦中,你是埋怨神,还是祷告寻求他的帮助?
\subsubsection*{挑战:}

\hspace{0.6cm}学习在困境中依靠神,而不是依赖自己的能力或外在的成就。

把每一天都当作神所赐的礼物,用感恩的心去活出他的旨意。
\subsection*{结论:智慧地数算自己的日子}
\begin{enumerate}
    \item 谨慎言语,不随意发怒或抱怨。

    \item 明白生命短暂,不虚度光阴。

    \item 把盼望放在神身上,而不是世上的成就。

    
\end{enumerate}


诗篇39篇提醒我们,生命有限,但若我们活在神的旨意中,我们的生命就充满意义。愿我们学习像摩西所祷告的:
“求你指教我们怎样数算自己的日子,好叫我们得着智慧的心。”(诗90:12)

\subsection*{结束祷告}
\textbf{亲爱的天父,}

我们感谢你透过诗篇39篇提醒我们,生命短暂,我们当谨慎言行,不虚度光阴。主啊,求你指教我们怎样数算自己的日子,让我们每一天都活在你的旨意中,不被世界的虚荣捆绑,而是单单仰望你。求你赦免我们的罪,帮助我们在困难中信靠你,在短暂的生命中活出你的荣耀。

奉主耶稣基督的名祷告,阿们!
%-----------------------------------------------------------------------------
\newpage
\section{诗篇第40篇:从绝望到得救}
\subsection*{引言}
\hspace{0.6cm}弟兄姐妹,你是否曾经历过人生的低谷?是否有过被困在困难、罪恶或痛苦之中的感觉?在这些时刻,你如何面对?今天,我们一同来学习诗篇40篇,看到神如何将我们从困境中拯救出来,并使我们的生命充满新的盼望。

诗篇40篇可以分为三部分:
\begin{itemize}
    \item 神的拯救(1-5节)——等候神,他必施恩

    \item 我们的回应(6-10节)——顺服神,活出他的旨意

    \item 再一次呼求(11-17节)——持续信靠神,依靠他到底

\end{itemize}

让我们一起进入今天的学习!

\subsection*{一、等候神,他必施恩(诗40:1-5)}
“我曾耐性等候耶和华,他垂听我的呼求。”(诗40:1)

1. 神如何拯救我们?

神听我们的祷告(1节):大卫在困境中“耐性等候”神,并经历了神的垂听。

神将我们从绝望中拯救出来(2节):
“从祸坑里、从淤泥中把我拉上来”——代表困境、罪恶、绝望的深渊。
“使我的脚立在磐石上”——神是我们坚固的倚靠。

神使我们有新的盼望(3节):
“他使我口唱新歌”——当神拯救我们,我们的生命会有新的敬拜和见证。

2. 我们如何等候神?

“耐性等候”并不是消极等待,而是积极地信靠,相信神的时间和计划。

\subsubsection*{现实应用:}
\hspace{0.6cm}你是否正处于低谷,不知道何去何从?

你是否因等待神的回应而感到焦虑?

神应许他会施行拯救,但他的时间不一定是我们的时间,我们要学会耐心等候。
\subsubsection*{挑战:}

\hspace{0.6cm}在等待神的拯救时,我们是否愿意信靠他,还是选择抱怨和焦虑?

行动建议:每天用5分钟时间祷告,安静在神面前,相信他的带领。
\subsection*{二、顺服神,活出他的旨意(诗40:6-10)}
“祭物和礼物,你不喜悦;你已经开通我的耳朵。”(诗40:6)

1. 神要的不是外在的宗教仪式,而是顺服的心(6-8节)

“祭物和礼物,你不喜悦。”(6节)——神看重的不是外在的敬拜,而是内心的真实顺服。

“我的神啊,我乐意照你的旨意行;你的律法在我心里。”(8节)——真正的敬拜是顺服神的旨意,并活出神的话语。

2. 见证神的作为(9-10节)

大卫不隐藏神的公义、信实和慈爱,而是向人述说神的恩典。

我们也当如此,不仅自己经历神,更要向别人分享他的救赎。

\subsubsection*{现实应用}
\hspace{0.6cm}你是否曾经只是“做宗教的事情”,却没有真正活出对神的顺服?

你是否愿意像大卫一样,在生活中勇敢地见证神的恩典?
\subsubsection*{挑战:}

\hspace{0.6cm}行动建议:每周找一个机会,与至少一个人分享神在你生命中的作为。
\subsection*{三、持续信靠神,依靠他到底(诗40:11-17)}
“耶和华啊,求你不要向我止住你的慈悲。”(诗40:11)

1. 生命的挑战仍会继续(12节)

虽然大卫已经经历神的拯救,但他仍然面临困难和仇敌的攻击。

“祸患围困我,不可胜数;我的罪孽追上了我。”(12节)——即使经历过神的救恩,我们仍然会面对罪和挑战。

2. 继续呼求神(13-17节)

“耶和华啊,求你快快搭救我!”(13节)——即使我们已经得救,我们仍需要持续依靠神。

“愿一切寻求你的,因你高兴欢喜。”(16节)——信靠神的人最终必得喜乐和满足。

\subsubsection*{现实应用}
\hspace{0.6cm}你是否在经历得救后,又回到了困境和挣扎之中?

你是否愿意每天持续祷告,倚靠神,而不是靠自己的能力?
\subsubsection*{挑战:}

\hspace{0.6cm}行动建议:每天写下你向神的祷告,并记录神如何在你的生命中回应你。
\subsection*{结论:从绝望到得救,活出信靠的生命}
\begin{enumerate}
    \item 等候神,他必施恩:不要害怕困境,神必定拯救你。

    \item 顺服神,活出他的旨意:不要只是外在敬拜,而是真心顺服。

    \item 持续信靠神,依靠他到底:即使得救了,也要不断依靠神。

\end{enumerate}

神今天邀请我们不要被环境压倒,而是选择信靠他,并在生命中勇敢见证他的作为!

\subsection*{结束祷告}
\textbf{亲爱的天父,}

我们感谢你,因为你是信实的神。你听见我们的呼求,从困境中拯救我们,使我们的脚立在坚固的磐石上。主啊,求你赐给我们耐心,让我们在等待中不灰心。求你帮助我们顺服你的旨意,活出你的话语。即使面对新的挑战,我们仍要依靠你,坚信你必带领我们到底。愿我们的生命成为你的见证,使更多人认识你的恩典。

奉主耶稣基督的名祷告,阿们!
%----------------------------------------------------------------------------
\newpage
\section{诗篇第41篇:在患难中经历神的恩典}
\subsection*{引言}
\hspace{0.6cm}弟兄姐妹,你是否曾经历过背叛、疾病或艰难的日子?你是否曾帮助别人,但在自己最需要帮助时,却感觉被人遗忘?在这些痛苦的经历中,我们如何面对?诗篇41篇是大卫在患病和被人背叛时的祷告,他在困境中仰望神,最终经历神的恩典和拯救。

今天,我们将透过诗篇41篇,思考以下三个方面:
\begin{itemize}
    \item 怜悯人的必蒙神的怜悯(1-3节)

    \item 面对背叛时,我们应当如何依靠神(4-9节)

    \item 神最终必拯救敬虔的人(10-13节)

\end{itemize}

让我们一起进入今天的学习!

\subsection*{一、怜悯人的必蒙神的怜悯(诗41:1-3)}
“眷顾贫穷的有福了!他遭难时,耶和华必搭救他。”(诗41:1)

1. 怜悯人是蒙福的途径

大卫在这里提到神特别眷顾那些帮助贫穷和困苦之人的人。

耶稣在马太福音5:7中说:“怜恤人的人有福了,因为他们必蒙怜恤。”

现实应用:你是否曾经帮助过有需要的人?在你自己遇到困难时,你是否经历过神的怜悯?

2. 神如何眷顾那些怜悯人的人?

“耶和华必保全他,使他存活。”(2节)

“他在病榻上,耶和华必扶持他。”(3节)

\subsubsection*{现实应用:}

\hspace{0.6cm}你是否愿意成为一个怜悯他人的人?

你是否曾经经历过神在你困难时的帮助?
\subsubsection*{挑战:}

行动建议:本周尝试主动帮助一个有需要的人,比如探访病人、资助贫困者、鼓励受伤的心灵。
\subsection*{二、面对背叛时,如何依靠神(诗41:4-9)}
“连我知己的朋友,我所倚靠、吃过我饭的,也用脚踢我。”(诗41:9)

1. 大卫面对的痛苦

大卫不仅生病,还受到仇敌的攻击(5节)。

最令人痛苦的是,他的知己朋友背叛了他(9节)。

这一节在新约中也应验在耶稣身上(约13:18),指的是犹大出卖耶稣。

2. 面对背叛,我们应该如何回应?

向神呼求(4节):“耶和华啊,求你怜恤我,医治我!”

信靠神,而不是倚靠人:诗118:8:“投靠耶和华,强似倚赖人。”

即使被人伤害,我们仍要倚靠神,因为他是信实的。
\subsubsection*{现实应用}

\hspace{0.6cm}你是否曾被你信任的人背叛?你如何处理这段伤痛?

你是否愿意像大卫一样,把你的痛苦交托给神,而不是陷入苦毒?
\subsubsection*{挑战:}

行动建议:如果你曾因被人伤害而苦毒,今天试着为那个人祷告,并求神医治你的心。
\subsection*{三、神最终必拯救敬虔的人(诗41:10-13)}
“你必扶持我,使我永远站立在你面前。”(诗41:12)

1. 神的信实与拯救
大卫深信神最终必救他脱离困境(10节)。

神不仅拯救他,还让他在敌人面前得胜(11节)。

“你必扶持我,使我永远站立在你面前。”(12节)——表明神的保守和恩典。

2. 赞美神的信实(13节)

诗篇41篇以敬拜和赞美结束:“耶和华以色列的神是应当称颂的,从亘古直到永远!”

即使经历背叛、痛苦和疾病,大卫仍然选择敬拜神。
\subsubsection*{现实应用}

\hspace{0.6cm}你是否相信神会在你困境中最终拯救你?

你是否愿意像大卫一样,即使在痛苦中,仍然选择信靠和敬拜神?
\subsubsection*{挑战:}

行动建议:每天写下你为神的信实感恩的事情,并在困难中坚持敬拜神。
\subsection*{结论:在患难中经历神的恩典}
\begin{enumerate}
    \item 怜悯人,必蒙神的怜悯——让我们活出神的爱,帮助有需要的人。

    \item 面对背叛,依靠神——不要陷入苦毒,而是把痛苦交托给神。

    \item 神最终必拯救我们——即使在苦难中,我们仍要信靠他,并赞美他的信实。

\end{enumerate}

愿我们都能在困境中仰望神,并最终经历他的拯救和恩典!

\subsection*{结束祷告}
\textbf{亲爱的天父,}

我们感谢你,因你是信实的神。你看顾怜悯人的人,并在我们遭遇困难、背叛和疾病时,不离开我们。主啊,求你赐给我们信心,让我们在等待拯救的过程中,仍然倚靠你。帮助我们在受伤害时,不陷入苦毒,而是选择宽恕。愿我们的生命成为你的见证,使人因我们而归荣耀给你。

奉主耶稣基督的名祷告,阿们!
%-----------------------------------------------------------------------------
\newpage
\section{诗篇第42篇:在干渴中寻求神——信靠与渴慕}
\subsection*{引言}
\hspace{0.6cm}弟兄姐妹,今天我们一同来思考诗篇42篇。它表达了一种深切的渴望和信靠,这种渴望不仅是生活中的困境和痛苦的反映,也是一颗渴望神临在、寻求神的心。大卫在这篇诗篇中通过比喻和诗意的语言,向我们展示了如何在生命的干渴和低谷中,持续仰望神,找到盼望和力量。

你是否曾在生命的某个阶段感到干渴,灵魂疲惫,像旱地中的荒土,渴望得着滋养?今天,诗篇42篇会为我们提供如何应对这类困境的答案——通过深切的渴望神、坚持信靠神、并在困境中等待神的作为。
\subsection*{一、灵魂的干渴与寻求(诗42:1-3)}
“神啊,我的心切慕你,如鹿切慕溪水。”(诗42:1)

\subsubsection*{1. 灵魂的干渴}
\hspace{0.6cm}大卫在这里用“如鹿切慕溪水”来比喻自己内心的渴望。这种渴望是一种深切的、迫切的,甚至是饥渴的寻求,就像一只长时间在旷野中奔波的鹿,渴望着溪水的滋润。

灵魂的干渴表现出我们内心深处的空虚和对神的渴望。或许在生活中,我们会遭遇到许多困境,无论是工作压力、关系问题,还是内心的孤独和痛苦,都可能让我们感到像是漫长的旷野之行,缺乏滋养。
\subsubsection*{2. 寻求神的回应}
\hspace{0.6cm}“我的心切慕你” —— 大卫不是渴望物质或外在的安慰,而是渴望神的同在和引导。
我们也许曾经历过生活的困境,失落或疲惫,在这些时刻,我们的心是否像大卫一样渴望神的帮助?

现实应用:在困境中,我们的心是否会向神呼喊,寻求他的安慰和力量?我们是否能够在痛苦中寻找到神的同在,而不是只依赖自己的力量去挣扎?
\subsubsection*{挑战:}

在每天的生活中,花时间安静在神面前,像大卫一样将你的干渴向神倾诉,并邀请他进入你的心灵深处,充满你的灵魂。
\subsection*{二、回忆神的恩典与在困境中的盼望(诗42:4-6)}
“我从前与众人同行… 我们欢呼赞美神的声音。”(诗42:4)

\subsubsection*{1. 回忆神的恩典}
\hspace{0.6cm}大卫在诗篇中提到自己曾经经历过神的恩典,回忆过去与神同行的时光。他怀念那时在圣殿中的敬拜,怀念那份与神亲密的关系。
在困境和低谷中,我们常常失去了对神恩典的感知,陷入了情绪的低落和灵魂的困倦。大卫的做法是回忆神过去的信实——回想神如何曾带领自己渡过难关,赐下安慰和力量。
\subsubsection*{2. 盼望神的作为}
\hspace{0.6cm}大卫的心情开始从消极转向盼望(6节)。他知道,虽然眼前有困苦,但神的恩典是未曾改变的。
在困境中,我们要学会回顾过去的恩典,重新建立信心,相信神的作为。正如诗篇103篇所说:“要称颂耶和华,不忘记他的一切恩惠。”

现实应用:你是否曾经历过神的恩典?即使在今天的困境中,你是否能够回忆起过去神如何在你的生命中动工,给予你力量与安慰?
\subsubsection*{挑战:}

在每天的祷告中,回顾神在你生命中的恩典,感谢他曾经带你走过的每一条路,并为今天的困境带着信心,依靠神。
\subsection*{三、坚定信心,仰望神的拯救(诗42:7-11)}
“我的心啊,你为何忧闷?为何在我里面烦躁?应当仰望神。”(诗42:11)

\subsubsection*{1. 生命的波动与神的掌权}
\hspace{0.6cm}大卫在这段经文中经历了内心的忧虑和困扰,他在问自己:“为何忧闷?” 这是每个人都会经历的内心挣扎,尤其是在面对困难和痛苦时。
尽管他感到困扰和痛苦,但他没有放弃信心,而是提醒自己:“应当仰望神。”他坚信,神是他的拯救和神是他的帮助。
\subsubsection*{2. 神的恩典是我们的力量}
\hspace{0.6cm}大卫的信心并不是建立在他看得见的环境上,而是建立在神的应许和信实上。

现实应用:在面对困难和不确定的未来时,我们常常感到困惑和无力。但大卫教导我们,不论环境如何变化,我们的信心应该建立在神的身上。我们可以仰望神,因为他是我们的拯救和力量。
\subsubsection*{3. 如何仰望神?}
\hspace{0.6cm}在困境中仰望神,并不是简单的等待,而是选择信任他的时间和计划。

我们要学会在困境中呼喊神的名字,坚定地相信他会在他的时间里赐下拯救。
\subsubsection*{挑战:}

行动建议:无论你正面临什么样的困境,都请你在每天的祷告中,选择信靠神,仰望他,祈求他的安慰、指引与力量。
\subsection*{结论:在干渴中寻求神,信靠他的救恩}
\begin{enumerate}
    \item 灵魂的干渴——我们每个人都会有灵魂干渴的时刻,感到疲惫和空虚,但这正是我们寻求神的时候。

    \item 回忆神的恩典——在困境中,我们要学会回顾神在过去的信实和恩典,重新建立我们的信心。

    \item 坚定信心,仰望神——面对困境,我们的信心应建立在神的应许上,坚定仰望他的拯救。

\end{enumerate}

愿我们每个人都能在灵魂的干渴中,寻找到神的滋养,在困境中仰望神,坚定信靠他的救恩!

\subsection*{结束祷告}

\textbf{亲爱的天父,}

我们感谢你,因为你是我们的生命之泉。我们承认我们的灵魂常常感到干渴,疲惫,甚至迷失,但感谢你在每个困境中呼召我们来寻求你。主啊,求你帮助我们回忆过去你在我们生命中的恩典,并在困境中坚定信心,仰望你的救恩。我们将一切的忧虑和痛苦交托给你,愿你成为我们坚固的帮助。

奉主耶稣基督的名祷告,阿们!

%----------------------------------------------------------------------------
\newpage
\section{诗篇第43篇:在困境中寻找神的光与真理——盼望与信靠}
\subsection*{引言}
\hspace{0.6cm}亲爱的弟兄姐妹,今天我们一同来学习诗篇43篇。这篇诗篇紧接着诗篇42篇,仍然是大卫在困境中的祷告,反映了他内心的挣扎与对神的深切呼求。在困苦中,面临疑惑、被冤屈、感到被抛弃时,大卫将目光从眼前的困境转向神,寻求神的光与真理。诗篇43篇带给我们一种深刻的信仰智慧,那就是:在困境中,我们不仅要向神呼求,还要坚定仰望神的光和真理,以信靠为盾,继续前行。
\subsection*{一、在困境中呼求神的帮助(诗43:1-2)}
“神啊,求你为我伸冤,拯救我脱离不虔诚的国民,脱离诡诈不义的人。”(诗43:1)

\subsubsection*{1. 大卫的困境:被冤屈和迫害}
\hspace{0.6cm}在这两节经文中,大卫向神呼求,求神为他伸冤。他正经历着被人冤屈、被不义之人陷害的困境。他深知,自己无法凭借人的力量解决这些问题,唯有依靠神的公义和拯救。
大卫所面临的困境并不仅仅是外在的敌人,也包括他内心的不安与孤独,因为他感觉到自己被遗弃,被无辜指责。这些经历让我们想起在现代社会中,许多人在面对社会压力、人际冲突或不公时,常常会感到无助和困惑。
\subsubsection*{2. 呼求神的帮助:信靠神的公义}
\hspace{0.6cm}大卫在困境中没有依靠自己去报仇或挣扎,而是将自己交托给神。他深信神是公义的,神必为他伸冤。

现实应用:在我们的生活中,我们也可能遭遇到误解、背叛或不公正的待遇。面对这些挑战时,我们是否愿意像大卫一样,将自己的问题交托给神,相信神必按他的公义来处理?
\subsubsection*{挑战:}

行动建议:当你感到冤屈或无助时,不要单凭自己的力量去抗争,而是转向神,向他呼求。将你的重担交托给神,相信他会为你伸冤。
\subsection*{二、寻求神的光和真理(诗43:3-4)}
“求你发出你的光和你的真理,引导我,领我到你的圣山、到你的居所。”(诗43:3)

\subsubsection*{1. 寻求神的光}
\hspace{0.6cm}在困境中,大卫不仅仅是向神呼求伸冤,他更深切地知道,他真正需要的是神的光。
神的光代表了神的带领、神的真理、神的启示。无论身处何种黑暗或困境,神的光能照亮我们的道路,帮助我们看到出路,带领我们走向正确的方向。

现实应用:我们常常在生活中遇到各种各样的黑暗时刻,无论是情感上的黑暗,还是精神上的压抑,甚至是对未来的迷茫。当这些黑暗笼罩我们时,我们是否渴望寻求神的光,依赖神的指引?
\subsubsection*{2. 寻求神的真理}
\hspace{0.6cm}神的真理是我们面对困境时的标准和指引。大卫知道,唯有神的真理能指引他走出困境,回到与神的亲密关系中。
神的真理不但指引我们的行为,还能安慰我们心灵的创伤。我们常常在困难中迷失方向,但神的真理像一盏明灯,照亮我们前行的路。

现实应用:我们如何通过神的话语和真理来面对生活中的挑战?我们是否在困境中依赖神的话语,还是更多依赖自己的理解和经验?
\subsubsection*{挑战:}

行动建议:每天花时间读经祷告,特别是在困难时,更要寻求神的话语和真理。让神的话语成为我们心中的光,指引我们的方向。
\subsection*{三、信心的坚持与盼望(诗43:5)}
“我的心啊,你为何忧闷?为何在我里面烦躁?应当仰望神,因为我还要称赞他,他是我面上的光和我的神。”(诗43:5)

\subsubsection*{1. 大卫的内心挣扎}
\hspace{0.6cm}诗篇43:5展现了大卫内心的挣扎,他对自己说:“为何忧闷?为何烦躁?”这是他对自己情绪的审视和呼召,提醒自己不要被困境压倒。
大卫意识到,他的信心不是建立在周围的环境上,而是建立在神的身上。当我们在困境中经历低谷时,我们的内心往往会感到忧虑和烦躁,这是人之常情,但大卫教导我们要通过信心来坚定我们的盼望。
\subsubsection*{2. 坚定仰望神,称赞神}
\hspace{0.6cm}尽管大卫正经历困境,他依然选择仰望神,将自己的希望放在神身上,并且称赞神的信实与慈爱。
信靠神的盼望使大卫能够超越眼前的困苦,走向一个充满信心和感恩的生活态度。

现实应用:在我们自己的困境中,是否能够像大卫一样,从内心转向神,坚定信靠神的拯救和恩典?无论外在环境如何变化,我们是否愿意像大卫一样在困境中坚持仰望神,并称赞他的名字?
\subsubsection*{挑战:}

行动建议:当你在生活中感到忧虑和烦躁时,停下来默想神的作为,通过祷告和赞美,重新调整你的视角,坚定你对神的信心。
\subsection*{结论:在困境中寻找神的光与真理}
\begin{enumerate}
    \item 在困境中呼求神的帮助——当我们面临冤屈、压力或痛苦时,我们要向神呼求,交托一切,信靠他的公义和拯救。

    \item 寻求神的光和真理——无论我们走得多远,神的光和真理总能为我们指引道路,帮助我们脱离黑暗。

    \item 信心的坚持与盼望——在困境中,我们需要坚持仰望神,信靠神的拯救,并以感恩的心称赞神。

\end{enumerate}

在我们的生活中,神的光和真理永远是我们前行的动力和指引。当你遭遇困境时,记得将眼目从困境转向神,寻求他的光与真理,并坚持信靠,仰望神的拯救。

\subsection*{结束祷告}
\textbf{亲爱的天父,}

我们感谢你,因为你是我们的光和真理。在我们遭遇困境、冤屈和痛苦时,求你为我们伸冤,帮助我们看见你的光,听见你的声音。主啊,求你赐给我们坚定的信心,让我们在困境中依靠你,仰望你,持续盼望你的拯救。我们将我们的重担交托给你,感谢你永不离弃我们,永远是我们的力量。

奉主耶稣基督的名祷告,阿们!
%-----------------------------------------------------------------------------
\newpage
\section{诗篇第44篇:在困境中保持信心与依靠——信仰反思}
\subsection*{引言}
\hspace{0.6cm}亲爱的弟兄姐妹,今天我们来研读诗篇44篇。诗篇44篇是一篇充满深刻情感和痛苦呼求的诗篇,表达了以色列人在遭遇困境时的困惑与挣扎。这篇诗篇中,大卫或作者回顾了以色列过去的得胜与神的恩典,却又面对眼前的失败和困苦,感到神似乎远离了他们。这篇诗篇不仅是对神的呼求,也是对信仰的一次深刻反思。

在我们的生活中,我们常常会经历这样的时刻:回想过去的恩典和得胜,却又在当前的困境中感到困惑、痛苦和无助。诗篇44篇帮助我们理解,在生命的低谷中,如何依然保持对神的信心,如何面对疑惑和失落,并坚定地依靠神的拯救与公义。

\subsection*{一、回顾神的作为与历史的恩典(诗44:1-3)}
% “神啊,我们耳中听见了,祖宗告诉我们,你在古时所做的事。”(诗44:1)

\subsubsection*{1. 以色列的历史回顾}
\hspace{0.6cm}在这几节经文中,诗人回顾了以色列历史中的神的恩典与拯救。以色列人曾在困境中看见神的伟大作为,神曾为他们战胜敌人、为他们开道路、赐下胜利。这是以色列信仰的根基,也是他们依靠神的力量所在。

现实应用:在我们的生活中,我们也常常会有过往的经历,可以回忆起神如何在我们的生命中作工,如何在困境中赐下帮助和安慰。回顾神的恩典和作为能帮助我们在当前的困境中找到信心与希望。
\subsubsection*{2. 纪念神的恩典带来的安慰与力量}
\hspace{0.6cm}诗人通过回忆过去神的作为,鼓励自己和全体以色列民继续信靠神。在生活中,当我们回顾过去的恩典时,我们的心会被坚定,面对眼前的挑战和痛苦时,我们会重新得着力量。

现实应用:我们可能在生活中遇到一些困难,眼前似乎看不到出路,但回忆神的信实与曾经的恩典,能帮助我们看到神的信实和大能,重新恢复信心。
\subsubsection*{挑战:}

在面对困境时,花时间回顾神在你生命中的恩典和历史,通过回忆过往神的作为来加强自己的信心。
\subsection*{二、当遭遇困境时的信心挣扎(诗44:4-8)}
% “你是我的王、我的神,你就是命令雅各得胜的。”(诗44:4)

\subsubsection*{1. 挣扎与困惑}
\hspace{0.6cm}诗人开始提到,虽然他们曾经历过神的帮助和拯救,但现在,他们正面临敌人的攻击和压迫。神似乎没有按他们所期望的方式介入,反而是他们在遭遇困境时,仿佛感到神的沉默与远离。这让他们开始挣扎,心中充满疑惑和痛苦。
在这段经文中,诗人表达了一种信仰的冲突:\textbf{为什么过去的神的恩典不再显现?为何神在如今的困境中没有帮助我们?}这是一种真实的信仰挣扎,是每一个信徒在面对苦难时都会经历的心理过程。
\subsubsection*{2. 依靠神的力量}
\hspace{0.6cm}即使在困境中,诗人仍然回到对神的信靠,他们坚信,神是他们的王,是他们得胜的源泉。即使眼前的环境看似无望,他们仍然坚信神必能使他们得胜。

现实应用:我们在遭遇痛苦、挫折和困境时,也会有这样的挣扎——我们曾经历过神的恩典,为什么现在似乎没有得到帮助?但我们要像诗人一样,继续依靠神的能力,而不是依赖自己的力量。无论眼前的困难如何,神的能力和恩典是我们信心的根基。
\subsubsection*{挑战:}

当你感到神的沉默时,依然坚定地将目光投向神,相信神的能力与公义,并在祷告中继续依赖他。
\subsection*{三、面对困境的呼求与祷告(诗44:9-26)}
% “我们因你的缘故终日遭遇死亡,被当作宰杀的羊。”(诗44:22)

\subsubsection*{1. 向神呼求}
\hspace{0.6cm}在诗篇的后半部分,诗人进入了深切的祷告与呼求。他们承认当前的困境,并深切呼求神的帮助。他们没有放弃信心,而是选择继续向神呼喊,请求神的拯救与介入。
这是一种真实的祷告,不掩饰痛苦与失望,而是坦诚地向神倾诉内心的挣扎。诗人说:“我们为你的缘故受苦,遭遇死亡。”这句话表达了他们对神的忠诚,尽管他们现在遭遇了困境,但他们仍然愿意将一切交托给神。
\subsubsection*{2. 坚持信心,等待神的拯救}
\hspace{0.6cm}尽管他们的困境未立即得到解答,但诗人并没有放弃,而是坚持信靠,呼求神显现他的拯救。信仰的真谛不在于短期的胜利,而在于持守信心,在困境中坚持仰望神。

现实应用:面对困境和痛苦时,我们需要有坚定的信心,无论环境如何变化,都要继续向神呼求,并相信他必在合适的时间施行拯救。神的时间表与我们的期待不同,但他总会按照最好的方式来回应我们的呼求。
\subsubsection*{挑战:}

即使在看似没有希望的时刻,坚持祷告,向神倾诉你的困惑与痛苦,并继续信靠神的时间与计划。
\subsection*{结论:在困境中保持信心与依靠}
\begin{enumerate}
    \item 回顾神的作为与历史的恩典——当我们在困境中时,不要忘记过去神如何帮助我们,回顾神在我们生命中的信实与作为,将加深我们对神的信靠。

    \item 当遭遇困境时的信心挣扎——面对困境时,我们会有挣扎和疑惑,但我们要学会像诗人一样,将信心建立在神的能力与公义之上,继续依靠他。

    \item 面对困境的呼求与祷告——无论遇到什么困难,我们都要坚持向神呼求,坦诚表达内心的痛苦与困惑,并继续信靠神,等待他的拯救。

\end{enumerate}



诗篇44篇提醒我们:在困境中,我们的信仰不能被环境动摇,我们要依靠神的恩典,继续持守信心,在神的时间里等待他的救恩。

\subsection*{结束祷告}
\textbf{亲爱的天父,}

感谢你在我们生命中的带领与恩典。当我们在困境中时,我们有时会感到迷茫和困惑,甚至质疑为何神没有按我们的方式回应我们的呼求。主啊,求你帮助我们在困境中保持信心,回顾你过去的信实,坚定依靠你,无论眼前的环境如何变化,都要相信你的拯救与公义。我们愿意将一切交托给你,等候你的作为。

奉主耶稣基督的名祷告,阿们!
%----------------------------------------------------------------------------
\newpage
\section{诗篇第45篇:神的恩典与婚姻的奥秘}
\subsection*{引言}
\hspace{0.6cm}亲爱的弟兄姐妹,今天我们将一同研读诗篇45篇,这是一首婚礼颂歌,描绘了一位王的荣耀与美丽,同时也隐含着神对婚姻的美好计划。这篇诗篇向我们展示了王的荣光与尊贵,也引发我们对婚姻关系的深刻反思,尤其是在神眼中,婚姻的意义与价值。尽管诗篇45篇的背景与古代的王室婚礼相关,但它也启示了我们现代婚姻中的一些核心价值观:忠诚、爱情、尊重与神的祝福。

\subsection*{一、王的荣耀与婚姻的象征(诗45:1-9)}

\subsubsection*{1. 王的尊贵和婚姻的象征}
\hspace{0.6cm}诗篇45篇的开篇以诗人对王的赞美为主,描绘了一位威风凛凛的王,光彩照人,能力无比。王的尊贵、权威和荣耀不仅象征着权力,也象征着婚姻中的责任与爱。在这一切描述中,我们看到神对婚姻的定义:婚姻是尊贵的、神圣的,充满了爱、责任与荣耀。
婚姻中的爱,并不仅仅是情感的交流,更是一种深沉的责任与尊重。神创造婚姻,是为了体现他在基督与教会之间那种深刻而荣耀的关系。
\subsubsection*{2. 婚姻是神的恩典}
\hspace{0.6cm}王的荣耀不仅仅在于他的外貌和威势,也在于他所代表的公义和恩典。婚姻也是如此:它是神所设立的美好礼物,充满了神的恩典与祝福。通过婚姻,夫妻之间要互相扶持、互相尊重,在爱中共同经历神的恩典与指引。

现实应用:我们常常把婚姻视为两个人的关系,但实际上,婚姻是神所设立的,带有深厚的神圣责任。在婚姻中,我们要像王一样,承担起尊贵的责任,爱护和尊重我们的配偶,像神对教会的爱一样,无条件地给予支持与关怀。
\subsubsection*{挑战:}

无论我们当前的婚姻状态如何,都要意识到婚姻是神的恩典和祝福,在婚姻中活出责任和荣耀,在爱中反映神的恩典。
\subsection*{二、婚姻中的忠诚与责任(诗45:10-17)}

\subsubsection*{1. 婚姻中的忠诚与尊重}
\hspace{0.6cm}在诗篇45的中段,诗人继续描述了王与他的配偶的关系。王后被形容为穿着华丽的衣服,站在王的右边,表现了婚姻中的忠诚与尊重。站在王的右边不仅是荣耀的象征,更是表示她对王的支持与顺从。这种顺从并不是盲目的,而是基于相互尊重和爱。
婚姻中的忠诚与责任是相互的:丈夫应该爱妻子,如同基督爱教会;妻子也应当尊敬丈夫,这种关系中既有爱,也有责任。没有忠诚与责任的婚姻,无法维持稳定与长久。
\subsubsection*{2. 婚姻中的共同使命}
\hspace{0.6cm}诗人提到王和王后共同承担使命和荣耀。婚姻不仅是为了满足个人的情感需要,它还有神圣的使命——在这个关系中,夫妻应该一起为神的荣耀和人类的益处工作。无论是通过生育后代,还是通过在社会中作为信徒的见证,婚姻是神设立的工作和祝福。

现实应用:我们在婚姻中的责任,不仅仅是情感上的支持,更是共同为神的荣耀和福音工作。夫妻之间应当在生活中共同支持彼此的呼召和使命,不断地为神的国度增添力量。
\subsubsection*{挑战:}

在婚姻关系中,不仅要注重情感的交流和支持,还要互相鼓励,共同承担家庭和神国的使命,为彼此的生命和神的荣耀共同努力。
\subsection*{三、婚姻关系中的神圣使命(诗45:18-17)}

\subsubsection*{1. 婚姻是彰显神荣耀的场所}
\hspace{0.6cm}诗篇45篇的结尾提到,王和王后的婚姻将带来荣耀,他们的名字将被传扬,世世代代的万民都要称赞他们。婚姻在神的眼中,不仅是个人的事情,而是反映神荣耀的重要途径。

婚姻关系中,夫妻应该在彼此的关系中活出基督的爱,彰显神的荣耀。婚姻不只是一个生活的安排,它也是一场持续的见证神爱的旅程。
\subsubsection*{2. 神在婚姻中的统治与主权}
\hspace{0.6cm}诗篇中的王象征着神对婚姻的主权与统治。每一段婚姻都应当在神的统治下,夫妻应当在神的旨意中行事。婚姻中的每一份荣耀与喜乐,都是来自神的恩典,因此我们在婚姻中应当始终以神为中心,按他的旨意生活。
\subsubsection*{现实应用:}

在我们的婚姻生活中,是否始终记得婚姻是神所设立的,带着神的使命和荣耀?我们是否在婚姻关系中彰显出神的爱与信实?婚姻不只是两个人之间的关系,它更是一个为神所设立的圣约,是神祝福我们,并通过我们来见证他的爱与公义。
\subsubsection*{挑战:}

将婚姻作为神的呼召与使命,在婚姻中活出神的荣耀,无论在顺境还是逆境中,始终记住婚姻关系中神的主权与恩典。
\subsection*{结论:婚姻中的恩典与责任}
\hspace{0.6cm}王的荣耀与婚姻的象征——婚姻是神设立的,充满神圣责任和荣耀。夫妻间的关系应当如同王和王后,彼此尊重、共同承担使命。
\begin{enumerate}
    \item 婚姻中的忠诚与责任——婚姻中充满忠诚与责任,夫妻应在神的爱中互相扶持,共同承担神所托付的使命。

    \item 婚姻关系中的神圣使命——婚姻不仅是个人的事,更是一个反映神荣耀和见证神爱的地方。夫妻在婚姻中的关系应该彰显神的荣耀,并在神的旨意中共同生活。

\end{enumerate}

在婚姻中,神给我们的是恩典与使命,婚姻的美好不仅仅是情感的满足,更是神在我们生命中的旨意和计划。愿我们每一个人都能在婚姻关系中,活出神的荣耀,并以神为中心,走在他的旨意中。

\subsection*{结束祷告}
\textbf{亲爱的天父,}

感谢你赐给我们婚姻这一美好的恩典。我们知道婚姻是你设立的,不仅是为了情感的满足,更是为了彰显你的荣耀和见证你的爱。求你帮助我们在婚姻关系中活出忠诚、尊重与责任,始终以你为中心,在你所设立的使命中同行。无论是顺境还是逆境,我们都愿意将婚姻交托给你,求你赐下智慧、忍耐和爱心,帮助我们共同荣耀你的名。

奉主耶稣基督的名祷告,阿们!
%---------------------------------------------------------------------------
\newpage
\section{诗篇第46篇:神是我们的避难所与力量——安慰与信心}
\subsection*{引言}
\hspace{0.6cm}亲爱的弟兄姐妹,今天我们一起学习诗篇46篇,这是一篇极具安慰与信心的诗篇。在这篇诗篇中,诗人宣告了神作为我们坚固的避难所和力量的伟大。诗篇46篇不仅仅是对神作为救主的颂扬,也是对我们在困境中的信心呼召。在每一个艰难的时刻,神是我们的庇护,他的能力和保护永远不变。

诗篇46篇提醒我们,神的临在与帮助,不是仅限于某些特殊的时刻或情境,而是无论在任何情况下,神始终是我们坚实的依靠和力量。
\subsection*{一、神是我们避难所与力量(诗46:1-3)}
“神是我们的避难所和力量,是我们随时的帮助。”(诗46:1)

\subsubsection*{1. 神是我们随时的帮助}
在这几节经文中,诗人首先宣告了神是我们的避难所和力量。当我们遭遇困难、恐惧和混乱时,神是我们最坚固的庇护所,是我们随时的帮助。无论我们面对何种困境,神都是我们最可靠的避难所。
诗人使用了“避难所”和“力量”这两个词。避难所代表了神是我们在危机中能够依靠的地方,而力量则是神能够在我们软弱时赐给我们坚强的支持。
\subsubsection*{2. 现实应用:信靠神的保护}
在现代社会中,我们常常面对各种压力和挑战:家庭的压力、工作上的挑战、健康问题、社会的不安等等。在这些时刻,我们容易感到无助、迷茫和疲惫。然而,诗篇46篇提醒我们:神随时愿意成为我们的帮助。我们应当学习在这些困境中,将焦点转向神,信靠他的保护和力量。
\subsubsection*{挑战:}

当你感到压力和困惑时,记住诗篇46篇所说的:神是我们的避难所和力量。在祷告中依靠神,相信神能在你的困境中为你提供力量。
\subsection*{二、神的保护无可撼动(诗46:4-7)}
“神在她的城中,神在其中,必不动摇;神必帮助她,到了早晨。”(诗46:5)

\subsubsection*{1. 神的同在与保护}
诗人进一步描绘了神在城中的保护,强调神的存在让这个城不动摇。这是一个非常强烈的象征,表明神的同在能让我们在动荡不安的环境中保持稳定与平安。即使在周围的世界充满震荡和动荡,神的同在依然是我们坚固的保障。
在生活中,我们也会遇到各种外在的压力与挑战,有时外部的情况让我们感到不安和恐惧。但是,诗篇46篇提醒我们:神的保护是不受任何外在环境的影响的。他的同在就是我们稳定和安息的根基。
\subsubsection*{2. 现实应用:无论外界如何动荡,神是我们的安稳}
在现代社会中,我们常常面临着动荡的政治、经济不稳定、社会不安等问题。这些因素可能会带来很大的恐惧和不安。然而,我们要记住,神的同在是无法被撼动的。他比一切外在的动荡更为强大,他的保护不依赖于外界的环境,而是完全依靠他的信实与大能。
\subsubsection*{挑战:}

面对生活中的动荡与不安时,保持冷静,将心思集中在神的同在和保护上。学习在每个不安的时刻,坚信神始终是我们稳定的源泉。
\subsection*{三、神的安慰与安稳(诗46:8-11)}
“你们要休息,要知道我是神。”(诗46:10)

\subsubsection*{1. 安静与安稳的呼召}
在诗篇的最后部分,神向我们发出呼召:“你们要休息,要知道我是神。”这是一个安慰的呼召,也是一个信心的呼召。当我们处于动荡和不安之中,神邀请我们进入他的平安与安稳中。他提醒我们,在困境中,休息与信靠是信仰的体现。
这个安静并不是消极的逃避,而是通过对神的信靠,让我们的内心不再受外在环境的影响,安静在他的同在中。
\subsubsection*{2. 现实应用:依靠神的安慰}
在日常生活中,特别是当我们面临压力和困扰时,我们容易陷入焦虑和紧张的情绪中。然而,诗篇46篇教导我们:当我们把心放在神身上时,神会赐给我们内心的安慰与安稳。神的同在能够帮助我们超越表面的困境,进入一种深层次的心灵平安。
\subsubsection*{挑战:}

每天花时间安静在神的面前,祷告并依靠他的安慰,让自己的心灵得以平静和恢复,不被外在环境所左右。
\subsection*{结论:神是我们的避难所与力量}
\begin{enumerate}
    \item 神是我们随时的帮助——在我们面临困境和挑战时,神是我们随时的帮助,他是我们最可靠的避难所。

    \item 神的保护无可撼动——无论外界如何动荡,神的同在始终是我们安稳的保障,他的保护不受任何环境影响。

    \item 神的安慰与安稳——在困境中,神邀请我们进入他的安稳与休息之中,依靠他的力量与安慰,超越眼前的压力与困扰。

\end{enumerate}

亲爱的弟兄姐妹,诗篇46篇给我们带来了极大的安慰与鼓励。无论我们面临什么困境,神是我们的避难所,是我们随时的帮助。他的保护是不动摇的,他的安慰能够带给我们内心的安静。让我们在每一个风浪中,都能紧紧依靠他,信靠他的大能与信实。

\subsection*{结束祷告}
\textbf{亲爱的天父,}

感谢你赐给我们诗篇46篇这宝贵的教导。我们承认,在生活中有许多挑战和困境,但你是我们的避难所,是我们的力量,是我们随时的帮助。求你帮助我们在每一个困境中依靠你,在动荡的环境中保持信心,相信你的保护与安慰是永远不变的。求你赐给我们安静的心,依靠你走过每一条艰难的道路。

奉主耶稣基督的名祷告,阿们!
%----------------------------------------------------------------------------
\newpage
\section{诗篇第47篇:万民当赞美神——敬拜与神的主权}
\subsection*{引言}
亲爱的弟兄姐妹,今天我们一起学习诗篇47篇,这篇诗篇充满了对神的赞美与敬拜,它宣告了神作为宇宙主宰的伟大与权能。诗篇47篇特别强调了神的统治和主权,不仅是对以色列的,还是对全地、对万民的。我们今天所面临的世界,充满了不确定性与挑战,但诗篇47篇教导我们,在神的主权下,我们应当坚定信心、欢喜赞美,因他掌管万物,所有的荣耀都归他。
这篇诗篇不仅是一篇关于神主权的颂歌,也是一篇关于我们如何回应神伟大的邀请——那就是通过敬拜与赞美,彰显神的荣耀。
\subsection*{一、向神欢呼——宣告神的伟大与荣耀(诗47:1-2)}
“万民啊,你们都要拍掌;要用欢声向神呼求。因耶和华至高,是可畏的,是治理全地的大王。”(诗47:1-2)

\subsubsection*{1. 向神欢呼与赞美}
诗人以一种喜乐和欢腾的语气开始这篇诗篇,呼吁所有的万民都要拍掌、欢呼。这里的“拍掌”不仅仅是对神的外在敬意,它代表了全心全意的欢庆与敬拜。当我们向神发出赞美时,我们的内心应当充满喜乐与感恩,因为神的伟大与荣耀是无与伦比的。
诗人指出,神至高无上,他的伟大是任何事物无法比拟的。作为神的子民,我们应当因他的伟大而感到荣耀与骄傲。
\subsubsection*{2. 现实应用:欢庆与赞美神的伟大}
在我们生活的世界中,我们时常会因为工作、家庭、健康等各方面的压力而感到焦虑或不安。然而,诗篇47篇提醒我们,不管周围环境如何,神始终是至高的,他是我们可以依靠和赞美的对象。无论发生什么事情,神的伟大与荣耀是不变的,我们应当因他的伟大而欢喜。
\subsubsection*{挑战:}

行动建议:无论在顺境还是逆境中,我们都应当从内心欢呼赞美神,因他是至高无上的主。当你感到压力或困扰时,花时间专注于赞美神的伟大,这将帮助我们调整心态,专注于神的主权。
\subsection*{二、神的统治——主权的宣告(诗47:3-4)}
“他叫万民服在我们下面,把列国交在我们脚下。他为我们选择了产业,就是他所亲爱的雅各的荣耀。”(诗47:3-4)

\subsubsection*{1. 神的统治是全球性的}
在这两节经文中,诗人描述了神的统治是如此广泛,他使万民都服从于他,列国都被交在以色列脚下。这里不仅指的是以色列的历史背景,也是对神普遍主权的宣告。神的统治超越了任何国家、民族与时空的限制,他是宇宙间唯一的主宰。
神不仅是我们个人生命中的主权者,也是在全球范围内主宰一切的神。无论是国家领导者,还是普通百姓,所有人都在神的掌握之中。
\subsubsection*{2. 现实应用:信靠神的主权}
我们今天生活在一个全球化的世界中,常常面临政治、经济等方面的不确定性。诗篇47篇提醒我们,尽管世界上有许多复杂和不安的因素,但神的主权从未改变。无论我们面对什么样的挑战,都应当坚信神依然掌管一切。
作为神的子民,我们应当在生活中表达对神主权的信靠。无论是国家的政治局势,还是家庭中的纷扰,我们都应当相信神的旨意最终会成就,他是我们稳定的依靠。
\subsubsection*{挑战:}

行动建议:每当你感到对社会、国家或世界局势的无力时,记住神是宇宙的主宰,他的计划不会落空。通过祷告和敬拜,重新将心放在神的主权上,让他的平安充满我们的心。
\subsection*{三、神的百姓与万民的应答(诗47:5-9)}
“神登上欢乐的宝座,耶和华登上高天,发出号声,吹号的声音,列国的君王,聚集在一起,联合为一。神的百姓也要在一起,赞美神的圣名。”(诗47:5-9)

\subsubsection*{1. 神的百姓与万民的应答}
诗篇47篇最后的几节描述了万民和神的百姓的回应。神登上宝座,显示出他是掌管万国的王。诗人展现了一幅壮丽的图画:神的百姓将与万民一同站在神面前,赞美他的圣名。神的主权并不仅仅是对以色列的,也适用于所有万国万民。
在神的统治下,全地都将响应神的主权,最终一切都要归向神,赞美他的圣名。这不仅是对神未来荣耀的预言,也是我们每个人现在应当响应的呼召:在地上,我们应当以敬拜和顺服回应神。
\subsubsection*{2. 现实应用:全地的敬拜与呼召}
神呼召我们做他的见证,无论是通过我们的言语,还是通过我们的生活方式,都要彰显出神的荣耀与主权。我们身边的每一个人,无论信与不信,都应当看到神的伟大与主权,因而在我们生命的见证中,神的名字得以被高举。
在日常生活中,我们如何回应神的呼召呢?是否愿意在工作、家庭、社交等方面,以神为中心,活出敬拜与荣耀呢?作为基督徒,我们不仅仅是通过口头赞美,更要通过行动去见证神的荣耀。
\subsubsection*{挑战:}

行动建议:让我们的生活成为对神主权的见证。无论在工作、家庭、社交还是个人生活中,都要以神为中心,活出敬拜与顺服。每当你与他人交往时,都要记得你的行为和态度应该反映神的荣耀。
\subsection*{结论:神的伟大与主权}
\begin{enumerate}
    \item 向神欢呼,宣告他的伟大与荣耀——我们应当在一切情况中,因神的伟大与荣耀而欢庆,表达对他的赞美与敬拜。

    \item 神的主权超越万国,稳定我们的信心——无论外界如何动荡不安,我们要相信神仍然掌管一切,他的主权是无可撼动的。

    \item 神的百姓要回应神的呼召,通过敬拜与见证彰显神的荣耀——我们要在个人生活中以神为中心,活出敬拜与荣耀,成为神主权的见证。

\end{enumerate}

亲爱的弟兄姐妹,让我们不忘诗篇47篇的呼召:在神的伟大和主权面前,我们要回应以赞美与敬拜的心,活出敬虔的生活。无论在顺境或逆境,神的伟大与主权都不改变,愿我们每天都能因他的荣耀而欢喜,坚定信心,勇敢活出神的呼召。

\subsection*{结束祷告}
\textbf{亲爱的天父,}

感谢你是至高的神,掌管万国万民,统治全地。感谢你赐给我们这篇诗篇,让我们在你的伟大和主权面前重新得到安慰和力量。求你帮助我们在生活中始终依靠你的主权,勇敢地回应你的呼召,活出赞美和荣耀。无论在任何困境中,我们都愿意以欢呼与赞美的心来敬拜你,因你是配得一切荣耀的神。
奉主耶稣基督的名祷告,阿们!
%-----------------------------------------------------------------------------
\newpage
\section{诗篇第48篇:神的城,神的荣耀——信心与希望}
\subsection*{引言}
亲爱的弟兄姐妹,今天我们一同学习诗篇48篇。这篇诗篇的主题是对神圣城耶路撒冷的赞美与歌颂,诗人通过对耶路撒冷这座城市的赞美,展示了神的伟大与荣耀,并且呼召神的百姓信靠神,因他是我们的保障。在这篇诗篇中,诗人不仅回顾了神在历史中的作为,更展望了神的城在未来的荣耀。它给我们带来的教训是:神的存在与保护是我们永远的依靠,神的荣耀最终会彰显在全地。
\subsection*{一、神的城是荣耀与美丽的象征(诗48:1-3)}
“耶和华是大应当称颂,在我们神的城,在他的圣山,是美丽的、欢乐的全地。”(诗48:1-2)

\subsubsection*{1. 神的城代表着神的荣耀}
诗篇48篇一开始,诗人高声赞美神,说:“耶和华是大,应当称颂。”这不仅仅是对神的一种称赞,也是对神在耶路撒冷这座城市中显现荣耀的一种表达。耶路撒冷作为神的圣城,是神荣耀与美丽的象征。
神的城不仅仅是一座物理意义上的城市,更是神与他百姓同在的象征。这座城市的美丽不仅体现在外在的建筑上,更是神的同在与祝福所带来的荣耀。耶路撒冷象征着神的信实与伟大,它提醒我们,神不仅在过去保护了他的百姓,今天仍然是我们的保护者,明天更会带来永远的荣耀。
\subsubsection*{2. 现实应用:神的荣耀在我们生命中的彰显}
在我们今天的生活中,虽然耶路撒冷作为神的圣城已经不再是唯一的焦点,但我们每个基督徒的生命都应当成为神荣耀的见证。我们的生命中也应该反映出神的荣耀与美丽,不管在工作、家庭、学校或任何场合,我们应当成为神恩典与荣耀的载体。
诗篇提醒我们,神的荣耀不是遥不可及的,它在我们身边,甚至就在我们心中。作为神的百姓,我们的生活应当彰显神的同在与荣耀。
\subsubsection*{挑战:}

行动建议:让我们在每一天的生活中都活出神的荣耀。无论是在职场上,还是在家人和朋友面前,以敬虔和爱心为神做见证,让别人通过我们的行为看到神的荣耀。
\subsection*{二、神在我们困境中的保护(诗48:4-8)}
“他们看见了、就惊讶、惊惶失措;急忙逃跑。”(诗48:5)

\subsubsection*{1. 神的保护与回应}
诗人接着描述了神如何保护耶路撒冷,抵挡敌人的进攻。敌人虽然聚集而来,充满威胁,但神的力量使他们四散逃跑。神在他的城中展示了他超越一切的能力,他的保护使他的百姓得以安然无恙。
诗篇48篇的作者提醒我们,当我们面临困境与挑战时,神的保护就是我们的盾牌和保障。正如神保护耶路撒冷一样,他也会保护我们,帮助我们度过生活中的风暴与挑战。
\subsubsection*{2. 现实应用:信靠神的保护与帮助}
在我们现代的生活中,我们常常面临种种困难与挑战:工作上的压力、家庭中的纷争、健康上的问题等。诗篇48篇提醒我们,不管外界如何动荡,神是我们坚强的后盾。我们不应当依赖自己的力量,而是要依靠神的保护与引导。
这不仅仅是一个属灵的安慰,也是一个实际的指引。神的保护并不是我们生活中的偶尔事件,而是神对我们每一天的看顾。我们每当感到无助和困惑时,可以祈祷,呼求神的帮助。
\subsubsection*{挑战:}

行动建议:当你面对困难和挑战时,将你的焦虑与担忧交托给神,并相信他会在你生命中展现出超凡的力量和保护。记住,神是我们的盾牌,我们可以安心地在他的怀抱中找到安慰。
\subsection*{三、神的城是万民的希望(诗48:9-14)}
“我们在神的殿中,思想他的慈爱。”(诗48:9)

\subsubsection*{1. 神的慈爱与我们同在}
诗人不仅反思了神的保护,还提醒我们,神的慈爱与信实是他百姓的依靠。神不仅在历史中保护了他的城,也在今天与我们同在,他的慈爱常常伴随着我们。在神的殿中,我们可以思想神的慈爱,感受到他与我们同在的真实与亲近。
神的慈爱带给我们信心,帮助我们面对生命中的起伏。当我们仰望神的慈爱时,我们能够在心中找到安慰与希望。
\subsubsection*{2. 现实应用:神的慈爱是我们不变的希望}
在我们的生活中,神的慈爱不仅仅是安慰,也是力量的源泉。无论我们身处何种困境,神的慈爱给我们带来无尽的希望。每当我们在困境中感到迷茫或恐惧时,回想神对我们的爱,就能重新获得力量。
神的慈爱并不依赖我们的表现,而是因为他的信实与大爱。他的爱是我们坚定不移的盼望,让我们在风雨中依然能站立。
\subsubsection*{挑战:}

行动建议:让神的慈爱成为我们生活的核心。无论面对怎样的环境,我们都应当常常思想神对我们的爱,并在每一次的祷告和敬拜中,回归到神对我们的无条件的爱与恩典。
\subsection*{结论:神的城是我们永远的希望与保障}
\begin{enumerate}
    \item 神的城代表着神的荣耀与美丽——耶路撒冷作为神的圣城,是神荣耀与美丽的象征。今天,神的荣耀应当在我们每个人的生命中得以彰显。

    \item 神在我们困境中的保护——无论外界如何动荡,神始终是我们的保护者。我们可以安息在他的怀抱中,信靠他的能力和帮助。

    \item 神的城是万民的希望——神的慈爱是我们坚定的希望,他的同在与帮助永不改变。我们应当常常思想神的慈爱,并以此为力量,走向未来。

\end{enumerate}

亲爱的弟兄姐妹,诗篇48篇教导我们,神的荣耀与保护不仅是历史中的事实,更是我们每天生活中的真实体验。无论我们面对何种挑战,神的慈爱、神的主权都永远不变。让我们在每一天的生活中,以敬拜和赞美回应神的荣耀,并以信靠和依赖回应神的保护与帮助。

\subsection*{结束祷告}
\textbf{亲爱的天父,}

感谢你是那位伟大的神,祢的荣耀在全地彰显。感谢你以慈爱与保护环绕我们,在我们的一生中,你从未离弃我们。求你帮助我们在每一个困境中依靠你,因你的同在我们不至于动摇。求你赐给我们坚强的信心,常常在你的殿中思想你的慈爱,活出你的荣耀。

奉主耶稣基督的名祷告,阿们!
%-----------------------------------------------------------------------------
\newpage
\section{诗篇第49篇:财富与永恒——警示与盼望}

\subsection*{引言}
\hspace{0.6cm}亲爱的弟兄姐妹,今天我们一同研读诗篇49篇。这篇诗篇的主题是对财富、权势和物质的反思,诗人警告人们不要将心思放在短暂的、易逝的财富上,而应当追求更为重要、永恒的东西。诗篇49篇给我们提供了一个深刻的启示:无论我们在世上多么富足,最终,唯有对神的敬畏和生命的义才是我们的真正财富。

在今天的社会,物质财富的追求似乎成了许多人生活的中心。无论是在职业、家庭还是日常生活中,我们常常受到“成功”与“财富”标准的压力和诱惑。然而,诗篇49篇提醒我们,物质的追求并不能带给我们永恒的满足和安宁。真正的生命财富是与神的关系,是追求神的国和他的义。

\subsection*{一、财富与死亡的无常(诗49:1-12)}
“听啊,列国的人,所有住在世上的人,都要侧耳而听,连贵贱贫富都要注意。”(诗49:1-2)

\subsubsection*{1. 财富与地位的短暂性}
\hspace{0.6cm}诗篇49篇开篇呼召所有的人,不分贵贱、贫富都应当听从这警示。诗人指出,无论是富人还是贫穷人,都面临着同一个问题——死亡。富贵在世的那些人,也终究无法逃避死亡的审判。

在古代社会,财富和地位常常被看作是个人成功和幸福的标志,然而诗人揭示了一个深刻的事实:无论一个人多么富有,他无法带走任何东西,死亡是财富无法逃避的终结。财富并不能延续生命,反而它往往使人迷失方向,依赖物质而忽略了更重要的属灵追求。
\subsubsection*{2. 现实应用:财富与生命的意义}
\hspace{0.6cm}在现代社会,我们往往将财富、职业和社会地位作为衡量成功的标准。然而,诗篇49篇提醒我们,财富是短暂的,最终不能带给我们真正的满足和安息。我们必须认识到,财富本身不是生命的意义,人生的价值不在于积累物质财富,而在于我们如何活出神的旨意,如何在神面前活出真诚的生命。

这并不是说我们要抛弃财富,而是提醒我们,财富不能成为我们生命的中心,不能取代与神的关系。我们应当理智看待财富,把它作为神赐予我们的工具,而不是生命的目标。
\subsubsection*{挑战:}

行动建议:无论我们目前的经济状况如何,都要思考生命的真正价值。将财富看作是神的祝福与责任,而不是一切的终极追求。以神为中心的生活会让我们在拥有财富时不迷失,也在缺乏时不失去平安。
\subsection*{二、神的救赎与永恒的希望(诗49:13-20)}
“但神必救赎我的生命,脱离阴间的权势,因为他必收纳我。”(诗49:15)

\subsubsection*{1. 神的救赎带来永恒的盼望}
\hspace{0.6cm}诗篇49篇的后半部分转向了一个更为积极的主题,那就是神的救赎。诗人强调,虽然财富和权势不能阻止死亡,但神的救赎能带来永恒的生命。人可以依赖财富,但唯有神能带给我们真正的拯救。死亡不能吞噬神的百姓,因为神在他的恩典中,给了我们脱离死亡的希望。

诗人坚信,神必救赎那些信靠他的人,他不会让人永远留在死亡的权势之下。神的救恩和永生超越了所有世间的财富与荣华,神的应许给我们带来无与伦比的安慰与盼望。
\subsubsection*{2. 现实应用:信靠神,活在永恒的盼望中}
\hspace{0.6cm}当我们生活在这个物质充斥的世界时,我们容易把焦点放在眼前的财富和成就上,而忽视了生命的真正意义——永恒的生命和与神的关系。诗篇49篇教导我们,无论今生的成就如何,我们最终的归宿不是财富的堆砌,而是神的救赎和他为我们预备的永恒国度。

神的救赎给我们带来了不朽的盼望,让我们不再依赖短暂的世界财富,而是将焦点放在与神的关系上,活出一个为永恒目标而活的生命。
\subsubsection*{挑战:}

行动建议:让我们从神的救恩中汲取力量,不再将生命的价值寄托在短暂的财富上,而是为永恒的盼望而活。每一天都要活在神的救赎和恩典中,将目光从今生的物质焦虑中转向永恒的神国目标。
\subsection*{结论:财富的虚无与神的永恒}
\begin{enumerate}
    \item 财富与地位的无常——诗篇49篇警告我们,财富是短暂的,它不能带给我们永恒的安全和满足。无论我们在世上多么富足,最终死亡将是我们所有人的共同结局。

    \item 神的救赎带来的永恒希望——唯有神能带给我们超越死亡的救赎,他的恩典与永生是我们真正的希望。财富无法为我们带来真正的安全,唯有神的救恩能为我们提供不朽的盼望。

    \item 活出永恒的价值——我们应当理性看待财富,将其视为神的赐福,并使用它为神的荣耀,而不是把它当作生命的中心。最重要的是,我们要追求的是与神的关系,活出为永恒国度而活的生命。

\end{enumerate}

亲爱的弟兄姐妹,让我们不要让短暂的财富迷惑了我们的心,追求世上任何东西都无法带给我们真正的生命意义。只有神的救赎和他的永恒应许,才能成为我们生命的真正依靠和盼望。愿我们每个人都能活出神的旨意,依靠神的恩典,走向那永不朽坏的天国。

\subsection*{结束祷告}
\textbf{亲爱的天父,}

感谢你赐下诗篇49篇,让我们从中看见财富与生命的真正价值。感谢你为我们预备了救赎,给我们带来了永恒的希望。求你帮助我们在这个物质充斥的世界中,保持清醒的心智,不被财富所迷惑,常常以你为我们生命的中心,活出为永恒目标而活的生命。愿我们在你面前得着真实的满足和安慰。

奉主耶稣基督的名祷告,阿们!
%----------------------------------------------------------------------------
\newpage
\section{诗篇第50篇:敬畏神的真诚与生命的顺服}

\subsection*{引言}
亲爱的弟兄姐妹,今天我们要一同学习诗篇50篇。诗篇50篇是神通过亚萨的口发出的警告,它揭示了神对伪善宗教行为的审判,并强调了真诚的敬拜和对神命令的顺服。通过这篇诗篇,神让我们明白:真正的敬拜不仅仅是外在的仪式或祭祀,而是内心的顺服与生活的公义。
在我们今天的生活中,常常会遇到很多压力与诱惑,甚至是出于外在的宗教行为来寻求神的祝福,却忽略了神更看重的是我们内心的态度与生活的公义。诗篇50篇给我们提供了关于敬拜的深刻反思,它不仅仅是对古代以色列人宗教行为的警告,也是对今天每一个基督徒生活的提醒。
\subsection*{一、神的审判与对假冒伪善的警告(诗50:1-6)}
“全地和其中所有的,都是属于耶和华的。”(诗50:12)

\subsubsection*{1. 神是宇宙的主宰,审判一切}
诗篇50篇一开始,神通过亚萨宣告他的主权和审判。诗篇描述了神从西安发出命令,召集全地的人来接受审判。神强调,全地和其中的万物都是他的,没有任何人或物能够逃脱神的审判。
神的审判并不仅仅是对外在行为的审视,神看的是内心的动机和生命的态度。我们所有的行为,包括我们的敬拜,都是在神的审查之下。神不是单纯地看我们的宗教仪式和外在行为,而是看我们心中的真诚与顺服。
\subsubsection*{2. 现实应用:神审判的是我们的内心}
在今天的生活中,许多人在外表上看似敬虔,但在内心却远离神。我们可能会出席教会活动,参加祷告会,甚至奉献,但如果我们的心没有真正转向神,依然充满着虚伪、骄傲、罪恶和不顺服,那么我们的敬拜就会失去意义。
诗篇50篇提醒我们,神不单看我们外在的宗教行为,而是看我们的内心。神的审判不是针对我们能否做宗教仪式,而是看我们是否真实地活在他的面前,是否真正地敬畏他、顺服他。
\subsubsection*{挑战:}

行动建议:让我们检查自己的内心,避免做假冒伪善的敬拜。无论是祷告、崇拜,还是服侍,所有的一切都应该从真诚的心出发。我们要问自己:今天的敬拜,是否是出于对神的爱与敬畏?我们的心是否完全交托给神,愿意顺服他的旨意?
\subsection*{二、神不需要人的祭物,但需要真诚的敬拜(诗50:7-15)}
“我不指责你们的祭物,凡你们的燔祭,我常在你面前。”(诗50:8)

\subsubsection*{1. 神不需要祭物,他只看我们的心}
在诗篇50篇的中段,神告诉我们,他并不需要人的祭物。神并非无所不知,但他不是单纯为了祭物而接受我们的敬拜。神说:“我不责怪你们的祭物,但我不需要你们的祭物”。神不是靠祭物来满足自己的需要,因为神本身就是万有的主宰。
神的本意是希望我们通过这些祭物,表达出对他的顺服与敬畏,而不是单纯的仪式。真正的敬拜并不在于外在形式,而在于内心顺服与对神旨意的理解与回应。
\subsubsection*{2. 现实应用:真正的敬拜是全人全心的顺服}
在现代的教会生活中,我们常常看重外在的敬拜形式,比如诗歌、奉献、聚会等,但如果我们忽略了内心的改变和与神的真实关系,那么这些外在行为便失去了意义。
真正的敬拜不是形式上的宗教行为,而是全人全心的顺服神的旨意。神要的是我们的心、我们的悔改、我们的顺服,而不是外在的献祭与仪式。我们应该通过我们的一生来展示对神的敬爱与顺服,这才是真正的敬拜。
\subsubsection*{挑战:}

行动建议:今天,当我们站在神面前,我们不应该仅仅依赖传统的宗教仪式来安慰自己。我们应当检视自己的内心,是否愿意全身心地顺服神,是否愿意在日常生活中实践敬拜,而不仅仅是在教堂的时光中。
\subsection*{三、假冒伪善的结果与神的审判(诗50:16-23)}
“凡心里邪恶的,我必显明他的罪孽。”(诗50:19)

\subsubsection*{1. 假冒伪善的结局:神的审判}
诗篇50篇最后部分,神警告那些在敬拜中充满虚伪、心存不顺服的人,他们的行为最终会受到神的审判。神不会容忍虚伪和伪善,尤其是在敬拜他的时候。如果我们在敬拜中心存假意,神必定显明我们心中的罪恶。
神不仅仅看外在的行为,他更注重的是我们心中是否真诚。如果我们表面上敬拜神,内心却充满恶意与罪恶,神必定会审判我们。神的审判是公义的,他看透一切,我们无法隐瞒任何东西。
\subsubsection*{2. 现实应用:我们要真实地悔改与顺服}
今天的教会生活中,我们可能面临假冒伪善的挑战。**我们是否在生命中有不愿悔改的罪?是否有在敬拜中带着虚伪的心态?如果有,诗篇50篇提醒我们,\textbf{神的审判必定会降临,只有真实的悔改与顺服,才能使我们与神保持亲密的关系。}
神要的是我们真诚的悔改和内心的顺服。如果我们愿意顺服神,神必定赐给我们真正的平安和喜乐。
\subsubsection*{挑战:}

行动建议:让我们在神的面前真诚悔改,放下虚伪,抛弃心中的恶意,追求真实的敬拜。我们要记住,神所看重的不是我们的外在表现,而是我们内心的悔改与顺服。今天,让我们向神献上一个真实的心,愿意为他活出真正的敬拜。
\subsection*{结论:真诚的敬拜与内心的顺服}
\begin{enumerate}
    \item 神的审判与警告——诗篇50篇教导我们,神是宇宙的主宰,他审判的标准不仅仅是我们的外在行为,而是我们内心的动机与生活的态度。

    \item 神不需要我们的祭物,他需要我们的真诚与顺服——真正的敬拜是从心出发,是全人全心的顺服神的旨意,而不仅仅是外在的宗教仪式。

    \item 假冒伪善的结局——神警告我们,如果我们带着伪善的心态敬拜他,必定会面临他公义的审判。唯有真实的悔改与顺服才能得着神的赐福。

\end{enumerate}

亲爱的弟兄姐妹,让我们从诗篇50篇中反思自己的敬拜生活,避免伪善与虚伪,而要追求一个真诚、顺服神的心。愿我们每一天都活出神喜悦的生命,用真实的心来敬拜他。

\subsection*{结束祷告}
\textbf{亲爱的天父,}
感谢你通过诗篇50篇提醒我们敬拜的真义。求你帮助我们在敬拜中不要依赖外在的形式,而是从内心开始顺服你、敬畏你。帮助我们在一切事情中活出真诚的敬拜,不让伪善占据我们的生命。愿我们每一天都为你活出真实的生命,愿你喜悦我们的敬拜。
奉耶稣基督的名祷告,阿们。
%----------------------------------------------------------------------------
\newpage
\section{诗篇第51篇:真实悔改,得蒙洁净}

\subsection*{引言:罪与悔改的呼唤}
\hspace{0.6cm}弟兄姐妹,今天我们要一起学习诗篇51篇,这是一首充满痛悔、祈求怜悯和渴望洁净的诗篇。大卫在犯下奸淫和谋杀的罪之后,被先知拿单指责,他的心被神的光照,深深地认识到自己的罪,并向神发出了真诚悔改的呼求。

这首诗篇不仅是大卫个人的祷告,更是神赐给我们每一个人的悔改之路。我们每个人都会软弱、跌倒,但神的怜悯是无限的,只要我们真诚悔改,神就愿意赦免,并赐下更新的生命。
\subsection*{一、承认自己的罪,寻求神的怜悯(1-6节)}
“神啊,求你按你的慈爱怜悯我,按你丰盛的慈悲涂抹我的过犯!”(诗51:1)

\subsubsection*{1. 罪带来的破碎}
大卫的祷告始于承认自己的罪,他没有找借口,也没有逃避,而是直接来到神面前,恳求神的怜悯。
他深知自己得罪的不仅是人,而是神本身。他说:“我向你犯罪,唯独得罪了你。”(诗51:4) 这句话提醒我们,所有的罪最终都是对神的冒犯。
\subsubsection*{2. 现实应用:我们是否意识到自己的罪?}
现代人往往容易忽视罪的严重性。我们可能会觉得“小错无伤大雅”,或者认为只要不伤害别人,神不会太在意。但诗篇51篇提醒我们,每一个罪都会影响我们与神的关系,并破坏我们的灵命。
悔改的第一步就是承认自己的罪,不找借口,不推卸责任,直接来到神面前寻求他的怜悯。
\subsubsection*{挑战与行动建议:}
\hspace{0.6cm}每日反思自己的言行:我们是否在某些事上亏欠了神?是否在生活中容忍了某些看似“无害”的罪?

用诗篇51篇的前6节作为每日祷告,向神承认我们的罪,求他洁净我们。
\subsection*{二、祈求洁净,渴望内心的更新(7-12节)}
“神啊,求你为我造清洁的心,使我里面重新有正直的灵。”(诗51:10)

\subsubsection*{1. 悔改不仅仅是认罪,更是寻求更新}
大卫不仅求神赦免他的罪,更渴望从内心被更新。他知道,单单被赦免还不够,他需要被改变,需要神的灵在他里面做工,使他能够远离罪恶。
真正的悔改不是“下次不再犯”,而是让神的恩典在我们里面工作,使我们从根本上被改变。
\subsubsection*{2. 现实应用:我们是否真正渴望被神更新?}
有时候我们认罪,但并没有真正想要改变。我们可能会对神说:“神啊,求你饶恕我。”但我们的内心仍然留恋罪,仍然不愿意离开旧的生活方式。
真正的悔改意味着我们愿意让神重塑我们的心,使我们远离罪,活出圣洁的生命。
\subsubsection*{挑战与行动建议:}
\hspace{0.6cm}祷告求神洁净我们的心,不仅仅是求神赦免,而是求神改变我们的思想、价值观和生活方式,使我们更加渴慕圣洁。

建立新的属灵习惯:阅读圣经、祷告、敬拜,让神的话语充满我们的心,使我们能够刚强地抵挡罪的诱惑。
\subsection*{三、悔改带来新的生命和真实的敬拜(13-19节)}
“主啊,求你使我张开嘴,我的口便传扬赞美你的话。”(诗51:15)

\subsubsection*{1. 被赦免的人会有新的生命}
大卫的悔改并没有让他停留在自责和懊悔之中,相反,他渴望用新的生命来荣耀神。他说:“我就把你的道指教有过犯的人,罪人必归顺你。”(诗51:13)
真正的悔改会带来生命的改变,并让我们愿意向人见证神的恩典。
\subsubsection*{2. 神所悦纳的敬拜是真诚的心}
诗篇51:16-17说:“你本不喜爱祭物,若喜爱,我就献上;燔祭你也不喜悦。神所要的祭,就是忧伤的灵;神啊,忧伤痛悔的心,你必不轻看。”
这告诉我们,神不看重外在的宗教仪式,他看重的是内心真正的敬拜。我们可以奉献、服侍、参与教会,但如果我们的心没有真实的敬畏与悔改,这些行为在神眼中毫无意义。
\subsubsection*{现实应用:我们的敬拜是否出于真诚的心?}
我们是否真的带着敬畏与感恩的心来到神面前? 还是仅仅是出于习惯、责任,甚至是为了得到神的祝福?
真正的敬拜不是外在的仪式,而是内心的顺服和爱神的回应。
\subsubsection*{挑战与行动建议:}
\hspace{0.6cm}以感恩的心敬拜神:每天数算神的恩典,让悔改后的新生命成为我们赞美神的动力。

向人分享神的恩典:如果你经历了神的赦免和更新,不要隐藏,让你的见证成为别人的祝福。
\subsection*{结论:真实悔改,得蒙洁净}
\begin{enumerate}
    \item 真实的悔改始于承认自己的罪,并寻求神的怜悯。
    \item 真实的悔改不仅是求赦免,更是求神更新我们的心,使我们远离罪恶。

    \item 真实的悔改会带来新的生命,让我们用真实的敬拜和见证荣耀神。

\end{enumerate}

弟兄姐妹,神是慈爱的,他愿意赦免一切真心悔改的人。让我们不只是口头的认罪,而是真正地让神在我们的生命中掌权,使我们活出圣洁的生命,成为神所喜悦的人!

\subsection*{结束祷告}
\textbf{亲爱的天父,}

我们感谢你,因你的怜悯,我们可以来到你面前寻求赦免。求你洁净我们的心,使我们不只是认罪,更是愿意被你更新。帮助我们远离罪恶,活出圣洁的生命,让我们的敬拜是真实的,生命是蒙你悦纳的。谢谢你赦免我们,使我们重新得力,愿你掌管我们的一生。

奉主耶稣基督的名祷告,阿们!
%----------------------------------------------------------------------------
\newpage
\section{诗篇第52篇:倚靠神的慈爱,胜过骄傲与恶行}
\subsection*{引言:倚靠神,还是倚靠自己的势力?}
\hspace{0.6cm}弟兄姐妹,我们每一天都在面临一个重要的选择:我们要倚靠神的慈爱,还是倚靠自己的能力、财富或权势?诗篇52篇向我们展示了两个不同的生命态度——一个是骄傲、欺压人的恶人,一个是谦卑倚靠神的义人。

大卫在这篇诗篇中斥责了恶人的虚假与狂傲,同时表达了自己对神坚定的信靠。他提醒我们,人的权势、财富和谎言终究都会消逝,唯有神的慈爱和公义永远长存。
\subsection*{一、恶人的特征:信靠自己,不敬畏神(1-4节)}
“勇士啊,你为何以作恶自夸?神的慈爱是常存的。”(诗52:1)

\subsubsection*{1. 以恶自夸,依靠权势}
\hspace{0.6cm}这里的“勇士”指的是多益(撒母耳记上22:9-19),他为了讨好扫罗王,出卖了大卫,并导致挪伯城的祭司被屠杀。
这是一种典型的恶人心态:他们依靠自己的势力,以为可以为所欲为,不敬畏神。

现实应用:在我们的生活中,我们是否也看到类似的现象?有些人依靠自己的财富、权力或口才欺压别人,以为可以掌控一切,却不知神的公义审判终必临到。
\subsubsection*{2. 口出谎言,行事诡诈}
“你的舌头邪恶诡诈,好像剃头刀,快利伤人。”(诗52:2)

恶人的一个显著特征是口出谎言,用言语伤害别人。

现实应用:在职场、家庭、社交媒体上,我们是否见过那些喜欢操纵、诋毁他人的人?他们可能为了自己的利益,不惜撒谎、操纵事实,甚至毁坏别人的名声。

挑战:我们是否也在不知不觉中,因着嫉妒或私心,说出伤害他人的话?让我们谨慎自己的言语,不要成为“快利的剃头刀”,而要用言语建立人,而不是拆毁人。
\subsubsection*{3. 喜爱恶事,远离神的真理}
“你喜爱恶多于善,又喜爱说谎,不喜爱说公义。”(诗52:3)

恶人的另一个特征是,他们不仅行恶,还乐在其中,不愿意听从神的真理。

现实应用:今天的世界充满了扭曲的价值观,很多人以欺骗、贪婪、骄傲为成功的手段,而不愿意遵循神的公义和慈爱。

挑战:我们要常常自省,是否在某些方面妥协了世界的价值观,而远离了神的教导?
\subsection*{二、恶人的结局:神的审判必临到(5-7节)}
“神也要毁灭你,直到永远;他要把你拿去,从帐篷中抽出,又从活人之地将你拔出。”(诗52:5)

\subsubsection*{1. 神的公义审判必定临到}
\hspace{0.6cm}恶人可能暂时兴盛,但他们的结局是毁灭。神的审判是彻底的,他要将恶人连根拔起,使他们无法再存留。

现实应用:历史上多少骄傲自大的统治者、企业家、政客,曾经一时风光,但最终都倒下了。世界上的权势和财富是暂时的,唯有神的国度永存。
\subsubsection*{2. 义人的喜乐:神必显明公义}
“义人要看见而惧怕,并要笑他。”(诗52:6)

义人看到神的审判,就知道公义终将得胜,因此他们存着敬畏神的心,并因神的作为而喜乐。

现实应用:当我们面对恶人猖獗的时候,不要气馁,因为神的公义终必显明,我们只需要耐心等候。
\subsubsection*{3. 依靠自己终将失败}
“看哪,这就是那不以神为他力量的人;他只倚仗丰富的财物,在邪恶上坚立自己。”(诗52:7)

这提醒我们,任何不倚靠神,而倚靠金钱、权势、自己的聪明才智的人,最终都会失败。

现实应用:在追求成功的过程中,我们是否把信心放在金钱、人脉、能力上,而不是放在神身上?
\subsection*{三、义人的回应:信靠神,扎根于他的慈爱(8-9节)}
“至于我,就像神殿中的青橄榄树;我永永远远倚靠神的慈爱。”(诗52:8)

\subsubsection*{1. 义人像青橄榄树,扎根于神}
\hspace{0.6cm}橄榄树象征着生命、繁荣、祝福。义人不是像恶人一样短暂的荣华,而是像橄榄树一样,根基稳固,长久结果子。

现实应用:我们的生命是否扎根于神?我们是否每天通过读经、祷告、敬拜来亲近他,使我们的信心稳固?
\subsubsection*{2. 信靠神的慈爱,不倚靠世界的势力}
\hspace{0.6cm}现实应用:在面对挑战、逼迫、不公时,我们是否仍然坚持信靠神?还是在困境中失去信心,转向人的方法?
\subsubsection*{3. 以感恩回应神的信实}
“我要称谢你,直到永远,因为你行了这事;我也要在你圣民面前仰望你的名,这名本为美好。”(诗52:9)

义人的生命是充满感恩的,因为他们知道神是信实的,他的名是美好的。

现实应用:我们是否常常数算神的恩典,并在众人面前见证他的美善?
\subsection*{结论:倚靠神的慈爱,远离骄傲与恶行}
\begin{enumerate}
    \item 恶人的特征:依靠自己,不敬畏神,口出谎言,最终必灭亡。

    \item 神的公义审判必定临到,恶人无法逃脱。

    \item 义人要倚靠神的慈爱,扎根于他,并以感恩敬拜他。

\end{enumerate}

让我们选择成为义人,不倚靠世界的势力,而是倚靠神的慈爱,活出一个荣耀神的生命!

\subsection*{结束祷告}
\textbf{亲爱的天父,}

我们感谢你,因你的慈爱,我们可以在你里面得着稳固的生命。帮助我们远离骄傲、自义和诡诈,让我们做一个真正倚靠你的人。无论环境如何,我们愿意扎根于你的恩典,持守信仰,成为你忠实的儿女。

奉主耶稣基督的名祷告,阿们!
%-----------------------------------------------------------------------------
\newpage
\section{诗篇第53篇:愚顽人心中的无神论}

\subsection*{引言:世界观的抉择}
弟兄姐妹,我们每一天都在做出信仰上的选择。是相信神的存在和主权,还是活得像没有神一样?诗篇53篇是一篇深刻揭示人心的诗篇,它不仅描述了世人的愚顽,也宣告了神的审判和义人的盼望。

\subsection*{一、愚顽人的心态:否认神,活在罪中(1-3节)}
“愚顽人心里说:‘没有神’。他们都是邪恶,行了可憎恶的罪孽,没有一个人行善。”(诗53:1)

\subsubsection*{1. “愚顽人”是谁?}
\hspace{0.6cm}这里的“愚顽人”不是指智商低下的人,而是指心灵刚硬、不敬畏神、道德败坏的人。他们不是因为缺乏知识而否认神,而是因为他们不愿意顺服神的权柄。

现实应用:今天的世界中,很多人嘴上可能不说“没有神”,但他们的生活方式却完全排斥神。他们追求自己的利益,活得像“没有神”一样。
\subsubsection*{2. 否认神的直接后果——道德败坏}
\hspace{0.6cm}当人心中没有神,他们就会任意妄为,不受道德的约束。“行了可憎恶的罪孽”意味着他们的行为使神厌恶。

现实应用:我们看到的社会乱象,如贪腐、欺骗、不公正、暴力等,正是人远离神的后果。当人心没有神,社会道德就会滑坡。
\subsubsection*{3. 人的普遍败坏}
“神从天上垂看世人,要看有明白的没有?有寻求他的没有?”(诗53:2)

神在寻找敬畏他的人,但\textbf{“没有寻求神的”},这表明世人的败坏是普遍的,罪影响了所有人。

现实应用:我们是否真正寻求神?还是只是表面上信仰,生活方式却远离神?
\subsection*{二、神的审判:神察看世人,定意惩治罪恶(4-5节)}
“作孽的没有知识吗?他们吞吃我的百姓如同吃饭一样,并不求告神。”(诗53:4)

\subsubsection*{1. 恶人的残暴与无知}
\hspace{0.6cm}这里描述了恶人“吞吃百姓如同吃饭”,表示他们毫无怜悯,任意践踏别人。

现实应用:在职场、社会中,有些人以剥削别人、欺压弱小为生。他们似乎无所畏惧,但神的审判不会迟到。
\subsubsection*{2. 神的审判使恶人恐惧}
“他们在那里大大害怕,因为神把那些安营攻击你之人的骨头打散了。”(诗53:5)

恶人可能一时嚣张,但当神的审判临到,他们必然惧怕、颤抖。

现实应用:有时候,我们会感到恶人昌盛、公义迟迟未见。但这节经文提醒我们,神的审判终必来到,我们不必灰心。
\subsection*{三、义人的盼望:救恩终将临到神的子民(6节)}
“但愿以色列的救恩从锡安而出!神救回他被掳的子民那时,雅各要快乐,以色列要欢喜。”(诗53:6)

\subsubsection*{1. 救恩的应许}
\hspace{0.6cm}诗人看到了神救赎的盼望,知道神终究会施行拯救。

现实应用:今天,我们也要持守这个盼望。无论世界如何败坏,神的救恩必然成就。最终,义人将得胜,神的公义必要完全彰显。
\subsubsection*{2. 义人的喜乐}
\hspace{0.6cm}当神施行拯救时,雅各和以色列(象征神的子民)都要欢喜。

现实应用:我们在等候神的公义时,不应充满焦虑和怨恨,而要带着信心、盼望和喜乐,相信神的救赎终必临到。
\subsection*{结论:你愿意成为愚顽人,还是义人?}
\begin{enumerate}
    \item 愚顽人否认神,导致道德败坏和最终的审判。

    \item 神察看世人,并且必定审判罪恶,使恶人惧怕。

    \item 义人要仰望神的救恩,持守信心,最终将得胜。

\end{enumerate}
\subsubsection*{问题挑战:}
\begin{itemize}
    \item 你是否在生活中真实地敬畏神,而不是口头上的信仰?

    \item 你是否愿意信靠神,等待他的公义,而不是靠自己的方法去报复或抱怨?

    \item 你是否有盼望,相信神的救恩终会来到?

\end{itemize}

让我们选择成为义人,敬畏神,持守信仰,等候他的救恩!

\subsection*{结束祷告}
\textbf{亲爱的天父,}

我们感谢你,因为你是公义的神,你鉴察人心,不容罪恶得胜。求你使我们远离愚顽,不否认你,而是在生活中真实地敬畏你。帮助我们在等候你的救恩时,依然持守信心,活出你的光。我们相信你的公义必定显明,愿你的名得荣耀!

奉主耶稣基督的名祷告,阿们!
%-----------------------------------------------------------------------------
\newpage
\section{诗篇第54篇:在困境中仰望神的拯救}

\subsection*{引言:当人生遭遇背叛与困境}
弟兄姐妹,你是否曾经历过被人误解、攻击甚至背叛的痛苦?在困境中,我们该如何应对?是凭自己的能力反击,还是转向神,寻求他的帮助?
诗篇 54 篇是大卫在危难时刻的祷告。当西弗人向扫罗泄密,出卖大卫的藏身之处(撒上 23:19-20),大卫被迫再次逃亡。在这样的背景下,他写下这首诗,向神呼求拯救,并坚信神的公义必然显现。

\subsection*{一、转向神,寻求他的帮助(1-3节)}
“神啊,求你以你的名救我,凭你的大能为我伸冤。”(诗 54:1)

\subsubsection*{1. 直奔神的名与能力}
\hspace{0.6cm}大卫没有先求人的帮助,而是直接向神祷告。他知道,“神的名”代表他的本质、权柄与信实,而“神的大能”则是他施行拯救的手段。

现实应用:当我们遭遇不公或困难时,我们是否第一时间祷告,而不是依靠自己的方法去解决?

\vspace{0.2cm}

“神啊,求你听我的祷告,留心听我口中的言语。”(诗 54:2)

这显示了大卫对神的信任,他确信神是听祷告的神。
\subsubsection*{2. 敌人的恶行与狂傲}
“因为外人起来攻击我,强暴人寻索我的命;他们眼中没有神。”(诗 54:3)

\textbf{“外人”}指的是那些与神的约毫无关系、不敬畏神的人,他们的行动完全是基于自私和野心。

现实应用:在职场、生活中,我们可能遇到不敬畏神的人试图伤害我们,甚至遭遇背叛。这时,我们需要像大卫一样,把一切交托给神,而不是陷入恐惧或愤怒。
\subsection*{二、相信神,倚靠他的公义(4-5节)}
“神是帮助我的,是扶持我命的。”(诗 54:4)

\subsubsection*{1. 神是我们的帮助者}
\hspace{0.6cm}大卫并没有陷入绝望,而是宣告神是他的帮助。他知道,人可以背叛他,但神永不离弃。

现实应用:当朋友或同事误解我们时,我们是否仍然相信神是我们的帮助?还是我们会选择用世界的方法去自卫?
\subsubsection*{2. 神必报应恶人}
“他要报应我仇敌所行的恶,求你凭你的诚实灭绝他们。”(诗 54:5)

大卫不是自己复仇,而是把审判的权柄交托给神。他相信,神的公义必定显明。

现实应用:我们是否愿意把那些伤害我们的人交托给神,而不是自己伸冤?罗马书 12:19 说:“伸冤在我,我必报应,这是主说的。”
\subsection*{三、感恩神,因他的救赎而欢喜(6-7节)}
“我要把甘心祭献给你,耶和华啊,我要称赞你的名,因为这是美好的。”(诗 54:6)

\subsubsection*{1. 感恩是信心的表达}
\hspace{0.6cm}大卫在尚未看到拯救前,就已经决定向神献上感恩的祭。

现实应用:当我们还在等待神的回应时,我们是否已经预备好感恩的心,提前相信神的作为?

\vspace{0.2cm}
“他从一切的急难中把我救出来;我的眼睛也看见了我仇敌遭报。”(诗 54:7)

这是大卫的信心宣告。他经历了神多次的拯救,所以他相信这次神仍然会施行拯救。

现实应用:当我们回顾神过去的恩典,我们就能坚定相信未来他仍然会带领我们走出困境。
\subsection*{结论:信靠神,在困境中得胜}
诗篇 54 篇教导我们,在人生的困境和背叛中,我们应该:
\begin{enumerate}
    \item 转向神,寻求他的帮助(而不是依靠人或自己的力量)。

    \item 相信神,倚靠他的公义(而不是自己伸冤)。

    \item 感恩神,因他的救赎而欢喜(即使拯救尚未显现,我们也可以凭信感谢他)。

\end{enumerate}
\subsubsection*{挑战问题:}
\begin{itemize}
    \item 你最近是否遇到不公正的对待?你是选择自我伸冤,还是交托给神?

    \item 你是否愿意在困境中依靠神,并且提前为即将到来的拯救献上感恩?

\end{itemize}

愿我们学习大卫的信心,在任何处境下都仰望神,相信他是我们的帮助者!

\subsection*{结束祷告}
\textbf{亲爱的天父,}

感谢你借着诗篇 54 篇教导我们,在困境中仍然要信靠你。主啊,帮助我们在被误解、被攻击或被背叛时,不是陷入苦毒,而是转向你,寻求你的拯救。愿我们像大卫一样,凭信心等候你的公义,并且用感恩的心来回应你的作为。谢谢你永远是我们的帮助者。

奉主耶稣基督的名祷告,阿们!
%----------------------------------------------------------------------------
\newpage
\section{诗篇第55篇:在背叛与重压中投靠神}
\subsection*{引言:人生的重担与心灵的伤痛}
弟兄姐妹,我们都曾经历人生的风暴,也许是朋友的背叛、家人的误解,或是生活的重担让我们感到无力。在这样的时刻,我们应该如何回应?诗篇 55 篇是大卫在极度痛苦中向神倾诉的祷告,他不仅面对外敌的攻击,更经历了来自至亲朋友的背叛。然而,在他的挣扎与哭求中,他学会了把一切交托给神。

\subsection*{一、向神倾诉,寻求他的聆听(1-8节)}
“神啊,求你留心听我的祷告,不要隐藏不听我的恳求!求你侧耳听我,应允我。”(诗 55:1-2)

\subsubsection*{1. 祷告是痛苦时的出口}
\hspace{0.6cm}大卫没有把苦毒埋在心里,而是向神倾诉。他求神“留心听”,因为他知道神是信实的。

现实应用:当我们遇到压力、困惑、伤害时,我们是向神倾诉,还是自己压抑、埋怨或求助于人?

\vspace{0.2cm}

“我心在我里面甚是疼痛,死的惊惶临到我。”(诗 55:4)

这是一种极深的痛苦,甚至让大卫恐惧死亡。他真实地表达自己的情绪。

现实应用:今天,许多人面临焦虑、抑郁、背叛的痛苦,但神愿意听我们的哀叹,我们可以自由地向他诉说。
\subsubsection*{2. 逃避的渴望}
“我说:但愿我有翅膀像鸽子,我就飞去得享安息。”(诗 55:6)

大卫渴望逃避痛苦,但他最终没有逃走,而是选择面对,并把一切交托给神。

现实应用:我们是否曾希望逃避现实,却发现只有面对它,交托给神,才能真正得安息?
\subsection*{二、信靠神,面对恶人的背叛(9-21节)}
\subsubsection*{1. 外敌的攻击}
“他们在城墙上昼夜绕行,在城内也是罪孽和奸恶。”(诗 55:10)

大卫看见敌人充满邪恶,但他知道,神最终会审判他们。

现实应用:社会中不公义的现象让我们焦虑,但我们要相信神掌管一切,不用害怕。
\subsubsection*{2. 朋友的背叛更令人痛苦}
“不是仇敌辱骂我,若是仇敌,我还能忍耐;也不是恨我的人向我狂大,若是恨我的人,我必躲避他。不料,是你,你原与我平等,是我的同伴,是我知己的朋友。”(诗 55:12-13)

朋友的背叛比仇敌的攻击更让人痛苦。大卫曾信任这个人,但最终他伤害了大卫。

现实应用:我们是否曾被信任的人伤害?当被亲近的人出卖时,我们该如何回应?
\subsubsection*{3. 神最终会报应恶人}
“至于我,我要求告神,耶和华必拯救我。”(诗 55:16)

大卫没有自己复仇,而是把审判交给神。

现实应用:当我们被冤枉或受到伤害时,我们是否愿意让神来伸冤,而不是自己报复?
\subsection*{三、交托重担,等候神的拯救(22-23节)}
\subsubsection*{1. 交托重担}
“你要把你的重担卸给耶和华,他必扶持你;他永不叫义人动摇。”(诗 55:22)

交托不是放弃,而是信任神。我们不能自己扛起生命中的一切重担,唯有神能扶持我们。

现实应用:我们是否真正把烦恼交给神,还是仍然紧抓不放?
\subsubsection*{2. 神的公义必然成就}
“神啊,你必使恶人下到灭亡的坑。”(诗 55:23)

最终,神的公义会显现,恶人不会永远昌盛。

现实应用:我们可以安心信靠神,而不是焦虑未来。
\subsection*{结论:在重压与背叛中,信靠神}
诗篇 55 篇教导我们,在面对人生的风暴时,我们要:
\begin{enumerate}
    \item 向神倾诉,寻求他的聆听(把我们的痛苦告诉神)。

    \item 信靠神,面对恶人的背叛(不自己伸冤,而是交托给神)。

    \item 交托重担,等候神的拯救(真正把忧虑交给神,并相信他掌管一切)。

\end{enumerate}
\subsubsection*{挑战问题:}
\begin{itemize}
    \item 你是否正经历来自身边人的伤害?你愿意将它交给神,而不是自己报复吗?
    \item 你是否愿意信靠神,等候他的拯救,而不是靠自己挣扎?

    \item 你是否愿意把你的重担完全卸给神,相信他会扶持你?

\end{itemize}

愿我们都学习大卫,在风暴中坚定信靠神!

\subsection*{结束祷告}
\textbf{亲爱的天父,}

感谢你借着诗篇 55 篇提醒我们,当我们被伤害、被误解,甚至被至亲朋友背叛时,我们仍然可以信靠你。主啊,我们把一切重担交托给你,求你安慰我们、扶持我们,使我们不被仇恨和苦毒吞噬。愿我们在你的怀抱中得安息,深知你掌管一切,并且必伸张公义。

谢谢你是信实的神,奉主耶稣基督的名祷告,阿们!
%---------------------------------------------------------------------------
\newpage
\section{诗篇第56篇:在惧怕中信靠神}
\subsection*{引言:当恐惧来临时,你会怎么做?}
弟兄姐妹,我们都曾经历害怕的时刻:也许是突如其来的疾病,也许是面对不确定的未来,或是遭遇他人的攻击。在恐惧面前,我们该如何回应?
诗篇 56 篇是大卫在被非利士人抓住时的祷告(撒上 21:10-15)。面对生死关头,他不是靠自己的聪明,而是选择信靠神。今天,我们要学习如何在恐惧和困境中抓住神的应许,并坚定信靠他。

\subsection*{一、向神呼求——在危难中,我们要向神祷告(1-4节)}
“神啊,求你怜悯我,因为人要把我吞了,终日攻击欺压我。”(诗 56:1)

\subsubsection*{1. 人的恐惧是正常的,但我们可以带到神面前}
\hspace{0.6cm}大卫在非利士人的城市迦特,孤立无援,被敌人包围,随时可能被杀。他的恐惧是真实的。

现实应用:当我们感到害怕时,我们可以向神祷告,而不是自己扛着不安。

\vspace{0.2cm}

“我惧怕的时候要倚靠你。”(诗 56:3)

这节经文告诉我们:害怕并不可耻,关键是当害怕时,我们倚靠谁?
大卫没有被恐惧击倒,而是选择倚靠神。

现实应用:你现在是否在害怕某件事?你愿意像大卫一样,把惧怕交给神吗?
\subsection*{二、坚定信靠——在攻击中,我们要相信神的应许(5-11节)}
\subsubsection*{1. 人的攻击不会得胜}
“他们终日颠倒我的话;他们一切的心思都是要害我。”(诗 56:5)

敌人用言语攻击大卫,曲解他的话,制造谎言。

现实应用:今天,我们是否也曾遭遇流言蜚语、不公的指控?我们可以像大卫一样,把一切交托神,而不是自己去报复。
\subsubsection*{2. 神纪念我们的眼泪}
“我几次流离,你都记数;求你把我眼泪装在你的皮袋里。这不都记在你册子上吗?”(诗 56:8)

神关心我们的痛苦,甚至连我们的眼泪都珍藏起来。

现实应用:你曾在痛苦中流泪吗?神都看见了。他不会让我们的痛苦白费。
\subsubsection*{3. 信靠神,就不惧怕}
“我倚靠神,必不惧怕。人能把我怎么样呢?”(诗 56:11)

大卫的信心来自神,而不是环境。他知道,人无法真正伤害他,因为神在掌权。

现实应用:当你面对恐惧时,你是相信神的保护,还是被恐惧控制?
\subsection*{三、带着感恩生活——在得胜中,我们要荣耀神(12-13节)}
\subsubsection*{1. 还愿与感恩}
“神啊,我向你所许的愿在我身上;我要将感谢祭献给你。”(诗 56:12)

大卫在困境中向神许愿,他得胜后不忘感恩。

现实应用:当神拯救我们后,我们是否仍然感恩,还是只在危难时才呼求神?
\subsubsection*{2. 活出神的光明之路}
“因你救我的命脱离死亡,你岂不是救我的脚不跌倒,使我在生命光中行在神面前吗?”(诗 56:13)

得胜后,我们要继续行在神的光中,荣耀他。

现实应用:你愿意每天活在神的光中,用生命见证他的恩典吗?
\subsection*{结论:在恐惧中,我们要信靠神}
诗篇 56 篇教导我们,在面对恐惧和攻击时,我们要:
\begin{enumerate}
    \item 向神呼求——不把恐惧憋在心里,而是交给神。

    \item 坚定信靠——相信神掌权,不被环境影响。

    \item 带着感恩生活——在得胜后,继续荣耀神。

\end{enumerate}
\subsubsection*{挑战问题:}
\begin{itemize}
    \item 你现在是否正面对恐惧?你愿意向神呼求,而不是自己抗争吗?

    \item 你愿意相信神纪念你的眼泪,并在困境中信靠他吗?

    \item 你愿意带着感恩活在神的光中,见证他的恩典吗?

\end{itemize}

愿我们都能在恐惧中选择信靠神!

\subsection*{结束祷告}
\textbf{亲爱的天父,}

感谢你通过诗篇 56 篇教导我们,无论何时,我们都可以信靠你。主啊,你知道我们的恐惧和忧虑,求你帮助我们把一切交托给你,不让恐惧胜过我们的信心。我们相信你掌权,知道你纪念我们的眼泪,也深信你必拯救我们。愿我们在得胜后,不忘感恩,继续在你的光中生活。

谢谢你是信实的神,奉主耶稣基督的名祷告,阿们!
%----------------------------------------------------------------------------
\newpage
\section{诗篇第57篇:在困境中仰望神}
\subsection*{引言:当我们身处绝境,我们仰望谁?}
\hspace{0.6cm}亲爱的弟兄姐妹,生活中我们难免遇到困境:有人遭受不公正的对待,有人被朋友背叛,有人身处危机之中,甚至感觉走投无路。这时候,我们的心态如何?我们会绝望吗?还是选择仰望神?

今天,我们要透过诗篇 57 篇来学习如何在困境中仰望神,并经历他的拯救。这首诗是大卫在被扫罗追杀,躲在洞里时写的(撒上 22:1)。面对生死危机,大卫没有怨天尤人,而是将目光定睛在神身上。

\subsection*{一、在苦难中,寻求神的怜悯(1-3节)}
“神啊,求你怜悯我,怜悯我!因为我的心投靠你;我要投靠在你翅膀的荫下,等到灾害过去。”(诗 57:1)

\subsubsection*{1. 投靠神,而不是环境}
\hspace{0.6cm}大卫当时躲在洞里,身体虽有遮蔽,但他知道真正的避难所是神的翅膀之下(象征神的保护与安慰)。

现实应用:当我们遇到危机时,我们是否更依赖自己的聪明、人际关系,还是首先来到神面前?
\subsubsection*{2. 相信神会拯救}
“我要求告至高的神,就是为我成全诸事的神。”(诗 57:2)

大卫深信神掌管一切,并且“为他成全诸事”。

现实应用:无论面对经济压力、人际关系问题,还是身体疾病,我们都可以倚靠神,相信他会成全一切。
\subsubsection*{3. 神从天上施行拯救}
“那要吞我的人辱骂我的时候,神必从天上施恩救我。”(诗 57:3)

大卫的信心不是建立在环境的改变上,而是建立在神的信实之上。

现实应用:我们是否愿意在困难中仰望神,相信他会施恩拯救?
\subsection*{二、在危机中,坚定信靠神(4-6节)}
“我的性命在狮子中间,我躺卧在性情猛烈的人中。”(诗 57:4)

\subsubsection*{1. 现实的危机是真实的}
\hspace{0.6cm}大卫的敌人就像狮子一样,随时要吞噬他。

现实应用:我们也会面对困难,比如职场中的不公、朋友的背叛,甚至信仰上的逼迫。我们要如何面对?
\subsubsection*{2. 不报复,而是仰望神}
“他们为我的脚设下网罗,压制我的心;他们在我面前挖了坑,自己反掉在其中。”(诗 57:6)

大卫没有靠自己的方法反击,而是让神来掌权。

现实应用:当我们受到不公正的对待时,我们愿意交给神,而不是自己动手解决吗?
\subsection*{三、在得胜后,颂赞神的名(7-11节)}
“神啊,我心坚定,我心坚定;我要唱诗,我要歌颂!”(诗 57:7)

\subsubsection*{1. 在困境中依然赞美}
\hspace{0.6cm}大卫仍然选择歌颂神,而不是抱怨。

现实应用:我们是否愿意在挑战中仍然感恩,歌颂神?
\subsubsection*{2. 让全地都知道神的荣耀}
“主啊,你当崇高,过于诸天!愿你的荣耀高过全地!”(诗 57:11)

大卫不仅自己信靠神,更希望让全世界都看到神的荣耀。

现实应用:我们的生命是否成为别人认识神的见证?
\subsection*{结论:在困境中,我们选择仰望神!}
诗篇 57 篇提醒我们:
\begin{enumerate}
    \item 在苦难中,我们要寻求神的怜悯,不要靠自己解决。

    \item 在危机中,我们要坚定信靠神,而不是抱怨环境。

    \item 在得胜后,我们要颂赞神的名,让人看到神的荣耀。

\end{enumerate}
\subsubsection*{挑战问题:}
\begin{itemize}
    \item 你现在是否处在困境中?你愿意寻求神的怜悯,而不是靠自己的方法吗?

    \item 你是否愿意像大卫一样,把一切交托给神,而不是自己解决?

    \item 你是否愿意无论顺境逆境,都选择赞美神?

\end{itemize}

愿我们都能在困境中仰望神,并经历他的得胜!

\subsection*{结束祷告}
\textbf{亲爱的天父,}

感谢你透过诗篇 57 篇教导我们,无论遇到什么困难,我们都可以投靠你。主啊,我们承认自己有时会害怕,会想用自己的方法解决问题,但今天我们愿意学习大卫的信心,把一切交托给你。主啊,求你帮助我们,不论环境如何,我们都能坚定信靠你,并在得胜后荣耀你的名。

谢谢你是信实的神,奉主耶稣基督的名祷告,阿们!
%------------------------------------------------------------------------------
\newpage
\section{诗篇第58篇:义人的盼望}
\subsection*{引言:面对不公,我们如何回应?}
\hspace{0.6cm}亲爱的弟兄姐妹,我们是否曾经历过不公?看到邪恶之人得势,恶人似乎毫无惧怕地行不义,我们的心是否感到愤怒和无助?我们是否也曾质疑:神真的掌权吗?公义真的会彰显吗?

诗篇 58 篇正是大卫对不公义社会的回应。他看到掌权者不按真理施行审判,反而作恶,因此,他向神呼求,求神按公义施行审判,并相信最终义人必得赏赐,恶人终必灭亡。

今天,我们要透过这篇诗篇,学习如何在面对不公时,依靠神、坚持公义,并充满盼望。

\subsection*{一、世界充满不公,但神鉴察一切(1-5节)}
“世人哪,你们默然不语,真按公义施行审判吗?世人哪,你们按正直判断人子吗?”(诗 58:1)

\subsubsection*{1. 世上的不公是真实的}
\hspace{0.6cm}大卫看到审判官不按公义判断,反而心怀恶念(2节)。他们不但自己行恶,还误导百姓,像毒蛇一样传播邪恶(4-5节)。

现实应用:在今天的社会,我们也看到许多不公义的现象,比如腐败、欺骗、剥削穷人。作为基督徒,我们不能麻木不仁,而要像大卫一样,为公义呼喊。
\subsubsection*{2. 神看见一切,不会无动于衷}
\hspace{0.6cm}\textbf{神是公义的神,他不会放任罪恶横行。}即使恶人看似得势,神仍在掌权。

现实应用:当我们面对不公时,不要灰心,要把问题交托给神,他必按公义行事。
\subsection*{二、恶人虽猖狂,但神必施行审判(6-9节)}
“神啊,求你敲碎他们口中的牙!”(诗 58:6)

\subsubsection*{1. 神的审判是真实的}
\hspace{0.6cm}大卫求神让恶人像水一般流失,像折断的箭无法伤人(7节)。他深信,恶人的势力不会长久,他们终将被毁灭。

现实应用:当我们看到邪恶势力猖狂时,不要害怕,因为神的审判必定来临。
\subsubsection*{2. 罪恶的结局是败亡}
“恶人未见火荆热,必被狂暴的旋风刮去。”(诗 58:9)

恶人的毁灭是迅速的,就像锅下的柴火未燃烧完,人已经被风刮走。

现实应用:我们不需要自己报仇,神自有公义的审判。
\subsection*{三、义人要存盼望,神的公义终将显明(10-11节)}
“义人见仇敌遭报就欢喜,要在恶人的血中洗脚。”(诗 58:10)

\subsubsection*{1. 义人最终必得胜}
\hspace{0.6cm}这里的“洗脚”是象征胜利,并非鼓励暴力,而是指神的公义得到了伸张,义人终于得到了安慰。

现实应用:在面对黑暗时,我们要坚定盼望,相信神最终会伸张公义。
\subsubsection*{2. 世人终将认识神的公义}
“必有人说:义人诚然有善报,在地上果有施行判断的神。”(诗 58:11)

神最终会显明他的公义,让世人知道他掌权。

现实应用:当我们行走在这不公义的世界,不要气馁,要相信神的公义最终会完全显明。
\subsection*{结论:面对不公,我们要信靠神的公义}
诗篇 58 篇提醒我们:
\begin{enumerate}
    \item 世界充满不公,但神鉴察一切——我们要勇敢发声,不向罪恶妥协。

    \item 恶人虽猖狂,但神必施行审判——神的公义不会缺席,我们不需自己伸冤。

    \item 义人要存盼望,神的公义终将显明——最终,义人必得赏赐,公义必得胜。

\end{enumerate}
\subsubsection*{挑战问题:}
\begin{itemize}
    \item 当你看到不公义的事情,你是选择沉默,还是勇敢为真理发声?

    \item 你是否相信神的审判是公义的,而不试图自己伸冤?

    \item 你是否愿意坚持走在公义的道路上,即使看起来暂时受苦?

    
\end{itemize}

愿我们都能持守信仰,等候神的公义显明!

\subsection*{结束祷告}
\textbf{亲爱的天父,}

我们感谢你,因你是公义的神。你鉴察世上的一切,你必不容罪恶长久得势。主啊,当我们面对不公、看到恶人猖狂时,求你赐给我们信心,让我们相信你的掌权。帮助我们在等待公义显明时,不灰心丧志,而是继续行走在正义的道路上。主啊,我们把一切的委屈和不平交托给你,愿你按公义施行审判,让世人都知道你是公义的神!

奉主耶稣基督的名祷告,阿们!
%-----------------------------------------------------------------------------
\newpage
\section{诗篇第59篇:在困境中仰望神的拯救}
% 讲章: —— 诗篇 59 篇
\subsection*{引言:当敌人围困时,我们如何回应?}
\hspace{0.6cm}弟兄姐妹,你是否曾经历过被人误解、攻击,甚至被恶意对待的时刻?当你发现周围充满敌意,甚至看似无路可逃时,你会选择用自己的方式反击,还是信靠神的拯救?

诗篇 59 篇 是大卫在被扫罗王派人追杀时写的。他本是忠心事奉神的人,却被无辜地追逼,面临极大的困境。然而,在这样的危机中,大卫没有绝望,而是向神祷告,寻求拯救,并最终坚信神的保护和公义。

今天,我们要从诗篇 59 篇中学习如何在困境中依靠神,经历他的拯救,最终得胜。

\subsection*{一、向神呼求拯救,不要靠自己伸冤(1-5节)}
“我的神啊,求你救我脱离仇敌,把我安置在高处,得脱那起来攻击我的人。”(诗 59:1)

\subsubsection*{1. 遇到困境,第一步是转向神}
\hspace{0.6cm}大卫并没有选择依靠自己的聪明和能力,而是首先求告神。

现实应用:当我们遭遇不公、被误解、被攻击时,我们的第一反应是什么?是愤怒、报复,还是向神祷告?
\subsubsection*{2. 神是我们的避难所}
\hspace{0.6cm}大卫知道,真正的安全不在于自己的力量,而在于神的保护。

现实应用:当我们在职场、学校或家庭中遭遇挑战时,我们是否愿意把自己交托给神,相信他的掌权?
\subsection*{二、信靠神的掌权,不惧怕仇敌的攻击(6-15节)}
“他们晚上转回,叫号如狗,围城绕行。”(诗 59:6)

\subsubsection*{1. 恶人虽狂妄,但神必掌权}
\hspace{0.6cm}大卫描述敌人如狗般围攻,昼夜不休,但他深知神的权柄高过一切。

现实应用:我们生活中可能遇到各种形式的“敌人”——无论是困难的环境,还是人的攻击,我们要相信神的掌权,而非被恐惧支配。
\subsubsection*{2. 依靠神,胜过恐惧}
“我要仰望你,耶和华是我的高台。”(诗 59:9)

现实应用:当我们面对考试压力、事业挫折或人际冲突时,我们要学习像大卫一样,把焦点放在神身上,而不是困境上。
\subsection*{三、因神的慈爱得胜,并歌颂他的拯救(16-17节)}
“但我要歌颂你的力量,早晨要高唱你的慈爱。”(诗 59:16)

\subsubsection*{1. 无论环境如何,都要敬拜神}
\hspace{0.6cm}大卫的敌人仍然存在,但他已经开始歌颂神的得胜。

现实应用:在困难中,我们是否愿意用敬拜来回应,而不是埋怨和恐惧?
\subsubsection*{2. 神的慈爱是我们最终的胜利}
“因为你作过我的高台,在我急难的日子作过我的避难所。”(诗 59:16)

现实应用:在人生风暴中,我们若紧紧抓住神,他的慈爱必定带领我们走向胜利。
\subsection*{结论:在困境中坚定信靠神}
诗篇 59 篇提醒我们:
\begin{enumerate}
    \item 向神呼求拯救,不要靠自己伸冤——信靠神,而不是自己的方法。
    \item 信靠神的掌权,不惧怕仇敌的攻击——相信神的权柄高过一切环境。

    \item 因神的慈爱得胜,并歌颂他的拯救——无论处境如何,都要以敬拜回应。

\end{enumerate}
\subsubsection*{挑战问题:}
\begin{itemize}
    \item 当你遭遇不公时,你是选择自己伸冤,还是交托给神?

    \item 你是否相信神会掌权,即使看起来环境不利?

    \item 你是否愿意在困境中敬拜神,而不是陷入抱怨和恐惧?

    
\end{itemize}

愿我们都能在困境中信靠神,经历他的拯救!

\subsection*{结束祷告}
\textbf{亲爱的天父,}

我们感谢你,因为你是我们的避难所,在困境中你是我们的高台。主啊,当我们面对挑战、压力和人的攻击时,求你帮助我们不靠自己的方式伸冤,而是完全交托给你。主啊,求你使我们心中充满信心,让我们知道你的慈爱必定带领我们得胜。无论环境如何,愿我们都能歌颂你的名,相信你的拯救。

奉主耶稣基督的名祷告,阿们!
%-----------------------------------------------------------------------------
\newpage
\section{诗篇第60篇:在困境中寻求神的得胜}
% 讲章: —— 诗篇 60 篇
\subsection*{引言:当失败降临,我们当如何回应?}
\hspace{0.6cm}弟兄姐妹,你是否曾经历过失败、挫折,甚至觉得神好像离你而去?无论是在事业、学业,还是人际关系中,我们都会遇到低谷。在这些时刻,我们如何面对?是自怨自艾、埋怨神,还是转向神,寻求他的帮助?

诗篇 60 篇 是大卫在战争失败后写的。他原本在征战中连连得胜,但这次却遭遇惨败,甚至觉得神已经离弃他们。然而,在这诗篇中,大卫没有沉溺于绝望,而是带着信心呼求神,最终重新得着力量。今天,我们要学习如何在失败和困境中转向神,并在他的带领下再次得胜。

\subsection*{一、承认自己的困境,向神呼求帮助(1-5节)}
“神啊,你丢弃了我们,使我们破败,你向我们发怒,求你使我们复兴!”(诗 60:1)

\subsubsection*{1. 失败不是终点,而是反思的机会}
\hspace{0.6cm}大卫并没有否认失败,而是坦然承认问题,并求神复兴他们。

现实应用:当我们遇到人生的低谷,比如考试失利、事业受挫、人际关系破裂,我们需要承认问题,并向神求问:“主啊,我需要你的帮助!”
\subsubsection*{2. 失败可能是神的提醒,要我们回到他的旨意}
\hspace{0.6cm}大卫意识到,他们的失败可能是因为神的愤怒,表示他们偏离了神的道路。

现实应用:当事情不顺利时,我们是否愿意安静下来,寻求神的心意,看看自己是否偏离了神的带领?
\subsubsection*{3. 依靠神,而非自己}
“你把困难的事指示给敬畏你的人,叫他们因真理得以逃脱。”(诗 60:4)

现实应用:失败时,不要只靠自己的方法解决问题,而是求神用他的真理引导我们找到出路。
\subsection*{二、坚信神的应许,不被环境动摇(6-8节)}
“神已经指着他的圣洁说:‘我要欢乐,我要分开示剑,丈量疏割谷。’”(诗 60:6)

\subsubsection*{1. 神的话语是我们得胜的关键}
\hspace{0.6cm}大卫在困境中,选择抓住神的应许,相信神依然掌权。

现实应用:当环境看似黑暗时,我们是否愿意抓住神的话,而不是被现实击垮?
\subsubsection*{2. 神是万国之主,他掌管一切}
“基列是我的,玛拿西是我的,以法莲是护卫我的头,犹大是我的杖。”(诗 60:7)

这节经文强调神对列国的主权,提醒我们:世界的一切仍然在神的掌控之中。

现实应用:在面对国家、社会动荡,或个人挑战时,我们可以相信神仍然掌权。
\subsubsection*{3. 在不确定的环境中,信靠神的带领}
“摩押是我的沐浴盆,我要向以东抛鞋,非利士啊,你们当因我欢呼。”(诗 60:8)

这节经文象征神对列国的审判,也代表神最终会战胜敌人。

现实应用:即使现在我们还没有看到胜利,我们要相信神最终会为我们成就他的旨意。
\subsection*{三、依靠神的能力,走向最终得胜(9-12节)}
“神啊,你不是丢弃了我们吗?你不和我们的军兵同去吗?”(诗 60:10)

\subsubsection*{1. 失败后,不是靠自己,而是靠神重新得胜}
\hspace{0.6cm}大卫知道,以色列不能靠自己的军力取胜,只有神与他们同在,他们才能得胜。

现实应用:当我们经历失败后,我们可以选择继续靠自己,或是谦卑下来,依靠神的智慧和能力。
\subsubsection*{2. 认识人的软弱,依靠神的帮助}
“求你帮助我们攻击敌人,因为人的帮助是枉然的。”(诗 60:11)

大卫明白,人的力量有限,但神的能力是无限的。

现实应用:无论是工作、学业还是人生的挑战,我们要承认自己的有限,把希望放在神身上,而不是单靠人的方法。
\subsubsection*{3. 最终的胜利属于神}
“我们倚靠神,才得施展大能,因为践踏我们敌人的就是他。”(诗 60:12)

这节经文是整篇诗的结论:得胜的关键不在于我们,而在于神。

现实应用:当我们把生命交托给神时,即使暂时失败,最终我们会在神的带领下得胜。
\subsection*{结论:从失败到得胜,关键在于信靠神}
诗篇 60 篇提醒我们:
\begin{enumerate}
    \item 承认自己的困境,向神呼求帮助——不要否认失败,而要在失败中寻求神。
    \item 坚信神的应许,不被环境动摇——无论环境如何,相信神仍然掌权。

    \item 依靠神的能力,走向最终得胜——不是靠自己,而是靠神才能真正得胜。

\end{enumerate}
\subsubsection*{挑战问题:}
\begin{itemize}
    \item 你是否在失败时转向神,而不是靠自己?

    \item 你是否在困难中相信神的应许,而不是被环境左右?

    \item 你是否愿意把人生的每一个挑战交给神,让他带领你最终得胜?

\end{itemize}

愿我们都能在困境中依靠神,经历他的大能和得胜!

\subsection*{结束祷告}
\textbf{亲爱的天父,}

我们感谢你,因为你是我们的拯救者。在失败和困境中,你仍然掌权。主啊,求你帮助我们在挫折中转向你,而不是倚靠自己的聪明和能力。求你坚固我们的信心,让我们紧紧抓住你的应许,不被环境动摇。主啊,我们相信,你是带领我们得胜的神,愿你引导我们走向最终的胜利。

奉主耶稣基督的名祷告,阿们!
%-----------------------------------------------------------------------------
\newpage
\section{诗篇第61篇:从绝望到坚固的磐石}
% 讲章: —— 诗篇 61 篇
\subsection*{引言:当心灵疲惫时,我们的避难所在哪里?}
\hspace{0.6cm}弟兄姐妹,人生的旅途上,我们都会经历疲惫、迷茫和挑战。有时候,我们会感觉自己站在生命的低谷,看不到希望;有时候,我们的心被重担压得喘不过气;有时候,我们甚至觉得神好像离我们很远。
在这样的时刻,我们应该何去何从?

诗篇 61 篇 是大卫在困境中所写的诗。当他感到疲惫、心力交瘁时,他没有沉溺于绝望,而是转向神,求神带领他进入更高之地,站立在坚固的磐石上。今天,我们要学习如何在生命的风暴中,像大卫一样转向神,找到真正的安全感和盼望。

\subsection*{一、向神倾诉心中的重担,寻求他的带领(1-2节)}
“神啊,求你听我的呼求,侧耳听我的祷告!我心里发昏的时候,要从地极求告你,求你领我到那比我更高的磐石。”(诗 61:1-2)

\subsubsection*{1. 人的软弱是神工作的起点}
\hspace{0.6cm}大卫在“心里发昏”的时候呼求神,这说明他已经筋疲力尽,甚至可能感到孤单无助。

现实应用:当我们感到迷茫、失落时,我们的第一反应是什么?是埋怨、逃避,还是像大卫一样,立刻向神祷告?
\subsubsection*{2. 祷告是通向磐石的桥梁}
\hspace{0.6cm}大卫没有选择靠自己解决问题,而是求神\textbf{“领我到那比我更高的磐石。”}

现实应用:在危机中,我们需要的是更高的视角——站在神的磐石上,才能看清问题的本质,并找到真正的出路。
\subsubsection*{挑战自己:}
你是否习惯在困难中第一时间转向神?
你是否愿意相信神能带领你到更高之地,而不是自己挣扎?
\subsection*{二、在神的同在中寻得安息和保护(3-5节)}
“因为你作过我的避难所,作过我的坚固台,好使我脱离仇敌。我愿永远住在你的帐幕里,投靠在你翅膀下的隐密处。”(诗 61:3-4)

\subsubsection*{1. 过去的经历是信心的根基}
“因为你作过我的避难所。”

大卫回忆起神过去的保守,因此在当前的困境中,他仍然相信神不会离弃他。

现实应用:当我们感到害怕、无助时,我们可以回顾神过去的恩典,从而坚固我们的信心。
\subsubsection*{2. 神是我们永恒的避难所}
\hspace{0.6cm}大卫渴望“永远住在神的帐幕里”,这象征着他对神的同在充满信心和渴慕。

现实应用:今天,我们的避难所是什么?是金钱、权力、人的安慰,还是神的同在?只有在神里面,我们的心灵才能真正得到安息。
\subsubsection*{3. 在神的翅膀下得安全感}
\hspace{0.6cm}“投靠在你翅膀下的隐密处” 这是充满安慰的画面,像母鸟保护小鸟一样,神用他的爱包围我们。

现实应用:当世界让我们害怕、焦虑时,我们要学会回到神的怀抱,依靠他,而不是靠自己的努力去掌控一切。
\subsubsection*{挑战自己:}
你是否愿意让神成为你唯一的避难所?
你是否在焦虑时,愿意回顾神过去的恩典,并选择信靠他?
\subsection*{三、持续信靠神,持守信仰直到永远(6-8节)}
“你要加添王的寿数,他的年岁必存到世世代代。他必永远坐在神面前,愿你预备慈爱和诚实保佑他!这样,我要歌颂你的名,直到永远,每日还我所许的愿。”(诗 61:6-8)

\subsubsection*{1. 持续信靠神,不只是短暂的祷告}

\hspace{0.6cm}大卫不仅在危机中寻求神,更希望\textbf{“每日还所许的愿”},这代表他愿意一直忠心跟随神,而不是只在困难时才亲近神。

现实应用:我们是否只在困难时呼求神,而在顺境时就远离他?真正的信仰是每天都与神同行,不论环境如何。
\subsubsection*{2. 让神的慈爱和诚实成为我们的保障}
\hspace{0.6cm}大卫祈求神用\textbf{“慈爱和诚实”}来保佑他,这代表神的信实是我们生命的依靠。

现实应用:世界上的一切都会改变,但神的爱永不改变,我们可以完全信靠他。
\subsubsection*{3. 以敬拜和顺服回应神}
\hspace{0.6cm}大卫最终的回应是\textbf{“我要歌颂你的名,直到永远。”}

现实应用:当我们经历神的拯救后,我们的回应应该是持续地敬拜、感恩,并活出信仰。
\subsubsection*{挑战自己:}
你是否愿意每天亲近神,而不仅仅是在困难时才寻求他?
你是否愿意用实际行动(祷告、敬拜、顺服)来回应神的恩典?
\subsection*{结论:在神的磐石上得胜}
诗篇 61 篇提醒我们:
\begin{enumerate}
    \item 向神倾诉心中的重担,寻求他的带领——当我们感到软弱时,不要逃避,而要向神祷告。

    \item 在神的同在中寻得安息和保护——真正的平安不是来自环境,而是来自神的翅膀之下。

    \item 持续信靠神,持守信仰直到永远——不论顺境或逆境,我们都要忠心跟随神,并用生命回应他的恩典。

\end{enumerate}

愿我们都能在生命的风暴中,站立在神的磐石上,经历他的得胜!

\subsection*{结束祷告}
\textbf{亲爱的天父,}
我们感谢你,因为你是我们坚固的磐石。在我们心里疲惫、迷茫时,你依然是我们的避难所。主啊,求你帮助我们,让我们学会第一时间向你祷告,而不是靠自己的力量挣扎。愿你的慈爱和信实保守我们,让我们一生敬拜你、跟随你。
奉主耶稣基督的名祷告,阿们!
%-----------------------------------------------------------------------------
\newpage
\section{诗篇第62篇:在动荡中信靠神}
\subsection*{引言}
我们生活在一个充满变数的世界里,无论是学业、工作,还是人际关系,都可能遭遇挫折和挑战。在面对失败、不公、压力或恐惧时,我们如何保持内心的平静?诗篇第62篇向我们展示了一个深刻的信仰真理:唯有在神里面,我们才能找到真正的安全感和依靠。
今天,我们将深入剖析这篇诗篇,结合实际生活,帮助我们在动荡的世界中坚固信心,学会完全信靠神。
\subsection*{一、单单依靠神,而非世俗的支撑(诗62:1-2)}
「我的心默默无声,专等候神;我的救恩是从他而来。惟独他是我的磐石,我的拯救;他是我的高台,我必不大大动摇。」(诗62:1-2)
\subsubsection*{1. 现实中的“支撑物”并不可靠}
我们往往会把安全感寄托在金钱、地位、人际关系、学业或能力上。但现实是,这些东西都可能在一夜之间崩塌。例如:
一次考试失利可能让我们怀疑自己的能力;
一段亲密关系的破裂可能让我们感到孤独;
经济的动荡可能让我们失去积蓄;
工作或事业的失败可能让我们感到前途渺茫。
这些外在的支撑都不可靠,唯有神是坚固的磐石。
\subsubsection*{2. 在神里面,我们得着真正的稳固}
大卫在诗篇62篇中强调:“惟独神是我的磐石,我的拯救,我必不大大动摇。”这里的“磐石”象征坚固、稳定、不动摇。真正的平安不是来自于外在环境的顺利,而是来自于内心对神的信靠。
\subsubsection*{应用:}
当面临挑战时,学会安静等候神,而不是慌乱地寻找人的帮助;
养成每日灵修、祷告的习惯,使我们的心在神的话语中得安慰;
遇到挫折时,不要急于找外部的答案,而要问自己:“我是否真正在依靠神?”
\subsection*{二、世人的诡诈与神的信实(诗62:3-8)}
「你们大家攻击一人,要把他杀害,如同歪斜的墙、将倒的壁一样;他们彼此商议,专要从他的尊位上把他推下。他们喜爱谎话,口虽祝福,心却诅咒。」(诗62:3-4)
\subsubsection*{1. 世界充满欺骗与不公}
大卫在这段经文中描述了一种普遍的现实:人会背叛、谎言充斥,嫉妒和权谋盛行。在今天的社会中,我们可能会遇到类似的情境:
职场中的尔虞我诈,有人为了利益而出卖朋友;
社交媒体上的虚假形象,人们口中称赞,心里却充满嫉妒和批评;
人际关系中的虚伪,表面上是朋友,背后却可能被算计。
\subsubsection*{2. 但神是信实可靠的}
面对人心的诡诈,大卫选择了什么?他没有倚靠人的帮助,而是宣告:
「我的心哪,你当默默无声,专等候神,因为我的盼望是从他而来。」(诗62:5)
大卫提醒自己,不要让人的恶行动摇自己的信心。人的话语可能会伤害我们,但神的话语永远是真实可靠的。
\subsubsection*{应用:}
当别人误解、攻击你时,学会交托给神,而不是急于自我辩护;
在社交关系中,做一个真实、正直的人,而不是随波逐流;
不要倚靠人的认可,而是寻求神的认可,因为人的评价会变,神的爱却永不改变。
\subsection*{三、真正的财富与能力来自神(诗62:9-12)}
「下流人真是虚空,上流人也是虚假;放在天平里,就必浮起;他们一共比空气还轻。不要仗势欺人,也不要因抢夺而骄傲;若财宝加增,不要放在心上。」(诗62:9-10)
\subsubsection*{1. 世俗的财富和权力不值得依靠}
大卫指出:无论是贫穷人还是富贵人,在神面前都不过是空气。 这提醒我们,财富、权力和地位都是短暂的,不能成为我们真正的安全感来源。
现实生活中,许多人追求金钱和名利,但最终却发现内心仍然空虚。许多事业成功的人,仍然感到孤独和焦虑,甚至抑郁。
\subsubsection*{2. 真正的力量来自神}
诗篇62:11-12说:「能力都属乎神,主啊,慈爱也是属乎你!」 这意味着:
一切成就最终都来自神,我们不能因自己的聪明才智而骄傲;
神掌管公义,他必按着各人的行为施行报应,所以我们要行在正道中,而不是急于求成、使用不义的手段。
\subsubsection*{应用:}
不要把生命的意义建立在财富、名声或成就上,而是建立在与神的关系上;
在追求目标的过程中,始终保持正直,信靠神的供应;
遇到不公时,相信神的公义,相信他的时间,不要因眼前的不公平而失去盼望。
\subsection*{结论:在不安的世界中,我们的安稳之道}
诗篇62篇给我们提供了一个在动荡世界中保持信心的秘诀:\textbf{单单依靠神,而非世俗的支撑}。
\begin{enumerate}
    \item 我们的安全感不是来自外在环境,而是来自对神的信靠;

    \item 世界可能充满不公和欺骗,但神永远信实可靠;

    \item 财富和权力不是终极目标,真正的能力和价值来自神。

\end{enumerate}

让我们在生活的挑战中,学会如诗人所说:“我的心哪,你当默默无声,专等候神。”愿神的话语成为我们人生的根基,使我们在风浪中依然稳固,在迷茫时依然有方向。
\subsection*{结束祷告}
\textbf{亲爱的天父,}

我们感谢你,让我们在这动荡的世界中找到真正的依靠。帮助我们在挑战中依然信靠你,不依赖世俗的力量,而是专心寻求你的引导。愿你的话语成为我们的力量,使我们在任何环境下都能刚强壮胆。奉

主耶稣基督的名祷告,阿们!
%-----------------------------------------------------------------------------
\newpage
\section{诗篇第63篇:在旷野中渴慕神}
% 讲章:——诗篇 63 篇
\subsection*{引言:当人生进入旷野,我们的渴望是什么?}
\hspace{0.6cm}弟兄姐妹,人生的道路上,我们都会经历“旷野时刻”——可能是孤独、挑战、失落,或者灵性的干旱。你是否曾经历过这样的时刻?在这些时刻,你的心渴望什么?是寻求世界的安慰,还是像大卫一样,渴慕神?

诗篇 63 篇是大卫在旷野中写的。他当时可能正在逃避仇敌,身处荒凉之地,物质匮乏,生命受到威胁。但在这样的环境下,他的第一反应不是抱怨、忧愁或恐惧,而是全心渴慕神。今天,我们要从大卫的诗篇学习如何在生命的旷野中寻求神,并经历他的满足。

\subsection*{一、在旷野中寻求神(1-2节)}
“神啊,你是我的神,我要切切地寻求你!在干旱疲乏无水之地,我的心渴想你,我的肉体切慕你。我在圣所中曾如此瞻仰你,为要见你的能力和你的荣耀。”(诗 63:1-2)

\subsubsection*{1. 旷野是考验人心的地方}
\hspace{0.6cm}大卫当时可能正被扫罗或押沙龙追杀,流亡在犹大的旷野。旷野是干旱、贫瘠、没有依靠的地方,但正是在这里,大卫对神的渴慕最为强烈。

现实应用:我们也会经历“旷野”——可能是事业的低谷、家庭的挑战、信仰的挣扎。在这些时刻,我们的心真正渴慕的是什么?是世界的帮助,还是神自己?
\subsubsection*{2. 真正的满足来自神,而非环境}
“我的心渴想你,我的肉体切慕你。” 大卫没有求水、求食物,而是首先渴望神的同在。

现实应用:在困境中,我们往往优先考虑如何解决问题,而不是如何亲近神。但真正能支撑我们的,不是物质,而是神的同在。
\subsubsection*{挑战自己:}
在困境中,你首先寻求的是什么?
你是否能在缺乏中仍然渴慕神,而不是埋怨环境?
\subsection*{二、在神里面得满足(3-8节)}
“因你的慈爱比生命更好,我的嘴唇要颂赞你。我还活的时候要这样称颂你,我要奉你的名举手。我在床上记念你,在夜更的时候思想你。因为你曾帮助我,我就在你翅膀的荫下欢呼。”(诗 63:3-7)

\subsubsection*{1. 认识神的慈爱,就能得着满足}
“你的慈爱比生命更好。” 这意味着即使失去一切,只要有神的爱,就已经足够了。

现实应用:我们是否真的相信神的爱比财富、成功、健康更重要?还是只有当一切顺利时,我们才觉得神的爱美好?
\subsubsection*{2. 敬拜神使我们的心得满足}
\hspace{0.6cm}大卫虽然身处困境,但他仍然用嘴唇赞美神、举手祷告。这表明他并不是在等待环境改变,而是在信心中敬拜。

现实应用:你是否能在困境中仍然敬拜神,而不是等问题解决后才感谢神?
\subsubsection*{3. 在神的翅膀下得安稳}
“我就在你翅膀的荫下欢呼。” 这是一个充满安全感的画面,神就像母鸟一样保护我们。

现实应用:你是否愿意放下焦虑,相信神的保护,而不是靠自己的努力去掌控一切?
\subsubsection*{挑战自己:}
你是否愿意在困境中仍然赞美神?
你是否愿意相信,神的爱比任何世界上的东西都更宝贵?
\subsection*{三、在神的信实中得胜(9-11节)}
“但那些寻索要灭我命的人必往地底下去。他们必被刀剑所杀,被野狗所吃。但王必因神欢喜;凡指着他发誓的,必要夸口,因为说谎之人的口必被塞住。”(诗 63:9-11)

\subsubsection*{1. 神的信实带来最终的得胜}
“但那些寻索要灭我命的人必往地底下去。” 这不是大卫自己的能力,而是他相信神会为他伸冤。

现实应用:当我们面对不公和逼迫时,我们是否愿意把公义交托给神,而不是靠自己的方式去报复?
\subsubsection*{2. 义人最终会因神喜乐}
“但王必因神欢喜。” 大卫相信,最终喜乐的不是恶人,而是那些依靠神的人。

现实应用:你是否愿意相信,神的时间和方式是最好的,哪怕短时间内你看不到公义的伸张?
\subsubsection*{3. 说谎之人的口必被塞住}
“说谎之人的口必被塞住。” 这意味着恶人的诡计最终会被神所制止。

现实应用:我们是否愿意相信,神比我们更了解真相?我们是否愿意在面对谎言和误解时,安静等候神的公义?
\subsubsection*{挑战自己:}
你是否愿意把你的冤屈交托给神,而不是自己伸张正义?
你是否愿意相信神的信实,而不是被眼前的困境动摇?
\subsection*{结论:旷野并不可怕,失去神的渴慕才可怕}
诗篇 63 篇提醒我们:
\begin{enumerate}
    \item 在旷野中,我们要寻求神,而不是被环境影响。

    \item 在神里面,我们才能得真正的满足,而不是依赖世界的供应。

    \item 神的信实终将带来得胜,我们要把公义交托给他。

\end{enumerate}

愿我们都能像大卫一样,在生命的旷野中仍然渴慕神、敬拜神,并最终经历他的得胜!

\subsection*{结束祷告}
\textbf{亲爱的天父,}

我们感谢你。你是我们生命的满足,你的慈爱比生命更好。在困境中,求你帮助我们渴慕你,而不是渴望世界的安慰。主啊,求你加添我们的信心,让我们相信你掌管一切,并在你的翅膀下得安息。

感谢你,奉主耶稣基督的名祷告,阿们!
%-----------------------------------------------------------------------------
\newpage
\section{诗篇第64篇:在暗箭中依靠神}
% 讲章:——诗篇 64 篇
\subsection*{引言:当暗箭射向你时,你依靠谁?}
\hspace{0.6cm}弟兄姐妹,我们都曾经历过别人的误解、攻击,甚至诋毁。有时候,这些攻击不是公开的,而是暗地里的,就像隐藏在黑暗中的箭,让人防不胜防。这可能是流言蜚语、职场上的陷害、朋友的背叛,甚至是网络上的恶意评论。当这些事情发生时,我们会怎么做?反击?愤怒?还是选择信靠神?

诗篇 64 篇就是大卫在面对暗中的攻击时所写的祷告。他没有求神让他立即得胜,而是呼求神的保护,坚定相信神的公义。今天,我们要从这篇诗篇学习如何在遭受攻击时,依靠神、交托给神,并最终看到神的公义彰显。

\subsection*{一、恶人的暗箭(1-6节):攻击是无形的,但神看见}
“神啊,我哀叹的时候,求你听我的声音!求你保全我的性命,不受仇敌的惊恐!求你把我隐藏,使我脱离作恶之人的暗谋和作孽之人的扰乱。”(诗 64:1-2)

\subsubsection*{1. 祷告是我们的第一反应,而不是最后的选择}
\hspace{0.6cm}大卫没有先去反击,而是 “求你听我的声音”,先把事情交托给神。

现实应用:当我们被误解、攻击时,我们的第一反应是什么?是去争辩、愤怒,还是来到神面前?
\subsubsection*{2. 暗中的攻击往往让人最无助}
“他们磨舌如刀,发出苦毒的言语,好像比准了的箭。”(诗 64:3)

这些攻击可能是言语的伤害,比如:
\begin{itemize}
    \item 流言蜚语:职场中的恶意造谣

    \item 诋毁中伤:亲友间的误解和背叛

    \item 网络攻击:社交媒体上的恶意评论

\end{itemize}

这些“暗箭”无形无影,但却能深深刺痛我们的心。
\subsubsection*{3. 恶人自以为隐藏,但神洞察一切}
“他们彼此商议设下网罗,说:谁能看见呢?”(诗 64:5)

现实应用:世界上的诡计、人心的恶念,虽然能暂时得逞,但神的眼目从未离开。我们是否相信神是全知的?

\subsubsection*{挑战自己:}
面对流言和诋毁时,我们是依靠自己的方法,还是交托给神?
我们是否也曾无意中伤害了别人,成为“暗箭”的射手?
\subsection*{二、神的公义之箭(7-9节):最终的伸冤在神手中}
“神必射他们,他们忽然被箭射伤。”(诗 64:7)

\subsubsection*{1. 终极的公义掌握在神手中}
“神必射他们。” 这里的“箭”代表神的公义。当人恶意攻击时,神必出手。

现实应用:或许你现在正经历不公,但不要灰心,因为神的箭比世人的暗箭更精准、更有力。
\subsubsection*{2. 恶人自食其果}
“他们必然绊跌,被自己的舌头所害;凡看见他们的,必都摇头。”(诗 64:8)

现实应用:历史上多少恶人最终都被自己的计谋害了?想想那些陷害他人的人,最终却被揭露,他们的恶行成为众人的笑柄。
\subsubsection*{3. 所有的人都会敬畏神}
“众人都要害怕,要传扬神的工作,并且明白他的作为。”(诗 64:9)

神的公义彰显时,世人都会看到神的作为,我们的信仰就成为真实的见证。

现实应用:你愿意相信神的时间、神的方式吗?还是你总想自己动手去报复?
\subsubsection*{挑战自己:}
你是否愿意忍耐,相信神的公义最终会显明?
你是否相信,人的计谋永远无法胜过神的智慧?
\subsection*{三、义人的回应(10节):信靠神并欢喜}

\subsubsection*{1. 义人的喜乐源于神,而不是环境}
\hspace{0.6cm}义人的喜乐不是因为问题消失,而是因为他们信靠神。

现实应用:当我们经历误解、诋毁时,我们仍然可以喜乐,因为神掌权。
\subsubsection*{2. 投靠神,而不是自己伸冤}
“并要投靠他。” 这意味着放下自己的计谋,把公义交给神。

现实应用:在面对逼迫时,我们愿意投靠神,还是靠自己争夺公平?
\subsubsection*{3. 以神的公义为夸耀}
“凡心里正直的人都要夸口。” 最后,神的公义一定会显明,我们可以见证神的作为。

现实应用:当神为你伸冤时,你愿意谦卑地见证他,而不是沾沾自喜吗?
\subsubsection*{挑战自己:}
你愿意因神的信实而喜乐,而不是因环境改变才喜乐?
你愿意放下自己的报复,把公义交托神?
\subsection*{结论:当暗箭来袭时,我们选择信靠神}
诗篇 64 篇提醒我们:
\begin{enumerate}
    \item 暗中的攻击虽然伤人,但神全然知晓,我们要选择祷告交托。

    \item 神的公义一定会显明,我们要耐心等候,不要自己动手报复。

    \item 最终,义人会因神欢喜,我们要信靠他,而不是被环境左右。
\end{enumerate}

愿我们都能在面对误解、攻击和不公时,选择像大卫一样,把一切交托给神,因他的信实而喜乐!

\subsection*{结束祷告}
\textbf{亲爱的天父,}
我们感谢你,因为你是公义的神。主啊,我们承认,在面对攻击和误解时,我们常常想自己伸冤,但今天你的话语提醒我们,真正的公义在你手中。主啊,求你保守我们的心,让我们不因恶人的暗箭而动摇,而是坚定地信靠你,等候你的作为。愿你的名在我们的生命中得荣耀!
奉主耶稣基督的名祷告,阿们!
%-----------------------------------------------------------------------------
\newpage
\section{诗篇第65篇:丰盛的恩典}

\subsection*{引言}
\hspace{0.6cm}亲爱的弟兄姐妹,今天我们要一起思想诗篇第65篇,这是一首充满感恩、颂赞和对神伟大作为的默想诗。诗人用优美的语言描述了神如何听祷告、赦免罪孽、赐下丰收,并眷顾世界各地。这首诗提醒我们,神不仅是创造的主,也是供应的神,是恩典的源头。

在这个快节奏、充满焦虑和挑战的时代,我们常常被现实问题压得喘不过气来,可能是学业的压力、工作的挑战、家庭的困难,甚至是对未来的担忧。诗篇65篇向我们展示了一种不同的生命态度:信靠神,感恩神,见证神的祝福。让我们一同来学习,从这篇诗篇中汲取信仰的力量。
\subsection*{一、神是聆听祷告、赦免罪孽的神(诗篇 65:1-4)}
“众民都要来到你面前。罪孽胜了我,至于我们的过犯,你都要赦免。”(诗篇 65:2-3)
\subsubsection*{人生的重担与神的赦免}
\hspace{0.6cm}我们每个人都带着自己的软弱和罪来到神面前。生活中,我们可能会因失败而感到羞愧,因人际关系的破裂而受伤,甚至因罪恶感而无法释怀。然而,诗篇65篇告诉我们,神是听祷告的主,他愿意赦免我们的罪。

现实生活中,有些人长期活在自责和悔恨中,觉得自己过去犯下的错误无法挽回。但圣经告诉我们,只要我们真心悔改,神就会赦免我们,使我们得自由。比如,一个曾经误入歧途的人,在回转向神后,经历了心灵的释放,开始了全新的生命。

神不仅是赦免人的神,也是欢迎我们亲近他的神。诗篇65:4说:“你所拣选、使他亲近你、住在你院中的,这人便为有福。” 这提醒我们:敬拜神、与神同行的人,是真正蒙福的人。
\subsection*{二、神是掌管万物、赐人满足的神(诗篇 65:5-8)}
“他以大能束腰,用力量安定诸山,使海浪翻腾,叫地极的人都惧怕。”(诗篇 65:6-8)
\subsubsection*{神掌权带来的平安}
\hspace{0.6cm}诗篇65篇不仅告诉我们神是赦罪的神,也让我们看见他的全能。他创造天地,掌管风浪,治理世界。他的能力远超我们的理解,这意味着我们可以放心地把自己的未来交托给他。

现实生活中,许多人在面对困难时会陷入焦虑,比如大学生面临学业压力、职场人士面对裁员风险、家长为孩子的成长忧虑。但当我们意识到神是掌管万物的神,我们就可以放下自己的焦虑,相信神会为我们开道路。

记得有一个姐妹,曾经因为经济困难而焦虑不已,但她祷告后看到神奇妙的供应,工作机会从意想不到的地方来,她的信心也因此增长。
神不仅掌管自然界,也掌管人类的历史。他的作为让“地极的人都惧怕”(诗篇65:8),也让信靠他的人得平安。
\subsection*{三、神是供应丰盛、眷顾大地的神(诗篇 65:9-13)}
“你眷顾地,降下透雨,使地大得肥美……你以恩典为年岁的冠冕,你的路径都滴下脂油。”(诗篇 65:9,11)
\subsubsection*{神的供应超乎所求所想}
\hspace{0.6cm}诗篇65篇的后半部分用极其生动的画面描绘了神如何赐下雨水,使土地丰收。这不仅指物质上的祝福,也象征着属灵的祝福。神的恩典像春雨一样,滋润我们的生命,使我们在信仰中成长、结果子。

现实生活中,许多时候,我们会担心自己的未来,比如大学生面对就业问题,年轻人考虑婚姻问题,父母为孩子的前途焦虑。但诗篇65:11提醒我们,神“以恩典为年岁的冠冕”,他的供应从不会迟到。

有一位弟兄,毕业时因经济不佳而找不到合适的工作,他祷告后神带领他进入一条他原本没有想到的职业道路,最后不仅收入稳定,还能帮助他人。他的经历让他深刻体会到神的供应与信实。

神的恩典不是一次性的,而是持续的。他的祝福如同“肥美的草场、欢呼的山岭”(诗篇 65:12-13),使我们的生命充满喜乐。
\subsection*{四、我们的回应:信靠、感恩、敬拜}
诗篇65篇不仅是一首感恩诗,更是提醒我们应当如何回应神的恩典:
\begin{enumerate}
    \item 信靠神,不忧虑未来 —— 神掌管万物,我们可以放心交托。

    \item 感恩神,不单顾自己 —— 认识到我们的一切祝福都来自神,因此愿意帮助他人,与人分享神的恩典。

    \item 敬拜神,活出荣耀他的人生 —— 诗篇65篇从敬拜开始,也以敬拜结束,这提醒我们,真正的满足不是从世界而来,而是从与神的关系而来。

\end{enumerate}
\subsection*{结语}
亲爱的弟兄姐妹,诗篇65篇提醒我们,神是听祷告的神,是掌管万物的神,是赐人丰盛的神。无论我们正面临何种挑战,我们都可以来到神面前,向他祷告,经历他的赦免、平安和供应。让我们用感恩的心来回应,信靠他、敬拜他,并将他的爱传递给更多的人。
愿神赐福大家,阿们!
\subsection*{结束祷告}
\textbf{亲爱的天父,}
% 亲爱的天父,

我们满怀感恩地来到祢面前,感谢祢在诗篇65篇中显明的丰盛恩典和奇妙作为。正如祢创造天地、使万物生长、赐下甘露滋润大地,祢也在我们的生命中不断施展奇妙恩惠。感谢祢的慈爱、信实与怜悯,使我们在每个季节、每个时刻都能感受到祢温柔的看顾。

求祢帮助我们常存敬畏之心,领受祢所赐的平安和喜乐,借着信心走在祢预备的道路上。愿我们在生活的每个角落都能见证祢的美善,成为祢恩典的见证人,将祢的光与爱传扬出去。

奉主耶稣基督的圣名祷告,阿们。
%-----------------------------------------------------------------------------
\newpage
% \section{诗篇第66篇:}
\section{诗篇第66篇:从试炼到赞美——经历神的拯救与信实}
\subsection*{引言:苦难让你远离神,还是更亲近神?}
\hspace{0.6cm}每个人在生活中都会经历挑战和试炼。当困难来临时,你的第一反应是什么?
是埋怨神,觉得他不爱你?
还是信靠神,相信他在掌权?

《诗篇》第66篇是一首赞美诗,诗人回顾了以色列的苦难和拯救,见证了神的信实,并呼吁万人一同敬拜神。今天,我们要学习如何在试炼中经历神,并以感恩回应他!

\subsection*{一、赞美神的奇妙作为(1-7节)}
\hspace{0.6cm}诗篇66:1-2 “全地都当向神欢呼!歌颂他名的荣耀!用赞美的言语将他的荣耀发明!”

诗篇66:5 “你们来看神所行的,他向世人所做之事是可畏的。”
\subsubsection*{1. 赞美神的大能}
\hspace{0.6cm}诗篇一开始,诗人邀请全地来赞美神,因为神的作为奇妙可畏!
神创造世界,管理万有。
神掌管历史,拯救他的子民。

我们是否常常因着神的作为而敬畏、感恩和赞美?
当你仰望星空,你是否会惊叹神的创造?
当你经历神的带领,你是否愿意向他献上感谢?

即使在困难中,我们仍然可以选择赞美神,因为他的信实永不改变!
\subsubsection*{2. 记念神的拯救}
诗篇66:6 “他使海变成干地,众民步行过河;我们在那里因他欢喜。”

诗人回顾以色列人过红海和约旦河的神迹(出埃及记14:21-22,约书亚记3:17),这些事件象征着神的拯救和带领。
今天,你是否还记得神曾如何拯救你?
\begin{itemize}
    \item 他是否曾在你软弱时扶持你?

    \item 他是否曾在你迷失时引导你?

    \item 他是否曾在你绝望时赐你希望?

\end{itemize}

如果你曾经历神的恩典,不要忘记感恩和见证!

\subsection*{二、试炼中的炼净与成长(8-12节)}
\hspace{0.6cm}诗篇66:10 “神啊,你曾试验我们,熬炼我们,如熬炼银子一样。”

诗篇66:12 “你使人坐车轧我们的头,我们经过水火,你却使我们到丰富之地。”
\subsubsection*{1. 试炼的目的:熬炼如精金}

\hspace{0.6cm}诗人坦言,以色列人曾经历极大的困苦,甚至“被人轧过头”,但神的试炼不是为了毁灭,而是为了炼净!
\textbf{金子要经过火炼,才能去除杂质;
信仰要经过试炼,才能更加纯净。}

你是否正在经历试炼?
工作、学业上的压力?
经济的困难?
人际关系的挑战?
灵命的低谷?

请相信,神不会白白让你受苦,他正在塑造你的生命,使你更加坚强!
\subsubsection*{2. 试炼的结果:进入丰富之地}
\hspace{0.6cm}诗人见证,以色列人虽然经历了痛苦,但最终神带领他们进入“丰富之地”。


\begin{itemize}
    \item 约瑟被卖为奴隶,经历监牢,但最终成为埃及宰相(创世记50:20)。

    \item 约伯失去一切,但后来神赐他加倍的祝福(约伯记42:10)。

    \item 大卫经历逃亡和苦难,但最终成为以色列的君王(撒母耳记下5:4)。

\end{itemize}

神允许试炼,但他的终极计划是祝福和丰盛!
你是否愿意在困难中依靠神,相信他会带你进入“丰富之地”?

\subsection*{三、以感恩回应神(13-20节)}
诗篇66:13-14 “我要用燔祭进你的殿,向你还我的愿,就是在急难时我嘴唇所发的,口中所许的。”
\subsubsection*{1. 用行动回应神}
\hspace{0.6cm}诗人不只是用嘴巴感谢神,而是带着燔祭进入圣殿,还清他的愿。
这提醒我们:当神拯救我们时,我们不应该只是口头感谢,而是要用行动回应!

你可以如何回应神?
\begin{itemize}
    \item 献上生命——将自己完全交托给神,不再活在罪中。

    \item 服侍神——用你的时间、恩赐、金钱来荣耀神。

    \item 见证神——向身边的人分享神在你生命中的作为。

\end{itemize}

不要只是在困难时祷告,得救后却忘记神!让你的感恩成为实际的行动!
\subsubsection*{2. 见证神的信实}
诗篇66:16 “凡敬畏神的人,你们都来听!我要述说他为我所行的事。”

诗人不只自己感恩,他更愿意向众人见证神的恩典!

你有向别人分享过神的恩典吗?
你曾如何经历神的供应?
你曾如何在试炼中得着神的安慰?
你曾如何看到神的带领?

见证神的作为,不仅能鼓励别人,也能让你的信心更加坚定!
\subsubsection*{3. 祷告必得应允}
诗篇66:19-20 “然而神实在听见了,他侧耳听了我祷告的声音。神是应当称颂的!他并没有推却我的祷告,也没有叫他的慈爱离开我。”

诗人亲身经历:神听祷告,他的慈爱永不改变!

你是否相信神会听你的祷告?
你是否愿意耐心等候神的时间?

即使祷告没有立刻得到答案,请相信神知道什么对你最好,他必定以他的方式成就!

\subsection*{结论:从试炼到赞美,经历神的信实!}
诗篇66篇告诉我们:

神的作为伟大,值得我们赞美!
试炼是熬炼,让我们更加信靠神!
神的带领最终会使我们进入“丰富之地”!
我们的回应是感恩、见证、和委身神的道路!

今天,你愿意在困境中仍然信靠神吗?
你愿意用生命回应神的恩典吗?
你愿意成为神祝福的见证人吗?

愿神赐福你,使你在试炼中成长,在祝福中成为别人的祝福!
阿们!
%-----------------------------------------------------------------------------
\newpage
% \section{诗篇第67篇:}
\section{诗篇第67篇:蒙福的生命,向列国见证神的荣耀}
\subsection*{引言:你如何理解“蒙福”?}
\hspace{0.6cm}在这个世界上,人们都渴望“祝福”,但对“祝福”的理解却大不相同。

有人认为祝福就是财富、事业成功、家庭美满。
也有人认为祝福就是健康、长寿、无灾无难。

但在《诗篇》第67篇中,诗人向我们展现了一幅更高的图画:神的祝福不仅仅是给个人的,更是为了使万国归向他!
今天,我们就来思想:“什么是真正的蒙福?为什么神要祝福我们?我们该如何回应?”

\subsection*{一、蒙福的源头:神的恩典与同在(1-2节)}
诗篇67:1-2 “愿神怜悯我们,赐福与我们,用脸光照我们,好叫世界得知你的道路,万国得知你的救恩。”

诗人一开篇就求神怜悯、赐福、光照,这让我们想到民数记6:24-26中祭司的祝福:

“愿耶和华赐福给你,保护你;愿耶和华使他的脸光照你,赐恩给你。”
\subsubsection*{1. 真正的祝福来自神,而不是世界}
\hspace{0.6cm}世界的“祝福”往往是短暂的,而神的祝福是永恒的。

财富可以失去,但神的恩典不会改变。
健康可能衰退,但神的同在是我们的安慰。
成就可能被遗忘,但神给我们的身份不会动摇。

你是否把祝福建立在短暂的事物上?还是把目光放在神的恩典上?
\subsubsection*{2. 神祝福我们,不仅仅是为了我们自己}
\hspace{0.6cm}诗篇67:2清楚地告诉我们:神的祝福不是让我们停留在自我满足,而是让“世界得知你的道路,万国得知你的救恩”。

你得到的恩典,是否让身边的人看见神?
你的成功,是否成为别人认识神的机会?

真正的蒙福,是让神的恩典成为我们生命的见证,让别人透过我们认识神。

\subsection*{二、万国的敬拜:神的荣耀临到全地(3-5节)}
\hspace{0.6cm}诗篇67:3 “神啊,愿列邦称赞你!愿万民都称赞你!”

诗篇67:4 “愿万国都快乐欢呼,因为你必按公正审判万民,引导世上的万国。”

诗人不仅渴望神祝福自己,更渴望全地都认识神,万国都来敬拜他!
\subsubsection*{1. 真正的喜乐来自神的公义}
\hspace{0.6cm}世界的快乐往往是短暂的,但神的公义带来真正的喜乐。今天的世界充满不公、战争、贫富差距、道德混乱。但诗篇告诉我们:神是公义的审判者,他必按公正审判万民。真正的公义不是靠人的努力,而是来自神的掌权。

你是否在面对世界的不公时,仍然相信神的公义?
你是否愿意在黑暗中成为光,让神的公义透过你彰显?
\subsubsection*{2. 你的生命是否带领别人归向神?}
“愿列邦称赞你!”(诗篇67:3)

当神祝福你,你的生命是否成为福音的桥梁,让别人因你而认识神?
你的言行是否荣耀神?
你的见证是否带领人归向神?

如果我们的生命只是享受神的恩典,而不去见证神的恩典,那就浪费了神给我们的祝福。

你愿意成为神祝福的管道,让你的家庭、朋友、甚至远方的人因你而认识耶稣吗?

\subsection*{三、丰收与感恩:以生命回应神的恩典(6-7节)}
诗篇67:6-7 “地已经出了土产;神,就是我们的神,要赐福与我们。神要赐福与我们,地的四极都要敬畏他。”

这里提到的“土产”,象征着神的供应和祝福,但这并不只是物质上的丰收,更是属灵上的丰收!
\subsubsection*{1. 你的生命是否结出果子?}
\hspace{0.6cm}在《新约》中,耶稣多次提到“结果子”的重要性:
约翰福音15:8 “你们多结果子,我父就因此得荣耀,你们也就是我的门徒了。”

你是否在生命中结出信心的果子?
你是否带领人认识神,结出福音的果子?

神的祝福不是让我们停留在享受恩典,而是让我们去影响这个世界!
\subsubsection*{2. 你的祝福是否带来敬畏神的心?}
“地的四极都要敬畏他。”(诗篇67:7)

敬畏神,不是惧怕,而是带着敬重与顺服。

你是否因着神的祝福,更加爱神?
你是否用感恩的心回应神的恩典?

\subsection*{结论:蒙福的生命,使命的呼召}
诗篇67篇告诉我们:真正的蒙福,是为了让世界认识神,让万国归向他!

1. 你如何看待祝福?
是停留在自我的满足,还是愿意成为祝福的管道?

2. 你的生命是否带领别人认识神?
你是否愿意用言行见证神的恩典?
你是否愿意为世界的灵魂祷告、奉献、行动?

3. 你是否愿意回应神的祝福?
让你的生命结出丰收的果子,成为荣耀神的器皿!

“神要赐福与我们,地的四极都要敬畏他!”(诗篇67:7)
愿我们都成为神祝福的器皿,使更多人因我们认识耶稣!
阿们!
\subsection*{结束祷告}
\textbf{亲爱的天父,}

感谢祢通过诗篇67篇向我们启示祢的恩典与慈爱,让万国在祢的光照中得以认识祢。我们感谢祢使大地丰收,赐下平安与喜乐,愿祢的救恩与真理在全世界彰显。求祢帮助我们成为祢恩典的传播者,让我们在生活中见证祢的公义与怜悯,将祢的爱带给每一个角落。

愿我们每个人的心中都燃起对祢无限的敬畏和感恩,使祢的名在万国之中被高举,祢的道路在地上显明。

奉主耶稣基督的圣名祷告,阿们。
%-----------------------------------------------------------------------------
\newpage
% \section{诗篇第68篇:}
\section{诗篇第68篇:在困境中依靠上帝的能力}
\subsection*{引言}
诗篇第68篇是一首充满力量的诗篇,它宣告了上帝的胜利、保护和供应。这首诗既回顾了以色列的历史,又预言了上帝对世界的掌权。今天,我们要从中学习,在现实生活的挑战中如何依靠上帝的能力,活出信仰的胜利。

\subsection*{一、上帝是困境中的拯救者(1-6节)}
“愿神兴起,使他的仇敌四散!”(诗篇68:1)

诗篇68篇一开头,就以极大的信心宣告神的得胜。作者以战斗的语言描述神的出现——当神兴起,仇敌就四散,恶人被吹散如烟,而义人因神的得胜而欢喜(2-3节)。
\subsubsection*{现实中的应用:}
在生活中,我们常常会遇到挑战,例如学业压力、人际关系问题、事业发展的困难,甚至疾病和家庭问题。这些困境就像仇敌围困我们,使我们软弱、无助。但是,这节经文提醒我们,神是我们的拯救者。当神介入我们的困境,问题不会再让我们绝望,而是成为我们信仰成长的机会。
\subsubsection*{见证或例子:}
有一位基督徒企业家,在创业初期面临巨大的财务困难,几乎破产。但他依靠神,不仅在祷告中寻求方向,还在实际行动中坚持正直经营。最终,他经历了神的供应,不仅企业存活下来,还成了行业的佼佼者。这正是神“兴起,使仇敌四散”的真实见证。
\subsection*{二、上帝是孤独者的父亲(5-10节)}
“神在他的圣所作孤儿的父,作寡妇的伸冤者。”(诗篇68:5)

诗篇68:5-6告诉我们,神特别关心孤独无助的人,他使孤独者有家,使被囚的得自由。
\subsubsection*{现实中的应用:}
在现代社会,人们常常感到孤独,即使身处人群,也可能因压力、焦虑、失败或过去的伤害而觉得无依无靠。许多大学生在初入大学时,远离家人,面对新环境,可能会感到孤独、迷茫。这节经文告诉我们,神不仅仅是创造者,更是爱我们的天父,愿意与我们建立亲密关系,成为我们的依靠。
\subsubsection*{实践方法:}
在孤独时,不要逃避,而是向神祷告,与他建立亲密的关系。
主动去关心身边同样感到孤独的人,参与教会团契、小组,彼此扶持。
\subsection*{三、上帝供应他子民的需要(11-19节)}
“你从丰富为困乏人预备的,神啊,这是你所预备的。”(诗篇68:10)

本段描述了神如何供应他的子民,以色列人出埃及时,神使旷野下雨(9节),并供应他们的食物和水。
\subsubsection*{现实中的应用:}
我们有时会忧虑自己的生活、前途,甚至信仰道路上的挑战。但这节经文提醒我们,神知道我们的需要,并且在合适的时间预备我们所需的一切。
\subsubsection*{实际例子:}
有些人在找工作时,经历了许多挫折,但后来回头看,发现神早已在预备更适合他们的道路。在人生的各个阶段,我们都要相信神的供应,即使暂时看不见,他仍在掌管一切。
\subsection*{四、上帝是得胜的君王(20-35节)}
“神是为我们施行诸般救恩的神。”(诗篇68:20)

诗篇的后半部分强调上帝的权柄,他不仅仅是以色列的神,更是全地的君王(32节)。
\subsubsection*{现实中的应用:}
当我们看到世界充满战争、不公、罪恶时,可能会质疑神的公义和能力。但这首诗提醒我们,神仍然在掌权,他的计划不会被世界的混乱所打乱。我们的信仰不应建立在环境的好坏上,而要建立在神永恒的应许上。
\subsubsection*{我们的回应:}
以祷告来信靠神的带领,而不是被恐惧左右。
在生活中做光做盐,活出神的公义和爱。
\subsection*{结论:依靠神,活出得胜人生}
诗篇68篇让我们看到,神是得胜的神,他能拯救、供应、保护和掌权。面对生活的挑战,我们可以放心地依靠他,因为:
神能在困境中拯救我们。
\begin{enumerate}
    \item 神是孤独者的依靠。

    \item 神会供应我们的需要。

    \item 神掌管一切,是最终的得胜者。

\end{enumerate}

愿我们都能在生活的每一天,将目光定睛在神身上,经历他的大能和恩典!
\subsection*{结束祷告}
\textbf{亲爱的天父,}

我们感谢你通过诗篇68篇向我们启示你的能力和爱。求你帮助我们在困境中信靠你,在孤独时投靠你,在缺乏时依赖你。愿我们的一生见证你的得胜!

奉耶稣基督的名,阿们!
%-----------------------------------------------------------------------------
\newpage
\section{诗篇第69篇:在困境中仰望神}
% 讲章:——诗篇 69 篇
\subsection*{引言:当你感到绝望时,你会向谁求助?}
\hspace{0.6cm}弟兄姐妹,你是否曾经历过被误解、被拒绝,甚至被人嘲讽?你是否有过呼求神却似乎没有立刻得到回应的时刻?当我们陷入极大的困境时,我们往往会感到孤立无援。

诗篇 69 篇是大卫在极度痛苦中写下的祷告诗。他经历了逼迫、羞辱、深深的绝望,但他依然选择仰望神,并且在苦难中信靠神的救恩。今天,我们要从这篇诗篇中学习,在苦难、误解和等待神的救恩时,我们该如何回应。

\subsection*{一、困境中的呼求(1-12节):当我们落入深水}
“神啊,求你救我!因为众水要淹没我。我陷在深淤泥中,没有立脚之地;我到了深水中,洪水漫过我身。”(诗 69:1-2)

\subsubsection*{1. 绝望的景象:水要淹没我}
大卫形容自己好像陷在深水里,无法站立,被洪水冲走。
现实应用:我们是否也曾感到无法自拔?或许是经济压力、家庭问题、学业的挑战,甚至是被误解的痛苦。
\subsubsection*{2. 祷告成为唯一的出路}
“我因呼求困乏,喉咙发干;我因等候神,眼睛失明。”(诗 69:3)
现实应用:我们是否曾经不断地向神呼求,但似乎没有立即得到答案?在这种时候,我们会继续等候,还是失去信心?
\subsubsection*{3. 人的逼迫和羞辱}
“无故恨我的,比我头发还多。”(诗 69:4)
“因我为你的殿心里焦急,如火烧着我;并且辱骂你人的辱骂,都落在我身上。”(诗 69:9)
现实应用:有时候,我们因为坚持信仰、追求正直,而遭遇别人的嘲笑和拒绝。你是否曾因信仰被误解或孤立?
\subsubsection*{挑战自己:}
当你感到孤立无援时,你的第一反应是抱怨,还是向神呼求?
你是否愿意为了神的荣耀,忍受短暂的羞辱和挑战?
\subsection*{二、信靠中的祷告(13-29节):等待神的拯救}
“但我在悦纳的时候,向你耶和华祈祷。神啊,求你按你丰盛的慈爱,凭你拯救的诚实应允我。”(诗 69:13)

\subsubsection*{1. 祷告带来希望}
“求你将我从淤泥中救拔出来,不叫我陷在其中。”(诗 69:14)
大卫在苦难中仍然选择向神呼求,他知道神是信实的,他的慈爱是丰盛的。
现实应用:我们是否相信,即使看不到答案,神仍然掌权?
\subsubsection*{2. 神比人的攻击更大}
“辱骂我的,我心里忧愁,忧愁难当。”(诗 69:20)
人的羞辱和拒绝让大卫极度痛苦,但他没有报复,而是把一切交托给神。
现实应用:当别人伤害你时,你会选择自己伸冤,还是让神来审判?
\subsubsection*{3. 耶稣在十字架上的预表}
“他们拿苦胆给我当食物,我渴了,他们拿醋给我喝。”(诗 69:21)
这节经文直接预表了耶稣在十字架上的经历(马太福音 27:34)。
现实应用:耶稣为我们的罪受苦,我们是否愿意在受苦时与他一同承担十字架?
\subsubsection*{挑战自己:}
当神的救恩似乎迟迟未到时,我们是否仍然选择相信?
你是否愿意在苦难中等候,而不是自己采取极端手段?
\subsection*{三、盼望中的赞美(30-36节):最终的得胜在神手中}
“我要以诗歌赞美神的名,以感谢称他为大。”(诗 69:30)

\subsubsection*{1. 赞美比献祭更蒙神悦纳}
“这便叫耶和华喜悦,胜似献牛,或是有角有蹄的公牛。”(诗 69:31)
现实应用:当我们在苦难中仍然敬拜神,这是真正的信心的表现。
\subsubsection*{2. 神必拯救他的子民}
“耶和华听见困苦人的需要,不藐视被囚的人。”(诗 69:33)
神不会忽视我们的痛苦,他的拯救会按他的时间来到。
\subsubsection*{3. 盼望最终的得胜}
“愿天和地、洋海和其中一切的动物,都赞美他。”(诗 69:34)
这是一幅末后的景象,表明神的公义必然得胜。
\subsubsection*{挑战自己:}
你是否愿意在痛苦中仍然敬拜神?
你是否相信,最终的得胜在神手中,而不是在人的手里?
\subsection*{结论:当困境来临,我们选择仰望神}
诗篇 69 篇提醒我们:
\begin{enumerate}
    \item 当我们陷入困境时,首先要向神呼求,而不是靠自己挣扎。

    \item 即使神的拯救尚未显现,我们仍然可以信靠他,因为他是公义的。

    \item 最终,神的公义必然彰显,我们要带着盼望敬拜他。

\end{enumerate}

愿我们都能在困难、误解和等待中,仍然选择信靠神,并用生命来赞美他!

\subsection*{结束祷告}
\textbf{亲爱的天父,}

我们感谢你,因为你是我们的避难所。在困境和羞辱中,我们常常软弱,但今天你的话语提醒我们,你是公义的神,你的救恩从不迟到。主啊,帮助我们在绝望时,仍然向你呼求;在等待中,仍然信靠你的应许;在得胜之前,就已经开始赞美你。愿我们的生命成为你荣耀的见证!

奉主耶稣基督的名祷告,阿们!
%-----------------------------------------------------------------------------
\newpage
\section{诗篇第70篇:在急难中呼求神}
% 讲章:——诗篇 70 篇
\subsection*{引言:当你急需帮助时,你首先想到的是什么?}
\hspace{0.6cm}弟兄姐妹,我们在生活中常常会遇到突如其来的危机:家庭问题、经济压力、健康状况、学业或工作中的挑战。在这些急难时刻,我们的第一反应是什么?是惊慌失措、求助于人,还是立刻转向神?

诗篇 70 篇是大卫在危难时刻所写的紧急祷告。这篇短短的诗篇向我们展示了一个属神的人在困境中的反应:立刻呼求神,并坚定信靠他的拯救。

今天,我们要从这篇诗篇中学习,在急难中如何祷告,并且如何在等待神的帮助时仍然持守信心和喜乐。

\subsection*{一、急难时,第一时间求告神(1-2节)}
“神啊,求你快快搭救我!耶和华啊,求你速速帮助我!”(诗 70:1)

\subsubsection*{1. 祷告的迫切性:快快帮助我!}
大卫没有兜圈子,而是直接向神发出最急切的请求。
他知道,人在危难时,最重要的不是人的方法,而是神的介入。
\subsubsection*{2. 现实应用:我们是否第一时间向神祷告?}
当困难来临时,我们往往先求助于人,或试图自己解决问题,等到无路可走时才想到神。
大卫的榜样告诉我们:神才是我们真正的拯救,应该首先转向他!
\subsubsection*{挑战自己:}

下次遇到紧急情况时,你是否愿意第一时间跪下祷告,而不是慌乱或埋怨?
\subsection*{二、面对敌人时,把公义的伸冤交给神(3节)}
“愿那些寻索我命的,抱愧蒙羞;愿那些喜悦我受害的,退后受辱。”(诗 70:3)

\subsubsection*{1. 交托神,而非自己伸冤}
大卫没有选择自己报复,而是把一切交给神。
现实应用:当我们遭受误解、诽谤或攻击时,我们很容易生气,想要为自己讨公道。
但圣经教导我们:“伸冤在我,我必报应。”(罗马书 12:19)
\subsubsection*{2. 神的公义最终必然彰显}
“愿那些喜悦我受害的,退后受辱。”(诗 70:3)
这表明神不会忽视我们的痛苦,他必然会按公义行事。
\subsubsection*{挑战自己:}

当你被人误解或伤害时,你是否愿意放下怒气,把一切交给神,而不是自己报复?
\subsection*{三、等待神的拯救时,持守信心和喜乐(4-5节)}
“愿一切寻求你的,因你高兴欢喜;愿那些喜爱你救恩的,常说:当尊神为大!”(诗 70:4)

\subsubsection*{1. 等待神的拯救,不是消极等待,而是持守信心}
“愿一切寻求你的,因你高兴欢喜。”
这里的“寻求”意味着持续不断地追求神,即使在困难中仍然相信神的美善。
现实应用:有时候,神的拯救不是立刻到来,我们需要在等待中继续敬拜、继续信靠。
\subsubsection*{2. 在苦难中仍然敬拜神}
“愿那些喜爱你救恩的,常说:当尊神为大!”
这意味着,即使神的拯救还未临到,我们仍然要宣告:神是伟大的!
现实应用:当你祷告后没有立刻得到答案,你会抱怨,还是继续敬拜神?
\subsubsection*{3. 神必然拯救依靠他的人}
“但我是困苦穷乏的,神啊,求你速速到我这里来!你是帮助我的,搭救我的,耶和华啊,求你不要耽延!”(诗 70:5)
大卫在承认自己的软弱时,仍然坚定相信神会拯救他。
现实应用:承认自己无助并不可耻,重要的是我们要依靠神,而不是靠自己。
\subsubsection*{挑战自己:}

当你的祷告没有立刻得到回应时,你是否愿意继续持守信心,喜乐地等候神?
\subsection*{结论:在急难中,仰望神才是我们的出路!}
诗篇 70 篇教导我们:
\begin{enumerate}
    \item 当我们陷入急难时,第一时间求告神,而不是自己惊慌或寻求人的帮助。

    \item 当我们遭受攻击时,不要报复,而是把公义的伸冤交给神。

    \item 即使神的拯救尚未显现,我们仍然要持守信心和喜乐,继续敬拜他。

\end{enumerate}

无论你现在正经历什么挑战,愿我们都能学习像大卫一样,在最困难的时候,第一时间转向神,并且在等待的过程中仍然信靠、喜乐、敬拜!

\subsection*{结束祷告}
\textbf{亲爱的天父,}

感谢你透过诗篇 70 篇教导我们,在急难时要立刻转向你,而不是惊慌失措或依靠自己。主啊,求你帮助我们,在困境中第一时间寻求你的拯救,并且在等待中仍然持守信心和喜乐。我们相信,你是公义的,你不会耽延,你的救恩必然来到。愿我们的生命尊你为大,愿我们的信心更加坚定!

奉主耶稣基督的名祷告,阿们!
%-----------------------------------------------------------------------------
\newpage
\section{诗篇第71篇:依靠上帝,度过生命的风暴}
% 诗篇第71篇:
\subsection*{引言}
人生如海,有风平浪静的时刻,也有狂风暴雨的挑战。当我们年幼时,依靠父母的怀抱;年长后,依靠自己的智慧和能力。然而,总有一些时刻,环境逼迫我们,人的力量显得渺小,我们需要一个坚固的避难所。诗篇第71篇是大卫在老年时的祷告,他回顾一生,依然坚定地倚靠上帝。今天,让我们从这篇诗篇中学习,在风暴中如何依靠神,度过生命的挑战。
\subsection*{一、上帝是我们的避难所(诗篇71:1-3)}
“耶和华啊,我投靠你;求你叫我永不羞愧。求你凭你的公义搭救我,救拔我,侧耳听我,拯救我!求你作我常住的磐石;你已经命定要救我;因为你是我的岩石,我的山寨。”(诗篇71:1-3)
\subsubsection*{1. 人生需要避难所}
每个人都会遇到困难,无论是学生面对学业压力,职场人士面对竞争,还是年长者面对健康衰退,我们都需要一个可靠的避难所。世界上的避难所——财富、权势、关系——有时会失去作用,唯有上帝是永恒可靠的。
\subsubsection*{2. 神是我们的磐石}
在圣经中,磐石象征稳固、信实和保护。当人生遇到风暴,我们可以随时投靠他,不会被击垮。大卫的一生经历过许多危机:被扫罗追杀、被儿子押沙龙背叛,但他始终把神当作避难所,从未绝望。
\subsubsection*{实际应用:}
当我们遇到失败和挫折时,不要只依靠人的帮助,而要祷告,把忧虑交托给神。
读神的话语,让圣经成为我们心灵的安慰和指引。
养成敬拜和祷告的习惯,让神成为我们随时可依靠的磐石。
\subsection*{二、回顾过去,坚定信心(诗篇71:5-9, 17-18)}
“主耶和华啊,你是我所盼望的;从我年幼,你是我所倚靠的。我从出母胎被你扶持;使我出母腹的是你;我必常常赞美你。”(诗篇71:5-6)
\subsubsection*{1. 过去的恩典是未来的信心}
大卫在回顾自己的一生时,发现神的恩典一直都在。他从小就是神所拣选的,在放羊时就经历过神的保守,击败狮子、熊,也靠着信心战胜歌利亚。他深知,神既然在过去带领他,也必定带领他走未来的路。
\subsubsection*{2. 生命中的每个阶段都可以倚靠神}
大卫强调“从年幼”到“年老”,神的恩典从未改变(诗篇71:17-18)。这提醒我们,不管是年轻、壮年,还是老年,我们都可以信靠神。很多时候,人到老年容易有“无用感”,但大卫仍然祈求:“神啊,到了年老发白的时候,求你不要离弃我!”(71:18)表明他愿意继续为神做见证。
\subsubsection*{实际应用:}
当你对未来感到迷茫时,回顾过去神如何带领你,会带来信心。
记录祷告蒙应允的经历,让自己在低谷时重新得力。
不管年龄大小,都可以服侍神——年少时可以传福音,年长时可以鼓励后辈。
\subsection*{三、面对挑战,持续赞美神(诗篇71:20-24)}
“你使我经历重大急难,必使我复活,从地的深处救上来。你必使我越发昌大,又转来安慰我。我的神啊,我要鼓瑟称赞你,称赞你的诚实;以色列的圣者啊,我要弹琴歌颂你。”(诗篇71:20-22)
\subsubsection*{1. 苦难不是终点,而是成长的机会}
大卫经历过许多苦难,但他相信神“必使他复活”,让他重新得力。在神的计划中,苦难不会毁灭我们,反而会塑造我们的信心,让我们更加依靠他。
\subsubsection*{2. 赞美是信心的表达}
即使身处困境,大卫仍然选择歌颂神。赞美神不是等到问题解决后才做,而是在困难中就开始,因为这表示我们相信神必带领我们度过。
\subsubsection*{实际应用:}
当你面临挑战时,不要只是埋怨,而是用祷告和赞美取代焦虑。
通过诗歌、读经、感恩日记,让自己在困难中保持积极的信心。
分享自己的信仰经历,鼓励其他正在经历低谷的人。
\subsection*{结论:在生命的风暴中,依靠神}
诗篇71篇向我们展示了一个信靠神的人,即使在困境中,依然可以充满盼望。他相信神是他的避难所,回顾过去的恩典,坚信神的信实,并在苦难中仍然赞美。
今天,不管你正经历什么——学业压力、工作瓶颈、家庭问题、健康挑战——都可以像大卫一样,把神当作避难所,回顾他的恩典,并持续赞美他。愿我们都在风暴中学会信靠,让生命的每一天都充满盼望与喜乐。
愿神祝福你!
\subsection*{结束祷告}
\textbf{亲爱的天父,}

在我们结束这段灵修时,如同诗篇71篇所启示的,我们满怀信心地来到祢面前。主啊,祢是我们永恒的避难所和坚固的保障,无论在困境或在衰老之时,我们都投靠祢。求祢在我们软弱时扶持我们,在风雨中成为我们的盾牌与拯救。

主啊,愿我们在祢无限的恩典中不断经历祢的大能,正如经上所写:“直到我将你的大能传给下一代。”求祢使我们的生命成为见证,将祢的奇妙作为和慈爱宣扬在每个时刻。无论前路如何,求祢的平安与指引常伴随我们,使我们在祢面前得着力量和希望。

奉主耶稣基督的圣名祷告,阿们。
%-----------------------------------------------------------------------------
\newpage
\section{诗篇第72篇:公义的君王}
% 讲章:——诗篇 72 篇
\subsection*{引言:我们需要怎样的领导者?}
\hspace{0.6cm}在这个世界上,人们常常在寻找一位真正正直、公义、慈爱且有智慧的领导者——无论是在国家、公司、教会,还是家庭。我们渴望看到一个能带来公平与繁荣的治理者。然而,现实往往让人失望,因为地上的领导者总是有缺陷的。

诗篇 72 篇是一篇关于“弥赛亚君王”的诗篇,所罗门王在此为王祈求公义与智慧,同时也预表了最终完全掌权的基督。今天,我们要借着这篇诗篇来看:神的国度如何与世界的治理不同,我们如何在日常生活中实践神国的价值观,并仰望基督这位完美的君王。

\subsection*{一、公义与公平:神所喜悦的领导(1-4节)}
“神啊,求你将你的判断赐给王,将你的公义赐给王的儿子!他要按公义审判你的民,按公平审判你的困苦人。”(诗 72:1-2)

\subsubsection*{1. 公义与公平:领导者的核心素质}
诗篇 72 篇一开始就祈求王有神的“公义”和“公平”。
这意味着真正的好领导必须依靠神的智慧,而不是人的私心。
现实应用:无论你是在领导团队、家庭,还是自己的生活,我们都需要按照神的公义行事,而不是凭着个人利益或情绪做决定。
\subsubsection*{2. 关心贫困人、维护弱势群体}
“大山要因公义使百姓得享平安。诸小山也要如此。他必为民中的困苦人伸冤,拯救穷乏之辈,压碎欺压人的。”(诗 72:3-4)

这里强调,神所喜悦的治理不是权力的彰显,而是为了使百姓得享平安,特别是帮助那些软弱无助的人。
现实应用:在社会中,神希望我们关心那些被忽略、被边缘化的人,例如贫穷者、孤儿、寡妇。
\subsubsection*{挑战自己:}

在你的生活中,有没有人为你的决策而受益?你是否在关注身边有需要的人?
\subsection*{二、神的国度带来平安与祝福(5-17节)}
“他要降临,像雨降在已割的草地上,如甘霖滋润田地。”(诗 72:6)

\subsubsection*{1. 神的统治带来丰盛的祝福}
这里的“雨”象征着滋润和更新,神的统治使土地繁荣、人民兴旺。
现实应用:当我们愿意按照神的原则行事,我们的家庭、工作、教会都会充满神的祝福和丰盛。
\subsubsection*{2. 他的国度要存到永远}
“他要执掌权柄,直到永远;在他之下的人民要生长如地上的禾稼。”(诗 72:7-8)

这里所描述的,不仅仅是地上的以色列王,更是预表着弥赛亚耶稣基督的国度。
现实应用:我们不能单单把希望寄托在地上的领导者身上,而是要把盼望放在基督永恒的国度里。
\subsubsection*{挑战自己:}

你是否愿意让基督在你的生命中掌权,让他的公义和爱影响你的一切决策?
\subsection*{三、神的荣耀要充满全地(18-20节)}
“耶和华——以色列的神是应当称颂的!惟独他行奇事。他荣耀的名也当称颂,直到永远!愿他的荣耀充满全地。”(诗 72:18-19)

\subsubsection*{1. 一切荣耀归给神}
诗篇 72 篇的结尾提醒我们,最终的荣耀不属于任何一个人,而是属于神自己。
现实应用:当你取得成就时,你是否愿意将荣耀归给神,而不是归功于自己?
\subsubsection*{2. 盼望弥赛亚完全掌权的那一天}
这个世界仍然充满不公义和混乱,但我们知道,最终基督会完全掌权,使万国万民敬拜他。
现实应用:我们应该活出神国度的价值观,预备迎接基督的再来!
\subsubsection*{挑战自己:}

你是否在生活中见证神的荣耀,让他成为你的一切?
\subsection*{结论:如何在日常生活中践行神国度的价值?}
\begin{enumerate}
    \item 在你的岗位上,按照神的公义行事
    
    无论你是学生、职员、父母、老板,都要在决策和行动中反映神的公义。

    \item 关心贫困人、帮助有需要的人
    
不要只关注自己的利益,要有基督的爱心。

    \item 在生活中预备迎接神的国度
    
每一天都以荣耀神为目标,而不是只追求个人的成功。

\end{enumerate}

愿我们都能在基督的掌权下,活出公义、恩典、与祝福的生命!

\subsection*{结束祷告}
\textbf{亲爱的天父,}

感谢你赐下诗篇 72 篇,让我们看见你公义的统治与国度的丰盛。主啊,帮助我们在生活中践行你的价值观,在我们的岗位上活出公义,关心弱势群体,并且期待你完全掌权的那一天。求你掌管我们的生命,使我们的一切决定都荣耀你!愿你的国度降临,愿你的荣耀充满全地!

奉主耶稣基督的名祷告,阿们!
%-----------------------------------------------------------------------------
\newpage
\section{诗篇第73篇:信仰的试炼与真实的祝福}
% 诗篇第73篇:
\subsection*{引言}
在我们的信仰旅程中,我们常会遇到一个问题:“为什么恶人昌盛,义人受苦?” 这不仅仅是一个古老的神学问题,更是我们现实生活中的困惑。当我们努力行善、敬畏神,却发现自己仍然遭遇困难,而那些不敬畏神的人似乎顺风顺水,我们会不会感到迷茫、愤怒,甚至开始质疑神的公义?
今天,我们来查考诗篇73篇,看看诗人亚萨如何经历信仰的挣扎,又如何在神的光照下得到答案。让我们带着真实的生命问题,一起进入这篇诗篇。
\subsection*{一、义人的迷茫:为何恶人昌盛?(1-14节)}
\subsubsection*{1. 诗人的困惑(1-3节)}
\hspace{0.6cm}诗篇73篇一开始,亚萨明确宣告:“神实在恩待以色列那些清心的人。”(1节)但随即,他就陷入了一个信仰的矛盾:“我见恶人和狂傲人享平安,就心生嫉妒。”(3节)

在现实生活中,我们可能也有这样的感受:
\begin{itemize}
    \item 职场上,一些同事用欺骗、奉承获得升职,而诚实守信的人反而受排挤;

    \item 学校里,一些人靠着作弊拿高分,而努力学习的同学却成绩不理想;

    \item 社会上,一些人靠着不正当手段发家致富,而普通人辛勤劳作仍然捉襟见肘。

\end{itemize}

这些现象让我们质疑:敬畏神真的有用吗?公义真的存在吗?
\subsubsection*{2. 观察到恶人的“顺遂”(4-12节)}
亚萨继续描述他所看到的景象:
\begin{itemize}
    \item 恶人没有疾病,身体健壮(4节)

    \item 他们骄傲、暴虐,却不受惩罚(6-7节)

    \item 他们藐视神,甚至否认神的存在(11节)
\end{itemize}

这让诗人产生了更深的疑问:“难道我洁净我的心,徒然无益吗?”(13节)他感到自己的信仰似乎没有带来实际的好处,反而成为了受苦的原因。

我们也可能会有类似的想法:
“我这么努力遵行神的道,却还是被现实打击。”
“那些不信神的人都过得很好,而我的信仰好像并没有让我更轻松。”

这样的思想很危险,因为它会让我们逐渐远离神,甚至跌倒。但亚萨没有停留在这里,他找到了解答的关键。
\subsection*{二、进入圣所,找到答案(15-20节)}
\subsubsection*{1. 进入神的圣所,重新得光照(16-17节)}
亚萨意识到,他的思想已经走到了危险的边缘。但当他“进入神的圣所”的时候,一切都改变了!
这意味着什么?进入圣所就是来到神的面前,重新用神的眼光来看世界:
当我们祷告、敬拜,神会调整我们的心思意念;
当我们查考圣经,我们会看到神的计划远远超越今世的成败。
当亚萨进入神的圣所后,他终于明白,恶人的昌盛只是暂时的,他们的结局是毁灭(18-20节)。就像一个站在滑坡上的人,表面看似稳固,实际上随时会跌落深渊。
\subsubsection*{现实应用:}
如果你在职场上看到有人用不正当手段得利,不要嫉妒,因为他们的成功可能只是短暂的。
如果你觉得自己受苦,不要灰心,因为神的恩典是永恒的,而世上的成功是短暂的。
我们要常常来到神的面前,让神调整我们的眼光,而不是被世界的表象所迷惑。
\subsection*{三、真正的祝福是什么?(21-28节)}
\subsubsection*{1. 诗人的悔悟(21-22节)}
亚萨承认,自己曾经的嫉妒和埋怨让他变得愚昧无知,像个“畜类”。我们有时候也是这样,只看到眼前的利益,而忽略了真正的价值。
\subsubsection*{2. 发现真正的福分(23-26节)}
亚萨在神面前得到了一个重要的启示:
“然而,我常与你同在;你搀扶我的右手。”(23节)
“除你以外,在天上我有谁呢?除你以外,在地上我也没有所爱慕的。”(25节)
“我的肉体和我的心肠衰残,但神是我心里的力量,是我的福分,直到永远。”(26节)
这才是真正的祝福!神的同在、神的引导、神的安慰,比世上的财富、健康、成功更加宝贵!
\subsubsection*{3. 选择亲近神(27-28节)}
诗人最终得出结论:“但我亲近神是与我有益;我以主耶和华为我的避难所,好叫我述说你一切的作为。”(28节)
这是一种成熟的信仰,不再以今生的成败来衡量神的公义,而是以上帝的同在作为最大的满足。
\subsection*{结论:如何在现实生活中应用?}
\begin{enumerate}
    \item 调整我们的眼光,不只看眼前,而是用永恒的视角看待事情。短暂的财富和成功不能定义人的价值,唯有神的同在才是最宝贵的。

    \item 不要嫉妒恶人的繁荣,而要坚守敬虔的道路。即使暂时看起来受苦,神的应许永不落空。

    \item 亲近神,让神成为我们的避难所。当我们感到迷茫、挣扎时,最重要的不是去和世界比较,而是来到神的面前,经历他的真实和信实。

    \item 活出真实的信仰,在职场、学校、家庭中做光做盐,让别人从我们身上看到神的荣耀。

\end{enumerate}

愿我们都能像诗人亚萨一样,从迷茫到信心,从嫉妒到满足,从质疑到坚定,最终能够说:“神是我的福分,直到永远!”
阿们!

\subsection*{结束祷告}
\textbf{亲爱的天父,}


感谢祢借着诗篇73篇向我们启示祢公义与慈爱的真理。在我们徘徊于疑惑和不解之中时,祢让我们明白:虽然眼前恶人的道路似乎宽阔、顺遂,但他们终究逃不出祢的审判。求祢赐我们智慧和坚固的信心,帮助我们不因暂时的困境或世事的不平而动摇,而是常存对祢无限恩典的仰望。

主啊,当我们心生嫉妒、徘徊于自我挣扎的深渊时,求祢如同那揭示真理的光,照亮我们的内心,使我们重新找回在祢面前的谦卑与依靠。愿我们每一天都能体会到祢的同在,经历那超越世俗纷扰的平安和安慰。

如今,我们将自己的疑虑、软弱与盼望都交托在祢手中。求祢以祢丰盛的恩典引导我们,使我们在行走人生路时,永不忘记依靠祢,直至永恒。

奉主耶稣基督的名祷告,阿们。

%-----------------------------------------------------------------------------
\newpage
\section{诗篇第74篇:在破碎中寻找神的拯救与信实}
% 诗篇第74篇:
\subsection*{引言}
\hspace{0.6cm}诗篇第74篇是亚萨所作的一篇诗,表露了他对神在国家灾难中的沉默和弃绝的困惑。这篇诗篇是一次集体的呼喊,表达了以色列民在被敌人侵略、圣殿被毁、神的荣耀被羞辱时的痛苦与失望。尽管如此,诗篇的核心却仍然传递出对神不变的信心与期待。亚萨在面对国家的破败、宗教的破碎、人民的流离失所时,并没有放弃,而是选择向神呼求,恳求神再度拯救。对于我们今天的信徒来说,诗篇74篇提供了在困境中呼求神、回望神作为的深刻教训。
\subsection*{一、痛苦的现实:国家的破败与神的沉默(诗篇 74:1-11)}
\hspace{0.6cm}诗篇的前半部分描述了以色列的苦难,亚萨直面神的沉默与以色列所面临的灾难。他说:“神啊,你为何永远丢弃我们?你为何向我们发怒如火呢?”(第1节)这是信徒在面对长时间的苦难时常有的疑问。当神似乎没有回应时,我们容易怀疑神的信实,甚至会觉得神对我们弃若敷衍。

亚萨继续列举了神百姓所经历的苦难:“你的圣殿被焚烧,圣所被践踏,你的仇敌在圣地高声呐喊。”(第7节)圣殿是以色列民族最重要的象征,是神同在的地方。然而,在外敌的侵略下,圣殿遭到摧毁,神的百姓四散逃亡,这无疑让信徒感到极大的绝望和迷茫。

在我们的生活中,难免也会遇到类似的困境。也许我们经历家庭的破裂、事业的失败、健康的危机,甚至是社会和国家的动荡。很多时候,困境让我们觉得神似乎远离了我们,我们可能会感到孤独、无助,甚至心生疑惑:“神到底在哪里?神为什么不回应我们的祷告?”

然而,诗篇74篇提醒我们,即使神似乎在沉默,他并未离弃我们。我们不应因为眼前的困境而怀疑神的存在和信实。神的沉默并不意味着神不在,神的拯救往往是按他的时机来临的。
\subsection*{二、向神呼求并回顾神的作为(诗篇 74:12-17)}
\hspace{0.6cm}在接下来的经文中,亚萨回顾了神过往在以色列历史中的作为,并且借此激发了对神的信心:“但神是我的王,是命令雅各得胜的神。”(第12节)亚萨回忆起神曾在以色列的历史中做出的伟大作为,回顾了神如何分开红海、击败敌人、在旷野中带领以色列百姓。在面对困境时,回顾神过去的作为是我们信仰的力量来源。

亚萨接着写到:“你拆毁了大海的头,打伤了怪兽……你使白昼与黑夜分开,设置了日月。”(第13-16节)这些表述是对神创造天地伟大作为的回顾,提醒我们神是全能的创造主,他的能力无可比拟。神曾经拆毁邪恶的力量,掌管自然界的一切。正因为神的伟大,亚萨相信神能够恢复破碎的局面,带来救赎。

在我们今天的生活中,当我们面临挑战时,回顾神的作为,特别是他在我们生命中的作为,是极为重要的。这种回顾能提醒我们神不曾改变,他依然是掌管一切的主。当我们在困境中无力自拔时,回望神的信实能够重新点燃我们的信心,让我们坚定地相信神的拯救最终必会到来。
\subsection*{三、呼求神的介入与拯救(诗篇 74:18-23)}
\hspace{0.6cm}接下来,亚萨继续为神的百姓呼求:“神啊,求你记念仇敌怎样亵渎你,愚昧的百姓怎样侮辱你的名。”(第18节)亚萨表达了对敌人亵渎神圣名的不满和愤怒,呼吁神不再容忍这些亵渎神的行为。接着,他恳求神再度显示他的能力,拯救受苦的百姓。

这一段经文让我们看到,呼求神的名字并不是单纯的悲叹,而是向神表白我们的困境并期待他采取行动。亚萨没有选择自暴自弃,也没有责怪神,而是选择把自己的愤怒和痛苦交给神,恳求神的介入。

今天,面对困境时,我们同样应当这样呼求神,带着信心和迫切的心情,恳求神的拯救。神知道我们的苦难,神会倾听我们的呼求。但这也要求我们信靠神会按照他的时间和方式行事。
\subsection*{四、信靠神的最终胜利(诗篇 74:23-23)}
\hspace{0.6cm}最后,诗篇74篇以亚萨对神不变信实的宣告作为结束:“愿你记得你的仇敌,愿你恢复对你的百姓的帮助!”这不仅是大卫自己对神信实的期待,也表达了他对神最终胜利的信心。

当我们在困境中感到绝望时,诗篇74篇鼓励我们不只看眼前的困境,更要看神的应许和过去的作为。虽然眼前的局面可能让人心碎,但神是公义的,他最终会胜过一切的敌人,带来恢复与平安。

\subsubsection*{实际应用}
\begin{itemize}
    \item 面对困境时,不要失去信心: 无论生活中的困难和挑战多么巨大,诗篇74篇教导我们,即使神暂时沉默,他依然在工作。在苦难中,我们可以选择依靠神,呼求神的帮助。

    \item 回顾神的作为,激发信心: 回顾神在我们生命中曾经施行的恩典和作为,能帮助我们从困境中看到神的信实与能力。这是我们信仰的重要资源。

    \item 在祷告中坚持信靠: 向神呼求是我们与神建立亲密关系的方式,不要因为看不到即时的结果而灰心丧气。继续在神面前坚持祷告,期待神按他的时间和方式回应。

    \item 坚信神的最终胜利: 无论眼前的挑战多么艰难,神的胜利是最终的。我们可以有信心,神会恢复破碎的事物,带来他的拯救和公义。

\end{itemize}

\subsection*{结论}
诗篇第74篇提醒我们,当我们遭遇人生的困境和国家的动荡时,我们不仅要面对痛苦的现实,更要记得回顾神过去的作为,信靠他不变的信实,呼求他的拯救。无论神的回应是否即时,神的最终胜利和公义一定会在合适的时刻显现。让我们在困境中坚守信仰,期待神的拯救,感受神带来的恢复和平安。
\subsection*{结束祷告}
\textbf{亲爱的天父,}

我们感谢你在困境中向我们显现你的信实和拯救。虽然我们有时面对眼前的困难感到无助,但我们知道你是全能的神,必定会在合适的时刻为我们带来拯救。求你帮助我们在困境中坚持信靠你,回顾你过去的恩典,并期待你最终的胜利。

奉耶稣基督的名祷告,阿门。
%-----------------------------------------------------------------------------
\newpage
\section{诗篇第75篇:在混乱世界中信靠神的公义与掌权}
% 诗篇第75篇:
\subsection*{引言}
在当今世界,我们常常面对不公正的现象:恶人似乎昌盛,正直的人却受苦;权势者滥用职权,弱势群体遭受欺压;社会充满腐败与不公。面对这样的处境,我们可能会质问:“神在哪里?为什么神不立即施行公义?”
诗篇第75篇正是回应这样的疑问。这是一首亚萨的诗,表达了对神公义审判的坚定信靠。诗人宣告:神是掌权者,世界虽然动荡,但神的公义不会迟延,他按他的时间施行审判,抬举谦卑人,降低骄傲人。这篇诗篇鼓励我们在动荡不安的世代,仍然信靠神的掌权,并以正确的态度面对人生。
\subsection*{一、神的名值得称颂,因为他的作为显明在世(诗75:1)}
神啊,我们称谢你,我们称谢你!因你的名相近,人都述说你奇妙的作为。
诗篇75篇一开始就是一首感谢的诗歌。诗人并没有先抱怨世界的不公,而是直接宣告神的名配得赞美。为什么?因为神的名“相近”,他的作为可以被人看见。
\subsubsection*{1. 生活中的应用:}
在现实生活中,我们可能会因眼前的不公而灰心,但诗人提醒我们,神的名一直与我们同在,他的作为并非隐藏的。例如:
当我们回顾自己的人生,会发现许多次神的保守与恩典——可能是在困难中突然得到帮助,可能是在迷茫时获得方向,可能是在绝望时体验到神的安慰。
历史上,许多邪恶政权最终崩溃,正义得以伸张,虽然过程可能漫长,但神的手一直在掌权。
\subsubsection*{2. 我们的回应:}
面对困境,我们应当先感恩,回顾神的作为,而不是只盯着眼前的困难。当我们学习感恩,我们的信心会更加坚定,相信神仍然在掌权。
\subsection*{二、神按他的时间施行公义(诗75:2-3)}
我到了所定的日期,必按正直施行审判。地和其上的居民都消化了;我曾立了地的柱子。
这节经文表明,神有一个确定的时间表,到时候,他必定按公义施行审判。虽然我们可能觉得恶人当下得势,但他们终究不能逃脱神的公义。
\subsubsection*{1. 为什么神的审判会延迟?}
\hspace{0.6cm}神的忍耐:神的时间表与人的时间表不同,他有时允许恶人暂时昌盛,是为了给他们悔改的机会。(彼得后书3:9)

神的计划:我们看到的只是短暂的片刻,而神掌管整个历史。就像农夫不会在种子刚发芽时就收割庄稼,神也按他的时间成就他的计划。
\subsubsection*{2. 现实生活中的例子:}
一些长期行恶的人最终受到法律的制裁,或遭遇意想不到的报应。
许多曾经遭受不公义对待的人,最终得到了公正的对待,虽然可能经历了很长时间,但神的公义终究显现。
\subsubsection*{3. 我们的回应:}
不要因为短暂的不公就灰心,神的审判虽不总是即时,但他的时间是最完美的。 我们要有信心,知道神在幕后掌权,在合适的时间会按正直施行审判。
\subsection*{三、骄傲者必被降卑,谦卑者必被升高(诗75:4-7)}
我对狂傲人说:‘不要狂傲’,对凶恶人说:‘不要举角’……惟有神断定:他使这人降卑,使那人升高。
这里的“角”象征力量和权势。诗人警告骄傲的恶人不要自恃甚高,因为神最终会让他们降卑,而抬举那些谦卑信靠他的人。
\subsubsection*{1. 现实中的骄傲与谦卑:}
历史上许多骄傲自大的统治者最终失败,如尼布甲尼撒王,他因狂妄自大,结果被神降为野兽般的生活(但以理书4:28-33)。
反之,许多谦卑倚靠神的人最终被提升,如约瑟,他被卖为奴隶,但最终成为埃及的宰相;大卫在牧羊时被神拣选,最终成为以色列王。
\subsubsection*{2. 生活中的应用:}
在职场上,谦卑工作的人通常会得到认可,而骄傲自大的人最终可能会遭遇失败。
在人际关系中,骄傲的人容易树敌,谦卑的人却能建立长久的友谊。
神的原则永不改变:骄傲的人终究会跌倒,而谦卑的人最终会被神提升。
\subsubsection*{3. 我们的回应:}
我们应当时刻保持谦卑,不自夸自己的成就,而是依靠神的带领。不要嫉妒那些暂时得势的骄傲者,因为最终神会按他的公义升高或降低人。
\subsection*{四、神的愤怒杯终必倒出(诗75:8-10)}
耶和华手里有杯,其中的酒起沫……地上的恶人都必喝这酒的渣滓,而且喝尽。
这里的“杯”象征神的愤怒与审判。恶人可能暂时兴盛,但最终他们必定要承担自己的罪孽,喝尽神审判的苦杯。
\subsubsection*{1. 现实生活的例子:}
许多长期作恶的个人或国家,最终都遭受毁灭。例如历史上的独裁者最终下场悲惨。
圣经中,法老曾经逼迫以色列人,但最终遭遇十灾,带领他的军队被红海吞灭。
\subsubsection*{2. 我们的回应:}
当我们看到世界上的邪恶时,不要以为神无所作为。神的审判终将临到,恶人必喝尽神忿怒的杯。因此,我们要敬畏神,远离恶事,选择公义的道路。
\subsection*{结论:信靠神的掌权,活出公义的生命}
诗篇75篇提醒我们:
\begin{enumerate}
    \item 神的名配得称颂,他的作为可见。

    \item 神有他的时间表,不要因为眼前的不公就灰心。

    \item 神抬举谦卑的人,降卑骄傲的人,要存敬畏的心生活。

    \item 恶人终将喝尽神愤怒的杯,我们应当选择走义路。

\end{enumerate}
\subsubsection*{我们如何应用?}
面对不公义,不要灰心,要信靠神的掌权。
遇到骄傲自大的人,不要嫉妒,要持守谦卑。
以公义、敬畏神的态度生活,相信神的最终审判与拯救。
愿我们在动荡的世界中,仍然信靠神的公义,持守正直,期待神按他的时间施行审判与拯救!
%-----------------------------------------------------------------------------
\newpage
\section{诗篇第76篇:神是全地的君王}
% 讲章:——诗篇 76 篇
\subsection*{引言:谁才是真正的掌权者?}
\hspace{0.6cm}在这个世界上,我们常常看到强权政治、军事冲突、社会动荡,许多人依靠武力、金钱和权势来建立自己的“国度”。然而,历史不断证明:人的国度终究会衰败,但神的国度永远长存。

诗篇 76 篇是一首赞美诗,描述了神如何在战胜敌人、审判列国时显出他的权能。这首诗提醒我们:\textbf{真正的君王不是地上的权势,而是掌管一切的神。}今天,我们要借着这篇诗篇思想:神的权能如何影响我们的生活,我们该如何回应他的统治?

\subsection*{一、神的同在使他的子民得胜(1-3节)}
“在犹大,神为人所认识;在以色列,他的名为大。”(诗 76:1)

\subsubsection*{1. 神的同在是得胜的关键}
诗篇 76:1-2 讲到,神的名在犹大和以色列是伟大的,因为他住在他们中间(“他的帐幕在撒冷”)。
这提醒我们,得胜的关键不在于人的能力,而在于神的同在。
现实应用:今天,我们面临属灵争战、生活挑战,我们要靠的是神的同在,而不是自己的聪明或资源。
\subsubsection*{2. 神粉碎敌人的武器}
“他在那里折断弓上的火箭,并盾牌、刀剑,和争战的兵器。”(诗 76:3)

神不仅与他的百姓同在,还亲自施行拯救,摧毁敌人的武器。
\subsubsection*{现实应用:}
我们常常依靠自己的方式去争战(比如依赖金钱、人脉、计划),但真正的胜利来自神。
在你面对困难时,你是否愿意先寻求神的帮助,而不是单靠自己?
\subsubsection*{挑战自己:}

你是否每天花时间亲近神,让他的同在成为你得胜的根基?
\subsection*{二、神的权能使列国战栗(4-10节)}
“你从有野食之山显为荣耀。心中勇敢的人都被抢夺,他们睡了长觉;没有一个英雄能措手。”(诗 76:4-5)

\subsubsection*{1. 人的骄傲在神面前毫无价值}
世上的强权者、军事将领,他们可能自认为无所不能,但在神的手中,他们不过是尘土。
现实应用:无论是国家领袖,还是我们自己,骄傲最终都会带来失败。我们必须谦卑在神面前,承认自己的有限。
\subsubsection*{2. 神是最终的审判者}
“你发怒的时候,谁能在你面前站得住呢?”(诗 76:7)

这个世界充满不公,但神终有一天会施行完全的公义审判。
现实应用:当我们看到社会不公、罪恶猖獗时,我们要相信神的公义,而不是焦急或报复。
\subsubsection*{挑战自己:}

你是否愿意把世界的不公交托给神,而不是让愤怒吞噬你?
\subsection*{三、神的统治值得我们敬畏和顺服(11-12节)}
“你们许愿,当向耶和华你们的神还愿。凡在他四面的人都当拿贡物献给那可畏的主。”(诗 76:11)

\subsubsection*{1. 既然神掌权,我们当敬畏他}
诗人呼吁百姓要向神许愿,并履行他们的承诺,表达他们对神的顺服。
现实应用:今天,很多人只在需要帮助时求告神,但当神拯救他们后,却忘记了神。
我们是否真正敬畏神,还是只在困难时才想起他?
\subsubsection*{2. 我们该如何顺服神的掌权?}
\begin{itemize}
    \item 献上贡物(感恩、敬拜、金钱、时间、生命)

    \item 履行对神的承诺(敬畏他、顺服他的旨意)

    \item 信靠神的公义,而不是倚靠人的方法

\end{itemize}
\subsubsection*{挑战自己:}

你是否在生活中真正顺服神,还是只在困难时才依靠他?
\subsection*{结论:如何回应神的权能?}
\begin{enumerate}
    \item 依靠神的同在,而不是人的方法(建立与神亲密的关系)

    \item 相信神的公义,而不是自己报复(交托你的愤怒与不公给神)

    \item 敬畏并顺服神的掌权(履行你的承诺,活出信仰)

    % \item 
\end{enumerate}

愿我们都能在神的掌权下,过一个敬畏神、顺服神、依靠神的生命!

\subsection*{结束祷告}
\textbf{亲爱的天父,}

感谢你借着诗篇 76 篇提醒我们,你是全地的君王,你掌权,你审判,你拯救。主啊,帮助我们在生活中更多依靠你的同在,而不是自己的聪明;帮助我们相信你的公义,而不是让愤怒控制我们;更帮助我们敬畏你,在日常生活中活出对你的顺服。愿你的国度降临,愿你的荣耀充满全地!

奉主耶稣基督的名祷告,阿们!
%-----------------------------------------------------------------------------
\newpage
\section{诗篇第77篇:在困境中寻求神——信仰的突破}
% 诗篇第77篇:
\subsection*{引言:当黑夜漫长时,你如何应对?}
我们每个人都会经历人生的低谷。也许是事业受挫、家庭破裂、疾病缠身,或者是长时间的祷告似乎没有回应。你是否曾在夜里辗转反侧,心中充满疑问:“神在哪里?他还在听吗?”
诗篇77篇是亚萨在痛苦与挣扎中写下的一首诗。他真实地表达了自己对神的疑惑与痛苦,但最终,他的思绪从困境转向神的信实,从迷茫进入信心的突破。这篇诗篇为我们提供了一个信仰的成长路径,帮助我们在黑暗中找到盼望。
\subsection*{第一部分:向神倾诉,不要封闭自己(诗77:1-3)}
我要向神发声呼求;我向神发声,他必留心听我。我在患难之日寻求主;我在夜间举手不住,我的心不肯受安慰。(诗77:1-2)
\subsubsection*{1. 诚实面对痛苦,不要压抑情绪}
亚萨在这里没有掩饰自己的痛苦。他坦率地向神呼求,不是冷漠地祷告,而是用迫切的心、不断地祈求。他没有选择沉默或假装坚强,而是向神倾吐自己的忧伤。
现实生活中,我们常常如何面对痛苦?
有人选择逃避,借助娱乐、工作、社交媒体来麻痹自己。
有人选择封闭自己,不愿与人分享内心的挣扎。
有人甚至远离神,觉得他似乎不在乎自己的处境。
但圣经告诉我们:当痛苦来临时,我们最应该做的就是向神倾诉,而不是逃避。 祷告不是给神信息,而是让我们自己在神的同在中得着释放。
\subsubsection*{应用:}
你最近是否有难以承受的压力?是否有未解决的困境?向神大声呼求吧!即使感觉不到神的回应,也不要停止祷告。
\subsection*{第二部分:当怀疑袭来时,转向神的信实(诗77:4-9)}
难道主要永远丢弃我,不再施恩吗?(诗77:7)
\subsubsection*{1. 信仰的低谷:当神沉默时怎么办?}
亚萨在这里表达了对神的深深疑问。他连续提出了六个问题,似乎在质问神是否已经遗忘了他。这是许多信徒在苦难中都会有的挣扎:
“神啊,你还爱我吗?”
“你为什么不回应我的祷告?”
“你曾经做过的神迹,现在为何不再显现?”
这表明,即使是属灵领袖,也会有信心软弱的时候。但是,信仰的成长不在于从不怀疑,而是在怀疑中依然愿意寻求神。
\subsubsection*{2. 神真的会丢弃我们吗?}
不,神的爱是不变的(罗8:38-39)。我们可能感觉不到他的同在,但这不代表他真的离开了我们。就像乌云遮住了太阳,但太阳仍然在发光。
\subsubsection*{应用:}
不要害怕向神倾诉你的疑问,他不会因为你的疑问而责备你。
在黑暗中,不要做决定。当情绪低落时,不要做重大决定,因为那时我们的判断容易受到负面情绪的影响。
\subsection*{第三部分:回顾神过去的作为,建立信心(诗77:10-15)}
我要提说耶和华所行的,我要记念你古时的奇事。(诗77:11)
\subsubsection*{1. 从回忆中找回信心}
亚萨开始改变焦点,不再专注于自己的痛苦,而是回忆神过去的作为。他想到神曾经带领以色列人出埃及,行过大能的神迹。这提醒我们:当我们看不到未来时,可以回顾神在过去的信实。
\subsubsection*{现实生活的应用:}
\hspace{0.6cm}记录神的恩典:许多基督徒在低谷中,会写下自己曾经历过的神迹和他的恩典。当你回顾这些时,会发现神从未离弃过你。

听见别人的见证:当你听见其他基督徒如何经历神的带领时,也会得到信心的激励。
\subsection*{第四部分:相信神仍然掌权(诗77:16-20)}
你的道在海中,你的路在大水中,你的脚踪无人知道。(诗77:19)
\subsubsection*{1. 神的道路高过我们的道路}
亚萨在最后,将眼光放在神的能力上。他回忆到神如何分开红海,带领以色列人走干地。虽然当时的人看不到神的脚踪,但神仍然在带领他们。
\subsubsection*{这意味着什么?}
我们不总是能理解神的作为,但这不代表他没有在工作。
神有时用我们意想不到的方式拯救我们。
他的带领不会总是显而易见,但我们可以相信他仍然掌权。
\subsubsection*{应用:}
即使你看不到神的手,也要信靠他的心。
神在沉默时,仍然在行动。
你可能不明白神的道路,但你可以信靠他的性格。
\subsection*{结论:如何在困境中经历信仰的突破?}
诗篇77篇给我们提供了一个信仰成长的路径:
向神倾诉,而不是封闭自己。
即使怀疑神,也要继续寻求他。
回顾神过去的作为,建立信心。
相信神仍然掌权,即使我们看不到他的作为。
\subsubsection*{挑战:}
你是否正在经历黑暗的时刻?今天就勇敢地向神呼求,不要隐藏你的痛苦。
花一些时间回顾神过去在你生命中的作为,让这些记忆成为你的信心支柱。
在等待神回应的过程中,不要放弃信靠,即使看不到出路,也要相信神仍在掌权。
\subsection*{结语}
人生的旅程并不总是一帆风顺,我们都会有低谷,但神的信实不会改变。愿我们在困境中,仍然学习信靠神,像亚萨一样,从痛苦到信心,从疑问到敬拜,从绝望到盼望!
\subsection*{结束祷告}
\textbf{亲爱的天父,}

在迷茫与困苦中,我们回想起祢曾经显明的大能。求祢帮助我们在疑惑中依然坚定信心,让祢的恩典和安慰充满我们的心。奉主耶稣的名祷告,阿们。
%-----------------------------------------------------------------------------
\newpage
\section{诗篇第78篇:从历史中学习信靠神}
% 诗篇第78篇:
\subsection*{引言:为何我们总是重复过去的错误?}
有句俗话说:“历史会重演。” 在我们的生活中,我们常常会看到同样的错误一再发生。无论是个人的失败,还是社会的问题,似乎我们很难真正从过去的经历中吸取教训。
诗篇78篇是亚萨写的一首历史诗歌,它回顾了以色列人的历史,提醒后代不要重蹈祖先的覆辙,而是要信靠神、顺服他的引导。这篇诗篇不仅是对以色列人的提醒,也是对我们今天信仰生活的深刻启示。
\subsection*{第一部分:传承信仰,避免遗忘神(诗78:1-8)}
“我的民哪,你们要留心听我的训诲,侧耳听我口中的话。我要开口说比喻,我要说出古时的谜语。”(诗78:1-2)
\subsubsection*{1. 为什么要回顾历史?}
亚萨以“比喻”和“谜语”开头,这意味着他要讲述的不只是过去的故事,而是带有深刻属灵意义的教训。他强调,信仰不仅是个人的事情,更是要世世代代传承的。
\subsubsection*{2. 如何避免“信仰断层”?}
诗篇78:4 说:“我们不将这些事向他们的子孙隐瞒,要将耶和华的美德和他能力,并他奇妙的作为,述说给后代听。” 这提醒我们:

家庭信仰教育至关重要:如果父母不向孩子传讲神的作为,孩子很可能会被世界的价值观塑造,渐渐远离神。

教会需要承担责任:教会的责任不仅是牧养当代信徒,也要培养下一代,使他们认识神的大能。
\subsubsection*{现实应用:}
你是否有意识地向下一代传承信仰?
你是否常常回顾并分享神在你生命中的恩典?
你是否在家庭或教会中承担教导的责任?
挑战: 让我们学习成为信仰的“桥梁”,而不是“断层”,用生命影响下一代!
\subsection*{第二部分:不信与悖逆的危害(诗78:9-41)}
“他们不遵守神的约,不肯照他的律法行,竟忘记他所行的和他显给他们奇妙的作为。”(诗78:10-11)
\subsubsection*{1. 以色列人的失败——不信神的能力}
以法莲人虽然装备精良,却在争战之日退后(诗78:9)。这说明,即使我们外在条件完备,如果缺乏信心,我们仍然会失败。
他们失败的原因是遗忘神的作为:
\begin{itemize}
    \item 忘记了神如何拯救他们出埃及(诗78:12-16)

    \item 忘记了神如何供应他们食物和水(诗78:23-29)

    \item 忘记了神如何惩罚他们的悖逆(诗78:30-31)

\end{itemize}
\subsubsection*{2. 我们是否也犯同样的错误?}
我们是否因眼前的难处就怀疑神?
我们是否在神供应时感恩,却在缺乏时埋怨?
我们是否只求神的祝福,而不愿意顺服他的旨意?
\subsubsection*{现实应用:如何避免重蹈覆辙?}
培养感恩的心:每天数算神的恩典,避免“灵性健忘症”。
在试炼中依靠神:不要因为暂时的困难就怀疑神的信实。
活出信仰,而不仅是知识:以色列人“知道”神的作为,但他们并没有真正信靠神。今天,我们需要的不只是头脑的知识,更是心灵的信靠和行动的顺服。
\subsection*{第三部分:神的管教与怜悯(诗78:42-64)}
“他们屡次试探神,惹动以色列的圣者。”(诗78:41)
\subsubsection*{1. 神的管教是出于爱}
以色列人的悖逆带来了神的惩罚,比如瘟疫、战争和被掳(诗78:59-64)。然而,神的管教并不是要毁灭他们,而是要唤醒他们的悔改。
我们是否也经历过神的管教?
也许是一些挫折,让我们重新回到神面前。
也许是神挪去了一些我们依赖的东西,使我们单单仰望他。
也许是某些计划被阻挡,神在引导我们走更好的道路。
\subsubsection*{2. 神的怜悯永远长存}
虽然神惩罚了以色列,但他并没有完全丢弃他们。神的心意不是毁灭,而是拯救。 他最终兴起大卫作王(诗78:70-72),预表着未来的弥赛亚——耶稣基督,他才是我们最终的拯救者!
\subsubsection*{现实应用:如何正确看待神的管教?}
不要抗拒,而是要顺服,因为神的管教是出于爱(来12:6)。
在失败后不要绝望,而要回转归向神,因为神的怜悯永不止息(哀3:22-23)。
\subsection*{结论:从过去的教训中活出信心的生活}
诗篇78篇告诉我们:
我们需要学习历史的教训,传承信仰,不要让下一代陷入同样的错误。
我们需要保持信靠,不要因困难就埋怨和悖逆,要记得神过去如何信实地带领我们。
神的管教是出于爱,而他的怜悯永远长存,即使我们失败,他仍然愿意拯救我们。
\subsubsection*{挑战与行动:}
这周每天写下3件神过去在你生命中的恩典,并向神感恩。
如果你在某些方面远离了神,今天就决定回转,重新信靠他。
主动向你的家人、朋友或下一代分享你的信仰经历,让神的作为成为他们的鼓励。
愿我们都能成为不忘记神作为的人,在现实生活中活出信心,顺服神的带领! 阿们!
\subsection*{结束祷告}
\textbf{亲爱的天父,}

我们感谢你,因你是信实的神,过去如何带领以色列人,你今天仍然带领我们。求你赐给我们一颗不忘恩的心,让我们时刻记住你的奇妙作为,在困境中依然信靠你。主啊,我们承认自己有时也像以色列人一样软弱,会怀疑、会悖逆,但你从未放弃我们。求你赦免我们的罪,帮助我们从失败中学习,从你的话语中得力量,坚定跟随你。愿我们的生命成为祝福,使更多人因我们而认识你的信实和慈爱。

奉主耶稣基督的名祷告,阿们!
%-----------------------------------------------------------------------------
\newpage
\section{诗篇第79篇:在苦难中仰望神}
% 讲章:——诗篇 79 篇
\subsection*{引言:苦难中的呐喊}
在我们的生命中,我们都会经历艰难的时刻——疾病、失业、家庭破裂、社会动荡,甚至是信仰的逼迫。在这些时候,我们可能会问:“神啊,你在哪里?为什么你容许这些事情发生?”
诗篇 79 篇是一首哀歌,由亚萨所写。当时,以色列经历了极大的灾难,圣城耶路撒冷被毁,圣殿被玷污,百姓被屠杀。他们身处绝望之中,向神发出呼求:“神啊,你为何容忍这些事?”

\subsection*{一、面对苦难,我们要带着信心向神呼求(1-5节)}
“神啊,外邦人进入你的产业,污秽你的圣殿,使耶路撒冷变成荒堆。”(诗 79:1)

\subsubsection*{1. 现实中的苦难:以色列的灾难}
这节经文描述了耶路撒冷被毁的惨状:

圣殿被玷污:神的居所被外邦人亵渎。

百姓被屠杀:尸体遍地,无人埋葬(2-3节)。

羞辱和痛苦:他们成为列国的笑柄(4节)。

这些情景让人痛心,也让我们想到今天世界各地的战争、迫害和信仰挑战。

\subsubsection*{2. 我们该如何回应?向神呼求!}
“耶和华啊,你要到几时呢?你要永远发怒吗?你的愤恨要如火焚烧吗?”(诗 79:5)

诗人并没有逃避痛苦,而是直接向神倾诉自己的疑问和苦楚。
这提醒我们,当我们面对苦难时,我们可以诚实地向神呼求,而不是埋怨或远离神。
\subsubsection*{现实应用:}
在你生命的低谷中,你是否向神祷告,还是只凭自己挣扎?
你是否相信神仍然掌权,还是被苦难的现实压垮?
\subsubsection*{挑战自己:}

在你的困难中,每天花时间向神祷告,带着信心交托给他。
\subsection*{二、面对罪恶,我们要真实悔改,求神怜悯(6-9节)}
“不要记念我们先祖的罪孽,向我们追讨;愿你的慈悲快迎着我们,因为我们落到极卑微的地步。”(诗 79:8)

\subsubsection*{1. 苦难的根源:个人与集体的罪}
以色列的苦难并非偶然,而是因为他们远离神,招致了神的管教(申命记 28 章)。
诗人认识到,他们不能只求神拯救,还要承认自己的罪,求神怜悯。
\subsubsection*{2. 真实的悔改}
“拯救我们,赦免我们的罪,为你名的缘故。”(诗 79:9)

诗人不只是为了自己的利益求神赦免,而是为了神的荣耀。
\subsubsection*{现实应用:}
当我们遭遇困难时,我们要自省:我们是否有未认的罪?
我们不能只是求神解决问题,而是要真正悔改,回归神的道路。
\subsubsection*{挑战自己:}

你是否愿意在祷告中承认自己的过犯,并寻求神的赦免?
你的祷告是“神啊,帮我度过难关”,还是“神啊,赦免我,让我活出你的旨意”?
\subsection*{三、面对敌人,我们要相信神的公义和拯救(10-13节)}
“为何容外邦人说:‘他们的神在哪里呢?’愿你使外邦人知道你在我们眼前伸你仆人流血的冤。”(诗 79:10)

\subsubsection*{1. 神是最终的审判者}
诗人向神呼求,愿神亲自为以色列伸冤,击败仇敌。
他相信,尽管现在他们受苦,但神最终会显明公义,惩罚恶人。
\subsubsection*{2. 我们当如何面对逼迫和不公?}
信靠神,而不是靠自己报复:“主说:‘伸冤在我,我必报应。’”(罗 12:19)
继续忠心于神,而不是灰心丧志:“至于我们,你的民,你草场的羊,要称谢你,直到永远。”(诗 79:13)
\subsubsection*{现实应用:}

在职场、生活、信仰上,我们可能会遭遇不公平的待遇,甚至是逼迫。
我们要忍耐,相信神掌管一切,并用感恩的心继续跟随他。
\subsubsection*{挑战自己:}

面对不公,你是选择自己报复,还是交托给神?
你是否愿意在苦难中仍然敬拜神,感谢神?
\subsection*{结论:苦难中的信仰之路}
从诗篇 79 篇,我们学到了:
\begin{enumerate}
    \item 当苦难来临,我们要向神呼求,而不是逃避或埋怨。

    \item 面对罪恶,我们要悔改,而不是仅仅寻求解脱。

    \item 面对敌人,我们要相信神的公义,而不是自己伸冤。

\end{enumerate}

愿我们在苦难中仍然坚守信仰,相信神的掌权,并活出敬畏神的生命。

\subsection*{结束祷告}

\textbf{慈爱的天父,}

感谢你通过诗篇 79 篇教导我们,在苦难中如何仰望你。主啊,我们承认自己的软弱,有时面对困难,我们容易埋怨,甚至怀疑你的作为。求你帮助我们,在痛苦中仍然信靠你,在罪恶中谦卑悔改,在不公中忍耐等候你的公义。主啊,愿你的旨意成就,愿你的名得荣耀!

奉主耶稣基督的名祷告,阿们!
%-----------------------------------------------------------------------------
\newpage
\section{诗篇第80篇:求神复兴我们}
% 讲章:——诗篇 80 篇
\subsection*{引言:当生命遭遇低谷}
在生活中,我们或许都经历过低谷和困境。无论是个人的失败、家庭的破裂,还是信仰上的挣扎,我们都会有软弱和迷失的时候。当我们发现自己好像远离了神,我们该如何回应?诗篇 80 篇是一首恳切的祈求,诗人哀求神复兴他的百姓,使他们从毁灭中被拯救。这篇诗篇不仅适用于以色列的历史,也深刻地触及我们个人生命和教会的光景。

\subsection*{一、呼求神的怜悯——神是我们真正的牧者(1-3节)}
“领约瑟如领羊群的以色列的牧者啊,求你侧耳而听!坐在二基路伯上的啊,求你发出光来!”(诗 80:1)

\subsubsection*{1. 认识神是我们的牧者}
诗人首先呼求神,以“牧者”来称呼他。神不仅是掌管万有的主,也是我们个人生命的引导者。
作为牧者,神照顾、保护和引导他的百姓(参诗篇 23 篇)。
但有时,我们可能会觉得自己像迷失的羊,远离了神的带领。
\subsubsection*{2. 祈求神的光照和拯救}
“神啊,求你使我们回转,使你的脸发光,我们便要得救!”(诗 80:3)

诗人不断祷告求神“使我们回转”。这表明百姓已经偏离了神,需要被神带回正道。
“使你的脸发光”意味着神的同在、恩典和祝福。如果神的脸向我们隐藏,我们就会陷入黑暗。
\subsubsection*{现实应用:}
我们是否觉得自己生命中的某些部分已经远离了神?
我们是否在寻求神的同在,还是只在遇到困难时才想到神?
\subsubsection*{挑战自己:}

每天花时间亲近神,寻求他的引导,不要等到困境才来找他。
\subsection*{二、 省察我们的现状——为何神的脸向我们转离?(4-13节)}
“耶和华万军之神啊,你向你百姓的祷告发怒,要到几时呢?”(诗 80:4)

\subsubsection*{1. 祷告无果?神为何向我们转脸?}
诗人哀叹,以色列的祷告似乎没有得到回应,反而遭遇更大的苦难。
他们“以眼泪当食物”,经历极大的痛苦(5节)。
这是因为他们偏离了神,不再顺服他的道路,所以神暂时掩面不听他们的祷告。
\subsubsection*{2. 生命的困境是否是神的提醒?}
罪使人与神隔绝(以赛亚书 59:2)。
当我们发现自己的属灵生命枯干、祷告无力时,我们需要反思自己是否偏离了神。
\subsubsection*{现实应用:}
在工作、家庭、信仰中,我们是否有时觉得神好像沉默不语?
这可能不是神的远离,而是我们的心刚硬、不愿顺服。
\subsubsection*{挑战自己:}

花时间省察自己,看看是否有隐藏的罪或不顺服神的地方。
认罪悔改,让神恢复与你的关系。
\subsection*{三、求神复兴我们——回到神的恩典中(14-19节)}
“万军之神啊,求你回转,从天上垂看,眷顾这葡萄树。”(诗 80:14)

\subsubsection*{1. 以色列像一棵葡萄树,需要神的修剪和眷顾}
诗人用葡萄树的比喻来描述以色列的过去和现在(8-13节)。
他们原本在神的恩典中繁荣,但后来因悖逆而遭到毁坏。
\subsubsection*{现实应用:}
你是否曾经历过一段亲近神的时期,但后来却变得冷淡?
你是否愿意求神再次浇灌你的生命,让你重新火热?
\subsubsection*{2. 神的拯救是我们唯一的希望}
“耶和华万军之神啊,求你使我们回转,使你的脸发光,我们便要得救。”(诗 80:19)

诗篇 80 篇三次重复这句话(3, 7, 19节),显示出这是一首“复兴的祷告”。
诗人深知,唯有神能让百姓回转,重新经历他的同在。
今天,我们个人的生命、家庭、教会乃至整个社会,也需要这样的复兴!
\subsubsection*{挑战自己:}

你愿意为自己和教会的属灵复兴祷告吗?
你愿意付上代价,回到神的道路上,重新活出对神的爱吗?
\subsection*{结论:向神发出复兴的祷告}
诗篇 80 篇教导我们:

在困境中,我们要呼求神的怜悯,因他是我们的牧者。
当我们感到神的脸向我们转离时,我们要省察自己的生命,是否有偏离。
我们要祈求神复兴我们的生命,使我们重新进入他的恩典中。
愿我们都能像诗人一样,发出真实的复兴祷告,经历神的同在和更新!

\subsection*{结束祷告}
\textbf{慈爱的天父,}

感谢你今天借着诗篇 80 篇对我们说话。主啊,我们承认自己时常偏离你的道路,使我们的祷告无果、生命枯干。但你是信实的牧者,我们求你再次复兴我们,使你的脸发光,好叫我们得救。主啊,我们愿意悔改,求你更新我们的心,让我们重新火热地爱你,顺服你的引导。

奉主耶稣基督的名祷告,阿们!
%-----------------------------------------------------------------------------
\newpage
\section{诗篇第81篇:回归神的呼唤}
% 诗篇第81篇:
\subsection*{引言:神在呼唤,我们在回应吗?}
你是否有过这样的经历?当你处在困难或迷茫中,神不断用环境、圣经、讲道甚至朋友的提醒来吸引你回转,但你却因忙碌或固执而忽略了?
诗篇81篇是一首呼唤人悔改、回归神的诗歌。它回顾了神对以色列的恩典,也指出了他们的悖逆和失落。今天,我们借着这篇诗篇,来思想神的恩典、人的回应以及如何重新回到神的心意中。
\subsection*{第一部分:欢庆神的救赎(诗81:1-7)——感恩神的恩典}
“你们当向神——我们的力量大声欢呼,向雅各的神发声快乐。”(诗81:1)
\subsubsection*{1. 赞美神的救赎与供应}
诗篇一开始就呼吁百姓要向神欢呼、快乐,并用诗歌、乐器赞美神。这是因为他们曾经是奴隶,如今被神拯救,应该满心感恩!

神是我们的力量(v.1):当我们软弱时,他是我们的倚靠。

神曾救赎以色列(v.6-7):他曾在埃及拯救他们,使他们不再做奴仆。

神在困境中回应(v.7):当以色列人遭遇困难,呼求神,神就听见了。

\subsubsection*{2. 我们是否记得神的恩典?}
以色列人被神拯救,却常常忘记神。我们是否也一样?
生活顺利时,我们容易忘记曾经在低谷中神如何帮助我们。
当问题解决后,我们很快忘记曾经向神流泪祷告的日子。
\subsubsection*{现实应用:}
每天回顾并感恩神的恩典,不要让感恩只停留在嘴上,而要成为生活的态度。
赞美神不仅在顺境,也要在逆境,因为他仍在掌权。
当我们呼求神时,要带着信心等候,他过去如何拯救以色列,今天也同样能拯救我们。
\subsection*{第二部分:人的悖逆与偏行己路(诗81:8-12)——不听神的声音的后果}
“我的民哪,你当听,我要劝戒你!以色列啊,甚愿你肯听从我。”(诗81:8)
\subsubsection*{1. 以色列人如何悖逆?}

\hspace{0.6cm}他们拒绝听神的话(v.8):神呼唤他们听从,但他们心刚硬。

他们拜假神(v.9):尽管神已拯救他们,他们仍去敬拜别的神。

他们固执己见(v.12):神任凭他们随自己的心意而行,结果自食恶果。
\subsubsection*{2. 为什么我们也会偏行己路?}

\hspace{0.6cm}信心软弱:我们觉得神的道路太难,就选择走自己的路。

属世的诱惑:金钱、名利、享乐让我们远离神,甚至把它们当作“偶像”。

骄傲与固执:我们认为自己的计划比神的更好,不愿意顺服。
\subsubsection*{3. 偏行己路的结果是什么?}
神任凭他们随自己的心意而行(v.12)——这是最严重的审判!
当我们一再抗拒神,神会任凭我们去经历错误的后果,直到我们愿意悔改。
这就像一个叛逆的孩子,父母劝了很多次,他仍然不听,最终只能让他自己去经历痛苦的教训。
\subsubsection*{现实应用:}
我们是否常常听神的声音,但却不去遵行?
是否有一些“偶像”在我们生命中占据了神的位置?
今天,我们愿意回转,顺服神的带领吗?
\subsection*{第三部分:神的应许与祝福(诗81:13-16)——回转就得福}
“甚愿我的民肯听从我,以色列肯行我的道。”(诗81:13)
\subsubsection*{1. 顺服神的道路会带来祝福}
如果以色列人愿意听神的声音,神就会:

制服他们的仇敌(v.14):神要为他们争战,而他们只需信靠。

使他们得着丰盛的供应(v.16):神要用“上好的麦子”和“磐石出的蜜”喂养他们。
\subsubsection*{2. 神的心意是要祝福,而不是惩罚}
神并不是要让我们痛苦,而是希望我们走在他的祝福中。
但如果我们远离神,我们就无法经历这些祝福。
神一直等待着我们的回转,当我们愿意顺服,他就为我们预备丰富的恩典。
\subsubsection*{3. 现实应用:如何回到神的祝福中?}
悔改并回转:如果你发现自己偏离了神,今天就是回转的机会!
重新委身神:每天花时间亲近神,听他的话,并照着去行。
信靠神的供应:不要靠自己的方法解决问题,而要信赖神的计划。
\subsection*{结论:今天,你要如何回应神的呼唤?}
诗篇81篇向我们发出三个重要的信息:
记得神的恩典,不要忘记他曾如何拯救你。
不要悖逆神的呼唤,不要让自己的骄傲和偶像夺走神在你生命中的位置。
回转并顺服神的道路,你将经历他丰盛的祝福!
\subsubsection*{挑战与行动:}
这周花时间回顾神在你生命中的作为,并写下你的感恩清单。
祷告求神显明你生命中是否有偶像,并决心除去它们。
如果你已经远离神,今天就决定回转,重新委身他的道路!
愿我们都能回应神的呼唤,不再固执己见,而是在他的带领下,走向丰盛的生命! 阿们!
\subsection*{结束祷告}
\textbf{亲爱的天父,}

我们感谢你,因为你是信实的神,过去如何带领以色列人,你今天仍然带领我们。主啊,你一直在呼唤我们回转,但我们常常选择自己的道路,远离你的心意。今天,我们愿意悔改,愿意重新将我们的生命交在你的手中。
主啊,求你帮助我们记住你的恩典,保守我们不被世界的诱惑所吸引,也不因自己的骄傲而悖逆你。我们愿意聆听你的声音,顺服你的带领,求你赐下上好的麦子和磐石出的蜜,使我们生命丰盛,荣耀你的名!

奉主耶稣基督的名祷告,阿们!
%-----------------------------------------------------------------------------
\newpage
\section{诗篇第82篇:公义的神与人的责任}
% 讲章:——诗篇 82 篇
\subsection*{引言:当世界充满不公}
在这个世界上,我们常常看到不公正的事情发生:贫穷、欺压、腐败、弱势群体被忽视……面对这些问题,我们应该如何回应?神在乎这些事情吗?诗篇 82 篇给了我们一个清晰的答案:神是公义的审判者,他要求地上的掌权者行公义,并呼吁他的子民站出来,成为世界的光和盐。
\subsection*{一、神是公义的审判者(1-2节)}
“神站在有权力者的会中,在诸神中行审判,说:‘你们审判不秉公义,徇恶人的情面,要到几时呢?’”(诗篇 82:1-2)

\subsubsection*{1. 神是宇宙的最高法官}
诗人以法庭的场景展开,神站在众“神”之中,宣布审判。
这里的“神”(希伯来文 elohim)指的是地上的掌权者、法官或领袖,他们受神的托付,应该管理世界秩序。
然而,他们却没有秉公行义,反而徇私枉法,伤害了无辜的人。
\subsubsection*{2. 神痛恨不公正的审判}
以色列的法官受命要按照神的律法审判,但他们却偏袒恶人,导致社会充满不公。
今天,我们也要问自己:
我们是否在自己的生活中行公义?
我们是否在工作、家庭、人际关系中偏袒某些人,而忽视了公正?
\subsubsection*{现实应用:}

在职场中,不因私交或利益而做出不公正的决策。
在家庭中,对待子女公平,避免偏心。
在社会中,关注弱势群体,不做冷漠的旁观者。
\subsection*{二、神要求掌权者行公义(3-4节)}
“你们当为贫寒的人和孤儿伸冤,当为困苦和穷乏的人施行公义。要救护贫寒和穷乏的人,救他们脱离恶人的手。”(诗篇 82:3-4)

\subsubsection*{1. 神关心贫寒、孤儿、困苦和穷乏的人}
圣经中,神一直特别关心社会中的弱势群体,如孤儿、寡妇、贫穷者和寄居者(申命记 10:18,箴言 31:8-9)。
神不仅自己施行公义,也呼召地上的掌权者和所有属神的人去维护公义。
\subsubsection*{2. 我们在社会中有责任去行公义}
这个世界的制度并不完美,但基督徒可以成为改变的力量。
我们可以在自己的岗位上做出公正的决定,也可以支持那些推动社会公义的事工。
\subsubsection*{现实应用:}

关心社会中的弱势群体,如留守儿童、被欺负的学生、无家可归者等。
参与公益活动,为不公正的事情发声,而不是袖手旁观。
在生活中,做一个守诚信、说真话、不欺压别人的人。
\subsection*{三、人的失败与审判(5-7节)}
“你们仍不知道,也不明白,在黑暗中走来走去;地的根基都摇动了。”(诗篇 82:5)

\subsubsection*{1. 人的无知导致世界的混乱}
这里描述了世界因人的不公而陷入黑暗和动荡。
当掌权者和普通人都忽视公义,整个社会都会受到影响。
\subsubsection*{2. 人终究要面对神的审判}
“我曾说:‘你们是神,都是至高者的儿子;然而,你们要死,与世人一样,要仆倒,像王子中的一位。’”(诗篇 82:6-7)
掌权者虽然暂时有权力,但终究要面对神的审判。
他们若行不义,就会像世上的凡人一样死去,无法逃脱神的公正审判。
\subsubsection*{现实应用:}

如果我们今天有权力(无论是在家庭、公司还是社会中),我们要谨慎使用它,因为最终我们要向神交账。
我们要常常提醒自己,不要滥用自己的影响力去压制别人,而要用它来祝福别人。
\subsection*{四、求神施行最终的公义(8节)}
“神啊,求你起来,审判世界!因为你要得万邦为业。”(诗篇 82:8)

\subsubsection*{1. 盼望神的最终审判}
诗篇 82 篇以一个强烈的呼求结束:神啊,求你起来!
这个世界虽然充满不公,但神最终会审判世界,使公义完全实现。
在新约中,我们知道耶稣基督就是那位最终的审判者,他将带来完全的公义和国度(启示录 19:11-16)。
\subsubsection*{2. 作为基督徒,我们当如何回应?}
信靠神的最终审判,而不是靠自己伸冤(罗马书 12:19)。
在地上做公义的使者,成为世界的光和盐(马太福音 5:13-16)。
为社会的公义祷告,并且采取行动,帮助有需要的人。
\subsubsection*{现实应用:}

面对社会不公,我们不应该绝望,而要相信神的最终审判。
我们要努力做公义的见证人,用爱和公义影响身边的人。
\subsection*{结论:我们要成为神的公义代言人}
神是公义的审判者,他痛恨不公义。
神要求掌权者和所有人都要行公义,特别是帮助弱势群体。
人的不公和无知会导致混乱,但最终神会审判世界。
基督徒要信靠神的公义,并在生活中活出公义。
愿我们都能成为神公义的代言人,在黑暗中发光!

\subsection*{结束祷告}
\textbf{亲爱的天父,}

我们感谢你,因为你是公义的神,你从不偏待人。主啊,求你使我们的心向你柔软,让我们愿意在生活中行公义,怜悯贫穷人,帮助受欺压者。求你也在这个世界上施行你的公义,让恶人无法继续行恶,让你的真理彰显。愿我们成为你国度的见证人,活出你的公义和爱。

奉耶稣基督的名祷告,阿们!
%-----------------------------------------------------------------------------
\newpage
\section{诗篇第83篇:在困境中求告神}
% 讲章:——诗篇 83 篇
\subsection*{引言:面对攻击,我们该如何回应?}
在这个世界上,基督徒会经历逼迫、误解、甚至来自敌人的攻击。面对这样的困境,我们应该如何回应?诗篇 83 篇是亚萨的诗歌,它表达了对敌人围攻的痛苦,同时向神呼求,求他伸手拯救。今天,我们要从这篇诗篇中学习如何在困境中依靠神,如何用信心回应挑战。
\subsection*{一、求神介入(1-4节)——当神似乎沉默时}
“神啊,求你不要静默!神啊,求你不要闭口,也不要不作声!因为你的仇敌喧嚷,恨你的抬起头来。他们同谋奸诈,要害你的百姓,彼此商议,要害你所隐藏的人。”(诗篇 83:1-3)

\subsubsection*{1. 神有时似乎沉默}
诗人一开始就向神呼喊:“神啊,求你不要静默!”
当我们面对攻击时,有时会觉得神没有回应,我们的祷告仿佛没有被听见。
但圣经告诉我们,神从不真正沉默,他在暗中掌管一切。
\subsubsection*{现实应用:}

当你在生活中遭遇挑战(职场的不公、家庭的冲突、人际关系的误解)时,不要以为神不在乎。
神允许困难发生,是为了炼净我们的信心,让我们更深地倚靠他。
\subsection*{二、敌人的联合(5-8节)——仇敌的狡诈与合谋}
“他们彼此商议同心合意,结盟要抵挡你。”(诗篇 83:5)

\subsubsection*{1. 敌人的诡计与联合}
诗人列举了一大批敌国的名字(以东人、以实玛利人、摩押人等),表明以色列正处于四面楚歌的危机中。
这些仇敌并不是单独行动,而是联合起来对抗神的子民。
\subsection*{2. 现实中的属灵争战}
今天,我们虽然不像以色列人一样面对刀剑的战争,但属灵争战仍然存在。
世界的价值观、错误的文化思潮、罪恶的诱惑,都会联合起来攻击神的子民。
\subsubsection*{现实应用:}

面对信仰上的挑战时,我们要知道,真正的争战并不是人与人之间的斗争,而是属灵的争战(以弗所书 6:12)。
我们要警醒祷告,靠着神的能力站立得稳,不被世界的潮流带走。
\subsection*{三、回顾神的作为(9-12节)——信心的见证}
“求你待他们如待米甸,如在基顺河待西西拉和耶宾一样。”(诗篇 83:9)

\subsubsection*{1. 过去的得胜是信心的凭据}
诗人回顾了神在历史上的得胜,例如基甸战胜米甸人(士师记 7 章)。
这些历史事件提醒我们:神曾经拯救他的子民,他今天仍然可以拯救我们!
\subsubsection*{现实应用:}

当我们面对挑战时,可以回顾神过去在我们生命中的恩典,这会帮助我们重新得力。
你是否记得你过去经历神的帮助?把这些经历写下来,作为你信心的见证。
\subsection*{四、求神彰显公义(13-18节)——最终的盼望}
“耶和华啊,愿他们永远羞愧惊惶,愿他们惭愧灭亡,使他们知道,惟独你名为耶和华的,是全地以上的至高者。”(诗篇 83:17-18)

\subsubsection*{1. 盼望神的最终审判}
诗人祈求神惩罚恶人,使他们认识到耶和华才是至高的神。
这不仅仅是为了复仇,而是让列国认识神的主权和公义。
\subsubsection*{2. 我们今天的盼望}
在新约中,我们知道最终的审判权在耶稣基督手中(启示录 19:11-16)。
神最终会审判恶人,公义必然得胜。
\subsubsection*{现实应用:}

当世界充满不公义时,不要灰心,而要仰望神的最终审判。
我们要做和平的使者,同时相信神会在他的时间里成就公义。
\subsection*{结论:在困境中信靠神}
\begin{enumerate}
    \item 神从未真正沉默,他掌管一切。

    \item 世界的攻击虽然猛烈,但神的能力更大。


    \item 回顾神过去的作为,可以坚定我们的信心。
    \item 最终的公义必然得胜,我们要存盼望的心,忠心事奉神。

\end{enumerate}
\subsection*{结束祷告}
\textbf{天父,}

我们感谢你,因为你是公义的神,虽然世界充满挑战,但你从未离开我们。主啊,求你在我们的困境中显现,帮助我们坚信你的同在。无论是生活的压力、工作的挑战,还是信仰上的争战,求你赐给我们刚强的心,使我们不被世界动摇。主啊,我们也求你在世界中彰显你的公义,愿更多的人认识你,归向你。

奉主耶稣基督的名祷告,阿们!
%-----------------------------------------------------------------------------
\newpage
\section{诗篇第84篇:渴慕神的同在}
% 诗篇第84篇:
\subsection*{引言:你最渴望的是什么?}
每个人心中都有渴望:有人渴望成功,有人渴望财富,有人渴望幸福的家庭。但你是否渴望神的同在?诗篇84篇是一首充满对神渴慕的诗,表达了诗人对神殿的热爱,以及住在神的同在中所带来的喜乐和祝福。今天,我们一起来思想:什么是真正的满足?我们如何经历神的同在?
\subsection*{第一部分:渴慕神的同在(诗84:1-4)——你的心在哪里?}
“万军之耶和华啊,你的居所何等可爱!我羡慕渴想耶和华的院宇,我的心肠、我的肉体向永生神呼吁。”(诗84:1-2)
\subsubsection*{1. 诗人为何如此渴慕神?}
诗人不是渴望去圣殿看建筑,而是渴望亲近神,因为神的殿象征他的同在。
他的渴望是发自内心的(“心肠、肉体”),不是表面的宗教行为。
连麻雀和燕子都在神的殿中找到安息(v.3),更何况是那些投靠神的人呢?
\subsubsection*{2. 我们是否也有这样的渴慕?}
我们是否在敬拜和祷告中真正渴望神,而不是仅仅完成宗教义务?
我们是否优先寻求神,还是世界上的事情更吸引我们的注意力?
我们是否像诗人一样,把亲近神当作生命的满足?
\subsubsection*{现实应用:如何培养对神的渴慕?}

\hspace{0.6cm}定期安静亲近神:每天花时间读经、祷告,不要让忙碌挤走对神的渴望。

参与敬拜:教会的敬拜是神同在的地方,我们需要主动融入。

思考神的美善:多回忆神的作为,感受他的恩典,渴慕就会增长。

挑战: 你最近一次真正渴望神的同在是什么时候?今天,求神恢复你对他的热情!
\subsection*{第二部分:靠神得力的旅程(诗84:5-7)——在旷野中经历神}
“靠你有力量,心中想往锡安大道的,这人便为有福!”(诗84:5)
\subsubsection*{1. 人生是一场朝圣之旅}
诗人提到“锡安大道”——这是以色列人上耶路撒冷圣殿朝见神的路。
这条路不总是容易的,他们要经过巴咖谷(流泪谷),象征人生的艰难和试炼。
\subsubsection*{2. 在困境中如何得力?}

\hspace{0.6cm}“靠你有力量”:真正的力量不在于环境,而在于是否依靠神。

“流泪谷变为泉源之地”(v.6):当我们依靠神,苦难会变成祝福。

“他们行走,力上加力”(v.7):在试炼中依靠神,我们会越来越刚强。

\subsubsection*{现实应用:如何在困难中经历神?}
\hspace{0.6cm}在祷告中寻求神的力量,而不是单靠自己。

相信神能把困境转化为祝福,即使环境没有立刻改变。

不断亲近神,你的信仰会越来越坚固,就像朝圣者“力上加力”。

挑战: 你现在是否处于“流泪谷”?今天,把你的难处交给神,让他把它变为泉源之地!
\subsection*{第三部分:住在神同在中的祝福(诗84:8-12)——最好的选择}
“在你的院宇住一日,胜似在别处住千日;宁可在我神殿中看门,不愿住在恶人的帐篷里。”(诗84:10)
\subsubsection*{1. 住在神里面比世界的一切都好}
诗人宁愿做圣殿的门卫,也不愿意离开神去享受罪中之乐。
这代表他的价值观:神的同在比财富、地位、享乐更重要。
\subsubsection*{2. 神的同在带来哪些祝福?}
\hspace{0.6cm}神是日头(v.11):他照亮我们的道路,指引我们前行。

神是盾牌(v.11):他保护我们,使我们在危险中得平安。

神赐下恩典和荣耀(v.11):他不仅供应我们现在的需要,也带领我们进入更丰盛的生命。
\subsubsection*{现实应用:我们是否真的以神为最重要的?}
\hspace{0.6cm}我们是否愿意放下世上的享乐,选择与神同行?

我们是否相信,跟随神比自己掌控人生更好?

我们是否愿意在每个选择上,把神的旨意放在首位?

挑战: 你现在是否愿意重新调整你的优先次序,把神放在第一位?
\subsection*{结论:神是我们最大的满足!}
\subsubsection*{诗篇84篇告诉我们:}
\begin{enumerate}
    \item 真正的满足不是来自物质或成就,而是来自神的同在。

    \item 人生的旅程可能充满困难,但靠着神,我们可以“力上加力”。

    \item 住在神里面的人是最有福的,因为神赐下恩典、荣耀和真正的喜乐!

    
\end{enumerate}
\subsubsection*{行动挑战:}

\hspace{0.6cm}这周,每天腾出时间安静亲近神,让他成为你的满足。

在困难中,选择相信神,而不是依靠自己的方法。

重新调整你的优先次序,把神的旨意放在生活的中心!

愿我们都成为渴慕神、依靠神、并经历神丰盛祝福的人! 阿们!
\subsection*{结束祷告}
\textbf{亲爱的天父,}

我们感谢你,因为你是我们真正的满足。求你恢复我们对你的渴慕,让我们像诗人一样,渴想你的同在。主啊,我们承认,我们常常被世界的事物吸引,忘记了你才是最值得追求的。今天,我们愿意回到你面前,把我们的心完全交给你。
主啊,求你帮助我们,在人生的流泪谷中,仍然依靠你,经历你的恩典和祝福。愿我们不再追求短暂的满足,而是渴望永恒的与你同在!

奉主耶稣基督的名祷告,阿们!
%-----------------------------------------------------------------------------
\newpage
\section{诗篇第85篇:复兴的盼望}
% 讲章:——诗篇 85 篇
\subsection*{引言:我们渴望复兴吗?}
在信仰的旅程中,我们常常会经历低谷,有时是个人灵性的衰微,有时是教会的冷淡,甚至是整个社会的败坏。然而,我们的神是一位施恩怜悯的神,他渴望复兴他的子民。诗篇 85 篇是一篇充满盼望的诗歌,它教导我们如何祈求神的复兴,并如何回应他的恩典。

\subsection*{一、回顾过去的恩典(1-3节)——神曾赐下复兴}
“耶和华啊,你已经向你的地施恩,救回被掳的雅各。你赦免了你百姓的罪孽,遮盖了他们一切的过犯。”(诗篇 85:1-2)

\subsubsection*{1. 复兴的基础是神的恩典}
诗人首先回顾神过去对以色列的恩典——他曾拯救他们,赦免他们的罪,使他们从被掳之地归回。
这提醒我们:神过去的作为,是我们今天可以信靠他的基础。
\subsubsection*{现实应用:}

你是否经历过神的恩典?是否曾在困难时得到他的帮助?
回顾过去的恩典,能帮助我们在当下的困境中仍然信靠神。
\subsection*{二、祈求新的复兴(4-7节)——求神再一次施恩}
“拯救我们的神啊,求你使我们回转,除去你的恼怒!”(诗篇 85:4)

\subsubsection*{1. 我们为什么需要复兴?}
以色列百姓虽然经历了过去的拯救,但他们仍然软弱,需要神再次更新他们。
诗人求神“使我们回转”,说明复兴的关键在于回转归向神。
\subsubsection*{2. 复兴不是自动发生的,需要祷告}
诗人向神祷告:“耶和华啊,求你使我们回转!”
复兴不是靠人的努力,而是神的工作,我们要向神祈求。
\subsubsection*{现实应用:}
\hspace{0.6cm}个人层面:你是否感觉信仰变得冷淡?是否对祷告、读经、敬拜失去热情?向神祷告,求他使你回转!

教会层面:今天的教会是否活在神的同在中?还是已经失去了起初的爱?让我们为教会的复兴祷告!

社会层面:当世界越来越败坏时,我们需要为国家、社会呼求复兴!
\subsection*{三、聆听神的应许(8-13节)——复兴的真正源头}
“我要听神——耶和华所说的话,因为他必应许将平安赐给他的百姓。”(诗篇 85:8)

\subsubsection*{1. 复兴的应许}
诗人说:“我要听神所说的话。”——复兴不是凭我们的感觉,而是要听神的应许。
神的应许是什么?
“慈爱和诚实彼此相遇,公义和平安彼此相亲。”(85:10)
“耶和华必将好处赐给我们,我们的地也要多出土产。”(85:12)
\subsubsection*{2. 复兴带来的果效}
\begin{itemize}
    \item 人与神之间和好(慈爱与诚实)
    \item 社会公义与和平同行(公义和平安彼此相亲)

    \item 地上的祝福(物质与属灵的丰盛)

\end{itemize}

\subsubsection*{现实应用:}

我们如何经历神的复兴?

聆听神的话语,回到圣经中寻找方向。

顺服神的带领,遵行他的公义。

期待神的作为,相信他必赐下祝福。
\subsection*{结论:复兴,从你我开始}
\begin{enumerate}
    \item 回顾过去的恩典,信靠神的信实。

    \item 向神祷告,求他使我们回转,带下复兴。

    \item 聆听神的应许,遵行他的道路,经历真实的复兴。

    \item 让我们每一个人,从自己开始,渴望并祈求神的复兴临到我们个人、教会和社会。

\end{enumerate}




\subsection*{结束祷告}
\textbf{天父,}

我们感谢你,因为你是昔在、今在、永在的神,你曾在过去拯救你的百姓,今天你仍然愿意复兴我们。主啊,我们承认我们常常远离你,我们的心灵容易冷淡,我们的国家、社会充满不公义和罪恶。求你施恩,赦免我们的罪,使我们回转归向你。愿你的公义和慈爱充满我们的生命,愿你的平安和圣洁掌管你的教会。

我们期待你的复兴临到,奉主耶稣基督的名祷告,阿们!
%-----------------------------------------------------------------------------
\newpage
\section{诗篇第86篇:在困境中寻求神的怜悯与引导}
% 诗篇第86篇:
\subsection*{引言:当你处在困境中,你如何回应?}
在生活中,我们都会遇到困难:经济压力、人际关系破裂、疾病、失败…… 当一切看似失控时,我们会选择依靠自己,还是转向神?
诗篇 86 篇是大卫在困境中向神呼求的一篇诗。他在痛苦、仇敌环绕、甚至生命受到威胁时,没有怨天尤人,而是转向神,呼求神的怜悯、引导和帮助。这篇诗篇不仅仅是他的祷告,也是我们的榜样——当我们陷入困境,我们该如何向神祷告?
今天,我们从诗篇 86 篇中学习如何在困境中寻求神,并经历他的信实与拯救。
\subsection*{第一部分:向神呼求,承认自己的需要(1-7节)}
“耶和华啊,求你侧耳应允我,因我是困苦穷乏的。”(诗篇 86:1)
\subsubsection*{1. 认识自己的有限,谦卑寻求神}
大卫在这里承认自己困苦穷乏,不是指物质上的贫乏,而是指自己在能力、智慧、力量上都不足,无法独自面对当前的挑战。
许多人遇到问题时,第一反应是依靠自己,试图控制局面,但当所有方法都无效时,才想到求神帮助。
真正的智慧不是等到山穷水尽才呼求神,而是从一开始就谦卑寻求他。
\subsubsection*{2. 持续向神祷告,相信神必应允}
“我在患难之日要求告你,因你必应允我。”(诗篇 86:7)
大卫深知,神不仅听见祷告,更是会回应祷告的神。
他的信心建立在神的信实上,而不是环境的改变。
现实生活中,我们常常因为祷告没有立刻得到回应就灰心,但大卫提醒我们:神会按他的时间和方式回应我们的呼求。
\subsubsection*{现实应用:当你遇到困难,你是否愿意第一时间转向神?}
你是依靠自己的方法,还是谦卑在神面前祷告?
你是否愿意持续祷告,相信神必在适当的时候回应?
挑战: 每天设定固定的时间,与神建立祷告的关系,学习在一切事上依靠他!
\subsection*{第二部分:认识神的性情,坚定信靠他(8-13节)}
“主啊,诸神之中没有可比你的,你的作为也无可比。”(诗篇 86:8)
\subsubsection*{1. 神是独一无二、满有能力的神}
大卫在这里赞美神的伟大,提醒自己:神的能力超越一切。
当我们认识神的伟大,我们的信心就不会被环境左右。
现实中,我们常常把目光放在问题上,而不是神的属性,这使得我们容易恐惧、焦虑。
\subsubsection*{2. 神是满有怜悯、乐意拯救的神}
“主啊,你本为良善,乐意饶恕,有丰盛的慈爱赐给凡求告你的人。”(诗篇 86:5)
神不仅是伟大的神,更是良善的神,他愿意拯救凡寻求他的人。
许多人在软弱、失败时,不敢来到神面前,觉得自己不配。但大卫提醒我们:神是满有怜悯的,他不会拒绝悔改归向他的人。
你是否曾因自己的软弱而不敢祷告?今天你要知道,神正在等待你回转!
\subsubsection*{3. 祈求神的引导,走在他的道路上}
“耶和华啊,求你将你的道指教我;我要照你的真理行。”(诗篇 86:11)
许多人在困境中只想要神帮助他们脱离,但大卫更进一步:他祷告的不仅仅是拯救,更是求神指引他当行的道路。
我们是否只想要神解决问题,而不愿意顺服他的引导?
\subsubsection*{现实应用:当你面对困难时,你是否愿意信靠神,而不是被环境影响?}
你是否真正认识神的性情,并以此坚定你的信心?
你是否愿意求神引导你,而不仅仅是解决问题?
挑战: 每天用一句话来宣告神的信实,比如:“主啊,我信靠你,你的道路高过我的道路!”
\subsection*{第三部分:经历神的拯救,并以感恩回应他(14-17节)}
“神啊,骄傲的人起来攻击我,又有一党强横的人寻索我的命,他们没有将你放在眼中。”(诗篇 86:14)
\subsubsection*{1. 在逼迫中依然倚靠神}
大卫面对的不是小问题,而是性命攸关的危机,但他依然选择信靠神。
在现实中,我们也会面对反对、批评,甚至是来自世界的逼迫。
当我们选择坚定依靠神,他必在最艰难的时刻扶持我们!
\subsubsection*{2. 见证神的怜悯和拯救}
“主啊,求你向我显出恩待我的凭据,叫恨我的人看见便羞愧。”(诗篇 86:17)
大卫不是求自己的荣耀,而是愿意让神的作为成为见证,使仇敌羞愧。
我们的生命若活在神的带领中,必能成为见证,让世人看到神的真实!
\subsubsection*{3. 以感恩回应神的恩典}
“主我的神啊,我要一心称赞你,我要荣耀你的名,直到永远。”(诗篇 86:12)
经历神的拯救后,我们要用感恩的心回应他,而不是把他当作“问题解决者”。
现实中,很多人在困难时祷告,但一旦问题解决,就忘记了神。
真正认识神的人,会一生赞美、荣耀他!
\subsubsection*{现实应用:你是否愿意让你的生命成为见证?}
你是否只在需要帮助时才寻求神,而忘记了他的恩典?
你是否愿意以感恩的生命回应神,而不仅仅是索取?
挑战: 每一天数算神的恩典,并用感恩的心来敬拜他!
\subsection*{结论:在困境中,寻求神、信靠神、经历神!}
\begin{enumerate}
    \item 向神呼求,承认自己的需要——谦卑祷告,不依靠自己。

    \item 认识神的性情,坚定信靠他——知道他是良善、信实的神。

    \item 经历神的拯救,并以感恩回应——让生命成为神的见证。

\end{enumerate}

愿我们在困境中,学会像大卫一样,寻求神、信靠神,并以感恩的心回应他! 阿们!
\subsection*{结束祷告}
\textbf{亲爱的天父,}

感谢你在困境中一直与我们同在。求你赐给我们信心,使我们在任何时候都依靠你。求你指引我们的道路,让我们活出你的旨意,并成为你荣耀的见证。

奉主耶稣基督的名祷告,阿们!
%-----------------------------------------------------------------------------
\newpage
\section{诗篇第87篇:荣耀之城}
% ——诗篇87篇的属灵启示

\subsection*{引言}
弟兄姊妹,今天我们要一起查考《诗篇》第87篇。这篇诗篇讲述了神拣选锡安(耶路撒冷),使之成为万国属神子民的归属之地。这不仅是对旧约时代耶路撒冷城的描述,更是指向新约中的属灵国度——天国,也与我们今天的信仰生活息息相关。让我们一起思想,这篇诗篇如何影响我们的生命,并从中找到属灵的智慧和实践的方向。

\subsection*{一、神所拣选的荣耀之城(1-3节)}
诗篇87:1-3说:
"耶和华所立的根基在圣山上。他爱锡安的门,胜于爱雅各一切的住处。神的城啊,有荣耀的事指着你说。"

\subsubsection*{1. 神的拣选,超乎人的理解}
\hspace{0.6cm}这节经文告诉我们,神拣选锡安作为他居所的地方,并非因为耶路撒冷本身有什么特别,而是出于神的旨意和爱。这让我们想起新约中彼得前书2:9所说的:
"唯有你们是被拣选的族类,是有君尊的祭司,是圣洁的国度,是属神的子民。"

今天,我们每一个重生得救的基督徒,都是神所拣选的子民,我们的生命就像锡安城,被神建造、坚立、赐福。这提醒我们要活出与蒙召之恩相称的生命,不再随从世界的价值观,而是以神的标准为中心,追求圣洁和公义。

\subsubsection*{2. 耶路撒冷的荣耀,指向天国}
\hspace{0.6cm}当圣经谈到“神的城”时,它不仅指物质上的耶路撒冷,更预表新天新地中的天国。启示录21:2-3描述新耶路撒冷降临,神亲自与人同住。今天,我们在地上的生活只是客旅,我们的真正归属是天上的耶路撒冷(腓立比书3:20)。这提醒我们:

不要过分倚靠今世的财富和成就,而要把眼光放在永恒的国度。

在世上为神作光作盐,使我们的生命散发属灵的荣光。

现实应用: 在当今社会,人们渴望找到归属感,渴望被认同。神在基督里给了我们最真实的归属——我们是天国的子民。因此,我们要在教会中彼此建立,不仅仅是“去教会”,而是活出“教会”的身份,使世人看到神的荣耀。

\subsection*{二、神的子民来自万国万邦(4-6节)}
"我要提起拉哈伯和巴比伦的人使他们认识我;看哪,非利士、推罗、古实,这些生于那里的人(或作:这一个生在那里)。论到锡安,必说:‘这一个、那一个都生在其中。’而且至高者必亲自坚立这城。"(诗篇87:4-5)

\subsubsection*{1. 神的救恩临到列国}
\hspace{0.6cm}这里提到的“拉哈伯”(指埃及)、巴比伦、非利士、推罗、古实,都是以色列的外邦敌人。然而,神的计划不是只拯救以色列,而是要使外邦人也归入他的国度。新约中,耶稣基督来到世上,为所有民族、所有罪人预备救恩,使他们成为神的儿女。

这对我们有很大的启发:神的救恩超越了国界、种族和文化,今天无论我们来自哪里,过去如何,只要信靠耶稣,就能在基督里成为新造的人。

\subsubsection*{现实应用:}

在全球化的今天,我们身边可能有来自不同文化背景的人,我们是否愿意去接纳他们、向他们传福音?
在教会中,我们是否仍然有偏见?是否只愿意与“合得来”的人交往,而忽视了神国度的普世性?
神的心意是让各国各民都能在他的国度中找到归属,我们作为神的儿女,也要有这样的胸怀,去关爱、接纳、祝福不同背景的人。

\subsection*{三、在神的城中有真正的喜乐(7节)}
"歌唱的、跳舞的都要说:‘我的泉源都在你里面!’"(诗篇87:7)

这节经文充满了喜乐的氛围。神的子民因着被拣选、因着在锡安城中的身份而欢喜快乐。这喜乐不是世上的短暂满足,而是来自神的恩典和同在。

\subsubsection*{1. 真正的满足在于神}
世界常常让我们相信,金钱、成功、地位能带来满足,但它们最终都会让人失望。真正的满足和喜乐,只有当我们与神建立亲密的关系时才能得到。正如诗篇16:11所说:
"在你面前有满足的喜乐,在你右手中有永远的福乐。"

\subsubsection*{2. 赞美与敬拜带来生命的更新}
诗篇87篇以敬拜和赞美为结尾。敬拜不是一种仪式,而是生命的流露。当我们真正认识到神是我们生命的源头时,我们的心自然会充满感恩和敬拜。

\subsubsection*{现实应用:}

在日常生活中,我们是否真正以神为满足?还是把自己的喜乐建立在短暂的成就上?
当遇到困难时,我们是否愿意用敬拜来转向神,而不是陷入抱怨和消极?
\subsection*{结论:我们的回应}
诗篇87篇提醒我们:
\begin{enumerate}
    \item 我们的身份——我们是神的子民,被拣选来归入他的国度。

    \item 我们的使命——神的国度是普世性的,我们要向世界见证他的荣耀。

    \item 我们的满足——真正的喜乐和生命的泉源在于神,而非世界的短暂享乐。

\end{enumerate}

愿我们今天都能重新思考自己的生命,是否活在神的荣耀中?是否关心神国的扩展?是否真正以神为满足?

\subsection*{结束祷告}
\textbf{慈爱的天父,}

感谢祢拣选我们成为祢的子民,使我们在基督里有真正的归属。我们承认,在这个世界上,我们常常被短暂的成就和欲望所吸引,却忘记了祢才是生命真正的泉源。主啊,求祢更新我们的心,使我们更多渴慕祢,在祢的国度里找到真正的喜乐。也求祢扩张我们的眼界,让我们看到普世的需要,愿意成为祢手中的器皿,把福音带给身边的人。愿祢的荣耀充满我们的生命,使我们的一生都为祢而活!

奉主耶稣基督的名祷告,阿们!
%-----------------------------------------------------------------------------
\newpage
\section{诗篇第88篇:黑暗中的信仰}
% 诗篇第88篇:
\subsection*{引言}
弟兄姊妹,今天我们要一起思想《诗篇》第88篇。这是一篇充满哀伤的诗篇,它不像其他哀歌诗篇那样,在结尾处展现转机或希望,而是以黑暗结束。它描述了一位遭遇极大痛苦的义人,在黑暗中向神呼求,却迟迟得不到回应。
也许你曾经经历过这样的时刻:祷告似乎没有回应,内心充满困惑和孤独,甚至怀疑神是否还在。这篇诗篇能帮助我们理解:即便在最深的黑暗中,我们仍要坚持信仰,因为神仍然与我们同在。
\subsection*{一、黑暗中的呼求——持守信仰的勇气(1-2节)}
"耶和华—拯救我的神啊,我昼夜在你面前呼吁。愿我的祷告达到你面前,求你侧耳听我的呼求!"
\subsubsection*{1. 祷告是一种信心的表达}
\hspace{0.6cm}诗人虽然处在极大的苦难中,但他仍称呼神为“拯救我的神”,并且昼夜不停地向神祷告。这表明,尽管诗人感到绝望,他仍然相信神是他的拯救者。

有时候,我们也会在生命中遭遇黑暗时期,比如事业的失败、亲人的离世、病痛的折磨,或是灵里的干渴。当我们感受不到神的同在时,我们是否仍愿意向他呼求? 这正是信仰的试炼——在看不见希望时,仍然选择信靠。
\subsubsection*{2. 神允许我们真实地表达痛苦}
\hspace{0.6cm}诗篇88篇的作者毫不掩饰自己的痛苦,甚至在整篇诗篇中充满了哀叹。这提醒我们:祷告不只是感恩和赞美,我们也可以向神倾诉内心的痛苦和挣扎。

现实生活中,我们可能习惯在人前隐藏自己的软弱,表现得坚强。但神愿意听我们最真实的声音,我们可以像诗人一样,带着眼泪来到神的面前,坦诚表达我们的忧伤。
应用: 你是否曾因神没有立刻回应你的祷告而感到灰心?你是否愿意继续祷告,即使在黑暗中?
\subsection*{二、苦难的重量——理解人生的低谷(3-9节)}
"因为我心里满了患难,我的性命临近阴间。他们把我放在极深的坑里,在黑暗地方,在深处。你的忿怒重压我身,你用一切的波浪困住我。"(3-7节)
\subsubsection*{1. 痛苦可能是长期的,而非短暂的}
\hspace{0.6cm}诗人不仅描述了自己的痛苦,而且表达了这种痛苦持续已久。他感到生命临近死亡,被神的忿怒所困。这与我们常听到的“信耶稣就会一直蒙福”形成了对比。事实上,信仰生活中有时会有长期的痛苦和试炼,我们必须正视这一点。

有时我们会觉得:“为什么我是个基督徒,却仍然遭遇这些苦难?” 但耶稣在世时也经历了极大的痛苦,甚至在十字架上呼喊:“我的神,我的神,为什么离弃我?”(马太福音27:46)神从未应许我们不会经历痛苦,但他应许在苦难中仍然掌权。
\subsubsection*{2. 黑暗可能带来孤独感}
\hspace{0.6cm}诗人感到自己被朋友、亲人远离,甚至觉得神也离弃了他。这正是许多陷入抑郁或极度痛苦中的人会经历的感觉。他们会觉得自己与世界隔绝,没有人理解,甚至连神都沉默。

今天,很多人正在经历类似的孤独感,比如:

事业上的失败,使你觉得无人可依靠。

亲人离世,使你陷入深深的悲伤。

长期的疾病,让你感到身体和心灵的双重折磨。

当我们面对这些黑暗时,我们如何回应?是选择埋怨神,还是像诗人一样,即便感到神远离,仍然呼求他?
应用: 你是否曾感到孤独无助?在那样的时刻,你是否仍愿意寻求神?
\subsection*{三、黑暗中的信仰——神仍然掌权(10-18节)}
"耶和华啊,你要行奇事给死人看吗?阴魂还能起来称赞你吗?"(10节)

诗篇88篇没有像其他哀歌诗篇一样,在最后给出一个转机,而是以“黑暗”作为结束。但这并不意味着没有希望,而是表明:即使在黑暗中,我们仍然可以持守信仰,因为神仍然掌权。
\subsubsection*{1. 神的沉默并不代表他不在}
诗人感觉不到神的同在,但他仍然在祷告。这表明,即使我们感觉不到神的回应,他仍然在我们身边。
就像约伯的经历,他在极大的痛苦中质问神,却没有立刻得到答案。但最终,他看见了神的荣耀(约伯记42:5)。有时候,神的沉默是一种试炼,目的是要让我们的信仰更加坚定。
\subsubsection*{2. 在黑暗中坚持信仰,是最深的敬拜}
诗篇88篇让我们看到:信仰不仅仅是在顺境中感恩和敬拜,更是在逆境中坚持信靠神。真正的信仰,并不是当神祝福我们时才信靠,而是在黑暗中仍然紧抓住他。
应用: 你是否曾因神的沉默而灰心?你是否愿意在黑暗中依然持守信仰?
\subsection*{结论:我们的回应}
\begin{enumerate}
    \item 即使在黑暗中,也要继续祷告——祷告不是要改变神的心意,而是要让我们更加靠近神。

    \item 不要害怕向神表达痛苦——真实的信仰,不是隐藏痛苦,而是在痛苦中依然信靠神。

    \item 神的沉默并不代表他不在——即使我们感觉不到神的同在,他仍然掌权,并在我们生命中动工。

\end{enumerate}

最终,我们的信仰不是建立在感受上,而是建立在神的信实上。愿我们在黑暗中,仍然持守对神的信靠!
\subsection*{结束祷告}
\textbf{慈爱的天父,}

我们感谢你,尽管在生命中有黑暗和痛苦,你仍然掌权。主啊,有时候我们会感到孤独,感到祷告没有回应,但求你帮助我们在黑暗中仍然信靠你。我们愿意像诗篇88篇的诗人一样,虽然感受到绝望,但仍然向你呼求。求你坚固我们的信心,使我们无论在顺境或逆境中,都能紧紧抓住你。感谢你一直与我们同在,愿你的爱和恩典充满我们的生命。

奉主耶稣基督的名祷告,阿们!
%-----------------------------------------------------------------------------
\newpage
\section{诗篇第89篇:神的信实与人的盼望——看神的应许与我们的信靠}
% 讲章:从诗篇89篇
\subsection*{引言}
弟兄姐妹,今天我们要来分享《诗篇》第89篇。这是一首关于神的慈爱和信实的诗篇,同时也表达了诗人对现实困境的疑问。在我们的人生中,我们或许也会遇到类似的挣扎:一方面,我们相信神是信实的,他的应许永不改变;另一方面,当环境变得艰难,似乎与神的应许相违时,我们会疑惑、甚至灰心。那么,面对这样的张力,我们该如何回应?今天我们就来探讨神的信实与我们的盼望。

\subsection*{一、神的慈爱与信实(1-4节)}

\subsubsection*{1. 赞美神的信实}
诗人以赞美开始,他宣告神的慈爱和信实是永恒的,不因环境改变,也不因人的软弱而动摇。这是我们信仰的根基——神的属性不会改变。

\subsubsection*{2. 神的应许不会落空}
神曾应许大卫,他的后裔要永远坐在宝座上(撒母耳记下7:12-16)。这应许最终在耶稣基督身上得到了完全的成就。即使在大卫的后代中出现过失败和背叛,神仍然信守他的承诺。这提醒我们,神的信实不依赖于人的表现,而是基于他自己的本性。

\subsubsection*{现实应用:}
你是否因环境的变化而怀疑神的信实?
你是否愿意坚定地相信神的应许,即使目前还没有看到成就?
\subsection*{二、神的权能与主权(5-18节)}

\subsubsection*{1. 神是全能的创造主}
诗人强调,天地万物都属于神,他有绝对的主权。这意味着,无论世界看起来多么混乱,神仍然掌权。

\subsubsection*{2. 神的统治是公义的}
有时我们会质疑,为什么恶人昌盛,义人受苦?但诗篇告诉我们,神的宝座是建立在公义和公平之上的。虽然我们现在可能看不清楚,但神最终会按着他的公义审判世界。

\subsubsection*{现实应用:}
你是否因世界的不公而灰心?
你是否愿意相信神最终的公义审判,而不是急于自己伸张正义?
\subsection*{三、信仰的挑战:当现实与信仰不符时(38-45节)}

\subsubsection*{1. 诗人的困惑}
诗篇的前半部分充满了对神信实的赞美,但到了后半部分,诗人开始表达自己的疑问:如果神的应许永远坚定,为什么现在大卫的后裔却遭遇困境?为什么神似乎隐藏了自己?

\subsubsection*{2. 当神的应许似乎“失效”时,我们如何回应?}
在我们的生活中,也会遇到类似的情况:我们相信神会看顾我们,但有时我们仍然经历疾病、失业、关系破裂等困难。在这种时候,我们的信心会受到极大的挑战。

\subsubsection*{现实应用:}
你是否曾因未见应许成就而对神失去信心?
你是否愿意在疑问和苦难中仍然坚持他的信实?
\subsection*{四、带着信心呼求神(46-52节)}

\subsubsection*{1. 真实的信仰包含疑问}
诗人并没有因为疑问而远离神,反而带着真实的心来到神面前。这提醒我们,信仰不意味着我们永远都不会有疑问,而是我们愿意带着疑问仍然来到神面前祷告。

\subsubsection*{2. 他的应许终必成就}
诗篇最后仍然以对神的称颂作为结束,表明诗人尽管有疑问,仍然愿意信靠神。这是一种成熟的信仰——即使暂时看不见神的作为,仍然相信他的信实。

\subsubsection*{现实应用:}
你是否愿意在苦难中仍然向神祷告?
你是否能在等候中仍然持守对神的信靠?
\subsection*{结论:坚定信靠神,持守他的应许}
《诗篇》第89篇告诉我们:
\begin{enumerate}
    \item 神是信实的,他的应许不会落空。

    \item 神是全能的,他掌管一切,即使世界看起来混乱。

    \item 信仰的挑战是真实的,但我们可以带着疑问来到神面前。

    \item 最终,我们仍要赞美神,因为他的信实直到永远。

\end{enumerate}

弟兄姐妹,无论你现在正处于顺境还是困境,愿你都能坚定信靠神的信实。因为他从不失信,他的应许在基督里总是“是的,阿们”(哥林多后书1:20)。

\subsection*{结束祷告}
\textbf{亲爱的天父,}

我们感谢你,因为你的信实直到永远。主啊,有时候我们的环境让我们疑惑,但求你帮助我们,即使在黑暗中,也能相信你的光仍然照耀。求你加添我们的信心,使我们无论在顺境或逆境中,都能坚定信靠你的应许。愿你的名被高举,愿我们一生都歌颂你的慈爱。

奉主耶稣基督的名,阿们!
%-----------------------------------------------------------------------------
\newpage
\section{诗篇第90篇:数算我们的年日,活出智慧人生}
\subsubsection*{引言:时间的珍贵与人生的智慧}
我们常常听到人们感叹:“时间过得真快!” 无论是年少的学生,还是步入中年的父母,或是进入老年的长者,都在不知不觉中被时间推着向前。诗篇 90 篇是摩西的祷告诗,他深刻地反思了人的短暂与神的永恒,并求神教导我们如何在有限的生命中活出智慧。今天,我们一起来思考如何数算自己的年日,使我们得着智慧的心。

\subsection*{一、神的永恒 vs. 人的短暂(1-6节)——有限人生的省思}
“主啊,你世世代代作我们的居所。诸山未曾生出,地与世界你未曾造成,从亘古到永远,你是神。”(诗篇 90:1-2)

\subsubsection*{1. 神是永恒的,我们是短暂的}
摩西在诗篇开头宣告了神的永恒:“从亘古到永远,你是神。”
但接着,他描述人的生命如晨雾、如草:“早晨发芽生长,晚上割下枯干。”(诗篇 90:5-6)
\subsubsection*{现实应用:}

我们习惯把自己当作世界的中心,然而我们的生命不过是短暂的一瞬间。
我们是否把信心建立在短暂的事物上(财富、名声、成就)?还是建立在神那永恒的居所中?
挑战:你是否愿意把焦点从短暂的事物转向永恒的神?

\subsection*{二、人生的困境与罪的影响(7-11节)——认识生命的真相}
“我们因你的怒气而消灭,因你的忿怒而惊惶。你将我们的罪孽摆在你面前,将我们的隐恶摆在你面光之中。”(诗篇 90:7-8)

\subsubsection*{1. 生命的短暂不仅仅是自然规律,更是罪的后果}
罪让人类的生命变得虚空,死亡成为人类的终局。
罪不仅让人短命,也让人心里充满愁苦和痛苦(90:9-10)。
\subsubsection*{现实应用:}

现代人常常以为“人生苦短,要及时行乐”,但摩西提醒我们:“人生苦短,更要敬畏神。”
我们面对苦难时,是否有意识地去省察自己的生命?是否愿意让神的光照亮我们的生命,使我们远离罪恶?
挑战:你是否认真对待自己的属灵生命,愿意向神悔改,活出一个圣洁的生命?

\subsection*{三、智慧的生活态度(12-17节)——如何活出有意义的人生}
“求你指教我们怎样数算自己的日子,好叫我们得着智慧的心。”(诗篇 90:12)

\subsubsection*{1. 数算日子,活出智慧}
“数算”意味着要珍惜每一天,不是浪费生命,而是用智慧去度过。
人生的智慧不是来自学问、财富,而是来自敬畏神(箴言 9:10)。
\subsubsection*{2. 祈求神的恩典,经历他的满足}
“求你使我们早早饱得你的慈爱,好叫我们一生一世欢呼喜乐。”(诗篇 90:14)
这说明真正的满足不在于外在环境,而在于神的爱。
无论环境如何变迁,认识神、经历他的恩典,才是生命的最大满足。
\subsubsection*{3. 祈求神的同在,让我们的工作有价值}
“愿主——我们神的荣美归于我们身上!愿你坚立我们手所做的工!”(诗篇 90:17)
任何努力,如果没有神的同在,最终都会归于虚空。
但若神坚立我们所做的工,我们的生命就不会白白浪费。
\subsubsection*{现实应用:}

每天清晨,我们是否祷告:“主啊,求你指引我的脚步,使我今日所行的合乎你的旨意。”?
我们的事业、家庭、服事,是否建立在神的同在之中?
挑战:你是否愿意邀请神进入你的工作、家庭和人生,让他坚立你所做的?

\subsection*{结论:活在永恒的光中}
人生短暂,但在神里面,我们可以找到真正的意义。
\begin{enumerate}
    \item 认定神的永恒性,不把人生焦点放在短暂的事物上。

    \item 面对罪的影响,学会敬畏神,寻求他的恩典。

    \item 数算自己的年日,活出智慧人生。

    \item 寻求神的同在,让我们的人生有真正的价值和意义。

    
\end{enumerate}

无论你现在处在哪个人生阶段,愿这篇诗篇提醒你,把握当下,以敬畏神的态度来过每一天!

\subsection*{结束祷告}
\textbf{天父,}

我们感谢你,因你是亘古到永远的神。你是我们的避难所,是我们的盼望。主啊,我们承认自己的生命短暂、有限,我们常常被世俗的事物所吸引,而忘记了你永恒的国度。求你赦免我们的软弱,使我们学会数算自己的年日,得着智慧的心。愿你的慈爱每天充满我们,使我们经历真正的满足。愿你坚立我们手所做的工,使我们的生命在你里面结出永恒的果子。

感谢你,奉主耶稣基督的名祷告,阿们!
%-----------------------------------------------------------------------------
\newpage
\section{诗篇第91篇:住在至高者隐密处}
\subsection*{引言:在不安的世界中寻求真正的平安}
我们生活在一个充满危机和不确定的世界里。疾病、战争、经济衰退、人际关系的破裂、心理压力等,常常让我们感到无助和恐惧。作为基督徒,我们如何在混乱和不安的环境中找到真正的安全感?诗篇 91 篇给了我们一个清晰的答案——当我们住在至高者的隐密处,我们就能得着神的保护、平安和拯救。今天,我们一同来思想如何活在神的遮盖之下,经历他的恩典与能力。

\subsection*{一、住在神的隐密处——真正的安全感(1-4节)}
“住在至高者隐密处的,必住在全能者的荫下。”(诗篇 91:1)

\subsubsection*{1. 什么是“住在至高者隐密处”?}
这不是指一个物理的地方,而是指与神建立亲密的关系,时刻活在他的同在中。
这个“隐密处”就像一个坚固的堡垒,只有那些亲近神、信靠神的人才能进入并得着真正的平安。
\subsubsection*{2. 住在神的荫下意味着什么?}
“我要论到耶和华说:他是我的避难所,是我的山寨,是我的神,是我所倚靠的。”(诗篇 91:2)
“荫下”意味着庇护、遮盖、保护,就像雏鸟躲在母鸟的翅膀底下(91:4)。
神不仅是我们的避难所,更是我们人生中随时的帮助(诗篇 46:1)。
\subsubsection*{现实应用:}

我们是否每天花时间亲近神,活在他的隐密处?
当遇到困难时,我们是先找神,还是先找世界的解决方案?
挑战:每天操练灵修、祷告,建立与神亲密的关系,让自己真正活在他的荫下!

\subsection*{二、胜过恐惧与攻击——神是我们的保护者(5-13节)}
“你必不怕黑夜的惊骇,或是白日飞的箭,也不怕黑夜行的瘟疫,或是午间灭人的毒病。”(诗篇 91:5-6)

\subsubsection*{1. 恐惧来自何处?}
\begin{itemize}
    \item 环境的威胁:战争、疾病、经济危机、天灾人祸。

    \item 属灵的攻击:撒旦的试探、信仰的挑战、世俗的压力。

    \item 心理的焦虑:对未来的不确定、对失败的恐惧、对人际关系的担忧。

\end{itemize}


\subsubsection*{2. 神的保护如何彰显?}
\begin{itemize}
    \item 神的信实是我们的盾牌和保障(91:4)——在困境中,神不会让我们孤单面对,而是亲自遮盖我们。

    \item 神使他的天使四围护卫(91:11-12)——就像但以理在狮子坑中被保护一样,神的使者也会在我们需要时帮助我们。

    \item 神赐给我们胜过仇敌的能力(91:13)——“你要践踏狮子和虺蛇。” 这是神赐给信靠他之人的属灵权柄。

\end{itemize}
\subsubsection*{现实应用:}

你是否活在恐惧之中?面对环境的威胁,你是否能用信心宣告神的保护?
当面对试探时,你是否知道神的信实是你的盾牌?
挑战:不要让恐惧捆绑你,学习在祷告中依靠神,凭信心宣告他的保护!

\subsection*{三、神的应许——爱他的人必得着拯救(14-16节)}
“神说:因为他专心爱我,我就要搭救他;因为他知道我的名,我要把他安置在高处。”(诗篇 91:14)

\subsubsection*{1. 谁能得着神的拯救?}
\begin{itemize}
    \item 专心爱神的人(91:14)——不是随意跟随,而是全心倚靠、真实委身的信徒。

    \item 认识神名的人(91:14)——“认识神的名”表示对神的信赖,意味着我们与他有深厚的关系。

\end{itemize}
\subsubsection*{2. 神的七重应许}
在 91:14-16,神给出七个应许:
\begin{enumerate}
    \item 搭救(拯救我们脱离困境)。

    \item 安置在高处(使我们稳固不动摇)。

    \item 应允我们的祷告(当我们呼求时,神必回应)。

    \item 在患难中与我们同在(即使我们面对苦难,神不会丢弃我们)。

    \item 搭救并使我们尊贵(从低谷提升我们)。

    \item 赐给我们长寿(活出丰盛的人生)。

    \item 让我们看见他的救恩(最终进入永恒的荣耀)。

\end{enumerate}
\subsubsection*{现实应用:}

你是否专心爱神,而不是在顺境时才亲近他?
你是否在遇到患难时,首先想到的是寻求神的帮助,而不是靠自己的力量?
挑战:立志做一个“专心爱神”的人,凭信心领受神的七重应许!

\subsection*{结论:活在神的同在中,经历真正的平安}
\begin{enumerate}
    \item 住在神的隐密处,让神成为你唯一的安全感。

    \item 胜过恐惧,凭信心宣告神的保护与得胜。

    \item 专心爱神,得着他七重的应许。

    % \item 
\end{enumerate}

无论你今天面对什么挑战,都让这篇诗篇成为你的安慰和力量。神是你的避难所,是你的盾牌,是你的高台,在他里面,你必定得着真正的平安!

\subsection*{结束祷告}
\textbf{天父,}我们感谢你,因你是我们随时的帮助和避难所。在这个充满不安和危机的世界里,我们愿意选择住在你的隐密处,投靠在你的翅膀荫下。主啊,我们承认自己的软弱,求你帮助我们胜过恐惧,赐给我们信心,使我们经历你的保护和供应。愿你坚固我们的信心,使我们专心爱你,得着你的应许。感谢你,奉主耶稣基督的名祷告,阿们!
%-----------------------------------------------------------------------------
\newpage
\section{诗篇第92篇:称谢耶和华,因他的信实与公义}
% 讲章:
% ——诗篇 92 篇

\subsection*{引言:感恩的生命}
诗篇 92 篇是一首安息日的诗歌,它提醒我们在忙碌的生活中停下来,向神献上感恩,并思考神的信实与公义。无论是顺境还是逆境,我们都可以找到感恩的理由。今天,我们要一同思想如何过一个感恩的生命,并在神的信实与公义中得着力量和盼望。

\subsection*{一、以感恩为祭,向神歌唱(1-4节)}
“称谢耶和华,歌颂你至高者的名;用十弦的乐器和瑟,用琴弹幽雅的声音,早晨传扬你的慈爱,每夜传扬你的信实。”(诗篇 92:1-3)

\subsubsection*{1. 感恩是敬拜的核心}
诗篇 92 开篇强调,称谢耶和华是“美好的”,意味着这是一件合宜的事。敬拜的核心是感恩,而感恩的生命带出真正的喜乐和满足。

\subsubsection*{2. 何时感恩?}
\begin{itemize}
    \item 早晨传扬神的慈爱:一天的开始,我们要带着信心仰望神,相信他的恩典够我们用。

    \item 夜晚传扬神的信实:回顾一天的经历,无论顺境逆境,我们都能看见神的带领和保守。

\end{itemize}
\subsubsection*{现实应用:}

你每天是否花时间向神感恩,而不是只在困难时才寻求他?
你的敬拜是否发自内心,带着真正的喜乐和感恩?
挑战:每天操练感恩,无论环境如何,都找出值得感谢神的理由!

\subsection*{二、神的意念高深,恶人终必灭亡(5-9节)}
“耶和华啊,你的工作何其大!你的心思极其深!”(诗篇 92:5)

\subsubsection*{1. 神的意念高过人的意念}
我们常常不能理解神的作为,但神的计划远远超越我们的想象。
即使世界充满不公,我们相信神最终必彰显他的公义。
\subsubsection*{2. 恶人看似兴旺,终将灭亡}
“愚顽人不晓得,愚昧人也不明白:恶人茂盛如草,作孽之人兴旺的时候,正是他们要灭亡,直到永远。”(诗篇 92:6-7)

罪恶之人虽然看似繁荣,最终必定败亡。
他们的兴旺如“野草”,虽短暂茂盛,却不能长存。
\subsubsection*{现实应用:}

你是否因眼前的不公平而怀疑神的公义?
你是否羡慕世人的成功,却忘了他们的结局?
挑战:信靠神的公义,不因短暂的不公而动摇信心!

\subsection*{三、义人如棕树发旺,生命结果不止息(10-15节)}
“义人要发旺如棕树,生长如黎巴嫩的香柏树。”(诗篇 92:12)

\subsubsection*{1. 义人的生命像棕树和香柏树}
\begin{itemize}
    \item 棕树:象征坚韧,在干旱中仍能结果。

    \item 黎巴嫩的香柏树:象征稳固、坚强、不被环境动摇。

\end{itemize}
\subsubsection*{2. 义人植于神的殿中,终身结果}
“他们栽于耶和华的殿中,发旺在我们神的院里。他们年老的时候仍要结果子,要满了汁浆而常发青。”(诗篇 92:13-14)

义人不只是短暂的兴盛,而是持续成长。
即使年老,仍然可以结出属灵的果子,影响下一代。
\subsubsection*{现实应用:}

你是否像棕树一样,生命坚韧,能够在干旱中仍然信靠神?
你的生命是否仍然结果,影响周围的人?
挑战:扎根神的话语,持续成长,不被环境动摇!

\subsection*{结论:活出感恩与信心的生命}
\begin{enumerate}
    \item 每天操练感恩,以敬拜神为生活的核心。

    \item 信靠神的公义,不因眼前的不公而灰心。

    \item 活出坚韧的信仰,如棕树和香柏树,持续结果子。

\end{enumerate}

让我们立志,成为那位真正信靠神的人,在一切环境中都能称谢他,见证他的信实和公义!

\subsection*{结束祷告}
\textbf{天父,}

我们感谢你,因你的信实和公义永不改变。求你赐给我们感恩的心,让我们在顺境和逆境中都能向你称谢。帮助我们不被世界的短暂荣华迷惑,而是扎根于你的话语,活出坚韧的信仰,结出丰盛的果子。

愿我们的生命荣耀你,奉主耶稣基督的名祷告,阿们!
%-----------------------------------------------------------------------------
\newpage
\section{诗篇第93篇:耶和华作王,他掌管一切}
% 讲章:
% ——诗篇 93 篇

\subsection*{引言:神仍然掌权}
在这个动荡不安的世界,我们常常被各种挑战、困境甚至不公正的事情所困扰。然而,诗篇 93 篇向我们宣告了一个坚定不移的真理——耶和华作王,他掌管一切。无论世界如何变动,神的宝座永远坚立。今天,我们要一同思想神的主权、能力和信实,并在他的话语中找到我们的安稳。

\subsection*{一、耶和华作王,他的权柄永远长存(1-2节)}
“耶和华作王,他以威严为衣;耶和华以能力为衣,以能力束腰;世界就坚定,不得动摇。你的宝座从太初立定,你从亘古就有。”(诗篇 93:1-2)

\subsubsection*{1. 神的主权——他以威严为衣}
这节经文告诉我们,神不是一个被动的神,而是主动掌权的神。他的威严如同国王的袍子,彰显他的荣耀和尊贵。
世界虽充满变数,但神的国度永不动摇。当我们看见世上的战争、天灾人祸、社会不公时,我们要记得,神仍然掌权,他的计划不会被打乱。
\subsubsection*{2. 神的能力——他以能力束腰}
束腰是古代武士和工人准备行动的姿态,表明神随时准备施行他的大能。
神的能力不会衰退,也不会受环境限制,我们可以完全信靠他。
\subsubsection*{现实应用:}

你是否在困难中怀疑神是否掌权?
你是否因生活的不确定性而焦虑?
挑战:相信神的主权,无论环境如何,都仰望他的大能!

\subsection*{二、世界虽起风浪,神仍然掌权(3-4节)}
“耶和华啊,大水扬起,大水发声,波浪澎湃。耶和华在高处大有能力,胜过诸水的响声,洋海的大浪。”(诗篇 93:3-4)

\subsubsection*{1. 世界的风浪无法超越神的能力}
“大水”和“波浪”象征世界的混乱,如战争、灾难、经济危机、疾病等。
但圣经告诉我们,神在高处,他的能力大过一切风浪。
\subsubsection*{2. 神的能力远胜过人所恐惧的}
当门徒在加利利海遇到风暴时,他们害怕得要死,但耶稣站起来斥责风浪,说:“住了吧!静了吧!”(马可福音 4:39)
风浪虽然可怕,但神的声音比风浪更大!
\subsubsection*{现实应用:}

你是否正在经历人生的风暴,比如失业、疾病、家庭危机?
你是否被生活的重担压得喘不过气来?
挑战:在风浪中仍然信靠神,因为他比风浪更大!

\subsection*{三、神的法度稳固,我们当信靠他(5节)}
“耶和华啊,你的法度最的确,你的殿永称为圣,是合宜的,直到永远。”(诗篇 93:5)

\subsubsection*{1. 神的话语永不改变}
世界会变,人心会变,但神的话不会变!
他的法度(律法、教导)是“最的确的”,意思是绝对可靠的。
\subsubsection*{2. 神的圣洁应当成为我们的追求}
既然神的殿(教会)是圣洁的,我们的生命也应当如此。
当世界混乱时,我们要回到神的话语,寻求他的引导。
\subsubsection*{现实应用:}

你是否愿意顺服神的话,而不是被世界的价值观影响?
你是否相信圣经是绝对可靠的指南?
挑战:每天阅读圣经,以神的法度作为人生的方向!

\subsection*{结论:在神的主权中得安息}
\begin{enumerate}
    \item 神的主权永远长存,不受环境影响。

    \item 世界虽有风浪,神仍然掌权,他比风浪更大。

    \item 神的法度稳固,我们要信靠并顺服他。

\end{enumerate}

无论你现在面对什么困境,都请记住——耶和华作王,他掌管一切!

\subsection*{结束祷告}
\textbf{天父,}

我们感谢你,因为你作王,你掌管世界,也掌管我们的生命。求你在我们的困难中赐下信心,让我们相信你的主权永不动摇。帮助我们在风浪中仍然仰望你,并顺服你的法度。

愿你的名在我们的生命中得荣耀,奉主耶稣基督的名祷告,阿们!
%-----------------------------------------------------------------------------
\newpage
\section{诗篇第94篇:信靠公义的神——看神的审判与安慰}
% 讲章:
\subsection*{引言:面对不公,我们何去何从?}
在这个世界上,我们常常目睹不公:邪恶之人兴旺,义人受苦,腐败、不公正充斥社会。面对这样的现实,我们是否会疑惑:神在哪里?他为何不伸张公义? 诗篇 94 篇就是对这种疑问的回应,它向我们展示了神的公义审判,也提醒我们在苦难中要信靠神的安慰和拯救。

\subsection*{一、神是公义的审判者(1-7节)}
“耶和华啊,你是伸冤的神;伸冤的神啊,求你发出光来!审判世界的主啊,求你使骄傲人受报应!”(诗篇 94:1-2)

\subsubsection*{1. 神掌管审判}
诗人呼求神显出他的公义,施行报应。这提醒我们:神绝不是漠视邪恶,而是掌管一切的审判者!
“伸冤的神”说明神必报应不公,只是时候未到。
\subsubsection*{2. 罪人的狂妄与恶行}
“耶和华啊,恶人夸胜,要到几时呢?”(诗篇 94:3)
我们可能也会问:恶人为什么常常得意?
他们欺压贫穷、残害无辜,以为神看不见(4-7节)。
\subsubsection*{现实应用:}

在面对社会不公或个人遭受不义时,我们要相信:神必伸张公义!
我们或许不明白神为什么暂时容忍恶人,但他的审判一定会来,我们当信靠他。
挑战:当你面对不公时,你是否仍然相信神是公义的?

\subsection*{二、神察看一切,必按公义施行审判(8-15节)}
“你们这民间的畜类人当思想,你们这愚顽人到几时才有智慧呢?造耳朵的,难道自己不听见吗?造眼睛的,难道自己不看见吗?”(诗篇 94:8-9)

\subsubsection*{1. 神必察看人心}
许多人以为他们可以随意犯罪,因为神“看不见”。但诗篇提醒我们:神造眼睛,他岂能不看见?神造耳朵,岂能不听见?
神完全了解世人的心思意念,没有任何恶行能逃脱他的目光。
\subsubsection*{2. 神的管教是出于爱}
“耶和华知道人的意念是虚妄的。”(诗篇 94:11)
人的智慧有限,神的智慧却无穷。神有时允许苦难临到,是为了管教我们,使我们归向他(12-13节)。
“耶和华必不丢弃他的百姓,也不离弃他的产业。”(诗篇 94:14)
神不会永远容忍恶人,也不会抛弃信靠他的人!
\subsubsection*{现实应用:}

你是否有时觉得神没有听见你的祷告?神不是不听,而是在他的时间成就他的旨意!
神的管教不是惩罚,而是为了塑造我们,帮助我们信靠他。
挑战:当神未立即伸张公义时,你是否仍愿意等候他?

\subsection*{三、神是困苦人的安慰与倚靠(16-23节)}
“耶和华若不是我的帮助,我就住在寂静之中了。”(诗篇 94:17)

\subsubsection*{1. 当人无助时,神是我们的拯救}
诗人承认,若非神的帮助,他早已绝望。这提醒我们:无论何时,我们都要投靠神,而不是依赖自己。
“我心里多忧多疑,你安慰我,就使我欢乐。”(诗篇 94:19)
生活中,我们或许会经历忧虑和挣扎,但神的安慰能使我们重新得力。
\subsubsection*{2. 神最终必施行公义}
“耶和华是我的高台,我的神是我投靠的磐石。”(诗篇 94:22)
无论环境如何,我们可以确信:神是我们的保护者和避难所!
恶人终将灭亡,神的公义必得胜(23节)。
\subsubsection*{现实应用:}

你是否正在经历困难和试炼?神的安慰足够支撑你!
在面对世界的邪恶时,我们要选择信靠神,而不是被恐惧或愤怒所支配。
挑战:你是否愿意在困境中,仍然宣告神是你的避难所?

\subsection*{结论:信靠公义的神,在他的安慰中得胜}
诗篇 94 篇提醒我们:
\begin{enumerate}
    \item 神掌管审判,恶人终必受报应。

    \item 神的眼目察看一切,他的管教是出于爱。

    \item 神是我们真正的倚靠者,他的安慰能使我们得胜。
\end{enumerate}

面对不公与试炼,我们或许不明白神的作为,但我们可以确信:神的公义永不动摇,他必然审判恶人,并保护他的子民!

\subsection*{结束祷告}
\textbf{亲爱的天父,}

感谢你是公义的神,你察看一切,你必按你的时间施行审判。主啊,当我们面对不公和苦难时,求你帮助我们不失去信心,而是坚定地依靠你。愿你的安慰充满我们的心,使我们在风浪中仍然得安息。主啊,我们相信你掌管万有,你是我们唯一的避难所。

奉主耶稣基督的名祷告,阿们!
%-----------------------------------------------------------------------------
\newpage
\section{诗篇第95篇:敬拜与顺服——生命之道}
% 讲章:
\subsection*{引言:为何敬拜神?}
在忙碌的生活中,我们是否曾经问过自己:“敬拜神真的重要吗?敬拜的意义是什么?”
诗篇95篇是一篇关于敬拜和顺服的诗,
今天,我们要思想:如何真正敬拜神,并在生活中活出敬拜的态度?

\subsection*{一、欢喜敬拜神——因他是伟大的神(1-7节)}
“来啊,我们要向耶和华歌唱,向拯救我们的磐石欢呼!”(诗篇95:1)

\subsubsection*{1. 敬拜是欢喜的回应}
诗人邀请我们来到神面前,以喜乐和感恩敬拜他。敬拜不是例行公事,而是从内心发出的欢呼!
\begin{itemize}
    \item “我们要来感谢他,用诗歌向他欢呼!”(2节)——敬拜是感恩的表达。

    \item “因为耶和华是大神,是大君王,超乎万神之上。”(3节)——敬拜是对神至高主权的承认。

\end{itemize}
\subsubsection*{现实应用:}

敬拜不仅仅是唱诗歌,而是发自内心的敬畏和尊崇。
无论我们生活如何,敬拜提醒我们神是配得荣耀的君王!
挑战:你是否习惯于带着喜乐和感恩来到神面前,而不是只在困难时寻求他?

\subsubsection*{2. 敬拜是认定神是创造主}
“地的深处在他手中,山的高峰也属他。海洋属他,是他造的;旱地也是他手造成的。”(诗篇95:4-5)

神是创造天地的主,万物都属于他。
我们的一切,包括生命的气息,都是神所赐。
\subsubsection*{现实应用:}

当我们看到自然界的伟大时,我们是否想到神的荣耀?
我们是否用神的创造来荣耀他,而不是只顾自己的利益?
挑战:你是否愿意承认神是生命的主权者,并在生活中尊荣他?

\subsection*{二、敬拜是顺服——不可硬着心(8-11节)}
“你们不可硬着心,像当日在米利巴,就是在旷野的玛撒。”(诗篇95:8)

\subsubsection*{1. 以色列人的失败——心硬悖逆}
“那时你们的祖宗试探我,探我,看我的作为。”(9节)
以色列人在旷野经历了神的供应(吗哪、水、云柱火柱),却仍然不信靠他,抱怨连连。
他们“试探”神,不是因为他们不了解神,而是他们选择不信。
\subsubsection*{现实应用:}

我们是否在困境中也像以色列人一样,对神失去信心,甚至怀疑他?
“心硬”不是指无知,而是指故意不愿意听从神。
挑战:你是否在某些生活领域仍然抗拒神的带领?

\subsubsection*{2. 神的警戒——不信的代价}
“所以我在怒中起誓,说:他们断不可进入我的安息。”(诗篇95:11)

以色列人因为不信和悖逆,最终40年漂流旷野,不能进入迦南地。
“安息”不仅指迦南地,也象征属灵的安息——即在神里面的平安和满足。
如果我们不顺服神,我们也无法真正经历神的安息。
\subsubsection*{现实应用:}

我们今天是否也因不信,而无法经历神的平安?
敬拜不仅仅是唱诗,而是全心顺服神的带领!
挑战:你是否愿意放下自己的固执,完全顺服神?

\subsection*{结论:敬拜神,并活出顺服的生命}
诗篇95篇给我们两个重要的提醒:
\begin{enumerate}
    \item 敬拜是欢喜的、感恩的回应,承认神是至高的创造主。

    \item 真正的敬拜不只是表面的歌唱,而是顺服神,避免像以色列人那样悖逆不信。

\end{enumerate}

如果我们心里刚硬,就无法进入神的安息。
愿我们用心灵和诚实敬拜神,并以顺服的心回应他!

\subsection*{结束祷告}
\textbf{慈爱的天父,}

感谢你通过诗篇95篇教导我们如何敬拜你。你是伟大的创造主,你配得我们全心的赞美。求你帮助我们,不只是用口来敬拜你,更用顺服的生命来荣耀你。主啊,若我们的心曾经刚硬,求你赦免我们,使我们愿意听从你的话,进入你所赐的安息。愿你的名在我们生命中被高举。

奉主耶稣基督的名祷告,阿们!
%-----------------------------------------------------------------------------
\newpage
\section{诗篇第96篇:向耶和华唱新歌——敬拜与宣扬}
\subsection*{引言:为何要向神唱新歌?}
在我们生活中,每天都有新的挑战、新的经历,我们是否也愿意以新的心态、新的敬拜来回应神的伟大?诗篇96篇是一首充满赞美和宣扬的诗,呼召我们在敬拜中更新自己,并向万国宣扬神的荣耀。
今天,我们要思考两个问题:
如何用真实的敬拜荣耀神?
如何将神的救恩传扬出去?
\subsection*{一、敬拜神——向他唱新歌(1-6节)}
“你们要向耶和华唱新歌!全地都要向耶和华歌唱!”(诗篇96:1)

\subsubsection*{1. 敬拜要充满新鲜感}
诗人呼召我们唱“新歌”,意味着:
\begin{itemize}
    \item 更新的心态:不把敬拜当成例行公事,而是用新的感恩敬拜神。

    \item 新的经历:神每天都在我们生命中做工,我们要以新鲜的敬拜回应他。

\end{itemize}
\subsubsection*{现实应用:}

我们是否习惯性敬拜,而缺少真实的感动?
每天经历神的恩典,我们是否愿意以新的态度来敬拜他?
挑战:今天你是否带着感恩的心,唱出“新歌”来敬拜神?

\subsubsection*{2. 敬拜要传扬神的救恩}
“要向耶和华歌唱,称颂他的名!天天传扬他的救恩!”(诗篇96:2)

敬拜不仅是唱诗,更是宣扬神的救恩!
“天天传扬”——敬拜不仅限于主日,而是每天的生命表达。
神的救恩是世界最宝贵的信息,我们是否愿意传扬?
\subsubsection*{现实应用:}

我们是否只在教堂里敬拜,却在生活中不敢提及神的救恩?
如果我们真的认识神的大能,我们愿意向人传扬他的救恩!
挑战:今天你是否愿意向身边的人分享神的恩典?

\subsubsection*{3. 敬拜要承认神的伟大}
“因耶和华为大,当受极大的赞美。他在万神之上,当受敬畏。”(诗篇96:4)

神是万神之上——在古代,人们敬拜偶像,但诗人宣告,唯有耶和华是真神!
敬拜是承认神的至高权柄!
\subsubsection*{现实应用:}

在现代社会,虽然我们不拜偶像,但是否把工作、金钱、名誉看得比神更重要?
敬拜神,就是在生活中把他放在第一位!
挑战:你是否愿意把神摆在生命的首位,不被世俗的“偶像”分心?

\subsection*{二、向万民宣扬神的荣耀(7-13节)}
“要将耶和华的名所当得的荣耀归给他。”(诗篇96:8)

\subsubsection*{1. 向列国宣扬神的荣耀}
敬拜不仅是个人的,更是世界性的!
神的荣耀应该被全地认识,我们有责任向外邦人传扬他的名。
\subsubsection*{现实应用:}

基督信仰不只是“我的信仰”,而是要影响世界!
我们是否愿意为世界代祷,让更多人认识神?
挑战:你是否愿意为未得之民祷告,并参与宣教行动?

\subsubsection*{2. 预备迎接神的审判}
“因为他来了,他要审判全地。他要按公义审判世界,按他的信实审判万民。”(诗篇96:13)

神的审判是公义的,也是信实的!
世界终有一天要面对神的审判,我们是否已经预备好?
\subsubsection*{现实应用:}

如果我们知道神要审判世界,我们是否更努力地活出圣洁的生命?
我们是否在传福音上更有紧迫感?
挑战:你是否愿意调整自己,活出合神心意的生活?

\subsection*{结论:用敬拜和宣扬回应神}
诗篇96篇教导我们:
\begin{itemize}
    \item 敬拜神要有新鲜感,真实感恩,承认神的伟大。

    \item 敬拜神不仅仅是个人的,而是要宣扬神的救恩,让万国认识他。

    \item 神的审判是公义的,我们要预备自己,并帮助更多人认识他。

\end{itemize}

让我们带着敬畏与欢喜,向神唱新歌,并活出敬拜的生命!

\subsection*{结束祷告}
\textbf{亲爱的天父,}

感谢你启示我们诗篇96篇的真理。求你更新我们的敬拜,使我们以真实的心来歌颂你,不只是嘴唇的敬拜,更是生命的献上。主啊,求你赐给我们勇气,让我们向世界传扬你的救恩,不惧怕,不退缩。求你帮助我们,预备迎接你的公义审判,使我们的生命合乎你的心意。

奉耶稣基督的名祷告,阿们!
%-----------------------------------------------------------------------------
\newpage
\section{诗篇第97篇:耶和华作王,愿地快乐}
% 讲章: ——诗篇97篇的启示
\subsection*{引言:你真正相信神在掌权吗?}
\hspace{0.6cm}当我们面对世界的混乱、社会的不公、生活的苦难时,我们很容易疑惑:\textbf{神真的在掌权吗?}诗篇97篇给了我们坚定的答案:“耶和华作王,愿地快乐!”(诗97:1)。

本篇诗篇是一首赞美神掌权的诗歌,提醒我们:

神的统治是公义和公正的,因此我们可以信靠他。

敬畏神的人要弃绝恶行,持守圣洁。

义人终得喜乐,神必保守他的子民。

让我们深入思想诗篇97篇,看看它如何影响我们的信仰与生活。

\subsection*{一、神掌权——全地当欢喜(1-6节)}
“耶和华作王,愿地快乐!愿众海岛欢喜!”(诗97:1)

\subsubsection*{1. 神的统治是普世性的}
“愿地快乐!愿众海岛欢喜!”
神的掌权不仅限于以色列,而是全地都应当因他的王权而喜乐。
无论世界如何动荡,神始终掌权。
\subsubsection*{现实应用:}

今天的世界充满不确定性(战争、疾病、经济危机),但我们要相信,神仍然坐在宝座上,他的国度永不动摇!
当世界恐惧时,我们是否愿意选择信靠神,而非焦虑?
挑战:今天你是否愿意在困境中依然选择喜乐,因为神掌权?

\subsubsection*{2. 神的统治是公义的}
“公义和公平是他宝座的根基。”(诗97:2)

神的掌权不是随意的,而是建立在公义和公平之上。
世上的政府可能不公正,但神永远公正!
\subsubsection*{现实应用:}

当我们看到社会的不公、腐败、欺压时,我们是否依然相信神最终会伸张公义?
我们是否愿意用公义的标准来行事,而不是随波逐流?
挑战:在生活中,你是否愿意站在公义的一方,即使要付出代价?

\subsection*{二、神必毁灭偶像与恶人(7-9节)}
“恨恶耶和华的,都必羞愧。神在万神之上。”(诗97:7,9)

\subsubsection*{1. 偶像的崩溃}
世界上有许多“假神”(金钱、权力、娱乐、科技)
但当神显现时,一切虚假的信靠都会崩溃,唯有他是至高的!
\subsubsection*{现实应用:}

我们是否也有自己“敬拜”的偶像?
我们是否过度依赖金钱、工作、社交媒体,而忽视了神?
挑战:今天你愿意彻底信靠神,不再被世俗的“偶像”捆绑吗?

\subsection*{三、义人的喜乐与保守(10-12节)}
“你们爱耶和华的,都当恨恶罪恶。”(诗97:10)

\subsubsection*{1. 义人当远离罪恶}
真正爱神的人,就会憎恶罪恶。
如果我们不厌恶罪恶,说明我们对神的爱还不够深。
\subsubsection*{现实应用:}

我们是否在某些罪上妥协?(撒谎、小偷小摸、色情、嫉妒、愤怒等)
我们是否真正厌恶罪,还是有时觉得它“没什么大不了”?
挑战:你是否愿意认真对待罪,并靠神的恩典胜过它?

\subsubsection*{2. 义人的光明和喜乐}
“义人必享光明和喜乐。”(诗97:11)

尽管世界充满黑暗,但神赐给义人光明。
神最终必带领敬虔的人进入永恒的喜乐。
\subsubsection*{现实应用:}

当你经历苦难时,是否仍相信神会带你走出黑暗?
你是否愿意用感恩的心去面对生活中的挑战?
挑战:今天,无论你的环境如何,你是否愿意因神的同在而喜乐?

\subsection*{结论:因耶和华作王而欢喜}
诗篇97篇提醒我们:
\begin{enumerate}
    \item 神掌权,全地都应当喜乐!
    \item 神的统治是公义的,我们要持守公义,弃绝恶行。

    \item 偶像终将崩溃,我们要单单敬拜神。

    \item 神必保守义人,使他们在黑暗中仍有光明和喜乐。

\end{enumerate}


无论世界如何改变,神的王权永不动摇!让我们存敬畏与喜乐的心,向他献上全然的敬拜!

\subsection*{结束祷告}
\textbf{亲爱的天父,}

感谢你借着诗篇97篇向我们启示你的王权。你掌管万有,求你赐给我们信心,使我们在动荡的世界中仍能喜乐。主啊,求你帮助我们远离罪恶,单单敬拜你。愿你的光照亮我们的生命,使我们在黑暗中仍能靠你喜乐。

奉耶稣基督的名祷告,阿们!
%-----------------------------------------------------------------------------
\newpage
\section{诗篇第98篇:以新歌颂扬耶和华的救恩}
% 讲章: ——诗篇98篇
\subsection*{引言:你最近有为神做过新事而欢喜吗?}
生活中,我们常常因为新事物感到兴奋,比如新手机、新工作、新的旅行体验。但我们是否同样因神的救恩、他的作为而欢喜、颂赞呢?
\textbf{诗篇98篇是一首欢呼神的诗歌,强调神奇妙的救恩、信实的公义,以及全地当如何敬拜他。}今天,我们要思考:什么是真正的敬拜?为什么要向神唱新歌?
让我们一同进入诗篇98篇,看神如何启示我们。

\subsection*{一、神的救恩值得我们用新歌颂赞(1-3节)}
“你们要向耶和华唱新歌,因为他行过奇妙的事。”(诗98:1)

\subsubsection*{1. 何为“新歌”?}
这里的“新歌”不仅仅指一首新的旋律或歌词,而是因神新的作为而发自内心的敬拜!
每天,我们都经历神新的恩典,因此,我们的敬拜也应当是新鲜的、充满感恩的!
\subsubsection*{现实应用:}

我们是否对神的作为感到麻木?是否只在遇到大事时才感谢他?
每天早晨,我们是否愿意以感恩的心赞美神?
挑战:今天,你愿意用新的感恩、新的敬拜,向神献上一首“新歌”吗?

\subsubsection*{2. 神的救恩是奇妙的}
“他的右手和圣臂施行救恩。”(诗98:1)

这里强调神亲自施行拯救,不是人努力得来的!
神的救恩在历史上最伟大的彰显就是耶稣基督的降生、受死和复活!
\subsubsection*{现实应用:}

你是否真正因基督的救恩而感恩?
还是觉得信仰只是一个传统,而不是你生命的真实经历?
挑战:今天,你是否愿意重新思想神的救恩,并以全新的热情来敬拜他?

\subsection*{二、神的公义值得全地欢呼(4-6节)}
“全地都要向耶和华欢呼。”(诗98:4)

\subsubsection*{1. 敬拜神不仅是个人的,也是普世的}
诗篇98篇让我们看到敬拜神不只是个人的事情,而是全地的事情!
天地万物都在宣告神的荣耀,我们更应当加入这场敬拜的盛宴!
\subsubsection*{现实应用:}

你是否在生活中以行动敬拜神?还是只有在教会敬拜时才有敬虔的心?
你是否愿意在工作、学习、家庭中,用生命见证神的荣耀?
挑战:敬拜不仅仅是唱诗,而是生活的态度。你今天是否用你的言行来荣耀神?

\subsubsection*{2. 用乐器和欢呼赞美神}
“用琴歌颂耶和华。”(诗98:5)

诗篇鼓励我们用各种方式敬拜神——音乐、诗歌、甚至欢呼!
这提醒我们:敬拜神应当是热情的、真实的!
\subsubsection*{现实应用:}

你在敬拜神时,是否只是机械地跟唱,还是全心全意地投入?
你是否愿意用各种方式敬拜神,如祷告、写诗歌、画画、舞蹈等?
挑战:你今天可以用什么方式来敬拜神,让你的敬拜更有生命力?

\subsection*{三、神的公义审判值得全地敬畏(7-9节)}
“他要按公义审判世界。”(诗98:9)

\subsubsection*{1. 神是公义的审判者}
诗篇98篇不仅谈到神的救恩,也提醒我们:神最终要审判世界!
这对敬畏神的人是安慰,对抵挡神的人是警告。
\subsubsection*{现实应用:}

你是否对世界的罪恶感到愤怒和无力?请相信,神必按公义审判一切。
你是否在自己的生活中行公义、好怜悯,活出神的心意?
挑战:你是否愿意活出公义的生命,预备迎接神的审判?

\subsection*{结论:以新歌迎接神的救恩与公义}
诗篇98篇提醒我们:
\begin{enumerate}
    \item 神的救恩是奇妙的,我们要以新歌颂赞他!

    \item 神的公义值得我们全心敬拜,无论是个人还是全地!

    \item 神的审判提醒我们要活出公义的生命,预备迎见他!

\end{enumerate}

今天,让我们以新的热情、新的感恩、新的敬拜,来荣耀我们的神!

\subsection*{结束祷告}
\textbf{亲爱的天父,}

感谢你赐下诗篇98篇,让我们看见你的救恩何等奇妙!主啊,求你更新我们的敬拜,使我们每天都用“新歌”赞美你,不让敬拜变成枯燥的习惯。愿我们的生命成为你的荣耀,愿万国万民都来向你欢呼。感谢你的公义,使我们在混乱的世界中仍然有盼望。求你帮助我们持守正直,迎接你的再来!

奉耶稣基督的名祷告,阿们!
%-----------------------------------------------------------------------------
\newpage
\section{诗篇第99篇:敬畏、尊崇至高的王}
% 讲章: ——诗篇99篇
\subsection*{引言:你如何看待神的圣洁?}
当我们提到“神是圣洁的”,你会有什么反应?是敬畏,还是无动于衷?
在这个世界,人们常常强调爱与宽容,却忽略了神的圣洁与公义。
诗篇99篇告诉我们:\textbf{耶和华是至高的王,他是圣洁的,配得全地敬畏与尊崇!}今天,让我们一同深入思想这篇诗篇,学习如何敬畏、尊崇这位圣洁的神。

\subsection*{一、神是掌权的王,万民当敬畏他(1-3节)}
“耶和华作王,万民当战抖!”(诗99:1)

\subsubsection*{1. 神的统治值得敬畏}
“基路伯上坐着”,表明神的同在与荣耀(出25:22)。
神不是一个遥远的观念,而是实际掌管世界、施行公义的王!
当神掌权时,世界不会混乱,万民当战抖、敬畏他!
\subsubsection*{现实应用:}

你是否真的相信神掌权?还是只在顺境时才信靠他?
当世界动荡(如战争、灾难、经济危机)时,你会恐惧,还是仍然信靠神?
挑战:今天,无论环境如何,信靠神是至高掌权的王!

\subsubsection*{2. 神的圣名值得尊崇}
“当称赞他大而可畏的名,他本为圣!”(诗99:3)

神的名字代表他的权柄、荣耀、圣洁。
但今天,世界对神的名字充满轻慢,不尊重、不敬畏。
\subsubsection*{现实应用:}

你是否在言语中敬畏神?还是随意使用“上帝”、“耶稣”作为口头禅?
你是否在生活中活出对神的尊崇?
挑战:用你的言行,向世人见证神的名是圣洁的!

\subsection*{二、神是公义的王,他必按正直施行审判(4-5节)}
“王有能力,喜爱公平。”(诗99:4)

\subsubsection*{1. 神的审判是公义的}
神不仅是大能的王,更是喜爱公义的王!
他在以色列中立定公平、施行公正,不会偏袒、不受贿赂!
\subsubsection*{现实应用:}

你是否也渴望在世界中看到公义?当你看到不公时,你如何回应?
你在自己的生活中,是否也秉持正直、公平,还是随波逐流?
挑战:效法神,成为正直公义的人!

\subsubsection*{2. 我们当尊崇神的圣洁}
“当尊崇耶和华我们的神,在他脚凳前下拜。”(诗99:5)

敬拜神不仅是唱歌,而是以行动尊崇他!
“在他脚凳前下拜”,表示完全降服在神的主权之下。
\subsubsection*{现实应用:}

你是否愿意顺服神,还是只愿意神按你的方式做事?
你是否以生命敬拜神,而不仅仅是口头上的信仰?
挑战:每天在生活中,以敬畏和顺服的心尊崇神!

\subsection*{三、神是圣洁的神,他愿与人亲近(6-9节)}
“耶和华我们的神啊,你应允他们;你是赦免他们的神。”(诗99:8)

\subsubsection*{1. 神的圣洁不等于遥远}
摩西、亚伦、撒母耳 都是神忠心的仆人,他们“求告耶和华,他就应允他们”(诗99:6)。
这表明,尽管神是圣洁的,但他愿意与人建立亲密的关系!
\subsubsection*{现实应用:}

你是否觉得神是遥不可及的?其实,他渴望与你建立亲密关系!
你是否愿意像摩西、撒母耳一样,常常向神祷告?
挑战:每天与神亲近,体验他的信实与恩典!

\subsubsection*{2. 神既公义,也满有怜悯}
“你是赦免他们的神,却按他们所行的报应他们。”(诗99:8)

神的公义和怜悯并不矛盾!他既愿意赦免人,也按公义审判罪恶。
今天,最大的赦免就是在耶稣基督里得到的救恩!
\subsubsection*{现实应用:}

你是否真正悔改,接受神的赦免?还是继续活在罪中?
你是否在对待别人时,也愿意饶恕和怜悯?
挑战:活在神的赦免中,并且饶恕他人!

\subsection*{结论:回应神的圣洁与公义}
诗篇99篇提醒我们:
\begin{enumerate}
    \item 神是掌权的王,我们当敬畏他!

    \item 神是公义的王,我们当顺服他!

    \item 神是圣洁的神,我们当亲近他!

\end{enumerate}

今天,让我们用敬畏、顺服和亲近的心,来回应神的圣洁!

\subsection*{结束祷告}
\textbf{亲爱的天父,}

感谢你藉着诗篇99篇提醒我们,你是至高掌权的王,你是公义审判的主,你也是圣洁但愿意亲近我们的神!主啊,我们愿意在你面前谦卑俯伏,求你炼净我们的心,使我们以敬畏的心尊崇你、顺服你、亲近你。帮助我们在这个世界活出你的公义,也让我们因着你的恩典而活在敬拜和感恩中。

奉耶稣基督的名祷告,阿们!
%-----------------------------------------------------------------------------
\newpage
\section{诗篇第100篇:以感恩和喜乐敬拜神}
% 讲章: ——诗篇100篇
\subsection*{引言:你是带着感恩还是抱怨来敬拜?}
在我们日常生活中,我们是否常常抱怨,觉得不够幸福?还是我们能常存感恩,喜乐地敬拜神?
诗篇100篇是一首感恩的诗歌,鼓励我们用喜乐、感恩和敬拜的心来到神的面前。今天,我们一同学习如何在生活中以感恩和喜乐敬拜神。

\subsection*{一、带着喜乐敬拜神(1-2节)}
“全地当向耶和华欢呼! 你们当乐意侍奉耶和华;当来向他歌唱!”(诗100:1-2)

\subsubsection*{1. 全地当向耶和华欢呼}
这里的“全地”不是指某个民族或群体,而是所有人类,都被呼召来敬拜神!
欢呼意味着充满喜乐和热情,而不是冷漠或勉强。
许多人敬拜神时,习惯于形式化、机械化,甚至是出于责任感,而非发自内心的喜乐。
\subsubsection*{现实应用:}

你是否常常以冷漠的心敬拜神?是否只是例行公事?
你是否因生活的难处,而失去了对神的喜乐?
挑战:即使在困难中,也选择用喜乐的心来敬拜神!

\subsubsection*{2. 带着乐意的心服侍神}
“乐意侍奉耶和华”,意味着我们敬拜神不是勉强的,而是甘心乐意的。
服侍神不是一种负担,而是发自内心的回应。
我们在教会或生活中服侍时,是否带着勉强的态度,还是甘心乐意?
\subsubsection*{现实应用:}

你在服侍神时,是否带着热情,还是觉得只是义务?
你是否看到敬拜和服侍神是特权,而不是负担?
挑战:调整你的心态,以甘心乐意的态度服侍神!

\subsection*{二、认定耶和华是神(3节)}
“你们当晓得耶和华是神!我们是他造的,也是属他的;我们是他的民,也是他草场的羊。”(诗100:3)

\subsubsection*{1. 确认神是我们的创造主}
我们敬拜的对象是谁?是自己的欲望、金钱、名声,还是创造我们的耶和华?
神是创造我们的主,我们是属他的!
\subsubsection*{现实应用:}

你是否真正相信神创造了你,并掌管你的生命?
你是否仍然把自己当作“人生的主宰”,而没有顺服神的带领?
挑战:每一天提醒自己,“耶和华是神,我是属他的!”**

\subsubsection*{2. 我们是神草场的羊}
这提醒我们:神是我们的牧者,我们是他的羊!
这意味着神看顾、供应、保护我们。
但作为羊,我们是否愿意顺服他的带领?还是常常偏行己路?
\subsubsection*{现实应用:}

你是否愿意信靠神,相信他的带领,而不是依靠自己的聪明?
你是否因着环境变化,就怀疑神的供应?
挑战:无论环境如何,\textbf{“我是神的羊,他必供应我!”}

\subsection*{三、以感恩进入神的同在(4节)}
“当称谢进入他的门;当赞美进入他的院。当感谢他,称颂他的名!”(诗100:4)

\subsubsection*{1. 感恩是进入神同在的钥匙}
这里的“门”和“院”指的是圣殿,象征着亲近神的方式。
感恩是亲近神的重要态度!
许多人来到神面前,带着抱怨,而不是感恩。
\subsubsection*{现实应用:}

你是以抱怨还是感恩的心来到神面前?
你是否能在生活中培养感恩的习惯?
挑战:每天花时间感谢神,即使在困难中,也找出值得感恩的事情!

\subsection*{四、神的慈爱永远长存(5节)}
“因为耶和华本为善,他的慈爱存到永远;他的信实直到万代。”(诗100:5)

\subsubsection*{1. 神的善良不会改变}
这节经文总结了前面的内容:神是良善的,他的爱永不止息!
无论环境如何,神的本质不会改变,他始终是良善的。
这意味着,即使我们经历苦难,神的爱仍然在我们身上!
\subsubsection*{现实应用:}

你是否曾因生活的挑战,而怀疑神的良善?
你是否在困难中,仍然选择相信神的信实?
挑战:今天,无论遇到什么困难,宣告“神是良善的,他的慈爱永远长存!”

\subsection*{结论:让我们以感恩和喜乐敬拜神!}
诗篇100篇教导我们:
\begin{enumerate}
    \item 用喜乐敬拜神!(不是冷漠或勉强)

    \item 认定耶和华是我们的神!(我们是属他的羊)

    \item 以感恩的心进入神的同在!(感恩是亲近神的钥匙)

    \item 信靠神的慈爱和信实!(他的良善永不改变)

\end{enumerate}

今天,就让我们选择带着感恩和喜乐的心敬拜神!

\subsection*{结束祷告}
\textbf{亲爱的天父,}

感谢你透过诗篇100篇提醒我们,要以喜乐、感恩和敬拜的心来到你的面前。主啊,我们承认有时我们敬拜时冷漠、服侍时勉强、面对挑战时抱怨,但你依然是掌权的神,你是良善的牧者,带领我们走义路。主啊,帮助我们成为甘心乐意敬拜你的人,愿我们的生命充满感恩,荣耀你的圣名。

奉主耶稣基督的名祷告,阿们!
%-----------------------------------------------------------------------------
\newpage
\section{诗篇第101篇:活出圣洁正直的生命}
% 讲章: ——诗篇101篇
\subsection*{引言:你如何在世上活出正直的生命?}
我们每天面对社会的诱惑、世界的价值观、人与人之间的纷争,如何才能持守正直、活出圣洁?诗篇101篇是大卫王在登基时所立的誓言,他立志以公义治理国家,并持守个人的圣洁与正直。今天,我们从这篇诗篇学习如何在生活中活出圣洁正直的生命,无论是在家庭、工作、社交,还是个人操守上,都遵行神的旨意。

\subsection*{一、以敬畏神的心立志过圣洁的生活(1-2节)}
“我要歌唱慈爱和公平;耶和华啊,我要向你歌颂!我要用智慧行完全的道。你几时到我这里来呢?我要存完全的心行在我家中。”(诗101:1-2)

\subsubsection*{1. 以敬拜开始每一天}
\hspace{0.6cm}大卫的第一句话是“我要歌唱慈爱和公平”,他明白敬拜神是活出正直生活的根基。
\textbf{慈爱(怜悯)和公平(公义)}是神的属性,也是我们当效法的。

现实应用:
你是否常常在日常生活中敬拜神,还是只有在聚会时才敬拜?
你是否愿意在家中、职场、社交圈里,活出神的慈爱与公义?
挑战:每天开始时,以敬拜和祷告献上你的心,并求神帮助你活出圣洁正直的生命!

\subsubsection*{2. 立志行完全的道}
\hspace{0.4cm}“我要用智慧行完全的道”:大卫渴望行事为人合神心意,而非随从世俗。

“我要存完全的心行在我家中”:真正的敬虔首先从家庭开始,一个人的属灵生命最真实的考验,就是他如何对待家人。

现实应用:
你是否在家庭中也活出圣洁正直的生命,还是在外面敬虔、在家里却放纵?
你是否愿意靠神的智慧,在家中、工作中、社交中,行完全的道?
挑战:今天就开始,在家人面前活出敬虔的生命!

\subsection*{二、拒绝罪恶,守护自己的心(3-4节)}
“我必不摆在我眼前邪僻的事;悖逆人所做的事,我甚恨恶,不容沾在我身上。弯曲的心思,我必远离;一切的恶人,我不认识。”(诗101:3-4)

\subsubsection*{1. 远离邪恶,不让罪影响自己}
\hspace{0.4cm}“我必不摆在我眼前邪僻的事”:我们眼睛所看的,会影响我们的思想和行为。

“悖逆人所做的事,我甚恨恶”:大卫表达了他对罪的厌恶,而不是妥协。

现实应用:
你是否在娱乐、社交媒体、电影、游戏等方面,容让邪恶的事进入你的眼目?
你是否有勇气拒绝妥协,而是在罪面前站立得稳?
挑战:今天起,决定远离一切可能使你远离神的事物!

\subsubsection*{2. 远离弯曲的心思}
\hspace{0.4cm}“弯曲的心思,我必远离”:罪常常从内心开始,我们需要保守自己的心思意念。

“一切的恶人,我不认识”:不是指完全不与罪人来往,而是不与他们同行,不认同他们的恶行。

现实应用:
你是否在朋友圈中受到不良影响,渐渐妥协?
你是否愿意选择远离一切带你走向罪恶的人或事?
挑战:求神洁净你的心,让你从思想上就远离罪恶!

\subsection*{三、坚持公义,选择与正直的人同行(5-8节)}
“暗中谗谤他邻舍的,我必将他灭绝;眼目高傲、心里骄纵的,我必不容他。我眼要看国中的诚实人,叫他们与我同住;行为完全的,他要侍候我。行诡诈的,必不得住在我家里;说谎话的,必不得立在我眼前。”(诗101:5-7)

\subsubsection*{1. 避免高傲与谎言}
\hspace{0.6cm}大卫拒绝骄傲的人和说谎的人,因为他们的行为与神的性情不合。
谎言、骄傲、诡诈,会破坏人与人之间的信任,也会使我们远离神。

现实应用:
你是否曾因害怕受责备,而选择说谎?
你是否曾因自己的才华、成就,而骄傲自满?
挑战:祷告求神赐你谦卑、诚实的心,远离谎言和骄傲!

\subsubsection*{2. 选择与敬虔的人同行}
\hspace{0.4cm}“我眼要看国中的诚实人,叫他们与我同住”:大卫选择与敬虔、诚实的人同行。
我们所亲近的人,决定了我们的生命方向。

现实应用:
你是否愿意选择那些能帮助你灵命成长的朋友,而非带你远离神的人?
你是否愿意加入一个敬虔的团契,与弟兄姐妹一起成长?
挑战:建立一个敬虔的朋友圈,彼此鼓励,彼此造就!

\subsection*{四、每日操练圣洁,持守到底(8节)}
“我每日早晨要灭绝国中所有的恶人,好把一切作孽的从耶和华的城里剪除。”(诗101:8)

\subsubsection*{1. 持续追求圣洁}
\hspace{0.4cm}“每日早晨”:活出圣洁正直的生命,不是一蹴而就的,而是每天的选择!
大卫不只是一次决定要圣洁,而是每日都立志远离罪恶。

现实应用:
你是否愿意每天花时间亲近神,求神帮助你活出圣洁?
你是否每天检查自己的行为,是否符合神的标准?
挑战:每天早晨,向神祷告,立志持守圣洁正直的生命!

\subsection*{结论:活出圣洁正直的生命}
诗篇101篇给我们四个重要的提醒:
\begin{enumerate}
    \item 敬拜神,立志过圣洁的生活(1-2节)

    \item 拒绝罪恶,守护自己的心(3-4节)

    \item 坚持公义,选择与正直的人同行(5-7节)

    \item 每日操练圣洁,持守到底(8节)

\end{enumerate}

今天,你愿意立志活出正直圣洁的生命吗?

\subsection*{结束祷告}
\textbf{亲爱的天父,}

感谢你透过诗篇101篇教导我们如何活出圣洁正直的生命。主啊,帮助我们每天选择敬拜你,拒绝罪恶,远离骄傲与谎言,并且与敬虔的人同行。求你保守我们的心,使我们在你的道路上持守到底,活出你的荣耀!

奉主耶稣基督的名祷告,阿们!
%-----------------------------------------------------------------------------
\newpage
\section{诗篇第102篇:从苦难到盼望——安慰与力量}
% 讲章:

\subsection*{引言}
\hspace{0.6cm}在我们的生命旅程中,难免会遇到低谷——挫折、孤独、疾病、失败,甚至绝望。在这些时刻,我们可能会像诗篇102篇的诗人一样,发出痛苦的呼喊:“耶和华啊,求你听我的祷告,容我的呼求达到你面前!”(诗102:1)

诗篇102篇是一篇“困苦人发昏的时候,在耶和华面前吐露苦情的祷告”。它真实地展现了人的痛苦,但更重要的是,它指引我们从痛苦中仰望神,寻找盼望。今天,我们就来剖析这篇诗篇,看看如何在现实生活中应用它,从困境中得着力量和安慰。

\subsection*{一、面对人生的苦难(1-11节)}
诗人一开始便在神面前倾诉自己的痛苦:

身体的痛苦(3-5节):他感到生命像烟一样消散,骨头焦烧如火,甚至因痛苦连饭都吃不下。

心理的折磨(6-7节):他感到孤独,像荒野的鹈鹕、废墟中的鸽子,无人理解。

外界的攻击(8节):仇敌辱骂他,让他雪上加霜。

神的隐面(10节):他觉得自己被神丢弃,完全无助。

\subsubsection*{现实生活中的应用:}
我们是否也有类似的经历?有时我们会经历家庭的破裂、事业的失败、疾病的折磨,甚至精神上的孤独。在这样的时刻,我们可能会觉得神远离我们,甚至怀疑他的存在。但诗篇102篇告诉我们:即使在最深的痛苦中,我们仍可以向神呼求!

\subsubsection*{实际应用建议:}

\hspace{0.6cm}在痛苦中学会倾诉:不要憋在心里,可以向神祷告,也可以找信赖的朋友、属灵的长辈分享你的挣扎。

正视情绪,不要逃避:像诗人一样,把真实的感受告诉神,而不是压抑。神不会因我们的软弱而远离,反而会更怜悯我们。
\subsection*{二、转向神的信实(12-22节)}
在痛苦的低谷中,诗人突然改变了视角,从自己转向神:“惟你耶和华必存到永远,你可记念的名也存到万代。”(诗102:12)

\subsubsection*{诗人意识到:}

\hspace{0.6cm}神是永恒的(12节):我们的生命短暂,但神永远不变。

神是怜悯的(13节):他必然起来,施恩给他的百姓。

神的计划远超我们(15-16节):神正在建造他的国度,即使我们看不到,他的手仍在运行。

神垂听人的呼求(17节):他不会忽略困苦人的祷告。

\subsubsection*{现实生活中的应用:}
当我们感到迷茫无助时,可以试着将眼光从自己的苦难转向神的信实。

学习数算恩典:每天写下神如何在小事上供应你,比如一次及时的帮助、一个安慰的言语,这些都是神的作为。

信靠神的时间表:神的拯救可能不会立刻发生,但他有最完美的计划。想想约瑟,他经历了13年的奴役和监牢,但最终成为埃及的宰相。
\subsection*{三、在神的永恒中得安慰(23-28节)}
诗篇的最后部分,诗人意识到自己的生命短暂,但神的年数存到永远。即便个人生命有限,神的国度仍会持续发展,神的信实不会改变。

\subsubsection*{现实生活中的应用:}

\hspace{0.6cm}接受生命的有限:我们无法掌控一切,但我们可以依靠那位掌管万有的神。

建立永恒的价值观:当面对人生抉择时,我们可以问自己:“这件事对永恒来说有价值吗?”比如,与家人和好的行动、帮助有需要的人、传扬福音等。
\subsection*{结语:如何在痛苦中抓住盼望?}
\begin{enumerate}
    \item 向神倾诉——不要隐藏自己的痛苦,勇敢向神祷告。

    \item 定睛神的信实——相信神的应许,即使环境没有改变,神仍然掌权。

    \item 活在永恒的盼望中——不要让短暂的苦难吞噬你的信心,神的国度仍在前行。

\end{enumerate}
\subsection*{结束祷告}
\textbf{亲爱的天父,}

感谢你透过诗篇102篇提醒我们,即使在最深的苦难中,你仍然聆听我们的呼求。主啊,我们有时会感到孤独、痛苦,甚至看不到希望,但求你帮助我们在绝望中仰望你,定睛于你的信实。你是永恒的神,你的应许不会改变,你的爱永远长存。

求你赐给我们力量,让我们在苦难中不丧志,而是更加依靠你。帮助我们学习数算你的恩典,在你永恒的国度中找到真正的盼望。无论环境如何,我们愿意相信你的带领,直到见你的那日。

奉耶稣基督的名祷告,阿们!
%-----------------------------------------------------------------------------
\newpage
\section{诗篇第103篇:神的恩典与人的回应}
% 讲章:从诗篇103篇看
\subsection*{引言:}
\hspace{0.6cm}亲爱的弟兄姐妹,今天我们一起来思想《诗篇》第103篇。这是一篇大卫的诗,充满了对神的赞美,提醒我们不要忘记他一切的恩惠。我们生活在一个快节奏、充满挑战的世界,常常会被忧虑和压力淹没,但大卫在这里呼吁我们的心灵回转,记念神的恩典,并活出感恩的生命。

让我们一起来阅读诗篇103篇1-5节:
\begin{quote}
    1 我的心哪,你要称颂耶和华!凡在我里面的,也要称颂他的圣名!
    
    2 我的心哪,你要称颂耶和华,不可忘记他一切的恩惠!
    
    3 他赦免你的一切罪孽,医治你的一切疾病。
    
    4 他救赎你的命脱离死亡,以仁爱和慈悲为你的冠冕。
    
    5 他用美物使你所愿的得以知足,以致你如鹰返老还童。
\end{quote}


\subsection*{第一部分:神的恩典何等浩大}

\subsubsection*{1. 他赦免我们的罪孽(第3节)}
在现实生活中,我们都会犯错,有时是言语上的伤害,有时是心里的不洁,有时是对神的远离。但神满有怜悯,愿意赦免我们一切的罪孽。\textbf{你是否曾经经历过神的赦免?}当我们真诚悔改时,他总是愿意接纳我们,就像浪子回头的故事(路加福音15章)。

\subsubsection*{2. 他医治我们的疾病(第3节)}
这里的“医治”不仅指身体的疾病,也包括心灵的医治。很多人虽然外表健康,但内心却被焦虑、恐惧和伤害捆绑。然而,耶稣曾说:“凡劳苦担重担的人,可以到我这里来,我就使你们得安息。”(马太福音11:28)我们要学会把内心的重担交托给神,相信他是医治者。

\subsubsection*{3. 他救赎我们的生命(第4节)}
罪的工价乃是死(罗马书6:23),但神却用他的爱救赎了我们,赐我们永生。许多人在生活的压力和罪的辖制下挣扎,甚至失去生命的意义。但神不仅拯救我们的灵魂,也赐我们新的生命,让我们在基督里成为新造的人(哥林多后书5:17)。

\subsubsection*{4. 他以慈爱和怜悯待我们(第4节)}
现实生活中,我们常常经历人的冷漠和不公,但神的爱却永不止息。他的爱像父亲怜悯儿女一样(诗篇103:13),不会轻易责备,反而时时引导我们走义路。

\subsubsection*{5. 他使我们如鹰返老还童(第5节)}
在生活中,我们有时会感到疲乏、失去盼望,但神能使我们重新得力。就像鹰在经历蜕变之后能重新展翅高飞一样,神能更新我们的生命,使我们在他的恩典中再次充满力量。

\subsection*{第二部分:我们的回应——活出感恩的生命}

\subsubsection*{1. 不忘记神的恩惠(第2节)}
人的本性容易健忘,我们常常记住痛苦,却忘记神的恩典。我们可以养成感恩的习惯,比如每天写下三件值得感恩的事情,并在祷告中向神感恩。

\subsubsection*{2. 用敬拜回应神的恩典(第1节)}
大卫提醒自己“我的心哪,你要称颂耶和华!”敬拜不只是唱诗,更是一种生活方式——用我们的言行、品格、服侍来荣耀神。

\subsubsection*{3. 效法神的怜悯,去爱别人(第8-13节)}
神宽容我们,我们也应当宽恕他人。现实中,我们可能遇到过不公平的对待,但神提醒我们:“他没有按我们的罪过待我们,也没有照我们的罪孽报应我们。”(第10节)我们要学习放下怨恨,以爱待人。

\subsubsection*{4. 信靠神,面对未来(第15-18节)}
诗篇103篇提醒我们,人的生命如草,短暂易逝,但神的慈爱却存到永远。因此,我们不必忧虑未来,而是要信靠神,专心跟随他。

\subsection*{结论:活在神的恩典中}

诗篇103篇是一首提醒我们感恩的诗,帮助我们在日常生活中时刻记住神的恩典。当我们回顾过去,会发现神的手一直牵引我们;当我们面对未来,我们可以充满信心,因为神的慈爱永不改变。愿我们每个人都活出感恩、敬拜和信靠的生命!

\subsection*{结束祷告}

\textbf{亲爱的天父,}

感谢你赦免我们的罪孽,医治我们的疾病,拯救我们的生命,并以慈爱和怜悯待我们。求你帮助我们不忘记你的恩惠,时时称颂你的名。无论在顺境还是逆境中,都愿我们的心紧紧依靠你,活出感恩的生命。愿你赐下平安,医治我们内心的伤痛,使我们重新得力,如鹰展翅上腾。

奉主耶稣基督的名祷告,阿们!
%-----------------------------------------------------------------------------
\newpage
\section{诗篇第104篇:赞美创造的主——看神的荣耀与供应}
% 讲章:
\subsection*{引言}
弟兄姊妹,今天我们要一同学习《诗篇》第104篇。这是一首充满赞美的诗篇,诗人用生动的语言描绘了神创造的伟大,他如何以智慧统管万有,并且供应一切受造物。这篇诗歌不仅让我们看到神的荣耀,也提醒我们如何在日常生活中回应他的恩典。
让我们带着敬畏的心,一起进入今天的信息。

\subsection*{一、神的创造显出他的荣耀(诗104:1-9)}
诗篇104篇开头就宣告:“我的心哪,你要称颂耶和华!耶和华—我的神啊,你为至大!你以尊荣威严为衣服。”(诗104:1)

\subsubsection*{1. 神的创造彰显他的智慧}
诗人描述了神如何用光为衣服,铺张穹苍,建立地的根基,甚至用水遮盖大地。这些话语让我们想起创世记第一章,神的话语带来秩序,使混沌变为有序。

\subsubsection*{2. 现代生活的反思}
在当今社会,我们习惯于依赖科技,却常常忽略了神的创造之美。我们每天看到日出日落,享受清新的空气,但我们是否曾停下来思想:这一切都是神的手所造?

\subsubsection*{应用:}
当我们在忙碌的生活中,不妨抬头看看天空,留意周围的自然景色,思考这位伟大的创造主,让我们的心充满敬畏和感恩。

\subsection*{二、神的供应显出他的恩典(诗104:10-30)}
\subsubsection*{1. 神供应一切受造物}
诗人接着描述神如何使泉水流入山间,滋润田野,让牲畜得饱足;他又使草生长,赐下五谷、酒、油和食物。这表明神不仅创造了世界,还一直在供应养育他的受造物。

\subsubsection*{2. 我们如何经历神的供应?}
在物质丰富的时代,人们常常以为财富、科技、努力才是生存的根本。但诗人提醒我们,真正的供应来自神。无论是我们的工作、食物,还是每天的呼吸,都是出于他的恩典。

\subsubsection*{应用:}

我们要学会依靠神,而不是单单依靠自己的能力。每天早晨,我们可以祷告:“主啊,感谢你赐给我今天的食物、空气和力量,帮助我信靠你。”

\subsection*{三、人的回应:赞美与顺服(诗104:31-35)}
诗篇104篇的最后,诗人用敬拜回应神的伟大:“愿耶和华的荣耀存到永远!愿耶和华喜悦自己所造的!”(诗104:31)

\subsubsection*{1. 赞美是自然的回应}
当我们真正认识到神的创造与供应,我们的心自然会充满敬拜。

\subsubsection*{2. 生活中的实际应用}

\hspace{0.6cm}培养感恩的心:每天数算神的恩典,感恩他的供应。

敬畏神,远离罪恶:诗人最后说:“愿罪人从世上消灭,愿恶人归于无有。”(诗104:35)这提醒我们,若我们敬拜神,就要活出圣洁的生命,拒绝罪恶。

\subsection*{结语与祷告}
\hspace{0.6cm}弟兄姊妹,今天我们学习了诗篇104篇,看见神的创造、供应和掌权。他是配得我们一生敬拜的神。让我们带着感恩的心,时刻仰望他,在忙碌的生活中仍然不忘他的恩典。

\subsubsection*{让我们一起祷告:}

\textbf{亲爱的天父,}

我们感谢你!你创造天地,使万物各按其时生长,你也供应我们一切所需。主啊,帮助我们不只是看到世界的美丽,更看到你的荣耀,使我们的心归向你。愿我们的生命成为你的见证,愿我们在生活中时刻感恩,并且依靠你。

奉主耶稣基督的名,阿们!
%-----------------------------------------------------------------------------
\newpage
\section{诗篇第105篇:记念主的作为,活出信实的生命}
% 讲章:——诗篇105篇的启示

\subsection*{引言}
亲爱的弟兄姊妹,今天我们要一起学习诗篇105篇,这是一篇充满感恩和敬拜的诗篇,提醒我们要记念神的作为,并将他的信实活出来。诗篇105篇不仅回顾了以色列人的历史,更教导我们如何在现实生活中信靠神,活出感恩与顺服的生命。
在这个快节奏、充满挑战的时代,我们常常容易忘记神在我们生命中的恩典,被生活的压力、焦虑和烦恼占据。今天,让我们透过诗篇105篇,从三个方面来思想如何在现实生活中记念神、依靠神,并活出信实的生命。

\subsection*{一、以感恩的心记念神的作为(诗篇105:1-7)}
“你们要称谢耶和华,求告他的名,在万民中传扬他的作为!”(诗105:1)

诗篇105篇以呼召我们感恩和赞美开始。大卫提醒以色列人要记得神的伟大作为,并不断地向人传扬他的奇妙工作。我们基督徒也应当培养感恩的习惯,不只是停留在言语上,更要从内心真正感受到神的恩典。

\subsubsection*{生活应用:}
\begin{enumerate}
    \item 建立感恩日记——每天写下神在你生活中的恩典,哪怕是微小的帮助和安慰。
    \item 在困难中仍然感恩——即使面对挑战,也要相信神掌权,并向他献上感谢。

    \item 用行动感恩——不仅仅口头感谢神,也要通过帮助他人、服侍教会和社会来表达对神的感恩。

\end{enumerate}
\subsubsection*{问题思考:}

你是否常常因生活的压力而忘记神的恩典?
你愿意每天操练感恩的态度,记念神在你生命中的作为吗?
\subsection*{二、信靠神的信实,即使在挑战中(诗篇105:8-22)}
“他记念他的约,直到永远,他所吩咐的话,直到千代。”(诗105:8)

诗篇105篇回顾了神如何带领以色列人,包括约瑟被卖、经历患难,最终却成为埃及的宰相,拯救了自己的家族。这提醒我们,神的道路高过人的道路,即使我们看不见未来,神仍然掌管一切。

\subsubsection*{生活应用:}
\begin{enumerate}
    \item 信靠神的时间表——有时候我们会觉得神的应许好像迟迟未兑现,但神从不误事,他的时间最完美。

    \item 不要让环境左右你的信心——约瑟在监牢里依然相信神,我们在困难中也要坚持祷告和依靠神。

    \item 培养长期信靠的心态——神的带领可能不会立刻显现,但我们要持守信心,等待神的工作完成。

\end{enumerate}
\subsubsection*{问题思考:}

你是否曾因环境的挑战而怀疑神的信实?
你愿意在等待中仍然相信神,并坚持忠心地生活吗?
\subsection*{三、回应神的恩典,活出顺服的生命(诗篇105:23-45)}
“他领自己的百姓欢然前行,领他的选民快乐而去。”(诗105:43)

神不仅拯救以色列人脱离埃及的捆绑,还供应他们在旷野的需要,最终带他们进入应许之地。神的带领不只是让我们蒙福,更是要我们回应他的爱,活出顺服的生命。

\subsubsection*{生活应用:}
\begin{enumerate}
    \item 顺服神的带领——当神呼召你去做某件事时,不要害怕,勇敢地顺服。

    \item 活出圣洁的生活——神救赎我们,不是让我们随意而行,而是要我们活出圣洁,荣耀他的名。

    \item 成为别人祝福的管道——神赐福以色列,是要他们成为万国的祝福,我们也要在家庭、职场、社会中作见证。

\end{enumerate}
\subsubsection*{问题思考:}

你是否愿意让神在你生命中掌权,而不是靠自己的计划和意志行事?
你愿意成为神的器皿,去影响和帮助身边的人吗?
\subsection*{结论与挑战}
诗篇105篇提醒我们:
\begin{enumerate}
    \item 常常感恩,记念神的作为——不让焦虑和压力蒙蔽我们的眼睛,而是以感恩的心数算主的恩典。

    \item 信靠神的信实,不被环境左右——即使在挑战中,也相信神的应许,持守信仰。

    \item 顺服神的引导,活出见证——让神的爱流淌在我们的生命中,成为别人祝福的管道。

\end{enumerate}

弟兄姊妹,你是否愿意在今天就开始操练这三方面呢?让我们每天记念神的恩典,相信他的信实,并活出顺服的生命!

\subsection*{结束祷告}
\textbf{亲爱的天父,}

感谢祢借着诗篇105篇提醒我们,要时常记念祢的作为,并活出信实的生命。求祢赐给我们一颗感恩的心,让我们在顺境和逆境中都能看到祢的恩典。主啊,帮助我们信靠祢的计划,即使在等待中也不失去信心。更求祢赐我们顺服的灵,让我们能成为祢的器皿,在这个世界上发光作盐。愿祢的名因我们的生命被高举,愿祢的国度在我们的生活中彰显!

感谢祢,奉主耶稣基督的名祷告,阿们!
%-----------------------------------------------------------------------------
\newpage
\section{诗篇第106篇:从失败中学习——警示与盼望}
% 讲章:
\subsection*{引言}

亲爱的弟兄姊妹,今天我们要一起查考诗篇106篇。这篇诗篇既是一篇悔改的诗,又是一篇警示的诗。它回顾了以色列人的历史,列举了他们如何一次次地悖逆神,但神仍然用他的慈爱与怜悯挽回他们。这篇诗提醒我们,不要重蹈以色列人的覆辙,而是要警醒、悔改,并坚定地信靠神。
在现实生活中,我们也常常软弱、失败,甚至偏离神的道路。但神的怜悯从不止息,他愿意接纳那些回转归向他的人。

\subsection*{一、认罪悔改——承认自己的失败(诗篇106:6-12)}
“我们与我们的祖宗一同犯罪,我们作了孽,行了恶。”(诗106:6)

诗篇106篇开头颂赞神的良善与信实,但很快作者就承认,以色列人一再悖逆神。他们在红海畔怀疑神的能力,但神仍然施行拯救。

\subsubsection*{生活应用:}
\begin{enumerate}
    \item 敢于承认自己的过犯——在生活中,我们可能因为骄傲、贪心、懒惰或私欲,做出违背神旨意的事情。第一步是勇敢承认自己的软弱,不自欺欺人。

    \item 不要活在过去的罪疚感中——神是信实的,若我们真心悔改,他就必赦免(约壹1:9)。

    \item 常常自省,向神祷告——每天花时间反思自己是否有得罪神的地方,并祈求神帮助我们悔改更新。

\end{enumerate}
\subsubsection*{问题思考:}

你是否有未向神承认的罪?
你是否常常为自己的错误找借口,而不是坦然面对?
\subsection*{二、警醒谨守——不要重蹈覆辙(诗篇106:13-39)}
“他们很快就忘了他的作为,不仰望他的指教。”(诗106:13)

诗篇106篇详细列举了以色列人在旷野中的失败:他们忘记神的作为(13节)、贪恋肉体的情欲(14节)、拜偶像(19-20节)、悖逆神的领袖(32节)、效法外邦恶俗(35-39节)。这些都是他们远离神的原因。

\subsubsection*{生活应用:}
\begin{enumerate}
    \item 保持属灵的警醒——不要因安逸或忙碌而远离神,定期读经、祷告,与神保持亲密关系。

    \item 警惕世界的诱惑——我们的时代充满了各种引诱:金钱、权力、虚荣、享乐,我们要小心别让这些偶像占据了我们的心。

    \item 避免消极群体影响——以色列人因受外邦文化影响而堕落,今天,我们也要谨慎交友,远离那些会带领我们远离神的环境。

\end{enumerate}
\subsubsection*{问题思考:}

你是否曾在顺境时忘记神,而只在困难时才想起他?
你的生活中是否有让你远离神的偶像?(如:金钱、名誉、享乐、手机社交等)
\subsection*{三、仰望神的恩典,活出新的生命(诗篇106:40-48)}
“然而,他因他们呼求的时候垂顾他们的急难;他也为他们记念他的约,照他丰盛的慈爱后悔。”(诗106:44-45)

虽然以色列人一次次失败,但神仍然因他的慈爱拯救他们。这让我们看到:人的失败不能拦阻神的恩典。今天,我们若愿意回转,神也必赦免我们,并赐我们新的生命。

\subsubsection*{生活应用:}
\begin{enumerate}
    \item 用信心仰望神的救赎——无论过去有多少失败,今天都可以重新开始,因为神的恩典比我们的罪更大。

    \item 成为神恩典的见证人——以色列人蒙神拯救后,应当向外邦人见证神的伟大。今天,我们也要向身边的人见证神的爱。

    \item 活出感恩与顺服的生命——既然神拯救了我们,我们就不应该再沉溺于罪,而是以顺服和感恩回应神的恩典。

\end{enumerate}
\subsubsection*{问题思考:}

你是否相信神仍然愿意赦免你的失败?
你愿意用行动回应神的恩典,活出新的生命吗?
\subsection*{结论与挑战}
诗篇106篇提醒我们:
\begin{enumerate}
    \item 认罪悔改,不逃避自己的失败——勇敢向神承认自己的软弱,他必赦免我们。

    \item 警醒谨守,不重蹈覆辙——不要让世界的诱惑和错误的群体影响我们,而是要紧紧跟随神。

    \item 仰望神的恩典,活出新的生命——无论过去有多少失败,今天都可以靠着神重新开始。

\end{enumerate}

弟兄姊妹,你是否愿意在今天就回转,重新寻求神?愿我们不再陷入以色列人曾犯的错误,而是学习警醒、悔改,并以顺服和感恩回应神的爱!

\subsection*{结束祷告}
\textbf{亲爱的天父,}

感谢祢借着诗篇106篇提醒我们,虽然我们软弱、失败,但祢的恩典永不止息。主啊,我们承认自己的过犯,求祢洁净我们的心,使我们不要再重蹈覆辙。求祢赐给我们警醒的灵,不被世界的诱惑所捆绑,而是时刻依靠祢的恩典而行。主啊,帮助我们活出新的生命,使我们的言行都能荣耀祢的名!愿我们的一生成为祢恩典的见证,让更多人因我们的生命而认识祢。

奉主耶稣基督的名祷告,阿们!
%-----------------------------------------------------------------------------
\newpage
\section{诗篇第107篇:在困境中经历神的救赎}
% 讲章:——诗篇107篇的启示
\subsection*{引言}
亲爱的弟兄姊妹,今天我们要学习诗篇107篇,这是一篇充满盼望和感恩的诗篇。它强调神的救赎之恩,告诉我们,无论遇到什么困境,神都愿意拯救那些呼求他的人。
在现实生活中,我们每个人都会经历不同形式的困境——迷失方向、缺乏资源、身处黑暗、甚至被罪恶捆绑。诗篇107篇用四种困境的比喻来描述人的光景,同时展示神如何施行拯救。这提醒我们:神的恩典大过我们的软弱,只要我们愿意向他呼求,他必拯救我们。
\subsection*{一、人的困境是现实的,但神的拯救更真实(诗篇107:4-32)}
诗篇107篇列举了四种困境,代表人类在属灵生命中的挣扎:

\subsubsection*{在荒漠中漂流(4-9节) —— 代表迷失方向的人}

以色列人在旷野漂流,没有水、没有食物,直到他们向神呼求,神引导他们找到安息之处。
现实生活中,我们可能迷失方向,不知道未来如何,也可能在学业、事业、家庭或情感中感到困惑。
但神应许要引导我们的脚步,只要我们愿意依靠他。
\subsubsection*{被黑暗捆绑(10-16节) —— 代表被罪恶辖制的人}

这些人因悖逆神,被黑暗和铁链捆锁,直到他们向神呼求,神打破他们的锁链,使他们得自由。
我们有时会被罪、成瘾、坏习惯、错误的关系捆绑,导致心灵黑暗、充满内疚和痛苦。
但神的能力能打破一切锁链,使我们得释放!
\subsubsection*{因罪受苦(17-22节) —— 代表因自己的过犯遭遇苦难的人}

这些人因愚妄的行为受苦,甚至到了生命的边缘,直到他们呼求神,神医治他们。
许多人因错误的选择或不顺服神,而经历失败、痛苦甚至疾病。但神愿意医治、恢复,并且赦免我们过去的过犯。
\subsubsection*{风浪中的惊惧(23-32节) —— 代表遭遇突如其来的风暴的人}

诗人描绘海上的人遇到狂风巨浪,生命几乎要倾覆,但当他们呼求神,神平息风浪,引导他们安全抵达。
我们的人生也会经历“风暴”——突如其来的疾病、失业、家庭问题,甚至是信仰的考验。
但神应许,即使在风暴中,他依然掌权,并且能够带领我们平安度过。
现实反思: 你是否曾在生活中经历类似的困境?你是否愿意相信神能拯救你?

\subsection*{二、无论在哪种困境中,向神呼求,他必搭救(诗篇107:6,13,19,28)}
在这四个困境中,诗篇107篇反复出现一句话:

“于是他们在苦难中哀求耶和华,他从他们的祸患中搭救他们。”(诗107:6, 13, 19, 28)

这节经文提醒我们:当人愿意在苦难中谦卑地呼求神,神必定施行拯救!
\begin{enumerate}
    \item 神的拯救是主动的 —— 他不只是被动地等着我们,而是渴望拯救我们。

    \item 神的拯救是完全的 —— 无论我们的困境多严重,他都能使我们脱离,并赐下真正的平安。

    \item 神的拯救是因着他的慈爱 —— 不是因为我们够好,而是因为他的怜悯与信实。

\end{enumerate}

\subsubsection*{生活应用:}

在困境中,不要单靠自己,而要向神祷告,求他指引出路。
当觉得无助时,记住神过去如何拯救你,增强信心。
在每一天的生活中,建立“呼求神”的习惯,不只是等到困难来了才找他。
现实反思: 你是否愿意在困境中,第一时间向神呼求,而不是依靠自己的聪明?

\subsection*{三、蒙恩后,当感恩回应,活出见证(诗篇107:1-3, 33-43)}
诗篇107篇的开头和结尾都强调感恩:

“你们要称谢耶和华,因他本为善,他的慈爱永远长存!”(诗107:1)

“凡有智慧的,必在这些事上留心,也必思想耶和华的慈爱。”(诗107:43)

当我们经历神的拯救后,不应该只是满足于自己的得救,而是要用行动回应神的恩典。

\subsubsection*{如何回应神的恩典?}
\begin{enumerate}
    \item 用感恩的心敬拜神 —— 不仅在教会里,更要在每天的生活中赞美神。

    \item 用生命活出神的见证 —— 让我们的生命成为神恩典的故事,使别人因我们而认识神。

    \item 用行动去帮助他人 —— 当我们经历神的恩典后,也要去帮助仍在困境中的人,使他们看见神的爱。

\end{enumerate}

现实反思: 你是否把神的恩典当作理所当然?你愿意成为别人生命中的祝福吗?

\subsection*{结论与挑战}
诗篇107篇提醒我们:
\begin{enumerate}
    \item 无论遇到什么困难,神都能拯救 —— 他的能力比我们的问题更大。

    \item 关键是要向神呼求,而不是单靠自己 —— 他应许回应一切寻求他的人。

    \item 得救后要感恩,并以生命回应神 —— 活出见证,成为祝福他人的器皿。

\end{enumerate}

亲爱的弟兄姊妹,你是否正处在人生的困境中?是否愿意向神呼求,经历他的拯救?让我们在今天就转向神,信靠他的大能,并且活出感恩的生命!

\subsection*{结束祷告}
\textbf{慈爱的天父,}

我们感谢祢,因为祢的慈爱永不止息。感谢祢在我们迷失、困苦、软弱、风暴中的时候,总是施行拯救。主啊,帮助我们在任何困境中都能向祢呼求,并相信祢的大能。也求祢赐给我们一颗感恩的心,让我们以生命回应祢的恩典,成为别人的祝福。愿祢的名因我们的见证而被高举!

奉主耶稣基督的名祷告,阿们!
%-----------------------------------------------------------------------------
\newpage
\section{诗篇第108篇:坚定信靠,得胜在望——盼望与信心}
% 讲章:
\subsection*{引言}
亲爱的弟兄姊妹,今天我们要查考诗篇108篇。这首诗是大卫的诗篇,包含了他对神坚定的信心,以及他在面对挑战时的祷告。这篇诗强调了信心、敬拜、祷告与得胜,提醒我们在任何环境下都要信靠神,并以赞美为武器,迎接神所赐的胜利。
在现实生活中,我们每个人都会遇到各种挑战——学业上的压力、工作中的难题、家庭的矛盾、甚至信仰的挣扎。但诗篇108篇告诉我们,当我们坚定依靠神、用敬拜来回应挑战时,神必带领我们得胜!

\subsection*{一、坚定的敬拜——即使在挑战中仍然赞美神(1-5节)}
“神啊,我心坚定,我口要唱诗歌颂!”(诗108:1)

诗篇108篇一开始,大卫就用坚定的信心宣告要赞美神。他的心已经准备好,无论环境如何,他都要用诗歌敬拜神。这给我们一个重要的提醒:敬拜不仅仅是顺境时的回应,更是困境中的选择。

\subsubsection*{现实生活中的挑战:}

你是否只在顺境时赞美神,而在困难时却埋怨?
你是否在面对考试、工作压力、家庭问题时,选择相信神仍然掌权?
你是否能在软弱时,仍然用诗歌敬拜神,让自己的信心得以坚固?
\subsubsection*{应用:如何在挑战中敬拜?}

\hspace{0.6cm}操练感恩的心 —— 在祷告中数算神的恩典,而不是专注于困难。

用诗歌敬拜 —— 无论环境如何,都让赞美成为生活的一部分。

坚定信靠神 —— 即使看不见出路,仍然相信神的美好旨意。
\subsubsection*{问题思考:}

你是否有一个“坚定敬拜”的心态?
你最近的一次赞美神,是在顺境还是在困境中?
\subsection*{二、迫切的祷告——倚靠神的应许寻求帮助(6-9节)}
“求你应允我们,好叫你所亲爱的人得救。”(诗108:6)

大卫不仅仅是敬拜,他也向神发出祷告。他知道,真正的帮助不是来自人,而是来自神的应许。

\subsubsection*{祷告的关键:}

\hspace{0.6cm}以神的应许为根基 —— 不是凭感觉,而是凭神的话语来祷告。

相信神的信实 —— 他曾应许帮助以色列人,今天也必不撇下我们。

带着信心祷告 —— 不是消极的诉苦,而是积极地求神介入,并期待他的作为。
\subsubsection*{生活应用:当你面临困境时,如何祷告?}

如果你在学业或工作上感到疲惫,不只是抱怨,而是求神赐下智慧和能力。
如果你在人际关系中受挫,不只是埋怨,而是求神教导你如何去爱、去宽恕。
如果你在信仰上遇到挑战,不只是消极退缩,而是求神坚固你的信心。
\subsubsection*{行动计划:建立一个迫切祷告的习惯!}

每天固定时间向神祷告(清晨或睡前)。
用神的话语祷告,比如:“神啊,你是我的避难所,我必不惧怕”(诗46:1)。
在群体中彼此代祷,不要独自面对困难。
\subsubsection*{问题思考:}

你的祷告是建立在神的应许之上,还是只是停留在自己的情绪中?
你最近的一次祷告,是带着信心,还是带着怀疑?
\subsection*{三、得胜的信心——相信神必带领我们跨越困境(10-13节)}
“我们倚靠神才得施展大能,因为践踏我们敌人的就是他。”(诗108:13)

诗篇108篇的最后部分,大卫宣告得胜的信心。他知道,真正的得胜不是靠人的力量,而是倚靠神。

\subsubsection*{得胜的信心意味着什么?}

\hspace{0.6cm}相信神已经掌权 —— 即使环境未改变,也相信神正在动工。

勇敢面对挑战 —— 不是逃避,而是带着信心迎战,知道神必帮助。

坚持到底 —— 许多时候,我们缺少的不是机会,而是坚持到底的信心。
\subsubsection*{生活应用:如何活出得胜的信心?}

在学习上,不因一次失败而灰心,而是相信神给你智慧去突破难题。
在工作上,不因挑战而退缩,而是带着信心去解决问题。
在信仰上,不因软弱而远离神,而是抓住他的应许,持续前行。
\subsubsection*{行动计划:用信心宣告得胜!}

当你面对挑战时,勇敢地说:“神必帮助我,我必得胜!”
在每天的祷告中,不只是求帮助,而是宣告神的得胜在你生命中成就。
养成“积极信靠”的习惯,不让负面情绪掌控你的人生。
\subsubsection*{问题思考:}

你是否真的相信,神可以带你跨越眼前的困境?
你的信心是建立在神的话语上,还是环境和感觉上?
\subsection*{结论与挑战}
诗篇108篇教导我们三个重要功课:
\begin{enumerate}
    \item 坚定敬拜 —— 无论环境如何,都要选择赞美神。

    \item 迫切祷告 —— 以神的应许为根基,寻求他的帮助。

    \item 得胜信心 —— 相信神必带领我们跨越困境,进入他所应许的得胜。

\end{enumerate}



亲爱的弟兄姊妹,今天你愿意选择信靠神吗?

你是否愿意在挑战中仍然赞美神?

你是否愿意建立一个恒切的祷告生活?

你是否愿意带着得胜的信心,勇敢面对人生的挑战?

愿神帮助我们,使我们成为一个\textbf{“无论环境如何,都坚定信靠神的人”}!

\subsection*{结束祷告}
\textbf{亲爱的天父,}

感谢祢的话语提醒我们,在任何环境中都要坚定信靠祢。主啊,帮助我们用敬拜来胜过恐惧,用祷告来寻求祢的帮助,并带着信心宣告得胜。求祢赐给我们刚强的心,使我们不被困难击倒,而是靠着祢,勇敢前行!愿我们的生命成为祢荣耀的见证!

奉主耶稣基督的名祷告,阿们!
%-----------------------------------------------------------------------------
\newpage
\section{诗篇第109篇:在逼迫中寻求神的公义}
% 讲章:——诗篇109篇的属灵功课
\subsection*{引言}
亲爱的弟兄姊妹,今天我们要查考诗篇第109篇。这首诗是大卫在极度痛苦和受逼迫时所写的,他呼求神伸张公义,为自己辩白,并求神对恶人施行审判。这首诗有强烈的情感,包含了大卫的痛苦、愤怒、祈求,以及对神最终公义的信靠。
在现实生活中,我们也可能会遭遇误解、被人诽谤,甚至受到不公平的对待。面对这些挑战,我们应该如何回应?诗篇109篇提醒我们:当遭受不公义时,不要自己伸冤,而要把一切交托给神,相信他的公义与救赎。

\subsection*{一、面对逼迫时,不要自己伸冤,而要向神倾诉(1-5节)}
“我所赞美的神啊,求你不要闭口不言。”(诗篇109:1)

诗篇109篇一开始,大卫就向神祷告,不是求神立刻报仇,而是求神听他的呼求。

大卫受到了极大的伤害。他说:“他们用谎言的舌头攻击我”(2节),又说:“他们用恶语围绕我,无故地攻击我”(3节)。这描述了他被人诽谤、陷害、遭受不公义对待的经历。

\subsubsection*{在我们的生活中,也可能会遇到类似的情况。}

在职场中,可能有人恶意中伤我们,甚至诽谤我们的人品。
在人际关系中,可能遭遇朋友的背叛,被误解甚至冤枉。
在信仰上,可能会因坚持真理而受到攻击或排斥。
当面对这样的情况时,我们容易用愤怒或报复来回应,但大卫给我们的榜样是:向神倾诉,而不是自己伸冤。

\subsubsection*{实际应用:}

遇到误解或伤害时,不要急着反击,而是先安静祷告,把事情交托给神。
以信心相信神知道一切,并且他会在合适的时间显明真相。
学习控制情绪,不让怒气支配我们的言行,而是用爱和智慧去应对挑战。
\subsubsection*{问题思考:}

当你被误解或伤害时,你的第一反应是什么?是愤怒、反击,还是向神祷告?
你是否愿意把你的委屈交托给神,让他来伸张公义?
\subsection*{二、在痛苦中,信靠神的公义,而非靠自己的愤怒(6-20节)}
“愿你派一个恶人辖制他,派一个对头站在他右边。”(诗篇109:6)

这段经文是诗篇109篇中最具争议的部分。大卫强烈地祈求神审判恶人,他希望那些诽谤他的人遭受报应。这让我们思考一个问题:基督徒应该如何看待这样的祷告?我们可以求神惩罚我们的敌人吗?

在新约中,耶稣教导我们要爱仇敌,为逼迫我们的人祷告(马太福音5:44)。但这并不意味着神的公义不再存在,而是要我们把审判的权利交给神,而不是自己报复。大卫的祷告并不是出于个人的仇恨,而是对神公义的呼求。他相信神是公义的,他求神来伸张正义,而不是自己动手。

\subsubsection*{实际应用:}

当有人伤害你时,先求神帮助你放下愤怒,而不是立刻反击。
相信神的公义,相信他比我们更清楚如何处理不义之人。
用祷告代替报复,把审判的权利交托给神,而不是靠自己的方法解决。
\subsubsection*{问题思考:}

你是否愿意放下自己的愤怒,相信神的公义?
你是否愿意为伤害你的人祷告,而不是求神报复他们?
\subsection*{三、即使在困境中,也要持守祷告和信心,等待神的拯救(21-31节)}
“主耶和华啊,求你为你的名恩待我;因你的慈爱美好,求你搭救我!”(诗篇109:21)

在这最后一部分,大卫转向神的恩典和拯救。他没有继续停留在愤怒和控诉中,而是把目光放在神的信实上。他知道,真正能帮助他、医治他的,不是人的报复,而是神的怜悯和拯救。

无论我们的处境多么艰难,我们都应该像大卫一样,选择继续信靠神,不让苦毒和怨恨控制我们,而是让神的恩典充满我们的心。

\subsubsection*{实际应用:}

在困境中,不要停止祷告,即使看不到改变,也要坚持祷告。
相信神的时间比我们的时间更完美,不要急于看到报应,而是相信神会以最合适的方式成就他的公义。
用感恩的心来面对困境,相信神最终会翻转一切,使我们经历他的美好旨意。
\subsubsection*{问题思考:}

你是否愿意在困境中仍然信靠神,而不是被痛苦和愤怒所掌控?
你是否愿意用祷告和信心,等待神的拯救,而不是靠自己的方法去伸冤?
\subsection*{结论与挑战}
诗篇109篇教导我们三个重要的功课:
\begin{enumerate}
    \item 面对逼迫时,选择向神倾诉,而不是自己伸冤。

    \item 信靠神的公义,而不是让自己的愤怒驱使我们去报复。

    \item 无论遭遇什么困境,都要持守祷告和信心,等待神的拯救。

\end{enumerate}

亲爱的弟兄姊妹,今天你是否正经历误解、逼迫或痛苦?你是否愿意像大卫一样,把一切交托给神,而不是自己伸冤?愿我们都能学习在困境中倚靠神,相信他的公义,并且以祷告和信心迎接神的拯救!

\subsection*{结束祷告}
\textbf{亲爱的天父,}

感谢祢的话语提醒我们,在遭受不公时,不要让愤怒和苦毒掌控我们,而是要把一切交托给祢。主啊,我们相信祢是公义的,祢必按着祢的时间成就祢的旨意。帮助我们在困境中仍然信靠祢,以祷告代替报复,以信心代替苦毒,愿我们的一生都能荣耀祢的名!

奉主耶稣基督的名祷告,阿们!
%-----------------------------------------------------------------------------
\newpage
\section{诗篇第110篇:基督的王权与祭司职分}
% 讲章:——诗篇110篇的启示
\subsection*{引言}
亲爱的弟兄姊妹,今天我们要查考诗篇第110篇。这是一首极具先知性意义的诗篇,被新约引用多次(如马太福音22:44,希伯来书5:6)。诗篇110篇描述了一位独特的君王,不仅坐在神的右边掌权,还身兼“麦基洗德等次的祭司”。在整本圣经中,这首诗篇明确指向弥赛亚耶稣基督,宣告他是君王、祭司和审判者。
无论我们今天面临什么样的挑战,诗篇110篇提醒我们:\textbf{耶稣已经得胜,他正在掌权,我们应当顺服、亲近并信靠他}!

\subsection*{一、耶稣是掌权的君王,我们要顺服他(1-3节)}
“耶和华对我主说:‘你坐在我的右边,等我使你的仇敌作你的脚凳。’”(诗篇110:1)

诗篇110篇的开头,直接引用了天父对弥赛亚(耶稣)的应许,表明耶稣不仅是大卫的子孙,更是大卫的“主”。在新约中,耶稣自己也引用这节经文,向法利赛人证明他的神性(马太福音22:41-46)。

\subsubsection*{现实生活中的应用:}
\begin{itemize}
    \item 顺服耶稣的主权——耶稣坐在神的右边,表明他的权柄高于一切。今天,我们是否真正顺服他的掌管?

    \item 面对挑战时,信靠神的得胜——无论我们遇到什么困难,耶稣已经得胜,最终他的仇敌都会被践踏。我们要有盼望,而不是被现实的困难所压倒。

    \item 在世作他的精兵——第3节说:“当祢掌权的日子,祢的民要以圣洁的装饰为衣,甘心牺牲自己。”我们作为基督的子民,要积极参与他的国度事工,活出圣洁的生命。

\end{itemize}


\subsubsection*{问题思考:}

你是否在生活的每个领域都顺服耶稣的掌权,还是仍然自己做主?
当你面对困难时,你是依靠自己的能力,还是信靠那位已经得胜的君王?
\subsection*{二、耶稣是永远的祭司,我们要靠近他(4节)}
“耶和华起了誓,决不后悔,说:‘你是照着麦基洗德的等次,永远为祭司。’”(诗篇110:4)

这节经文是整本圣经中非常重要的启示,它表明耶稣不仅是君王,更是祭司。而且他的祭司职分不像利未人,而是像麦基洗德——一个超越律法体系、永恒的祭司(希伯来书7:17)。

\subsubsection*{现实生活中的应用:}
\begin{itemize}
    \item 耶稣是我们的中保,我们要亲近他——作为祭司,耶稣在神与人之间搭起了桥梁,我们可以随时向他祷告,不必再依靠人的祭司或献祭。

    \item 耶稣已经献上自己,成为我们永远的救赎——利未祭司需要不断献祭,但耶稣一次献上自己,就使我们得以完全(希伯来书10:14)。今天,我们不需要靠行为称义,而是靠信心接受耶稣的救恩。

    \item 我们在基督里也是祭司,要活出圣洁——彼得前书2:9告诉我们,信徒是“有君尊的祭司”,我们应该以祷告、敬拜、服事的方式,成为神与世人的桥梁。

\end{itemize}
\subsubsection*{问题思考:}

你是否每天亲近耶稣,承认他是你的大祭司?
你是否仍然试图靠自己的行为取悦神,而不是完全信靠耶稣的恩典?
\subsection*{三、耶稣是最终的审判者,我们要倚靠他(5-7节)}
“主要在他发怒的日子打伤列王。他要在列邦中施行审判,尸首就遍满各处。”(诗篇110:5-6)

诗篇110篇的最后几节描述了耶稣的再来和最终的审判。虽然今天世界仍然充满不公义,但终有一天,基督会再来,彻底击败罪恶和不义,施行公义的审判。

\subsubsection*{现实生活中的应用:}
\begin{itemize}
    \item 我们不需要为公义伸冤,神最终会审判一切不公——如果你今天在职场、家庭、社会中受到不公平对待,不要心存苦毒,因为最终神会按公义施行审判。

    \item 不要被世界的罪恶吸引,而要持守圣洁——当世界越来越混乱、价值观扭曲时,我们要坚守真理,持守信仰,因为耶稣最终会审判世界。

    \item 活出敬畏神的生命——既然我们知道基督最终会审判世界,我们就当存敬畏神的心,活出合神心意的生命,而不是随从世界的标准。

\end{itemize}
\subsubsection*{问题思考:}

你是否因世上的不公义而灰心,忘记了神最终的审判?
你是否活在敬畏神的生命中,期待耶稣的再来?
\subsection*{结论与挑战}
诗篇110篇清楚地指向耶稣基督,启示他是君王、祭司和审判者。
\begin{enumerate}
    \item 顺服君王——让耶稣掌管我们的生命,不再自己做主。

    \item 亲近祭司——每天与耶稣建立亲密关系,经历他的恩典和带领。

    \item 信靠审判者——面对不公义时,不报复、不绝望,而是相信耶稣最终会施行公义。

\end{enumerate}

亲爱的弟兄姊妹,你是否愿意今天就重新调整你的生命,顺服基督、亲近他,并以敬畏的心等候他的再来?

\subsection*{结束祷告}
亲爱的天父,感谢祢赐下诗篇110篇,让我们看见耶稣基督是那位至高掌权的君王、永远的祭司和最终的审判者。主啊,我们愿意顺服祢的掌管,在生活的每个领域都让祢作主。帮助我们更加亲近祢,相信祢的恩典够我们用。无论世界如何动荡,我们要坚信祢的公义最终会成就。愿我们一生都活在祢的旨意里,预备自己迎接祢的再来!奉主耶稣基督的名祷告,阿们!
%-----------------------------------------------------------------------------
\newpage
\section{诗篇第111篇:敬畏耶和华,智慧之始}
% 讲章:——诗篇111篇的启示
\subsection*{引言}
亲爱的弟兄姊妹,今天我们要查考诗篇第111篇。这篇诗篇是一首赞美诗,特别强调神的作为、信实与公义,并以\textbf{“敬畏耶和华是智慧的开端”}作为总结(诗111:10)。这首诗提醒我们:神的作为何等伟大,他的恩典何等信实,我们要用感恩的心敬畏他,并行在智慧之中。
无论我们面对怎样的环境,这篇诗篇鼓励我们:当以感恩的心数算神的恩典,信靠他的供应,敬畏他的名,行在智慧中!

\subsection*{一、神的作为伟大奇妙,我们要常存感恩(1-4节)}
“我要一心称谢耶和华,在正直人的大会中,并在会众中。”(诗111:1)

诗人一开始就宣告自己要全心赞美神,而且要在众人面前公开见证神的作为。这提醒我们:感恩不仅是个人的心态,更应该成为我们信仰生活的一部分。

\subsubsection*{神的作为何等奇妙?}

\hspace{0.6cm}创造之工(2节):“耶和华的作为本为大,凡喜爱的都必考察。”——天地万物都是神手所造,我们应当因神的创造而赞叹。

救赎之恩(3-4节):“他所行的是尊荣和威严,他的公义存到永远。”——神不仅创造万物,更救赎了我们,使我们得享永生的盼望。
\subsubsection*{现实生活中的应用:}
\begin{itemize}
    \item 培养感恩的习惯——不要只在顺境中感谢神,而要在任何环境中都数算神的恩典(帖前5:18)。

    \item 在众人面前见证神——诗人说,他在会众中赞美神。我们是否也愿意在朋友圈、家庭、教会中分享神的恩典?

    \item 认真考察神的作为——不要对神的恩典视而不见,而要思想他的作为,带着敬畏的心学习他的话语。

\end{itemize}
\subsubsection*{问题思考:}

你有多少次因为忙碌而忘记感恩?
你是否愿意在别人面前分享神的作为,而不是只在私下默默感恩?
\subsection*{二、神的供应充足信实,我们要倚靠顺服(5-9节)}
“他赐粮食给敬畏他的人,他必永远记念他的约。”(诗111:5)

这几节经文强调神的供应和信实。神不仅创造世界,更持续供应我们的需要,并以永不改变的约持守对他子民的承诺。

\subsubsection*{神如何供应我们?}

\hspace{0.6cm}物质上的供应(5节):神供应我们日用的饮食,不让我们缺乏。

属灵上的供应(6-7节):神赐下他的话语,使我们明白他的心意。

救赎的供应(9节):“他向百姓施行救赎,命定他的约,直到永远。”——最终,神在基督里赐下了最伟大的救恩,使我们得享永生。
\subsubsection*{现实生活中的应用:}
\begin{itemize}
    \item 信靠神的供应,不为明天忧虑——我们是否因经济压力、未来的不确定而焦虑?神是信实的,他会供应我们所需的(太6:33)。

    \item 遵行神的诫命,回应他的信实——既然神信实守约,我们是否也愿意忠心遵行他的诫命,不随从世界?

    \item 感谢神的救赎,活出圣洁——诗人特别提到神的救赎,我们是否珍惜这份恩典,过一个圣洁讨神喜悦的生活?

\end{itemize}
\subsubsection*{问题思考:}

你是否因生活的需要而忧虑,忘记了神的供应?
你是否忠心遵行神的诫命,回应他的信实?
\subsection*{三、敬畏神是智慧的开端,我们要行在真理(10节)}
% “敬畏耶和华是智慧的开端,凡遵行他命令的是聪明人。”(诗111:10)

\subsubsection*{何谓敬畏神?}

不是害怕神,而是尊崇他,顺服他的旨意。
智慧的开端——真正的智慧不是世界的聪明,而是明白神的旨意,顺服他的带领。
\subsubsection*{现实生活中的应用:}
\begin{itemize}
    \item 用神的标准来做决定——我们是否常常用世俗的眼光看问题,而不是按神的话语来行?

    \item 在生活中尊崇神——无论是工作、学习、家庭,我们是否都把神放在第一位,还是让自己的欲望主导我们?

    \item 活出智慧的生命——智慧不仅是知识,更是实践。智慧人会遵行神的命令,而不是仅仅知道却不去行。

\end{itemize}
\subsubsection*{问题思考:}

你是否愿意让神的话语来指引你的人生,而不是世俗的标准?
你是否仅仅知道神的话语,而没有真正去实践?
\subsection*{结论与挑战}
诗篇111篇告诉我们:敬畏神并感恩他的作为,是我们信仰生活的核心。
\begin{enumerate}
    \item 感恩神的伟大作为——学习数算神的恩典,不论顺境逆境都存感恩的心。

    \item 信靠神的信实供应——面对挑战时,不要惧怕,而是相信神必然带领。

    \item 敬畏神并行在智慧中——用神的标准做决定,让神的话语塑造我们的生命。

\end{enumerate}

亲爱的弟兄姊妹,你是否愿意今天就开始重新调整你的生命,成为一个感恩、信靠、敬畏神的人?

\subsection*{结束祷告}
\textbf{亲爱的天父,}
感谢祢启示我们诗篇111篇,让我们看见祢的作为何等伟大,祢的供应何等信实,祢的智慧何等宝贵。主啊,帮助我们成为感恩的人,不因环境而动摇;帮助我们成为信靠的人,知道祢必供应我们所需;帮助我们成为敬畏祢的人,行在智慧中,遵行祢的诫命。
愿我们的生命能荣耀祢,奉主耶稣基督的名祷告,阿们!
%-----------------------------------------------------------------------------
\newpage
\section{诗篇第112篇:敬畏耶和华,福乐人生}
% 讲章:——诗篇112篇的启示
\subsection*{引言}
亲爱的弟兄姊妹,今天我们要查考诗篇第112篇。这篇诗篇描述了敬畏神的人如何蒙福,并将敬畏耶和华的义人与恶人形成鲜明对比。它清楚地告诉我们:真正的福乐不是来自财富或地位,而是来自对神的敬畏与顺服。
在当今社会,许多人追求财富、名誉、安逸的生活,但诗篇112篇提醒我们,真正的福气不是短暂的,而是从神而来的。让我们一起学习,如何成为敬畏神、蒙神祝福的人。

\subsection*{一、敬畏神的人必蒙福(1-3节)}
“你们要赞美耶和华!敬畏耶和华、甚喜爱他命令的,这人便为有福!”(诗篇112:1)

诗篇112篇一开始就强调:“敬畏耶和华的人是有福的!”这个“福”不是世界所定义的“财富、舒适、成功”,而是从神而来的真正满足和祝福。

\subsubsection*{敬畏神的人有哪些祝福?}

\hspace{0.6cm}家庭蒙福(2节)——“他的后裔在世必强盛,正直人的后代必要蒙福。”
一个敬畏神的人,不仅自己得福,他的家庭、子孙也会受益。家庭的兴旺不是来自财富,而是来自神的恩典。

物质蒙福(3节)——“他家中有货物,有钱财;他的公义存到永远。”
这节经文不是说敬畏神的人一定会成为百万富翁,而是说神会供应他们的需要,使他们在物质上不会缺乏。

\subsubsection*{现实生活中的应用:}

你是否真正敬畏神,并把神放在你生命的首位?
你是否用神的原则管理你的家庭和财务,而不是随从世界的标准?
\subsubsection*{问题思考:}

你是否在生活中寻求神的引导,还是依靠自己的聪明?
你如何在家中树立敬畏神的榜样,使你的后代也蒙福?
\subsection*{二、敬畏神的人刚强不惧怕(4-8节)}
“正直人在黑暗中,有光向他发现。他有恩惠,有怜悯,有公义。”(诗篇112:4)

诗篇112篇告诉我们,敬畏神的人即使在黑暗中,也不至于绝望,因为神是他们的光。

\subsubsection*{敬畏神的人有哪些特质?}

\hspace{0.6cm}他们慷慨待人(5节)——“施恩与人,借贷与人,他的事情要按公平办理。”
他们不是只想着自己的利益,而是愿意帮助有需要的人。

他们不惧怕坏消息(7节)——“他必不怕凶恶的信息;他心坚定,倚靠耶和华。”
他们的信心建立在神的应许上,而不是环境的好坏。

他们内心坚定,不被动摇(8节)——“他的心坚固,必不惧怕,直到他看见敌人遭报。”
他们知道神掌权,所以无论遇到什么挑战,都不会轻易被打倒。

\subsubsection*{现实生活中的应用:}

面对困难时,你是惊慌失措,还是相信神会带领你走出困境?
你是否愿意慷慨待人,成为神祝福他人的管道?
\subsubsection*{问题思考:}

你最近是否因为某些坏消息而感到焦虑?你是否愿意交托给神?
你是否用神的标准来处理金钱和人际关系,而不是依靠自己的智慧?
\subsection*{三、恶人最终被羞辱(9-10节)}
“恶人看见便恼恨,必咬牙而消化,恶人的心愿要归灭。”(诗篇112:10)

这节经文告诉我们:恶人虽然暂时看似成功,但最终会失败。

他们的心愿会落空——因为他们的根基不在神里面。

他们会恼怒义人的成功——因为他们看不到真正的福气来自神,而不是来自财富或势力。
\subsubsection*{现实生活中的应用:}

你是否有时看到恶人成功而感到不公?记住,神的公义最终会显明。
你是否愿意坚持敬畏神的道路,而不是羡慕世俗的成功?
\subsubsection*{问题思考:}

你是否因看到恶人昌盛而灰心?你愿意相信神最终的审判吗?
你是否愿意用敬畏神的标准来衡量成功,而不是世界的标准?
\subsection*{结论与挑战}
诗篇112篇提醒我们:真正蒙福的人是敬畏神、喜爱神命令的人。
\begin{enumerate}
    \item 敬畏神,必得福——无论是家庭、物质,还是属灵生命,敬畏神的人都必蒙福。

    \item 坚固信心,不惧怕——即使面对困难,敬畏神的人仍然坚定不移,因他们信靠神。

    \item 坚持公义,最终得胜——恶人虽然暂时昌盛,但最终会败坏;敬畏神的人最终必得胜。

\end{enumerate}

亲爱的弟兄姊妹,你是否愿意选择敬畏神的道路,让你的生命充满神的祝福、坚定的信心和真正的智慧?

\subsection*{结束祷告}
\textbf{亲爱的天父,}

感谢祢赐下诗篇112篇,让我们明白敬畏祢的人是何等有福。主啊,帮助我们常常敬畏祢,不是口头上的敬畏,而是从内心深处尊崇祢,并遵行祢的诫命。主啊,求祢坚固我们的信心,使我们在困难中不惧怕,知道祢掌权。求祢赐给我们慷慨的心,让我们乐意帮助他人,成为祢祝福的管道。主啊,帮助我们不羡慕恶人的成功,而是单单仰望祢的公义。愿我们的生命活出敬畏神的智慧,成为祢所喜悦的儿女!

奉主耶稣基督的名祷告,阿们!
%-----------------------------------------------------------------------------
\newpage
\section{诗篇第113篇:从卑微到尊荣}
% 讲章:——诗篇113篇的启示
\subsection*{引言}
亲爱的弟兄姊妹,今天我们要查考诗篇113篇。这篇诗篇是赞美诗,重点强调神至高无上,却顾念卑微。它提醒我们:神不仅是至高的君王,也是满有怜悯的父,他提升卑微的人,使他们得着荣耀。
这篇诗篇特别适用于今天的社会。很多人感到自己渺小、无助,甚至被社会边缘化。但诗篇113篇告诉我们:无论我们的处境如何,神都看见我们,他能使卑微的人升高,使失望的人充满希望。

\subsection*{一、从日出到日落都当赞美神(1-3节)}
“你们要赞美耶和华!耶和华的仆人哪,你们要赞美,赞美耶和华的名!”(诗篇113:1)

诗篇一开始就以呼召敬拜作为主题:“你们要赞美耶和华!”这个呼召不只是针对某些特定的人,而是针对所有神的子民。

\subsubsection*{为什么我们要赞美神?}

\hspace{0.6cm}因为他的名是配得称颂的(2节)——神的圣名永远配得敬畏和尊崇。

因为他的赞美是持续的(3节)——“从日出之地到日落之处,耶和华的名是应当赞美的。”意思是,无论早晨还是夜晚,无论在哪个国家、哪个文化中,神都值得被称颂。

\subsubsection*{现实生活中的应用:}

我们是否常常以感恩和赞美的心来面对每一天?还是习惯于抱怨?
我们的敬拜是否只是停留在周日的教堂里,还是每天都活在敬拜中?
\subsubsection*{问题思考:}

你每天是否有固定时间感谢和赞美神?
你是否在困难中仍然选择赞美,而不是只在顺境时赞美?
\subsection*{二、神至高无上却顾念卑微(4-6节)}
“耶和华超乎万民之上,他的荣耀高过诸天。”(诗篇113:4)

这几节经文强调神的至高无上,他高过万民,高过诸天。然而,令人惊奇的是,这位至高的神却愿意顾念卑微的人(6节)。

\subsubsection*{神如何顾念卑微的人?}

\hspace{0.6cm}他不仅是伟大的王,更是慈爱的父——他看到那些被人忽视、被人轻看的小人物,并亲自提升他们。

他的荣耀不影响他的怜悯——世界上的权势者往往只顾自己的利益,但神不同,他愿意俯身看顾我们。

\subsubsection*{现实生活中的应用:}

你是否曾感到自己渺小、被忽视?神看见你,并且顾念你。
你是否也愿意像神一样,关心那些社会上的弱势群体,如贫穷者、孤儿、病患?
\subsubsection*{问题思考:}

你是否曾因觉得自己微不足道而感到灰心?今天神要告诉你:他看见你!
你是否也愿意成为神手中的器皿,去关心那些卑微困苦的人?
\subsection*{三、神使贫寒者升高,使不孕者得喜乐(7-9节)}
“他从灰尘里抬举贫寒人,从粪堆中提拔穷乏人,使他们与王子同坐,就是与本国的王子同坐。”(诗篇113:7-8)

这两节经文让我们看到神的恩典如何改变人的生命。

\subsubsection*{神的作为如何彰显?}

\paragraph*{他提升贫寒人,使他们得尊荣(7-8节)}
在古代社会,贫寒人和乞丐通常被看作是无用的,但神却从灰尘中抬举他们,让他们与王子同坐。
这不仅仅是物质的改变,更是身份的提升。神让那些世界看不起的人,在他的国度中拥有尊贵的地位。
\paragraph*{他使不孕的妇人得喜乐(9节)}
在旧约时代,不孕常被视为羞辱,但神有能力使不孕者成为快乐的母亲。
这不仅指生育的祝福,也可以象征属灵的结果子,神能使我们的生命充满意义和喜乐。
\subsubsection*{现实生活中的应用:}

你是否正处于人生的低谷?请相信,神能改变你的处境,让你从卑微中被提升。
你是否愿意以信心仰望神,而不是被眼前的环境所限制?
\subsubsection*{问题思考:}

你是否曾因贫穷、失败或过去的经历而感到自卑?今天神要告诉你:他能提升你!
你是否相信神能使你生命结出丰盛的果子,即使现在还看不到?
\subsection*{结论与挑战}
诗篇113篇提醒我们:神是配得赞美的,他是至高的神,却又愿意顾念卑微的人。
\begin{enumerate}
    \item 持续赞美神,不论环境如何——从日出到日落,神都配得称颂。

    \item 相信神看顾你,即使你觉得自己渺小——神不仅是荣耀的君王,他也是慈爱的父。

    \item 信靠神能翻转人生——他能提升贫寒者,使他们得尊荣;他能使不孕者结出果子,满心喜乐。

\end{enumerate}

亲爱的弟兄姊妹,你是否愿意相信神的能力,赞美他,并依靠他改变你的生命?

\subsection*{结束祷告}
\textbf{亲爱的天父,}

感谢祢通过诗篇113篇提醒我们,祢是至高无上的神,却又满有怜悯。主啊,我们愿意每天都赞美祢,不论环境如何,都不停止感恩。主啊,求祢帮助我们相信,祢看顾卑微的人,我们的价值在于祢,而不在于世界的评价。主啊,我们也求祢翻转我们的生命,使我们从低谷中被提升,经历祢的恩典与祝福。

愿我们用一生来荣耀祢,奉主耶稣基督的名祷告,阿们!
%-----------------------------------------------------------------------------
\newpage
\section{诗篇第114篇:神的同在,带来自然与生命的震动}
% 讲章:——诗篇114篇的启示
\subsection*{引言}
亲爱的弟兄姊妹,今天我们要查考诗篇第114篇。这篇诗篇回顾了以色列人出埃及的神迹,并描述了当神的同在降临,大自然如何震动,海洋如何退却,磐石如何出水。这不仅是历史事件的回顾,更是对我们今日属灵生命的提醒:当神的同在进入我们的生命时,我们的环境、我们的困难、我们的生命都会经历震动和更新。
今天,我们一同思考:当神的同在在我们生命中彰显,我们是否愿意被改变?我们是否愿意让神在我们的困境中行奇事?

\subsection*{一、神带领他的百姓脱离捆绑(1-2节)}
“以色列出了埃及,雅各家离开说异言之民;那时犹大为主的圣所,以色列为他所治理的国度。”(诗篇114:1-2)

这两节经文回顾了以色列人出埃及的伟大历史,他们从奴役中被释放,成为神的选民,进入神的治理之下。

\subsubsection*{神拯救以色列的历史对我们今天的启示:}

\hspace{0.6cm}从世界的捆绑中得自由——埃及象征罪恶与世界的辖制,而出埃及象征神的救赎。今天,我们因耶稣基督的救恩,也得以脱离罪的捆绑,进入神的国度。

神的百姓归属于神——以色列成为神治理的国度,我们今日作为神的儿女,也被神带入他的同在,成为属神的子民。

神的同在是我们的圣所——“犹大为主的圣所”,意味着神住在他的百姓当中。今天,神也愿意住在你我的生命中,引导我们。

\subsubsection*{现实生活中的应用:}

你是否仍然被世界的价值观捆绑,还是已经进入神的自由?
你是否让神真正作你生命的王,还是仍然靠自己掌控人生?
\subsubsection*{问题思考:}

你是否因过去的罪和失败而觉得自己无法被神使用?请记住,神拯救以色列,不是因为他们完美,而是因着他的恩典。
你是否真正相信,神的同在比世界的成就更重要?
\subsection*{二、神的同在震撼自然界(3-8节)}
“大海看见就奔逃,约旦河也倒流;大山踊跃,如公羊,小山跳舞,如羊羔。”(诗篇114:3-4)

这段经文描述了神的同在如何使红海分开、约旦河倒流、群山震动。这些神迹表明:当神临到时,无法逾越的障碍都会让路,甚至自然界都会颤抖!

\subsubsection*{神的同在带来哪些震撼?}

\hspace{0.6cm}拦阻会退去(3节)——红海原本是拦阻以色列人的障碍,但在神的同在中,红海却分开,让他们得以通过。
今天,神的同在也能使你生命中的困境、障碍退去,关键是你是否相信并顺服。

旧有模式会改变(4节)——大山跳跃、约旦河倒流,意味着神的能力可以翻转不可能的局面。
今天,神仍然能改变你的环境、医治你的创伤、翻转你的生命。

磐石出水,供应我们的需要(8节)——以色列人在旷野没有水喝,神却使磐石流出水来供应他们。
今天,我们在干渴、疲乏时,神仍然是我们生命的活水,他必供应我们的一切需要。

\subsubsection*{现实生活中的应用:}

你是否正面临人生的红海?请相信,神能使你的困境让路!
你是否有生命中的“旷野”——疲乏、干渴、迷失?神能在最坚硬的磐石中为你流出活水!
\subsubsection*{问题思考:}

你是否愿意完全信靠神,即使你的环境看起来不可能改变?
你是否在困难中仍然选择赞美神,而不是只在顺境时感谢他?
\subsection*{结论与挑战}
诗篇114篇提醒我们:当神的同在降临,一切都会改变。
\begin{enumerate}
    \item 神的同在带来自由——他拯救我们脱离罪恶,带领我们进入他的国度。

    \item 神的同在带来震动——拦阻会退去,环境会改变,我们的生命会被翻转。

    \item 神的同在带来供应——即使在旷野,神仍然能从磐石中流出活水,满足我们的一切需要。

\end{enumerate}

亲爱的弟兄姊妹,你是否愿意邀请神进入你的生命,让他掌管你的道路,使你的生命经历神迹?你是否愿意在困境中信靠神,相信他必开道路?

\subsection*{结束祷告}
\textbf{亲爱的天父,}

感谢祢透过诗篇114篇提醒我们,祢是至高无上的神,祢带领以色列人出埃及,翻转他们的命运,祢也能翻转我们的生命。主啊,求祢进入我们的生命,使我们从罪的捆绑中得自由,使我们经历祢的神迹。主啊,面对人生的红海,我们愿意信靠祢,相信祢必开道路。面对旷野的干渴,我们愿意来到祢面前,求祢赐下属灵的活水,滋润我们的心灵。愿我们的生命常常被祢的同在充满,成为见证祢荣耀的人!

奉主耶稣基督的名祷告,阿们!
%-----------------------------------------------------------------------------
\newpage
\section{诗篇第115篇:荣耀归于神}
% ——诗篇115篇的信仰实践
% 经文: 诗篇115篇

\subsection*{引言:}
弟兄姊妹,我们常常会面对这个世界的挑战,身处自我主义、金钱至上、偶像崇拜横行的时代,人们容易迷失方向,被世俗价值观影响。然而,诗篇115篇向我们指明一条坚定的道路:荣耀唯独归于神。今天,我们要从这篇诗篇中学习如何在现实生活中活出信仰,如何抵挡偶像的诱惑,并且如何真正信靠神。

\subsection*{一、唯独神配得荣耀(1节)——摆正我们人生的焦点}
% “耶和华啊,荣耀不要归于我们,不要归于我们,要因你的慈爱和诚实归在你的名下。”

\subsubsection*{1. 抵挡自我中心,谦卑归荣耀给神}
在现实生活中,我们容易陷入自我荣耀的陷阱,比如在工作中追求个人成就,在社交媒体上渴望他人的认可,甚至在侍奉中也可能希望别人称赞我们。但诗篇115:1提醒我们:荣耀唯独归于神,因为他的慈爱和信实才是我们人生的根基。

\subsubsection*{实际应用:}

在工作中,不是为自己的名声而奋斗,而是以神的公义和诚信去影响他人。
在生活中,凡事先求神的国和神的义,而不是被人认可的焦虑捆绑。
在服侍中,不是为了被看见,而是单单为了讨神喜悦。
\subsubsection*{问题反思:}

我是否有时候过于在意别人的评价,而忽略了神的荣耀?
我能否在生活中真实地荣耀神,而非荣耀自己?
\subsection*{二、偶像的虚无(2-8节)——远离世界的偶像}
% “他们的偶像是金的,银的,是人手所造的。”(4节)

\subsubsection*{1. 现代社会的“偶像”是什么?}
圣经中的偶像是金银木石雕刻的神像,而今天,偶像的形式更为隐蔽和普遍。它可能是金钱、权力、科技、成功、甚至某些人。人若倚靠这些,就像诗篇所说的,那些偶像\textbf{“有口却不能言,有眼却不能看,有耳却不能听”}(5-6节),终究不能拯救人。

\subsubsection*{实际应用:}

\hspace{0.6cm}金钱:我们是否过分依赖金钱,觉得财富能带来安全感?圣经提醒我们:“依靠耶和华强似倚赖世人。”(诗118:8)

科技:人工智能、社交媒体是否占据了我们的心,影响我们与神的关系?

名声和成就:我们是否用世俗的成功标准来衡量自己,而忽视神在我们生命中的旨意?

\subsubsection*{2. 偶像的影响——成为像它一样}
诗篇115:8说:“造它的要和它一样,凡倚靠它的也要如此。” 这意味着,人若倚靠虚无的事物,就会变得和它们一样无生命、无意义。现代社会有太多人被金钱、娱乐、名利吞噬,失去了属灵的生命和真实的喜乐。

\subsubsection*{问题反思:}

我生命中是否有任何偶像,使我无法全然信靠神?
我是否把工作、学业、财富、社交地位看得比神更重要?
\subsection*{三、倚靠耶和华(9-15节)——得神的帮助和赐福}
% 诗篇115篇接下来三次呼唤:“以色列啊,你们要倚靠耶和华!”(9-11节)

\subsubsection*{1. 信靠神带来的平安}
真正的信心不是嘴上说信,而是在困难中仍然坚信神掌权。例如,生活中的经济压力、学业挑战、人际关系问题,都可能让我们焦虑。但神应许我们:“倚靠耶和华的人,耶和华是他们的帮助和盾牌。”(11节)

\subsubsection*{实际应用:}

面临困难时,先祷告交托,而不是先求人的帮助。
习惯依靠神,而非仅仅依赖自己的努力。
在每个选择中,让神的话语成为引导,而非世俗的标准。
\subsubsection*{问题反思:}

我是否愿意在困难中相信神的带领,而非焦虑?
我是否每天花时间亲近神,培养对他的信靠?
\subsection*{四、神的祝福临到敬畏他的人(12-18节)——活出信仰影响世代}
诗篇115:12-13说:“耶和华向来眷念我们;他还要赐福给我们,要赐福给以色列的家,赐福给亚伦的家。”

\subsubsection*{1. 神的祝福不是仅仅指物质,而是生命的丰盛}
今天很多人误解祝福,以为信神就一定会发财、升职、顺遂。然而,神真正的祝福是我们灵里的富足、平安,以及他的同在。

\subsubsection*{2. 影响我们的后代(14-15节)}
神的祝福是世世代代的(14节),所以我们要把信仰传承给后代,不仅是教导他们,更是以生命见证神的信实。

\subsubsection*{实际应用:}

在家庭中,以身作则,引导家人敬畏神。
在职场、校园,以公义和爱心成为祝福的管道。
用生活方式影响下一代,而不仅仅是教条式的传授信仰。
\subsubsection*{问题反思:}

我是否真正渴望神的祝福,而非仅仅世俗的好处?
我如何影响我的家人、朋友,让他们认识神?
\subsection*{结论:荣耀神,拒绝偶像,完全信靠}
诗篇115篇给我们三个重要的提醒:

1. 荣耀唯独归于神,生活中不要让自我中心取代神。
2. 远离现代社会的偶像,不让金钱、名声、成就成为我们的依靠。
3. 信靠神,得他的帮助,并成为祝福的器皿,影响我们的家庭和后代。
愿我们每一天都活在对神的敬畏和信靠中,让荣耀归于神!

\subsection*{祷告}
\textbf{天父,}我们感谢你,因为你是独一真神,配得一切荣耀。求你赦免我们有时过分追求自我的荣耀,或倚靠世界的偶像。帮助我们全心信靠你,在生活的每个方面都归荣耀给你。愿我们成为你忠心的仆人,在家庭、职场、校园中见证你的信实,使更多人认识你。奉主耶稣基督的名祷告,阿们!
%-----------------------------------------------------------------------------
\newpage
\section{诗篇第116篇:“我爱耶和华”——看信徒的感恩与回应}

% 经文: 诗篇116篇

\subsection*{引言:}
弟兄姊妹,我们都曾经历人生的风暴,也曾在软弱中向神呼求。每当我们经历神的拯救,心中便会充满感恩。然而,我们该如何回应神的恩典?诗篇116篇是一首充满个人情感的诗歌,表达了诗人因经历神的拯救而深深地爱慕神,并愿意用一生来回应他。今天,我们要借着这篇诗篇学习:如何在现实生活中经历神的恩典,并以实际行动来回应神的爱。

\subsection*{一、神垂听祷告(1-4节)——患难时向神呼求}
% “我爱耶和华,因他听了我的声音和我的恳求。”(1节)

\subsubsection*{1. 诗人的经历:从困境到拯救}
诗篇116篇的作者深知神的恩典,因为他曾经历极大的苦难。他形容自己“死亡的绳索缠绕我,阴间的痛苦抓住我”(3节)。然而,在痛苦中,他没有绝望,而是向神呼求:“耶和华啊,求你救我的灵魂!”(4节)

\subsubsection*{2. 现实中的应用:我们如何面对困境?}

当你身处生活的低谷时,你会选择向谁倾诉?是依靠自己,还是倚靠神?
当焦虑、压力、疾病、家庭问题临到时,我们是否愿意像诗人一样,第一时间来到神的面前,迫切呼求他?
\paragraph*{实际应用:}

养成每日向神祷告的习惯,不只是遇到困难才呼求他。
在疾病、困境中,先祷告,再行动,而不是先埋怨或焦虑。
见证神的垂听:当神回应我们的祷告时,要记得感恩,并与他人分享神的作为。
\paragraph*{问题反思:}

我是否只有在遇到困难时才向神祷告?
我是否相信神真的听祷告,并且愿意倚靠他?
\subsection*{二、神是慈爱公义的(5-11节)——信靠神的性情}
% “耶和华有恩惠,有公义,我们的神以怜悯为怀。”(5节)

\subsubsection*{1. 神的属性:恩惠、公义、怜悯}
诗人见证说:神不是冷漠的,而是有恩惠、有公义、满有怜悯的神。这意味着:

他愿意施恩(即使我们不配)。

他行事公义(他必报应罪恶)。

他满有怜悯(他体恤我们的软弱)。
\subsubsection*{2. 现实中的应用:信靠神的性情,而非环境}

当世界充满不公时,信靠神的公义:面对社会不公、职场压力、甚至被人误解,我们可以选择相信神的公义,而不是自己伸冤。
当人生遭遇低谷时,信靠神的怜悯:即使我们软弱、失败,神仍然怜悯我们,不撇弃我们。
\paragraph*{实际应用:}

遇到不公对待时,选择交托,而非报复。
在软弱和失败时,不要逃避神,而是回到他面前,承认自己的需要。
在每天的生活中,操练信靠神的良善,即使环境看起来不理想。
\paragraph*{问题反思:}

我是否真正相信神是恩惠、公义、怜悯的神?
我是否愿意在不公义的环境中忍耐,等候神的公义彰显?
\subsection*{三、向神还愿(12-14节)——感恩的回应}
% \subsubsection{“我拿什么报答耶和华向我所赐的一切厚恩?”(12节)}

\subsubsection*{1. 诗人的感恩之心}
诗人经历了神的拯救之后,没有把神的恩典当作理所当然,而是思考如何回应神。他的答案是:

举起救恩的杯(13节):意味着不断感恩,颂扬神的救赎。

向神还愿(14节):意味着用实际行动回应神的恩典。
\subsubsection*{2. 现实中的应用:如何以行动感恩神?}

\hspace{0.6cm}用言语感恩:每天操练感谢神,即使环境不好,仍然数算恩典。

用行动回应:用生命回应神,例如投身服侍、帮助有需要的人。

用忠心持守承诺:如果曾向神许愿,就要认真履行,不可轻易忘记。

\paragraph*{实际应用:}

养成写感恩日记的习惯,每天记录神的恩典。
在教会或社区服侍,将神的恩典传递出去。
在奉献上忠心,以金钱、时间、精力回应神的恩典。
\paragraph*{问题反思:}

我是否容易忘记神的恩典,而没有感恩的心?
我是否愿意用行动回应神,而不仅仅是口头说感谢?
\subsection*{四、活着是为神(15-19节)——奉献自己给神}
% “在耶和华眼中,看圣民之死极为宝贵。”(15节)

\subsubsection*{1. 诗人的觉悟:生命属于神}
诗人认识到,他的生命完全属于神,即使死亡也不能将他与神隔绝。因此,他愿意献上自己,为神而活(16节)。

\subsubsection*{2. 现实中的应用:如何活出为神而活的人生?}

\hspace{0.6cm}活出圣洁的生命:不要随从世界,而要行在神的旨意中。

勇敢为神做见证:不惧怕人,而是愿意在职场、家庭、学校见证基督。

将一生交托神:不再为自己而活,而是愿意按神的计划走人生道路。
\paragraph*{实际应用:}

在选择事业、婚姻、未来时,先问:这是否荣耀神?
在生活中,努力活出基督的样式,让别人因我们而认识神。
以顺服的心态,愿意接受神的带领,即使他的计划与我们的想法不同。
\paragraph*{问题反思:}

我是否真的愿意将生命献给神,而不是只在需要他的时候才靠近他?
我是否在生活的每个方面,都努力活出为神而活的生命?
\subsection*{结论:爱神、信靠神、回应神}
诗篇116篇教导我们:

1. 在困境中呼求神,相信他垂听祷告。
2. 信靠神的性情,不被环境左右。
3. 以感恩的心回应神,并忠心履行向神的承诺。
4. 活着是为了荣耀神,将自己献上当作活祭。
愿我们一生都能说:“我爱耶和华!”

\subsection*{祷告}
\textbf{天父,}我们感谢你,因为你是听祷告的神。你从困境中拯救我们,又以你的恩惠、怜悯和公义引导我们。求你帮助我们,不仅在口头上爱你,更愿意在行动中回应你的恩典。愿我们的生命成为你的荣耀,愿我们的心常常存感恩。奉主耶稣基督的名祷告,阿们!
%-----------------------------------------------------------------------------
\newpage
\section{诗篇第117篇:普世的赞美——神的慈爱与信实}
% 《》
% ——诗篇117篇讲章

% 经文: 诗篇117篇

\subsection*{引言:}
诗篇117篇是整本圣经最短的篇章,只有两节经文,但其信息却极其深远。它不是个人的祷告,而是一首全球性的呼召,呼吁万国万民都来颂赞神。这篇诗篇不仅展现了神的普世性,也向我们揭示了赞美的根基——神的慈爱与信实。
今天,我们要透过这短短的两节经文,深入剖析为什么全世界都应当赞美神?我们又该如何在现实生活中活出赞美?

\subsection*{一、全地当赞美神(1节)——赞美的普世性}
“万国啊,你们都当赞美耶和华!万民哪,你们都当颂赞他!”(诗篇117:1)

\subsubsection*{1. 赞美的对象:耶和华}
\hspace{0.6cm}诗人清楚地指出,我们的赞美不是建立在人或环境之上,而是专属于耶和华——创造主、救赎主、全地的掌权者。

在现实生活中,人们常常把赞美的焦点放在错误的对象上:
有人崇拜金钱,以为财富能带来安全感;
有人崇拜权势,追求社会地位;
有人崇拜科技和理性,认为人类智慧可以解决一切问题;
但诗篇117篇提醒我们,唯有神配得我们的敬拜,因为他才是全地的主宰。

\subsubsection*{2. 赞美的范围:万国万民}
诗篇117篇打破了狭隘的民族主义,表明神的救恩不是只给以色列,而是给全世界。这在当时是一个非常突破性的观念,因为犹太人曾认为自己是神的选民,而外邦人与神无关。但在这里,神向所有民族发出邀请,表明他的恩典不受种族、文化、国界的限制。

\subsubsection*{3. 现实应用:如何在生活中实践普世的赞美?}
\hspace{0.6cm}培养全球视野:关注世界各地的宣教工作,为未得之民祷告,而不是只关心自己的生活。

尊重不同文化的人:无论对方是什么背景,我们都当以基督的爱待人。

在任何环境下都愿意见证神:无论是在家、在学校、在职场,都不隐藏我们的信仰,而是用行动和言语颂赞神。
\paragraph*{问题反思:}

我是否认为神只是“我的神”,还是“全世界的神”?
我是否愿意让我的生命成为见证,使更多人认识神?
\subsection*{二、赞美的根基:神的慈爱与信实(2节)}
“因为他向我们大施慈爱,耶和华的诚实存到永远。你们要赞美耶和华!”(诗篇117:2)

\subsubsection*{1. 神的大爱(慈爱)}
\hspace{0.6cm}这里的“慈爱”(hesed,希伯来文)指的是神坚定不移的爱,是一种永不改变、充满怜悯、无条件的爱。

现实中的应用:

在失败时,神仍然爱我们——我们不因犯错就失去神的恩典。

在软弱时,神仍然扶持我们——即使我们对神不忠,他仍然信实。

在迷失时,神仍然寻找我们——他如同好牧人,寻找失丧的羊。

我们不配得神的爱,但他却主动将他的慈爱赐给我们,这是我们应当赞美他的原因!

\subsubsection*{2. 神的信实(诚实)}
\hspace{0.6cm}诗人强调神的诚实(信实)存到永远。人会改变、背叛、失败,但神的信实永不改变。

现实中的应用:

在不确定的未来中,我们可以信靠神——无论世界如何变化,神的应许永不落空。

在困境中,我们可以相信神会带领——即使道路崎岖,神仍然掌管一切。

在诱惑和挑战中,我们可以坚持信仰——因为神的真理不会动摇。

\subsubsection*{3. 现实生活的应用:如何回应神的慈爱与信实?}
\hspace{0.6cm}每日操练感恩:每天记录神的恩典,避免抱怨和负面情绪。

坚定信靠神:无论遭遇什么,不动摇信仰,而是持守神的应许。

以行动见证神:在生活中活出神的爱,关心身边的人。


\paragraph*{问题反思:}

我是否常常数算神的恩典,而不是专注于缺乏?
在困难时,我是否仍然相信神的信实,而不是靠自己?
\subsection*{结论:活出敬拜的生命}
诗篇117篇虽然简短,但信息极为丰富。它提醒我们:

神配得万国万民的赞美——他不是某个民族的神,而是全地的主。

神的慈爱无可比拟——他以坚定不移的爱待我们,即使我们不配。

神的信实永不改变——无论环境如何,他的应许永远有效。

愿我们不只是口头赞美神,而是用一生来回应他的爱!

\subsection*{祷告}
\textbf{天父,}我们感谢你,因你是全地的神,配得万国万民的敬拜。你的慈爱长存,你的信实永远不变,我们在任何环境中都愿意赞美你!求你帮助我们,在日常生活中活出信仰,不仅在教会里赞美你,也在世界中见证你的荣耀。愿我们的生命成为你的器皿,使更多人认识你。奉主耶稣基督的名祷告,阿们!
%-----------------------------------------------------------------------------
\newpage
\section{诗篇第118篇:信靠耶和华的人有福了}
% 《》——诗篇118篇讲章
% 经文:诗篇118篇

\subsection*{引言}
\hspace{0.6cm}诗篇118篇是一首充满感恩与得胜的诗篇,它不仅是个人的见证,也是整个信仰群体的宣告。许多学者认为,这首诗是以色列人在圣殿中高声歌唱的诗歌,特别是在庆祝神的拯救与信实时。

这一篇诗篇不仅在犹太人的敬拜中被使用,在新约中也多次被引用。耶稣进耶路撒冷时,百姓高喊的“奉主名来的是应当称颂的”(26节),正是出自这篇诗篇! 使徒彼得也在使徒行传4:11引用“匠人所弃的石头,已成了房角的头块石头”(22节)来指向基督。

今天,我们要透过诗篇118篇,来思想神的慈爱、得胜的信心,以及我们如何在现实生活中依靠神。

\subsection*{一、因神的慈爱,我们当称谢(1-4节)——感恩是信仰的核心}
“你们要称谢耶和华,因他本为善,他的慈爱永远长存!”(1节)

\subsubsection*{1. 诗人的呼吁:全体信仰群体都要感恩}
诗人不仅自己感恩,还呼召以色列民(2节)、祭司(3节)、敬畏耶和华的人(4节)都来感谢神。
这表明感恩不是个人的选择,而是整个群体的责任。
\subsubsection*{2. 现实中的应用:我们为什么要感恩?}
在生活中,我们常常专注于缺乏和问题,而忽略了神的恩典。然而,诗人提醒我们:无论环境如何,神的慈爱永不改变,我们当常存感恩的心。

\paragraph*{如何培养感恩的生活方式?}

每天记录神的恩典,避免把祝福当作理所当然。
在祷告中先感恩,再求告,让心灵被神的爱充满。
在困难中仍然感谢神,相信他掌管一切。
\paragraph*{问题反思:}

我是否容易抱怨,而不是数算神的恩典?
我如何在日常生活中操练感恩的习惯?
\subsection*{二、信靠神胜过倚靠人(5-9节)——信心的选择}
“倚靠耶和华,强似倚赖人;倚靠耶和华,强似倚赖王子。”(8-9节)

\subsubsection*{1. 诗人的经历:神在困境中的拯救}
诗人曾被急难围困(5节),但他呼求神,神就使他宽广。
即使仇敌围绕,他仍不惧怕(6节),因为知道神站在他这边。
他明白:人是有限的,唯有神是可靠的(8-9节)。
\subsubsection*{2. 现实中的应用:我们依靠什么?}
在现代社会,人们容易依靠金钱、人际关系、权势,但诗篇118篇提醒我们:只有神是完全可靠的。

\paragraph*{实际应用:}

在面对考试、工作挑战时,先祷告交托神,而不是只靠自己努力。
在做重大决定时,寻求神的引导,而不是只依赖人的建议。
在困境中,相信神比倚靠人更可靠,不被环境左右。
\paragraph*{问题反思:}

我是否更相信人的方法,而不是信靠神?
在困境中,我是焦虑害怕,还是安稳在神的信实中?
\subsection*{三、因神得胜,我们当喜乐(10-21节)——经历神的得胜}
“这是耶和华所定的日子,我们在其中要高兴欢喜!”(24节)

\subsubsection*{1. 诗人的经历:从危机到得胜}
诗人曾经历极大的压力(10-13节),甚至几乎被敌人毁灭(13节)。
但他宣告:“耶和华是我的力量,是我的诗歌,他也成了我的拯救。”(14节)
“义人的帐棚里有欢呼拯救的声音”(15节),表明神的子民都因他的作为而欢喜。
\subsubsection*{2. 现实中的应用:如何在生活中经历神的得胜?}
在软弱中,依靠神的力量,而不是只靠意志力撑下去。
在失败中,继续相信神能翻转困境,而不是放弃。
在顺境时,归荣耀给神,而不是归功自己。
\paragraph*{问题反思:}

我是否相信神能在困境中带来突破?
在胜利时,我是否把荣耀归给神,而不是归于自己?
\subsection*{四、神是我们的救恩,基督是我们的盼望(22-29节)}
“匠人所弃的石头,已成了房角的头块石头。”(22节)

\subsubsection*{1. 诗人的预言:被弃绝的石头成为根基}
这节经文预示了耶稣基督的降临和得胜。
耶稣被人弃绝,但神使他成为救恩的根基。
新约引用这节经文,表明耶稣是神计划的核心(太21:42,徒4:11)。
\subsubsection*{2. 现实中的应用:如何在基督里建立稳固的信仰?}
不被世界的价值观动摇,坚定跟随基督。
在患难中,知道基督已经得胜,不惧怕未来。
向世界见证耶稣,让更多人认识这位被弃绝的石头,成为他们的救主。
\paragraph*{问题反思:}

我是否真正以基督为生命的根基?
我是否愿意传扬基督,让更多人认识他?
\subsection*{结论:信靠神,活出感恩的生命}
诗篇118篇提醒我们:

神的慈爱永远长存,我们当感恩。

倚靠神胜过倚靠人,我们当信靠。

神是我们的拯救,我们当喜乐。

基督是我们的根基,我们当跟随。

让我们一生敬拜神,在他的得胜中欢喜快乐!

\subsection*{祷告}
\textbf{天父,}我们感谢你,因为你的慈爱永远长存!你是我们的力量,是我们的拯救,在困境中,你是我们唯一可靠的依靠。求你帮助我们,不倚靠人,不倚靠世界,而是全心信靠你。让我们的生命充满感恩,活出信仰,见证基督的荣耀!奉主耶稣基督的名祷告,阿们!
%-----------------------------------------------------------------------------
\newpage
\section{诗篇第119篇:喜爱神的话语,行走在光中}
% 《》
% ——诗篇119篇讲章

% 经文:诗篇119篇

\subsection*{引言:诗篇119篇的独特性}
\hspace{0.6cm}诗篇119篇是圣经中最长的诗篇,也是最强调神话语的重要性的一篇。整篇诗篇以\textbf{字母顺序诗(Acrostic)}的形式写成,每八节以希伯来字母顺序开头,显示出诗人对神话语的极大尊重。

本篇诗篇共有176节,不断地提到律法、法度、训词、命令、典章、话语等不同词汇,都是指向神的道。

\textbf{为什么神的话语如此重要?}

世界充满变化,但神的话永不改变。
人容易迷失,但神的话是指引方向的光。
我们常软弱,但神的话带来力量和安慰。

今天,我们要透过诗篇119篇,来思想神的话语如何塑造我们的生命,使我们走在光中,活出丰盛的信仰。

\subsection*{一、爱慕神的话语:敬虔生命的基础(1-16节)}
“行为完全、遵行耶和华律法的,这人便为有福!”(1节)

\subsubsection*{1. 遵行神话语的人有福了(1-3节)}
诗人一开始就宣告:敬畏神、遵行神律法的人是有福的! 这里的“有福”不仅指属灵的福气,也包括生活的稳固与满足。

\paragraph*{现实中的应用:}

我们每天如何安排时间? 是否愿意在忙碌的生活中花时间读圣经,而不是只依赖世界的知识?
我们是否单单听道,而没有行道? 许多人知道圣经的教导,却没有活出来。
\paragraph*{反思问题:}

我是否真正把神的话语放在心上,还是只是在需要时才翻开圣经?
我是否愿意调整生活习惯,让神的话成为我生命的准则?
\subsection*{二、神的话是我们的智慧(97-104节)}
“我何等爱慕你的律法,终日不住地思想。”(97节)

\subsubsection*{1. 默想神的话,使人有智慧(98-100节)}

\hspace{0.6cm}比仇敌更通达(98节):因神的命令,我们能明辨是非,不被敌人欺骗。

比师傅更聪明(99节):当一个人真正敬畏神,神会赐给他超越世俗的智慧。

比长老更明白(100节):真正的智慧不是来自年纪,而是来自遵行神的话语。
\subsubsection*{2. 现实生活中的智慧应用}
面对诱惑时,神的话语提醒我们不要犯罪。(101节)
面对决策时,神的话语指引我们正确的道路。
面对人际关系时,神的话语教导我们如何待人接物。
\paragraph*{反思问题:}

我是否真正将圣经作为生活的智慧来源?
在面对挑战时,我是否先寻求圣经的指引,而不是先问人?
\subsection*{三、神的话是我们脚前的灯(105-112节)}
“你的话是我脚前的灯,是我路上的光。”(105节)

\subsubsection*{1. 神的话语照亮前方的道路}
世界充满黑暗,许多时候我们会感到迷茫,不知道前面的路如何走。但诗篇119:105告诉我们:神的话就像一盏灯,指引我们前进的方向。

\paragraph*{现实生活的应用:}

在职业选择上,神的话语提醒我们追求公义,而不是唯利是图。
在人际关系上,神的话语提醒我们要饶恕,而不是记恨。
在道德抉择上,神的话语提醒我们持守圣洁,而不是随波逐流。
\paragraph*{反思问题:}

我是否愿意让神的话语来指引我每一天的生活?
当我面临选择时,我是优先参考圣经,还是世界的标准?
\subsection*{四、神的话是我们的盾牌与盼望(113-120节)}
“你是我藏身之处,又是我的盾牌,我甚仰望你的话语。”(114节)

\subsubsection*{1. 神的话语带来保护和安慰}
诗人宣告:神是他的藏身之处,是他的盾牌(114节)。
在危难和逼迫中,神的话语成为他的盼望(116节)。
\subsubsection*{2. 现实生活的应用:如何在困境中依靠神的话?}
面对压力时,用神的话语来鼓励自己,而不是陷入焦虑。
面对试探时,用神的话语来抵挡诱惑,而不是妥协。
面对逼迫时,持守神的话语,而不是随波逐流。
\paragraph*{反思问题:}

当我遇到困难时,我首先想到的是向神祷告,还是依靠自己的办法?
我是否愿意把神的话语作为我的安全感,而不是靠外在的环境?
\subsection*{结论:如何在生活中实践诗篇119篇的教导?}
\subsubsection*{1. 建立每日读经的习惯}
固定时间:每天清晨或睡前花时间读经。
固定计划:可以使用圣经阅读计划,系统性学习神的话语。
\subsubsection*{2. 养成默想神话语的习惯}
记忆经文:每天背一节圣经,使神的话扎根在心中。
随时思想:在工作、学习、行路时,默想神的话语,让它影响我们的思想和行为。
\subsubsection*{3. 在生活中行出神的话语}
面对试探时,用神的话来胜过诱惑。
面对决策时,参考圣经的教导,而不是只看环境。
面对挑战时,相信神的话语是我们的力量和安慰。

诗篇119篇不仅仅是一首诗歌,更是一封呼吁我们敬畏神、爱慕神的话语,并将其付诸实践的宣言!

\subsection*{祷告}
\textbf{慈爱的天父,}感谢你赐下你的话语,使我们的人生不再迷茫。你的话是我们脚前的灯,是我们路上的光,求你帮助我们
爱慕你的话,时常默想你的律法,在生活中实践你的教导。求你赐我们智慧,让我们胜过试探,坚守真道,活出合你心意的生命。愿你的话成为我们的盾牌,使我们在困境中仍然信靠你。感谢你,奉主耶稣基督的名祷告,阿们!
%-----------------------------------------------------------------------------
\newpage
\section{诗篇第120篇:在困苦中呼求神}
% 《》
% ——诗篇120篇讲章

% 经文:诗篇120篇

\subsection*{引言:诗篇120篇的背景}
\hspace{0.6cm}诗篇120篇是上行之诗的第一篇(120-134篇),这些诗篇是以色列人在朝圣上耶路撒冷敬拜时所唱的歌。它表达了敬拜者的心路历程:\textbf{从困苦到盼望,从世界到神的圣山}。

这首诗篇由六节经文组成,是一个充满痛苦呼求的祷告。诗人生活在一个充满谎言和冲突的环境,但他没有绝望,而是选择向神呼求。今天,我们也常面对不公正、诽谤、争斗,但神的话语告诉我们:当倚靠神,而不是被环境压倒!

今天我们要从诗篇120篇学习如何在困苦中倚靠神、如何面对恶劣环境,并如何保持信仰的坚守。

\subsection*{一、在困苦中向神呼求(1节)——祷告是出路}
“我在急难中求告耶和华,他就应允我。”(诗120:1)

\subsubsection*{1. 诗人的困境:急难中的祷告}
诗人面对极大的压力,这种“急难”可能是被误解、被攻击、被逼迫。
他没有去找人诉苦,而是直接向耶和华呼求,并且经历了神的应允。
\subsubsection*{2. 现实生活中的应用:当困境来临时,我们依靠什么?}
遇到困难时,我们的第一反应是什么?是抱怨、愤怒,还是来到神面前?
现代人面对困境时,常依赖人的方法(找关系、诉诸网络、用愤怒回应),但诗人提醒我们:最有效的出路是向神祷告!
\paragraph*{实际行动:}

建立稳定的祷告生活,无论顺境或逆境,都常常亲近神。
不要只在绝望时才找神,而是让他成为你生命的第一优先。
\paragraph*{反思问题:}

遇到问题时,我是否愿意第一时间向神呼求?
我是否相信神会垂听并回应我的祷告?
\subsection*{二、面对诡诈之人的攻击(2-4节)——神是公义的}
“耶和华啊,求你救我脱离说谎的嘴唇和诡诈的舌头!”(诗120:2)

\subsubsection*{1. 诗人的痛苦:被谎言中伤}
他被说谎的嘴唇和诡诈的舌头所攻击,可能遭受了恶意诽谤、误解和中伤。
这让他感到痛苦,但他没有用自己的方法反击,而是求神伸张公义。
\subsubsection*{2. 现实生活中的应用:当我们遭遇不公时}
现代社会充满假新闻、网络暴力、职场斗争、人际诽谤,基督徒也会遇到类似的伤害。
当你被误解、被攻击时,你是否能像诗人一样,把委屈交给神?
“诡诈的舌头”最终会被神审判(3-4节),我们要相信神的公义,而不是靠自己的愤怒来回应。
\paragraph*{实际行动:}

学习安静等候神,不要急于用愤怒回应别人。
凡事交托给神,让他来伸张公义,而不是用人的手段报复。
用诚实和爱心待人,即使别人诽谤我们,我们仍要持守真理。
\paragraph*{反思问题:}

当我被误解时,我是用愤怒回击,还是祷告交托神?
我是否也曾用言语伤害他人?
\subsection*{三、渴望远离争战(5-7节)——神是我们的平安}
“我寄居在米设,住在基达帐棚之中。”(诗120:5)

\subsubsection*{1. 诗人的痛苦:被困在好争斗的环境中}
米设:在黑海北部的外邦地,象征偏远、野蛮之地。
基达:以实玛利的后代,以游牧战争著称,象征暴力、冲突之地。
诗人感到自己被围绕在充满争斗、充满敌意的环境里,而他渴望和平(7节)。
\subsubsection*{2. 现实生活中的应用:如何在纷争中持守平安?}
在工作、家庭、人际关系中,我们常被卷入冲突,但神呼召我们成为和平的使者(太5:9)。
这个世界喜欢争论、对抗、竞争,但基督徒要成为带来和平的人。
有时候,我们无法改变周围环境,但我们可以选择如何回应。
\paragraph*{实际行动:}

避免无意义的争论,学习温柔和忍耐(提后2:23-24)。
用爱心化解冲突,而不是火上浇油。
为周围的环境祷告,求神赐下真正的平安。
\paragraph*{反思问题:}

我是否常常卷入无谓的纷争?
在紧张的环境中,我是制造冲突的人,还是带来平安的人?
\subsection*{结论:如何在困境中活出信仰?}
\subsubsection*{诗篇120篇教导我们:}

\hspace{0.6cm}当困苦来临时,要向神祷告,而不是靠自己的方法解决。

当遭受恶意攻击时,相信神的公义,而不是以恶报恶。

在充满纷争的环境中,努力做一个带来和平的人。
\subsubsection*{实际行动}
\hspace{0.6cm}每天花5-10分钟安静祷告,把困难交托神。

面对误解和诽谤时,选择用爱回应,而不是愤怒。

在家庭、职场、朋友圈里,努力成为一个带来和平的人。


\subsection*{祷告}
\textbf{天父,}我们感谢你,因为你是垂听祷告的神。当我们身处困境时,你应允我们的呼求。求你帮助我们,不被环境左右,而是坚定依靠你。让我们在诡诈的人群中,仍持守诚实;在充满纷争的世界里,仍成为和平之子。求你帮助我们凡事交托,相信你的公义,并以你的爱去影响周围的人。奉主耶稣基督的名祷告,阿们!
%-----------------------------------------------------------------------------
\newpage
\section{诗篇第121篇:神是我们的保护与帮助}
% 《》
% ——诗篇121篇讲章

% 经文:诗篇121篇

\subsection*{引言:诗篇121篇的背景与主题}
\hspace{0.6cm}诗篇121篇是上行之诗中的一部分,是以色列朝圣者在前往耶路撒冷的途中所唱的歌。这首诗篇以信靠与依赖神的保护为主题,向我们展示了神作为帮助者和保护者在我们生命中的重要性。诗人通过这首诗歌,提醒我们在生活中无论遇到任何挑战、困难或试炼,都能从神那里得到坚固的帮助和看顾。

诗篇121篇的主题是\textbf{“神是我们的保护与帮助”},无论在白昼还是黑夜,神都会保守我们。今天,我们会深入探讨这一真理,并反思如何在现实生活中,实践完全的依靠神。

\subsection*{一、神是我们的帮助(1-2节)}
“我向山举目,我的帮助从何而来?我的帮助从造天地的耶和华而来。”(诗121:1-2)

\subsubsection*{1. 诗人的疑问与回应}
诗人在面对人生的困难时,首先发出疑问:“我的帮助从哪里来?”
他看向高山,可能是寻找保护的象征,然而他很快得到的回答是:帮助来自神——那位创造天地的主。
\subsubsection*{2. 现实生活中的应用:我们的帮助从哪里来?}
在现代社会,我们常常感到迷茫或无助。经济压力、家庭困扰、事业瓶颈、情感挑战等问题,常常让我们觉得无法应对。
诗篇提醒我们,当我们面临困境时,不要把希望寄托在人的身上(如财富、关系、社会地位),而是要看向神,神是我们的真正帮助。
\paragraph*{实际行动:}

在困境中,将焦点从问题转移到神。
祷告:求神为你指引道路,赐给你智慧、力量。
信靠神的供应,相信他会在你需要的时刻,按他的方式帮助你。
\paragraph*{反思问题:}

当我遇到挑战时,我是否先想到寻求神的帮助,还是依赖自己或他人?
我是否经常在祷告中,向神呈上我的困境,并等待他的指引和帮助?
\subsection*{二、神是我们的守护者(3-6节)}
“耶和华是你的看护者,耶和华在你右边庇佑你。”(诗121:5)

\subsubsection*{1. 神守护我们的生命}
神不仅仅是我们的帮助者,更是我们的守护者。
诗人在这里强调神是一个全方位的守护者,无论是白天还是黑夜,他都时刻看顾我们(3-6节)。
神“不打盹,也不睡觉”,他对我们的保守是完全、持续、不间断的。
\subsubsection*{2. 现实生活中的应用:神的看护与保护}
在现代生活中,我们常常面临各种各样的危险:无论是身体健康的挑战,还是心理压力的困扰,或者社会和工作中的威胁。
诗篇121篇提醒我们,在这些困境中,我们不必担心,因为神是我们的看护者,他一直在我们身边保护我们。
神的保护是全方位的:不仅保守我们免受肉体的伤害,还保护我们的心理、情感和灵魂不受敌人的攻击。
\subsubsection*{实际行动:}

\hspace{0.6cm}信心的表现:无论在什么环境下,都要深信神是你的守护者,不会让你失落或迷失。

不断提醒自己:在任何情况下,神始终陪伴着你,他为你看顾一切。

感谢神的保护:每天早晨起床和晚上睡觉前,都可以感谢神守护你的一天,并祈求神继续看顾你。
\paragraph*{反思问题:}

在生活中,我是否意识到神时刻守护着我,保护我的生命免受危险?
当我遇到困境时,我是否相信神会保护我,且在每一个时刻都关注我的需要?
\subsection*{三、神是我们的保守者(7-8节)}
“耶和华要保守你出入,从今时直到永远。”(诗121:8)

\subsubsection*{1. 神对我们的全方位保守}
在诗篇的最后,诗人提到神的保守不仅仅是一时的,而是从今时直到永远。
神保守我们出入,指的是在日常生活的方方面面,他都看顾我们,从我们离开家门,到我们回到家里,神都在。
神的保守没有限期,他对我们的保护是无时无刻不在的。
\subsubsection*{2. 现实生活中的应用:神的保守是全面的}
我们的出入、行走、工作、学习、家庭等方面,神都在保守。
神的保守不仅仅是对我们肉体的保护,还包括对我们的情感、心智和灵魂的看顾。
当我们面对不确定的未来时,可以深信神的保守,知道无论我们的道路如何,神都掌权并看顾我们的每一步。
\subsubsection*{实际行动:}

\hspace{0.6cm}每一天都相信神在保守:无论是外出工作,还是回家休息,心中都可以安静地知道神在保护。

感谢神的保守:当我们经历平安时,不要忘记感谢神的保守与看顾。

依靠神的保守:当未来看起来不确定时,选择信靠神,知道他是保守我们走过每一步的力量。
\paragraph*{反思问题:}

我是否意识到神在我生命的各个方面都在保守着我?
我是否愿意在每个阶段的生活中,都依靠神的保守,而不是自己去挣扎?
\subsection*{结论:神是我们的帮助、守护与保守}
诗篇121篇提醒我们,神是我们一生的帮助和守护。无论面临怎样的挑战,我们都可以坚定信靠神,因为神会时刻保守我们,让我们在困境中得着帮助,在危险中得着保护,在未知的未来中得着保守。

\subsubsection*{实际应用}
\hspace{0.6cm}建立与神的亲密关系:每天在祷告中向神寻求帮助和保护。

信靠神的保守:无论今天面对什么困境,都要相信神必保守你,走在他的引导中。

传递信心与平安:在周围的人中,传递你从神得到的帮助与安慰。

\subsection*{祷告}
\textbf{亲爱的天父,}感谢你是我们的帮助、守护和保守。无论我们走到哪里,你的眼目常常看顾我们。求你帮助我们在生活的每一天,信靠你的保护和指引,依靠你胜过所有的困难和挑战。让我们在风雨中不动摇,心中坚定相信你的看顾。奉耶稣基督的名祷告,阿们!
%-----------------------------------------------------------------------------
\newpage
\section{诗篇第122篇:爱耶和华的殿,祈求耶路撒冷的平安}
% 讲章:——诗篇122篇的启示
\subsection*{引言}
亲爱的弟兄姊妹,今天我们要一起查考诗篇第122篇。这首诗是“大卫上行之诗”之一,是以色列百姓朝圣上耶路撒冷敬拜时所唱的诗歌,充满了喜乐、敬拜和对神的殿的热爱。
今天,我们可以从这首诗中学习如何爱神的教会、追求合一、并成为和平的使者。

\subsection*{一、喜乐地进入神的殿(1-2节)}
“人对我说:‘我们往耶和华的殿去。’我就欢喜。”(诗篇122:1)

大卫表达了他对敬拜的热爱。当听到有人邀请他去神的殿时,他的心充满喜乐。这种敬拜的渴望和热情,提醒我们要珍惜在神面前的敬拜时光。

\subsubsection*{现实生活中的应用}
\hspace{0.6cm}你是否渴望敬拜神?

许多基督徒对聚会和敬拜已经习以为常,甚至觉得可有可无。
但大卫告诉我们,敬拜神是一种特权,是我们心灵的喜乐源泉。

你是否带着喜乐的心来到神的面前?

你是否在敬拜中分心,心思仍然放在世俗事务上?

我们应当像大卫一样,珍惜每一次敬拜,把神放在生命的首位。
\subsubsection*{问题思考}
你是否把敬拜当作负担,而不是喜乐?
你是否愿意调整你的态度,让敬拜成为你生命中最期待的时刻?
\subsection*{二、耶路撒冷的合一与公义(3-5节)}
“耶路撒冷被建造,如同连络整齐的城。”(诗篇122:3)

\paragraph{耶路撒冷象征神的居所和属灵的合一。它不仅仅是以色列的首都,更是神设立的治理中心(5节)。}

\paragraph*{神的家要合一}
耶路撒冷的建筑“连络整齐”——象征教会要合一,不可分裂。

今天的教会常常因意见不同、文化差异而产生冲突,但神希望他的百姓合而为一。
耶稣在约翰福音17章为教会的合一祷告,合一的教会能彰显神的荣耀。
公义的治理(4-5节)——耶路撒冷不仅是敬拜的地方,也是神设立公义治理的地方。

在神的国度中,合一和公义是分不开的。
作为基督徒,我们也要在家庭、工作、教会中活出公义,维护神的秩序。
\subsubsection*{现实生活中的应用}
你是否在教会中制造纷争,还是努力追求合一?
你是否在工作和家庭中活出公义,成为光和盐?
\subsubsection*{问题思考}
你是否因为人与人之间的差异而产生隔阂?你是否愿意学习包容和合一?
你是否愿意为教会的合一和公义祷告?
\subsection*{三、为耶路撒冷的平安祷告(6-9节)}
“你们要为耶路撒冷求平安。”(诗篇122:6)

这节经文强调为耶路撒冷的平安祷告,在新约时代,这也象征我们要为神的国度、教会、国家祷告。

\subsubsection*{平安的意义}

\hspace{0.6cm}外在的平安——指国家、社会、教会的和睦与安全。

内心的平安——即便世界动荡不安,神仍然赐下属天的平安。

关系的平安——基督徒之间要彼此相爱,避免纷争。

\subsubsection*{现实生活中的应用}
你是否每天为教会、国家、世界的平安祷告?
你是否努力在你的家庭、职场、人际关系中成为和平的使者?
\subsubsection*{问题思考}
你是否愿意成为和平的见证人,不在冲突中火上浇油?
你是否愿意花时间为他人祷告,让神的平安临到?
\subsubsection*{结论与挑战}
诗篇122篇提醒我们:

珍惜敬拜的机会,带着喜乐进入神的同在。

在神的家中追求合一与公义,避免纷争和分裂。

为耶路撒冷、教会、世界的平安祷告,并成为和平的使者。

亲爱的弟兄姊妹,愿我们每一个人都珍惜敬拜、追求合一、为平安祷告,让神的荣耀在我们的生命中彰显!

\subsubsection*{结束祷告}
\textbf{慈爱的天父,}

感谢祢赐下诗篇122篇,让我们明白敬拜的喜乐、合一的重要、以及为平安祷告的责任。主啊,求祢帮助我们,不把敬拜当作例行公事,而是以喜乐的心来到祢面前。求祢使我们的教会合一,让我们彼此相爱,不让纷争和误解拆毁祢的家。主啊,我们也为教会、国家和世界的平安祷告,求祢施行拯救,使世界充满祢的和平。愿我们的生命成为祢的见证,使更多人认识祢。

奉主耶稣基督的名祷告,阿们!
%-----------------------------------------------------------------------------
\newpage
\section{诗篇第123篇:仰望神的怜悯}
% 讲章:——诗篇123篇的属灵启示
\subsection*{引言}
亲爱的弟兄姊妹,今天我们要查考诗篇第123篇。这首诗是“大卫上行之诗”之一,是以色列人朝圣前往耶路撒冷时所唱的诗歌。诗人用充满谦卑和信靠的语气,表达了对神的完全仰望,尤其是在困境、藐视和逼迫中,向神祈求怜悯。
今天,我们也会经历困难、讽刺、不公甚至逼迫。那么,当我们面对挑战时,我们是否愿意像诗人一样,全然仰望神的怜悯?

\subsection*{一、仰望神——我们的眼目单单注视他(1-2节)}
“坐在天上的主啊,我向祢举目!”(诗篇123:1)

诗人一开始就表明他的信心不是在地上的权势,而是在天上的神。这个举目仰望的姿态,是谦卑的表达,是对神的完全依靠。

\subsubsection*{1. 仰望神的供应}
“看哪,仆人的眼睛怎样望主人的手,使女的眼睛怎样望主母的手,我们的眼睛也照样望耶和华——我们的神,直到他怜悯我们。”(诗篇123:2)

诗人用“仆人仰望主人”来形容我们的依靠:

仆人依赖主人供应食物和保障,我们也要依靠神供应我们的需要。
仆人等待主人的吩咐,愿意顺服,我们也要聆听神的引导,遵行他的旨意。
\subsubsection*{现实生活中的应用}
当你遇到困难时,你是先依靠自己,还是第一时间转向神?
你是否像诗人一样,在等候神的旨意,而不是急躁地走自己的路?
\subsubsection*{问题思考}
你是否真正相信神是你的供应者,还是把希望放在金钱、能力和人的帮助上?
你是否愿意像仆人仰望主人一样,顺服神的计划,即便不符合你的期待?
\subsection*{二、呼求神的怜悯——面对嘲讽与逼迫(3-4节)}
“耶和华啊,求祢怜悯我们,怜悯我们!因为我们被藐视,已经到了极处。”(诗篇123:3)

诗人不仅仅是单单仰望神,更是在困境中迫切地向神呼求怜悯。他描述自己和百姓受到了极大的轻视和羞辱。

\subsubsection*{1. 受藐视的现实}
“那安逸人的讥诮和骄傲人的藐视,已经充满了我们的心。”(诗篇123:4)

诗人提到\textbf{“安逸人”和“骄傲人”},指的是那些不认识神、以自己为中心、嘲笑敬虔之人的群体。

他们可能是那些不信神的人,看不起依靠神的人。
他们可能是那些骄傲自满的人,以财富、权势或世俗成功来讥笑属神的人。

\subsubsection*{在今天的世界,我们可能也会经历类似的情况:}

\hspace{0.6cm}你可能因为坚持信仰,被人嘲笑为“愚昧”或“落伍”。

你可能因为选择诚实、正直而被人排挤。

你可能因为不愿意随波逐流,而被身边的人误解甚至讽刺。

但诗人给了我们榜样:当我们被人轻视、被世界误解时,不是与人争辩,而是转向神,求他怜悯!

\subsubsection*{现实生活中的应用}
你是否曾因为信仰而遭受讽刺或误解?
你是否愿意像诗人一样,把痛苦带到神的面前,而不是陷入愤怒或自怜?
\subsubsection*{问题思考}
你是否把世界的看法看得比神的看法更重?
你是否愿意选择坚守信仰,即使会被人讥讽?
\subsection*{结论与挑战}
诗篇123篇给了我们三个重要的属灵功课:

单单仰望神,不倚靠自己或世界的资源。

像仆人仰望主人一样,等候神的带领,并愿意顺服他的旨意。

在遭遇逼迫和嘲讽时,向神祈求怜悯,而不是靠自己报复或抱怨。

亲爱的弟兄姊妹,你是否在面对挑战时,第一时间就向神举目?你是否愿意把你的痛苦、羞辱带到神的面前,相信他会施行怜悯?

今天,愿我们都学习诗人的榜样,不论在顺境还是逆境中,始终仰望神、顺服神、信靠神!

\subsection*{结束祷告}
\textbf{慈爱的天父,}

感谢祢透过诗篇123篇教导我们,如何在困难和嘲讽中单单仰望祢。主啊,我们承认,很多时候我们太倚靠自己的聪明,太在意别人的眼光,而不是专心依靠祢。求祢帮助我们,使我们的眼目时刻定睛在祢的身上,如仆人仰望主人的手。

主啊,我们也求祢怜悯我们,在面对世界的嘲讽和挑战时,不让我们的心被苦毒充满,而是让我们更多地信靠祢,寻求祢的帮助。求祢赐给我们刚强的信心,使我们无论环境如何,都能持守祢的真理。

愿祢的怜悯临到我们,使我们的生命成为祢恩典的见证。奉主耶稣基督的名祷告,阿们!
%-----------------------------------------------------------------------------
\newpage
\section{诗篇第124篇:若不是耶和华帮助我们——信心之歌}
% 讲章:
\subsection*{引言}
\hspace{0.6cm}亲爱的弟兄姊妹,今天我们要查考诗篇124篇,这是一首大卫的上行之诗,表达了对神拯救的大能和信实的感恩。诗人回顾过去,承认若不是神的帮助,以色列早已灭亡。

我们的人生旅途也充满挑战,有时甚至感觉像是被仇敌追赶、被困境吞噬。但大卫提醒我们:神是我们的拯救者,他帮助我们脱离仇敌,赐给我们自由和胜利!

\subsection*{一、回顾神的恩典:若不是耶和华帮助我们(1-5节)}
“以色列人要说,若不是耶和华帮助我们,若不是耶和华帮助我们,当人起来攻击我们,向我们发怒的时候,就把我们活活地吞了。”(诗篇124:1-3)

\subsubsection*{1. 承认我们的软弱,数算神的恩典}
大卫两次重复“若不是耶和华帮助我们”,强调人本身无力抵挡仇敌。如果没有神,以色列早就被敌人吞灭了。

人起来攻击我们——象征现实生活中我们遭遇的困难、逼迫、甚至属灵争战。

被活活吞了——描述仇敌的凶猛,就像吞噬猎物的猛兽一样。

洪水泛滥(4-5节)——表示灾难的猛烈,让人感到无助。

\subsubsection*{现实生活中的应用}

我们都曾经历过“差点被毁灭”的时刻,也许是健康危机、经济困难、人际冲突,甚至是信仰低谷。
但回头看,我们会发现:若不是神的恩典,我们可能早已倒下!
你是否愿意回顾自己的生命,数算神的恩典,承认他是你的帮助?
\paragraph*{问题思考}

你是否经常只看到困难,而忘记神已经做的奇妙工作?
你愿意像大卫一样,带着感恩的心,承认“若不是耶和华帮助我,我早已被吞灭”吗?
\subsection*{二、宣告神的拯救:我们如雀鸟逃脱(6-7节)}
“耶和华是应当称颂的!他没有把我们当野食交给他们吞吃。我们好像雀鸟,从捕鸟人的网罗里逃脱;网罗破裂,我们逃脱了。”(诗篇124:6-7)

\subsubsection*{1. 神的拯救是主动的、完全的}

\hspace{0.6cm}耶和华是应当称颂的!——大卫从感恩进入赞美,因为他知道拯救完全是出于神。

我们如雀鸟逃脱——捕鸟人精心设下陷阱,但神使网罗破裂,让我们自由。
\subsubsection*{2. 我们的现实处境}
有时候,我们就像被捆绑的雀鸟,落入网罗,毫无反抗之力:

罪的网罗:我们常常被试探、软弱和失败困住,但神释放我们。

环境的困境:人生充满压力和挑战,似乎无路可走,但神打开了一条出路。

仇敌的攻击:撒但试图控告、恐吓我们,但神保护我们,使我们得胜。

\subsubsection*{现实生活中的应用}

你是否曾经经历神的拯救,如同从网罗中被释放?
你是否愿意相信,即便现在还在困境中,神依然掌权,他会打破网罗,让你自由?
\paragraph*{问题思考}

你是否愿意不再靠自己挣扎,而是转向神,寻求他的拯救?
你是否愿意带着信心和盼望,相信神必带你脱离困境?
\subsection*{三、信靠神的保护:我们的帮助在乎耶和华(8节)}
“我们的帮助,是在乎造天地之耶和华的名。”(诗篇124:8)

\subsubsection*{1. 认定神是唯一的帮助}
这个世界有许多“帮助”:金钱、人脉、聪明才智,但只有神的帮助才是可靠的。
神是造天地的主,意味着他掌管一切,他的能力无限,他的保护是彻底的。
\subsubsection*{2. 持续信靠神}
我们的帮助“在乎”神的名——表示我们需要持续仰望、依靠神,而不是一时的信心。
即便环境没有立刻改变,我们仍然要相信:神的名是我们的保障,他不会撇下我们。
\subsubsection*{现实生活中的应用}

当你感到无助时,你是向谁求帮助?
你是否真正相信神比任何环境都大,他掌管一切?
\paragraph*{问题思考}

你是否愿意放下自己的焦虑,把你的帮助交托给神?
你是否愿意在任何环境中,宣告:“我的帮助在乎耶和华” ?
\subsection*{结论与挑战}
诗篇124篇提醒我们:

回顾过去的恩典,承认“若不是耶和华帮助我们”,我们早已灭亡。

宣告神的拯救,像雀鸟逃脱网罗一样,我们是被神释放的。

坚定信靠神,唯独他是我们的帮助,他的能力超过一切困境。

亲爱的弟兄姊妹,你是否愿意像大卫一样,不论环境如何,仍然相信神的保护、感恩神的作为,并持续信靠神的帮助?

愿我们都成为常常数算恩典、见证神拯救、坚定信靠神的人!

\subsection*{结束祷告}
\textbf{慈爱的天父,}

感谢祢的话语提醒我们:若不是祢的帮助,我们早已被吞灭。主啊,我们承认,我们的力量有限,我们的智慧不足,我们的环境充满挑战,但我们知道,祢是拯救我们的神。

感谢祢在过去的日子里保护我们,使我们逃脱仇敌的网罗。求祢加添我们的信心,让我们不依靠自己,而是单单仰望祢。主啊,我们相信,祢的名是我们永远的帮助,祢必带领我们得胜!

奉主耶稣基督的名祷告,阿们!
%-----------------------------------------------------------------------------
\newpage
\section{诗篇第125篇:像锡安山,永不动摇——信心之路}
% 讲章:
\subsection*{引言}
亲爱的弟兄姊妹,今天我们一起来查考诗篇第125篇。这是大卫所写的上行之诗,描述了那些信靠耶和华的人必如锡安山,永不动摇。
我们生活在一个充满不确定性的世界,经济起伏、健康问题、人际关系的挑战、甚至信仰的逼迫,都可能让我们感到摇摆不定。但神在这首诗篇中应许:信靠他的人必如锡安山,稳固不移,得享神的保护和公义的带领!

\subsection*{一、坚定的信心——如锡安山,永不动摇(1节)}
“倚靠耶和华的人,好像锡安山,永不动摇,存到永远。”(诗篇125:1)

\subsubsection*{1. 何为“倚靠耶和华”?}
“倚靠”在原文中有“依赖、信任、交托”的意思。它意味着:

不是只在顺境中信靠神,而是在风暴中依然坚信神掌权。

不是靠自己的聪明、经验,而是将生命的主权交给神。
\subsubsection*{2. “好像锡安山”——坚固稳妥的信仰}
锡安山象征神的同在和稳固不移的信仰。它屹立千年,无论风吹雨打,仍然屹立不倒。
真正信靠神的人,不会被环境左右,而是站立得稳。即使世界变化无常,神的信实却从不改变。
\paragraph*{现实生活中的应用}
你是否在顺境时容易感恩,但在困境中却开始怀疑神?
你是否在面对挑战时,仍能坚定信靠神,而不是依靠自己的方法?
\paragraph*{问题思考}

你的信心是否像锡安山一样稳固,还是容易被环境动摇?
你是否愿意真正把人生的重担交给神,让他成为你的磐石?
\subsection*{二、神的保护——他环绕我们,直到永远(2节)}
“众山怎样围绕耶路撒冷,耶和华也照样围绕他的百姓,从今时直到永远。”(诗篇125:2)

\subsubsection*{1. 神的保护如山环绕}
耶路撒冷四面环山,使城池易守难攻,同样地,神用他的大能围绕保护属他的儿女。
这不是暂时的保护,而是\textbf{“直到永远”}的应许。
\subsubsection*{2. 属灵意义——神是我们的盾牌}
世界可能充满危机,但我们知道,神始终在掌权。
神的保护不仅仅是物质上的,更是属灵的。他使我们在试探、苦难中站立得稳。
\paragraph*{现实生活中的应用}
你是否曾经因环境的压力而感到害怕或不安?
你是否愿意相信,不论外在世界如何变化,神的同在始终不离不弃?
\paragraph*{问题思考}

你是否活在惧怕和焦虑中,还是活在神的应许和保护中?
你是否愿意学习每日默想神的话语,以他的应许来坚固自己的信心?
\subsection*{三、神的公义——赐福义人,除灭恶人(3-5节)}
“恶人的杖不常落在义人的份上,免得义人伸手作恶。”(诗篇125:3)

\subsubsection*{1. 神不允许恶人长期掌权}
“恶人的杖”代表恶人对义人的压迫,可能是社会的不公、信仰的逼迫,甚至是属灵争战。
但神应许:邪恶不会永远掌权,神最终要施行公义!
\subsubsection*{2. 神必定审判恶人,赐福义人}
“耶和华必使善人和心理正直的人享受平安。”(诗篇125:4-5)

义人最终会得到神的祝福与平安。
恶人最终会被神审判,不得在神的百姓中站立得住。
\paragraph*{现实生活中的应用}
你是否曾因世上的不公义感到愤怒或灰心?
你是否愿意相信,即使暂时看不到,神终究会施行公义?
\paragraph*{问题思考}

你是否愿意在面对不公时,仍然持守正直,而不是被环境同化?
你是否愿意把公义的伸张交托给神,而不是自己去报复?
\subsection*{结论与挑战}
诗篇125篇提醒我们:

真正信靠神的人,如锡安山,永不动摇。

神用他的大能围绕保护他的子民,从今时直到永远。

神是公义的,他必定审判恶人,赐福义人。

亲爱的弟兄姊妹,你是否愿意把你的信心建立在神的话语上,而不是世俗的安全感?你是否愿意在面对挑战时,仍然相信神的保护与公义?愿我们都成为如锡安山一样的信徒,坚定信靠神,得享他的保护与祝福!

\subsection*{结束祷告}

\textbf{亲爱的天父,}

感谢祢透过诗篇125篇提醒我们,信靠祢的人必如锡安山,永不动摇。主啊,我们承认,许多时候我们被环境动摇,被恐惧捆绑,被世界的变化影响,但今天,我们愿意单单信靠祢。

主啊,我们感谢祢的保护,如同众山围绕耶路撒冷一样,祢也围绕着我们,直到永远。求祢帮助我们,在困境中不害怕,在试探中不跌倒,在逼迫中仍然坚定跟随祢。

主啊,求祢彰显公义,赐福行善和心理正直的人,使我们能够持守正直,不因恶人的压迫而动摇。愿我们的一生都建立在祢的话语上,如锡安山一样坚固!

奉主耶稣基督的名祷告,阿们!
%-----------------------------------------------------------------------------
\newpage
\section{诗篇第126篇:流泪撒种,欢呼收割——盼望之歌}
% 讲章:
\subsection*{引言}
亲爱的弟兄姊妹,今天我们要一同查考诗篇126篇。这是一首充满盼望和喜乐的诗歌,描述了神如何带领以色列百姓从被掳之地归回,也展现了在困苦中的信心与期待。
诗篇126篇提醒我们:神是翻转困境的神,他能够使我们的眼泪变成喜乐,使我们的撒种带来丰收。 
\subsection*{一、经历神的拯救,满心喜乐(1-3节)}
“当耶和华将那些被掳的带回锡安的时候,我们好像做梦的人。”(诗篇126:1)

\subsubsection*{1. 过去的捆绑与神的拯救}
这节经文描述了以色列人被掳到巴比伦七十年后,神终于让他们归回圣地。这种经历就像是“做梦一样”,他们原本以为自己再也回不去了,但神却奇妙地带领他们回到锡安。
同样地,在我们的人生中,也有许多时候我们感到“被掳”:

也许是罪的捆绑,让我们陷在无助和黑暗中;

也许是生活的艰难,让我们感到希望渺茫;

也许是人际关系的破裂,让我们觉得孤立无援。

但感谢神,他是拯救我们的神! 只要我们依靠他,他能带领我们脱离困境,使我们重新得着喜乐。

\subsubsection*{2. 神的作为带来喜乐与见证}
“那时,我们满口喜笑,满舌欢呼;外邦中就有人说:‘耶和华为他们行了大事!’”(诗篇126:2)

当神拯救我们时,他不仅让我们喜乐,也让周围的人看见他的作为。
你的生命是否成为别人的见证? 当神赐福你时,你是否愿意将他的荣耀分享出去?
\paragraph*{现实生活中的应用}

你是否还在过去的失败和痛苦中挣扎?
你是否愿意相信,神能够带你进入一个新的季节?
\subsection*{二、祷告呼求神,求他翻转(4节)}
“耶和华啊,求你使我们被掳的人归回,好像南地的河水复流。”(诗篇126:4)

\subsubsection*{1. 生命的荒漠需要神的复兴}
这里提到的“南地”指的是犹大南方的旷野,平时干旱无水,但当雨季来临时,河水就会复流,使整个荒漠焕然一新。
这象征着神的复兴和恩典,能够让我们枯干的生命再次充满活力。
\subsubsection*{2. 祷告带来突破}
诗人并没有满足于过去的拯救,而是继续祷告求神完全复兴他的百姓。
你是否有持续为你的生命、家庭、教会、甚至国家的复兴祷告?
祷告不是单单求神赐福,而是求他在我们的生命中运行,带来真实的改变。
\paragraph*{现实生活中的应用}

你是否在干旱无望的生命状态中?
你是否愿意向神呼求,让他的活水充满你?
\subsection*{三、流泪撒种,必欢呼收割(5-6节)}
“流泪撒种的,必欢呼收割。那带种流泪出去的,必要欢欢乐乐地带禾捆回来。”(诗篇126:5-6)

\subsubsection*{1. 撒种的辛劳与期待}
这里的撒种指的是辛勤的努力和信心的投资。农夫在干旱的土地上撒种时,可能会流泪,因为他不知道这些种子是否能发芽。
但神应许,凡是带着信心撒种的人,最终必定欢呼收割!
\subsubsection*{2. 属灵的撒种}
我们在生活中也会经历“流泪撒种”的时刻:

为家人信主祷告多年,却迟迟不见果效;

为事业和未来努力,却一次次遇到挫折;

在事奉中摆上,却经历不被理解甚至被拒绝。

但神的应许是:只要你坚持撒种,总有一天,你会带着喜乐收割!

\paragraph*{现实生活中的应用}

你是否在撒种的过程中灰心?
你是否愿意相信神的信实,不论环境如何,仍然忠心耕耘?
\subsection*{结论与挑战}
诗篇126篇告诉我们:

神是翻转生命的神,他能让我们的痛苦变为喜乐。

当我们处在干旱困境中,要持续祷告,求神带来复兴。

流泪撒种的人,最终必定欢呼收割。

亲爱的弟兄姊妹,你的生命是否仍在流泪撒种的阶段? 你是否愿意相信神的应许,即使暂时看不到果效,仍然继续撒种、祷告、信靠他? 愿我们都能经历神的翻转与丰收,成为一个喜乐收割的人!

\subsection*{结束祷告}
\textbf{亲爱的天父,}

感谢祢的话语提醒我们,祢是翻转生命的神。无论我们的过去如何,祢都能使我们的眼泪变为喜乐。

主啊,我们向祢祷告,求祢复兴我们的生命,如同南地的河水复流,让干旱的心灵重新充满祢的活水。

主啊,我们承认,我们仍在撒种的过程,很多时候我们灰心、疲惫、甚至想要放弃。但今天,我们愿意抓住祢的应许,继续流泪撒种,相信总有一天,我们会欢呼收割。

求祢坚固我们的信心,让我们不被环境动摇,而是坚定地相信祢的信实。奉主耶稣基督的名祷告,阿们!
%-----------------------------------------------------------------------------
\newpage
\section{诗篇第127篇:凡事倚靠主,劳碌才有价值}
% 讲章:——诗篇127篇的智慧
\subsection*{引言}
亲爱的弟兄姊妹,今天我们要一起学习诗篇127篇。这首诗是所罗门所写,告诉我们真正的成就、家庭的兴旺和人生的意义都必须建立在神的根基之上。
现代社会强调努力、拼搏、奋斗,但诗篇127篇提醒我们:如果没有神的参与,我们的劳碌就毫无意义。
\subsection*{一、若不倚靠神,一切努力都是徒然(1-2节)}
“若不是耶和华建造房屋,建造的人就枉然劳力;若不是耶和华看守城池,看守的人就枉然警醒。”(诗篇127:1)

\subsubsection*{1. 劳碌但无果效的生活}
在今天的世界,人们拼命工作、加班、投资、经营事业,以为越努力就能越成功。但圣经告诉我们,如果神不在其中,我们所有的努力都会变得毫无意义。

事业的成功不是单靠个人的努力,而是神的赐福。

家庭的稳固不是单靠管理,而是神的保守。

城市的安全不是单靠警戒,而是神的看顾。

如果我们只靠自己,就像在沙滩上建房子,一场风暴就能摧毁一切。唯有让神成为我们的建造者,我们的劳碌才有意义。

\subsubsection*{2. 过度焦虑 vs. 在神里面的安息}
“你们清晨早起,夜晚安歇,吃劳碌得来的饭,本是枉然;唯有耶和华所亲爱的,必叫他安然睡觉。”(诗篇127:2)

很多人忙到无法休息,晚上睡不着,担心事业、家庭、未来。但神要我们学习信靠他,知道最终的结果在神的手里。

\paragraph*{现实生活中的应用}

你是否常常为了工作、事业焦虑,却忽略了神的引导?
你是否愿意将你的计划交给神,相信他会带领你?
\subsection*{二、神是家庭的建造者,儿女是他的产业(3-5节)}
“儿女是耶和华所赐的产业,所怀的胎是他所给的赏赐。”(诗篇127:3)

\subsubsection*{1. 儿女是神的礼物,而不是负担}
现代社会,很多人把养育孩子看作是一种压力,而不是神的恩典。

一些人觉得孩子会影响自己的事业发展;
另一些人觉得孩子需要太多投资和付出。
但圣经告诉我们:儿女是神赐的产业和赏赐!

“产业” 表示神把孩子交托给我们管理,而非完全属于我们。
“赏赐” 表示孩子是神的礼物,我们应该珍惜和感恩。
\subsubsection*{2. 儿女要在主里成长,成为得胜者}
“少年时所生的儿女,好像勇士手中的箭。”(诗篇127:4)

箭是用来射向目标的,父母的责任是塑造孩子,使他们走向神的道路。
这意味着我们要教导孩子认识神,而不是仅仅关注他们的学业和成就。
如果我们的孩子敬畏神,他们就能成为神国度的勇士,在世界中为神作光作盐。
\paragraph*{现实生活中的应用}

你是否用属世的价值观来教育孩子,还是用神的话语来塑造他们?
你是否愿意相信,神能带领你的家庭,使你的孩子成为他所喜悦的人?
\subsection*{三、安息在神的恩典中,不做徒然的工}
诗篇127篇的核心信息是:人生的努力要建立在神的根基上,否则一切都是徒然的。

房屋的建造,需要神的同在。

城市的安全,需要神的保护。

工作的劳苦,需要神的赐福。

家庭的成长,需要神的看顾。

\subsubsection*{1. 你是否信靠神,而不是只靠自己?}
如果你正忙于事业,是否愿意把你的公司、事业交给神,让他成为真正的老板?
如果你正为家庭操劳,是否愿意把你的孩子交托给神,让他成为孩子的导师?
\subsubsection*{2. 你是否愿意在神里面得享安息?}
神爱他的儿女,不希望我们活在焦虑和压力中。
当我们把一切交托给神,我们才能真正“安然睡觉”,因为我们知道,神掌管一切!
\subsection*{结论与挑战}
诗篇127篇提醒我们:

人若不依靠神,一切努力都是枉然的。

家庭的稳固、儿女的成长,都必须交托在神的手中。

唯有在神里面,我们才能真正得享安息,不至于徒劳无功。

亲爱的弟兄姊妹,你是否愿意调整你的生命次序,让神成为你的中心? 你是否愿意学习信靠神,而不是过度焦虑、依靠自己?愿神帮助我们,让我们的一生不再是“徒然劳力”,而是充满神的祝福和果效!

\subsection*{结束祷告}
\textbf{亲爱的天父,}

感谢祢通过诗篇127篇提醒我们,唯有在祢里面,我们的劳碌才有意义。

主啊,帮助我们,不要只依靠自己的努力,而是让祢成为我们生命的建造者。

我们把我们的家庭、事业、儿女、未来交托给祢,求祢带领,使我们的生命充满祢的荣耀,而不是徒然劳碌。

主啊,求祢教导我们在祢里面得享安息,让我们不再忧虑,而是信靠祢的供应和看顾。

奉主耶稣基督的名祷告,阿们!
%-----------------------------------------------------------------------------
\newpage
\section{诗篇第128篇:敬畏神的人有福了}
% 讲章:——诗篇128篇的智慧
\subsection*{引言}
亲爱的弟兄姊妹,今天我们要一起来学习诗篇128篇,这是一首充满祝福的诗篇,告诉我们敬畏神的人必得福乐。
在这个世界上,每个人都在追求幸福:

有人认为幸福是事业有成、财富丰厚;

有人认为幸福是家庭美满、儿孙满堂;

也有人觉得幸福是身体健康、无忧无虑。

然而,真正的幸福不是来自外在的物质,而是来自神的祝福。 诗篇128篇告诉我们,敬畏神的人才是真正蒙福的人,他们的工作、家庭、子孙、人生都会经历神的恩典和祝福。

\subsection*{一、敬畏神的人在工作上蒙福(1-2节)}
“凡敬畏耶和华、遵行他道的人便为有福!你要吃劳碌得来的;你要享福,事情顺利。”(诗篇128:1-2)

\subsubsection*{1. 敬畏神是蒙福的根基}
敬畏神是什么意思?

敬畏神不是害怕神,而是尊崇神、顺服神,把他放在生命的首位。
这意味着我们在工作上、道德上、决策上都愿意遵行神的道,不凭自己做决定,而是寻求神的旨意。
\subsubsection*{2. 劳碌得来的祝福}
这世界上有两种工作方式:

一种是靠自己拼搏,但充满焦虑和不安;

另一种是敬畏神、依靠神,在劳碌中仍然享受神的祝福。

\textbf{“你要吃劳碌得来的”——这意味着神不是让我们懒惰,而是祝福我们所付出的努力。}

如果你在职场中敬畏神,神会赐你智慧,使你的工作顺利。

如果你在事业中遵行神的原则,神会使你手中的工有果效。
\paragraph*{现实生活中的应用}

你是否在工作中敬畏神,还是只依靠自己的聪明?
你是否在金钱、道德、职业选择上遵行神的道路?
你是否愿意相信,神会祝福你的劳碌,使你享福?
\subsection*{二、敬畏神的人在家庭中蒙福(3-4节)}
“你妻子在你内室,好像多结果子的葡萄树;你儿女围绕你的桌子,好像橄榄栽子。”(诗篇128:3)

\subsubsection*{1. 夫妻关系的祝福}
诗人用“多结果子的葡萄树”来形容敬畏神之人的妻子,象征家庭的喜乐、满足和兴旺。
这意味着:

敬畏神的丈夫会尊重和爱护妻子,使家庭充满神的恩典;

敬畏神的妻子会成为家中的祝福,使家人得满足。

在现实生活中,许多家庭的问题是因为夫妻双方没有敬畏神,导致婚姻充满争吵、冷漠和不信任。但如果夫妻双方都敬畏神,神会使他们的婚姻稳固,并充满祝福。

\subsubsection*{2. 儿女的祝福}
“你儿女围绕你的桌子,好像橄榄栽子。”(诗篇128:3)

橄榄树是一种极其宝贵的树木,象征生命力和祝福。
这意味着敬畏神的家庭,儿女也会被神所祝福,成为有影响力的下一代。
\paragraph*{现实生活中的应用}

你是否在家庭中敬畏神,还是只凭自己的方式经营婚姻?
你是否用神的话语来教导儿女,而不是用世界的标准?
你是否愿意相信,神能祝福你的家庭,使你的家成为爱的避风港?
\subsection*{三、敬畏神的人在一生中蒙福(5-6节)}
“愿耶和华从锡安赐福给你!愿你一生一世看见耶路撒冷的好处!愿你看见你儿女的儿女!愿平安归于以色列!”(诗篇128:5-6)

\subsubsection*{1. 一生一世的祝福}
诗篇128篇的祝福不仅限于工作和家庭,还延续到一生的每一天。
敬畏神的人不只是短暂蒙福,而是“一生一世”都得神的恩典。
他们能看到神的国度兴旺,也能看到下一代持续在神的恩典中成长。
\subsubsection*{2. 代代相传的祝福}
“愿你看见你儿女的儿女。”

这意味着敬畏神的家庭,会把信仰传承下去,影响子孙后代。
现代社会,很多家庭的信仰只停留在一代,但神的祝福是要延续到千代的(出埃及记20:6)。
\paragraph*{现实生活中的应用}

你是否愿意活出敬畏神的榜样,使你的子孙蒙福?
你是否愿意为神的国度摆上,让神的旨意成就在你的家族中?
\subsection*{结论与挑战}
\subsubsection*{诗篇128篇告诉我们:}

\hspace{0.6cm}敬畏神的人在工作上蒙福,享受神所赐的果效。

敬畏神的人在家庭中蒙福,婚姻稳固,儿女成为祝福。

敬畏神的人在一生中蒙福,经历神的看顾,并影响子孙后代。

\subsubsection*{亲爱的弟兄姊妹,你是否渴望这样的祝福?}

\hspace{0.6cm}敬畏神不是一种形式,而是一种生活态度。

敬畏神的人才能真正享受神的祝福,过上丰盛的人生。

你愿意从今天开始,活出敬畏神的生活吗?

\subsection*{结束祷告}
\textbf{亲爱的天父,}

感谢祢赐下诗篇128篇,让我们明白敬畏祢的人才是有福的。

主啊,我们愿意把我们的工作交托给祢,求祢赐福我们手中的劳碌,使我们的事业荣耀祢的名。

求祢祝福我们的家庭,使我们的婚姻充满爱,儿女成为祢的产业,代代敬畏祢。

主啊,求祢一生一世与我们同行,使我们成为祢祝福的器皿,让我们的生命充满祢的恩典和荣耀。

奉主耶稣基督的名祷告,阿们!
%-----------------------------------------------------------------------------
\newpage
\section{诗篇第129篇:在患难中坚定信靠神}
% 讲章:——诗篇129篇的启示
\subsection*{引言:人生的苦难与信仰的坚持}
亲爱的弟兄姊妹,你是否经历过生命中的苦难?

是否遭遇过他人的伤害、逼迫或误解?

是否在生活、学业、工作或家庭中经历过挑战,感觉被世界压迫?

是否觉得自己在面对困难时信仰受到了考验?

诗篇129篇是一首上行之诗,是以色列人在上耶路撒冷敬拜时所唱的诗歌。它回顾了以色列民族长期受敌人逼迫的历史,同时宣告神的信实和拯救。今天,我们同样可以从这首诗篇中得到安慰和力量,在人生的苦难中坚定信靠神。

\subsection*{一、苦难是人生的一部分,但神必定拯救(1-4节)}
“从我幼年以来,敌人屡次苦害我——以色列要这样说——从我幼年以来,敌人屡次苦害我,却没有胜了我。
如同扶犁的在我背上扶犁而耕,耕沟甚长。耶和华是公义的;他砍断了恶人的绳索。”(诗篇129:1-4)

\subsubsection*{1. 苦难从“幼年”开始——信仰不会让我们免受苦难}
诗人提到“从我幼年以来”,意味着以色列的历史一直充满逼迫和苦难。
我们的信仰生活也常常如此,许多基督徒从信主之初就经历挑战:

可能来自家庭的不理解,甚至逼迫;

可能来自社会的压力,面对道德与信仰的冲突;

可能是生活中的种种挫折,觉得自己被神遗忘。

但诗人坚定地说:“却没有胜了我!”

苦难不会摧毁真正信靠神的人,因为神一直掌权!
仇敌的攻击不会彻底毁灭我们,神有最终的胜利!
\subsubsection*{2. 受苦如同“扶犁而耕”——伤害虽深,神仍掌权}
诗人用“如同扶犁的在我背上扶犁而耕”来形容敌人的残酷逼迫。
这画面非常痛苦,意味着伤害深入肌肤,甚至灵魂。
然而,这并不是结局!神是公义的,他砍断恶人的绳索,最终必定拯救他的子民。
\paragraph*{现实生活中的应用}

你是否在生活中感受到压力、逼迫或不公?
你是否觉得自己像被犁耕过一样,满是伤痕?
你是否愿意相信,神仍然掌权,他最终必定拯救你?
\subsection*{二、仇敌最终必失败,而敬畏神的人必得胜(5-8节)}
“愿恨恶锡安的都蒙羞退后!愿他们像房顶上的草,未长成而枯干!”(诗篇129:5-6)

\subsubsection*{1. 仇敌的结局是“蒙羞退后”}
诗人用祷告的方式宣告,所有反对神、逼迫神子民的人,最终都要蒙羞、失败。
他们可能暂时得势,但神不会让恶人永远昌盛。
\subsubsection*{2. 恶人如“房顶上的草”}
在中东,房顶上常常积尘,但不适合植物生长,偶尔长出的草很快就会枯干。
诗人用这个比喻来说明恶人不会长久兴旺,他们的生命短暂、无根基,最终会被神除灭。
这提醒我们:不要害怕恶人的嚣张,他们不过是短暂的草,不会永远得胜!
\paragraph*{现实生活中的应用}

你是否曾因恶人暂时得势而感到困惑?
你是否愿意相信,神的公义最终必然彰显,恶人不会长久昌盛?
\subsection*{三、在患难中,我们要坚持信靠神的公义和恩典}
诗篇129篇虽然讲述了受苦和仇敌的逼迫,但它的核心信息是:
\begin{itemize}
    \item 苦难不能战胜神的子民

    \item 恶人不会长久昌盛

    \item 神的公义最终会得胜

\end{itemize}

这提醒我们,在任何环境中,都要坚定信靠神!

\paragraph*{现实生活中的挑战与鼓励}

当我们遭遇挫折和不公时,我们不要灰心,而要坚定信靠神。
当我们面对恶人的逼迫时,不要惧怕,而要相信神的公义。
当我们看见世界的邪恶时,不要丧志,而要记住神的国度终必得胜!
\subsection*{结论:信心的胜利}
诗篇129篇给我们三点宝贵的属灵功课:

苦难是人生的一部分,但神必定拯救他的子民。

仇敌最终必失败,恶人的道路如同屋顶上的草,必定枯干。

在患难中,我们要坚持信靠神的公义和恩典。

\paragraph*{亲爱的弟兄姊妹,你是否正在经历苦难?}

不要失去盼望,因为神没有离弃你!
不要害怕恶人,因为他们的结局是失败!
坚持信靠神,神必使你最终得胜!
\subsection*{结束祷告}
\textbf{亲爱的天父,}

感谢祢的话语,让我们看到在苦难中的盼望。

主啊,我们承认人生充满挑战,我们也会遇到逼迫和困境,但我们相信祢是公义的,祢不会让仇敌得胜,祢最终必拯救我们。

求祢赐给我们信心,在苦难中仍然信靠祢;赐给我们忍耐,在不公中仍然仰望祢的公义;赐给我们勇气,让我们在世界的挑战面前站立得稳。

主啊,我们也为那些伤害我们的人祷告,愿祢的公义彰显,愿他们悔改归向祢,否则愿祢按祢的公义审判他们。

主啊,求祢帮助我们,使我们在苦难中仍然坚定信仰,因我们知道,最终的胜利属于祢!

奉主耶稣基督的名祷告,阿们!
%-----------------------------------------------------------------------------
\newpage
\section{诗篇第130篇:从深处呼求,得着神的恩典}
% 讲章:——诗篇130篇
\subsection*{引言:当我们陷入困境时,我们该怎么办?}
弟兄姊妹,你是否曾有过这样的经历:

在人生低谷中,感觉自己被罪恶、痛苦、失败或焦虑包围?

觉得神好像离你很远,祷告似乎没有回应?

心里充满自责,觉得自己不配来到神面前?

诗篇130篇是一首“求赦免的诗”,也是“上行之诗”之一,代表着敬拜者从深处向神发出的呼求。 这篇诗篇教导我们,当我们陷入困境时,不是靠自己的努力,而是要转向神,呼求他的怜悯,并且相信他的救赎。

\subsection*{一、在深处,我们应当向神呼求(1-2节)}
“耶和华啊,我从深处向你求告!主啊,求你听我的声音!愿你侧耳听我恳求的声音!”(诗篇130:1-2)

\subsubsection*{1. “深处”象征极大的困境}
“深处”在诗篇中常用来形容极大的痛苦、危机或灵魂的低谷。
这可能是外在的困境(如疾病、经济问题、人际冲突),也可能是内心的挣扎(如罪恶感、忧郁、灵命低沉)。
现实中的“深处”有哪些?

犯罪后,感到羞愧、不敢来到神面前。

生活中遇到困难,觉得神没有听我们的祷告。

经历失败,觉得自己没有希望了。

但诗人没有沉溺于“深处”,而是向神呼求!
他没有试图靠自己爬出来,而是直接向神求助。
这提醒我们,无论我们身处多么黑暗的处境,都要向神呼求,神必听我们的声音!
\subsection*{二、神满有怜悯,愿意赦免我们的罪(3-4节)}
“主耶和华啊,你若究察罪孽,谁能站得住呢?但在你有赦免之恩,要叫人敬畏你。”(诗篇130:3-4)

\subsubsection*{1. 若神按公义审判,我们无人能站立}
我们每个人都犯过罪,若神按公义来审判我们,我们都无法承受。
“因为世人都犯了罪,亏缺了神的荣耀。”(罗马书3:23)
“罪的工价乃是死。”(罗马书6:23)
\subsubsection*{2. 但神的怜悯使我们得赦免}
神不只是公义的审判者,他更是慈爱的父,愿意赦免我们的罪。
诗人说:“在你有赦免之恩”,表明神的怜悯远超过我们的过犯。
赦免的结果是让我们\textbf{“敬畏神”},而不是滥用他的恩典。
\subparagraph*{现实中的应用}

当你犯了罪,是否因自责而远离神?
你是否愿意相信神的赦免是真实的,只要我们诚心悔改?
真正的敬畏神,不是害怕,而是因他的恩典而敬爱他、顺服他。
\subsection*{三、等候神的人,必得盼望和更新(5-6节)}
“我等候耶和华,我的心等候;我也仰望他的话。我的心等候主,胜于守夜的等候天亮,胜于守夜的等候天亮。”(诗篇130:5-6)

\subsubsection*{1. 等候神不是消极的,而是充满盼望的信靠}
诗人用“守夜的等候天亮”来形容对神的盼望。
黑夜虽长,但天亮一定会来;困境虽深,但神的拯救必临到!
\subsubsection*{2. 以神的话语为盼望的根基}
诗人说:“我也仰望他的话”,神的话是我们信心的根基。
神的应许从不落空,他必成就他的计划!
\paragraph*{现实中的应用}

你是否曾在等候神的过程中感到焦虑?
你愿意像守夜的人一样,相信神的救恩必定到来?
你是否在等候中,更多地读神的话,坚固你的信心?
\subsection*{四、神是我们的救赎,他能完全救拔我们(7-8节)}
“以色列啊,你当仰望耶和华!因他有慈爱,有丰盛的救恩。他必救赎以色列脱离一切的罪孽。”(诗篇130:7-8)

\subsubsection*{1. 神的慈爱是我们得救的保障}
“他有慈爱”,表明神不是冷酷的审判者,而是充满恩典的父亲。
“他有丰盛的救恩”,意味着神的救赎足够拯救一切投靠他的人!
\subsubsection*{2. 神的救赎是完全的}
不仅救我们脱离外在的困难,也救我们脱离内心的罪恶。
不仅是暂时的帮助,更是永恒的救恩!
\paragraph*{现实中的应用}

你是否愿意完全信靠神,不只是为了解决问题,更是为得着他的救赎?
你是否愿意把一切重担、罪恶和软弱交托给神,相信他的恩典?
\subsection*{结论:从深处到救赎,神一直在听}
诗篇130篇带给我们三个重要的属灵功课:

在深处,我们要向神呼求,因为他必垂听。

神的怜悯超过我们的罪,他愿意赦免我们,让我们得自由。

等候神的人,必定经历他的救赎和丰盛的恩典。

亲爱的弟兄姊妹,无论你现在身处何境,都要仰望神,他的慈爱不会离开你!

\subsection*{结束祷告}
\textbf{亲爱的天父,}

感谢祢的话语,提醒我们在困境中仍然可以向祢呼求。

主啊,我们承认自己软弱无助,常常被罪和痛苦困住,但感谢祢,因祢满有怜悯,愿意赦免我们的过犯!

求祢赐给我们信心,使我们能够忍耐等候祢的拯救,像守夜的人等候天亮一样,坚信祢的应许永不落空!

主啊,我们仰望祢的慈爱和丰盛的救恩,求祢救赎我们,使我们在祢的恩典中得自由、得平安!

奉主耶稣基督的名祷告,阿们!
%-----------------------------------------------------------------------------
\newpage
\section{诗篇第131篇:安息在神的怀抱}
% 讲章:——诗篇131篇
\subsection*{引言:学会安静的心}
弟兄姊妹,你是否常常感到焦虑、烦躁、压力过大?
有时候,我们想掌控一切,但却发现自己力不从心。
我们担心未来,害怕失败,心里充满不安。
甚至,我们在信仰上也会焦虑,觉得自己灵性不够,祷告不够,服侍不够……
诗篇131篇是一首简短却极具深意的诗篇,只有三节经文,却充满深刻的属灵智慧。 这是大卫的诗,他向我们展现了一颗安静、谦卑、倚靠神的心。

\subsection*{一、谦卑的心 vs. 骄傲的心(1节)}
“耶和华啊,我的心不狂傲,我的眼不高大,重大和测不透的事,我也不敢行。”(诗篇131:1)

\subsubsection*{1. 大卫的谦卑:承认自己的有限}

\hspace{0.4cm}“我的心不狂傲”:骄傲的人以自己为中心,依靠自己的聪明才智,而不是倚靠神。

“我的眼不高大”:傲慢的人喜欢轻看别人,认为自己比别人更重要、更有能力。

“重大和测不透的事,我也不敢行”:这不是逃避责任,而是承认自己有局限,交托神掌管。
\subsubsection*{2. 我们的现实挑战}
这个世界推崇“自我奋斗”、“掌控一切”,让我们误以为\textbf{“一切靠自己”}才是成功之道。
但大卫提醒我们,真正的智慧是认识到自己的有限,愿意谦卑地倚靠神。
\paragraph*{应用问题:}

你是否常常陷入\textbf{“凡事想掌控”},却让自己压力越来越大?
你是否愿意承认自己的有限,让神来引导你的道路?
\subsection*{二、安静的心 vs. 焦虑的心(2节)}
“我的心平稳安静,好像断过奶的孩子在他母亲的怀中;我的心在我里面真像断过奶的孩子。”(诗篇131:2)

\subsubsection*{1. “断过奶的孩子”意味着什么?}
\hspace{0.6cm}断奶之前,婴儿常常焦虑、哭闹,依赖母亲喂奶,无法自己安息。

断奶之后,孩子虽然仍然需要母亲,但学会了安静地信靠母亲的爱,不再因饥饿而焦躁不安。

大卫的心就像这样——从焦虑到安息,从烦躁到平静,从依赖世界到单单信靠神!
\subsubsection*{2. 现代人的焦虑:为什么我们的心不安?}
担心经济问题、健康状况、人际关系、事业发展……
害怕未来,担心自己做得不够好,害怕失败。
甚至在信仰上,我们也容易焦虑:“我是否足够属灵?是否足够圣洁?”
\subsubsection*{3. 如何培养“安静的心”?}
\hspace{0.6cm}学会交托:“你们要休息,要知道我是神。”(诗篇46:10)

学会知足:“敬虔加上知足的心便是大利了。”(提摩太前书6:6)

学会默想神的话:“你的话是我脚前的灯,是我路上的光。”(诗篇119:105)

\paragraph*{应用问题:}

你是否像“未断奶的孩子”,常常因环境的变化而焦躁不安?
你是否愿意像大卫一样,安息在神的怀抱,单单信靠他的爱?
\subsection*{三、盼望的心 vs. 绝望的心(3节)}
“以色列啊,你当仰望耶和华,从今时直到永远!”(诗篇131:3)

\subsubsection*{1. 盼望建立在神的信实之上}
诗人最后不是只谈自己的感受,而是呼召整个以色列都要仰望神!
他的盼望不是建立在环境的改变,而是建立在神的信实上!
\subsubsection*{2. 我们盼望什么?}
世界的盼望是短暂的:金钱、地位、人的认可,都会过去。
真正的盼望在于神:他的应许永不改变,他的慈爱永远长存!
\subsubsection*{3. 如何保持盼望?}
回顾神的恩典:数算主恩,记得他过去如何带领你。
坚定信靠他的应许:“但那等候耶和华的必重新得力。”(以赛亚书40:31)
坚持祷告,不住仰望:“当将你的事交托耶和华,并倚靠他,他就必成全。”(诗篇37:5)
\paragraph*{应用问题:}

你是否常常因环境的变化而失去盼望?
你是否愿意从短暂的盼望转向永恒的盼望——神的信实?
\subsection*{结论:在神里面得着真正的安息}
诗篇131篇短小却深刻,教导我们:

谦卑的心:承认自己的有限,依靠神,而非依靠自己。

安静的心:像断奶的孩子一样,不再焦躁,而是完全信靠神。

盼望的心:无论环境如何,都要坚定仰望神的信实。

弟兄姊妹,愿我们在这焦虑的世界中,学会谦卑、学会安息、学会盼望!

\subsection*{结束祷告}
\textbf{亲爱的天父,}

感谢祢的话语,教导我们如何在祢里面得着真正的安息。

主啊,我们常常被世界的压力和焦虑捆绑,求祢帮助我们像大卫一样,谦卑自己,不依靠自己的聪明,而是全然信靠祢。

主啊,求祢使我们的心安静,不再被环境动摇,而是像断过奶的孩子一样,完全安息在祢的怀抱中。

主啊,求祢赐给我们信心,让我们从今时直到永远都仰望祢,因祢是信实的神,祢的慈爱永不改变!

奉主耶稣基督的名祷告,阿们!
%-----------------------------------------------------------------------------
\newpage
\section{诗篇第132篇:寻求神的同在与应许}
% 《》
% ——诗篇132篇讲章

% 经文:诗篇132篇

\subsection*{引言:神的同在与他的应许}
\hspace{0.6cm}诗篇132篇是一首上行之诗,被认为是以色列人在朝圣的过程中所吟唱的诗歌。它主要回顾了大卫对神的心意,以及神对大卫的应许。大卫竭力为神预备安息之所,而神应许要坚定大卫的后裔,最终这个应许在耶稣基督里得到了完全的实现。

这首诗篇向我们展示了信仰的两个核心:人对神的渴慕与神对人的信实。大卫寻求神的同在,而神也回应了他的渴望,赐下恩典和祝福。同样,在我们的生活中,我们如何寻求神的同在?我们是否相信神的应许永不改变?这首诗篇会帮助我们深入思考这些问题,并指导我们如何实践在生活中。

\subsection*{一、大卫对神的渴望(1-5节)}
“耶和华啊,求你记念大卫所受的一切苦难。他怎样向耶和华起誓,向雅各的大能者许愿,说:‘我必不进我的帐幕,也不上我的床榻;我不容我的眼睛睡觉,也不容我的眼皮打盹,直等我为耶和华寻得所在,为雅各的大能者寻得居所。’”(诗132:1-5)

\subsubsection*{1. 大卫对神的热切追求}
诗人回顾大卫王的心志,他渴望为神预备圣殿,让神的同在充满以色列。
他不以自己的安逸为优先,而是先求神的荣耀,甚至愿意牺牲自己的舒适。
\subsubsection*{2. 现实生活中的应用:我们是否渴望神的同在?}
在现实生活中,我们是否有像大卫一样的渴望,愿意优先寻求神的国和他的义?
我们是否愿意付上代价,在忙碌的生活中腾出时间来亲近神?
现代社会充满各种诱惑,我们可能更容易花时间在工作、娱乐、社交媒体上,而不是寻求神的面。
\paragraph*{实际行动:}

设定每日固定的灵修时间,无论多忙,都要腾出时间与神相交。
在生活的决策上,先求问神的旨意,而不是只靠自己的判断。
参与教会的服事,让自己的生活与神的国度紧密相连。
\paragraph*{反思问题:}

我是否愿意像大卫一样,优先寻求神的同在?
我是否愿意放下一些属世的舒适,以便更多地亲近神?
\subsection*{二、神的回应与应许(6-12节)}
“耶和华向大卫凭诚实起了誓,必不反悔,说:‘我要使你所生的坐在你的宝座上。你的众子若守我的约和我所教训他们的法度,他们的子孙必永远坐在你的宝座上。’”(诗132:11-12)

\subsubsection*{1. 神信实的应许}
神向大卫起誓,要使他的后裔坐在宝座上,这个应许最终在耶稣基督里得到了完全的实现。
神的应许是基于他的信实,而不是人的行为。
\subsubsection*{2. 现实生活中的应用:相信神的应许}
许多时候,我们会怀疑神的应许是否真的会成就,尤其是在经历困难和等待的时候。
但诗篇提醒我们,神的信实不会改变,他的应许终必实现,即使时间和环境似乎不利于我们。
\paragraph*{实际行动:}

读圣经,了解神对我们的应许,并在祷告中宣告他的信实。
在困境中,不要被环境左右,而是要坚定信靠神的话语。
当神的应许还未实现时,保持忍耐,知道他的时间永远是最好的。
\paragraph*{反思问题:}

我是否在等待神的应许时缺乏耐心?
我是否愿意相信,即使环境不利,神仍然掌权?
\subsection*{三、神的同在带来的祝福(13-18节)}
“因为耶和华拣选了锡安,愿意当作自己的居所,说:‘这是我永远安息之所,我要住在这里,因为是我所愿意的。’”(诗132:13-14)

\subsubsection*{1. 神同在的祝福}
神拣选了锡安,愿意住在那里,意味着他的同在要成为百姓的祝福。
诗篇提到神的同在会带来丰盛的供应、公义、喜乐和拯救(15-16节)。
\subsubsection*{2. 现实生活中的应用:如何经历神的同在?}
神愿意住在他的子民中间,今天我们每一个信徒就是神的殿(林前6:19)。
但我们是否真正渴望神的同在,还是仅仅满足于宗教仪式?
\paragraph*{实际行动:}

通过敬拜、祷告、读经,培养与神的亲密关系,使他的同在成为我们生活的一部分。
在生活中追求圣洁,让我们的生命成为神的居所。
让神的同在影响我们的家庭、职场、社交关系,使我们成为祝福他人的管道。
\paragraph*{反思问题:}

我是否渴望经历神的同在,而不仅仅是听关于神的事情?
我的生活方式是否反映出神的同在?
\subsection*{结论:渴慕神、信靠神、经历神}
诗篇132篇提醒我们:

我们要渴慕神的同在,像大卫一样,愿意付代价来亲近他。

我们要信靠神的应许,因为他的信实永不改变。

我们要经历神的同在,并让他的同在成为我们生活的祝福。

愿我们在生活中真实经历神的同在,并将这种祝福带给身边的人。

\subsection*{祷告}
\textbf{亲爱的天父,}

感谢你在诗篇132篇中向我们启示了你的信实与同在。求你赐给我们像大卫一样的渴慕之心,让我们在生活中优先寻求你。无论环境如何,我们都愿意相信你的应许,并在你的同在中得着满足。主啊,帮助我们每天亲近你,让我们的生命成为你喜悦的居所。愿你的荣耀充满我们的家庭、工作、教会和社会,

奉耶稣基督的名祷告,阿们!
%-----------------------------------------------------------------------------
\newpage
\section{诗篇第133篇:合一的福分}
% 《》
% ——诗篇133篇讲章

% 经文:诗篇133篇

“看哪,弟兄和睦同居是何等地善,何等地美!
这好比那贵重的油浇在亚伦的头上,流到胡须,又流到他的衣襟;
又好比黑门的甘露降在锡安山;
因为在那里有耶和华所命定的福,就是永远的生命。”(诗篇133篇)

\subsection*{引言:合一的重要性}
\hspace{0.6cm}诗篇133篇是一首短小却充满力量的诗歌,被称为上行之诗,是以色列人在朝圣时所吟唱的诗篇之一。它强调了合一的美好,并用两个生动的比喻来形容合一所带来的祝福——膏油和甘露。

在今天的世界里,无论是家庭、教会,还是社会,分裂和冲突随处可见。然而,诗篇133篇提醒我们:合一是神所喜悦的,并且合一会带来极大的祝福。让我们从三方面来深入剖析这首诗,并思考如何在现实生活中实践合一。

\subsection*{一、合一的美善(1节)}
“看哪,弟兄和睦同居是何等地善,何等地美!”

\subsubsection*{1. “善”与“美”——神所喜悦的合一}
“善”代表合一的道德价值,合一符合神的旨意,是正确的、正直的。
“美”代表合一带来的喜乐与和谐,使人与人之间的关系更加甜美。
这不仅仅是外在的和平,而是发自内心的和睦,是出于彼此相爱、尊重和理解的关系。
\subsubsection*{2. 现实生活中的应用:我们如何促进合一?}
在家庭中,夫妻之间、父母与子女之间是否彼此相爱、尊重?
在教会里,是否存在纷争、论断,还是彼此扶持、合一服事?
在职场和社交中,我们是制造冲突,还是促进和谐的人?
\paragraph*{实际行动:}

在家庭:主动表达爱与关心,避免争吵,以恩慈、温和的态度相处。
在教会:放下个人恩怨,以基督的爱彼此包容,专注于福音使命。
在人际关系中:学会倾听,不急于批评,尽量站在对方的角度思考问题。
\paragraph*{反思问题:}

我是否愿意为了合一而牺牲自己的骄傲和偏见?
在生活中,我是否主动成为一个和平的使者?
\subsection*{二、合一的比喻(2-3节)}
诗人用两个生动的比喻来形容合一的祝福——膏油和甘露。

\subsubsection*{1. 膏油:合一带来的圣洁与恩膏}
“这好比那贵重的油浇在亚伦的头上,流到胡须,又流到他的衣襟。”(2节)

膏油是用于祭司的膏抹(出埃及记30:30),代表神的拣选、圣洁和祝福。
这种油不是一点点,而是丰盛地流下,象征神丰沛的恩典。
祭司是百姓和神之间的中保,膏油预示着耶稣基督作为我们的大祭司,他带来了真正的和睦(以弗所书2:14-16)。
\paragraph*{现实生活中的应用:}

合一的群体能够承载神的恩膏,神的祝福会倾倒在其中。
如果教会合一,神的能力和圣灵的同在会更强烈地彰显。
在家庭和团队中,合一能够带来圣洁和力量,使人更容易经历神的恩典。
\paragraph*{实际行动:}

祷告求神洁净自己的心,愿意在合一上付出努力。
在服事中,避免嫉妒和比较,而是彼此鼓励和扶持。
\subsubsection*{2. 甘露:合一带来的生命与滋润}
“又好比黑门的甘露降在锡安山。”(3节上)

黑门山是以色列北部最高的山,终年湿润,水气充足,而锡安山位于南方,气候干燥。
诗人用这个比喻来说明:合一带来的祝福,就像滋润干渴之地的甘露,带来生命与复兴。
\paragraph*{现实生活中的应用:}

合一的家庭充满温暖,成员彼此扶持,生活就有甘甜的滋润。
合一的教会就能成为属灵的泉源,让人得着鼓励、更新和成长。
合一的社会会更加和谐,人与人之间的关系更加美好。
\paragraph*{实际行动:}

在家庭、教会、职场中,成为鼓励者,让人感受到温暖。
当看到分裂时,积极努力促进和睦,而不是制造纷争。
反思问题:

我的生活是否带来甘甜和滋润,还是让人感到干枯和苦涩?
我是否愿意主动修复关系,而不是等别人先改变?
\subsection*{三、合一的结果(3节下)}
“因为在那里有耶和华所命定的福,就是永远的生命。”(3节下)

\subsubsection*{1. 神命定的祝福}
合一不仅仅是人际关系的和谐,更是神的命定。
只有在合一的地方,神的福气才能够完全彰显。
“永远的生命”不仅指今生的祝福,也指向在基督里的永恒生命。
\subsubsection*{2. 现实生活中的应用:领受合一的祝福}
合一带来神的喜悦和丰盛(马太福音18:19-20)。
合一带来影响力:当基督徒彼此相爱,世人就能认出我们是主的门徒(约翰福音13:35)。
合一带来神的同在:当我们同心合意,神的能力会更大地运行(使徒行传2:1-4)。
\paragraph*{实际行动:}

每天在祷告中为关系的修复和合一祷告。
在教会中,主动促进合一,避免纷争和分裂。
在职场和社会,成为和平的使者,影响更多的人。
\subsection*{结论:成为合一的见证者}
诗篇133篇提醒我们:

合一是美善的,是神所喜悦的。

合一带来圣洁和滋润,使我们经历神的恩膏和祝福。

合一的群体会领受神所命定的福,就是永远的生命。

愿我们都成为促进合一的基督徒,在家庭、教会和社会中活出神的爱,让世界因我们的见证而看见神的荣耀!

\subsection*{祷告}
\textbf{天父,}

我们感谢你,因为你呼召我们活在合一之中。求你赐给我们柔和谦卑的心,使我们愿意为合一付上代价。求你洁净我们,让我们成为和平的使者,把你的爱和祝福带给身边的人。愿你的教会合一,家庭合一,我们的社会合一,使你的名得着荣耀!

奉耶稣基督的名祷告,阿们!
%-----------------------------------------------------------------------------
\newpage
\section{诗篇第134篇:夜间仍要称颂耶和华}
% 《》
% ——诗篇134篇讲章

% 经文:诗篇134篇

“看哪,耶和华的仆人,夜间站在耶和华殿中侍奉的,你们当称颂耶和华!
你们当向圣所举手,称颂耶和华!
愿造天地的耶和华,从锡安赐福给你们!”(诗篇134篇)

\subsection*{引言:黑夜中的敬拜}
\hspace{0.6cm}诗篇134篇是上行之诗的最后一首,也是一首呼召敬拜的诗歌。在这首短短的诗篇中,诗人鼓励那些夜间在圣殿中服事的人,无论环境如何,都要称颂耶和华。

在现实生活中,我们每个人都会经历“黑夜”——人生的低谷、试炼、孤独、等待的时刻。但诗篇134篇提醒我们,即使在黑夜,我们仍要称颂耶和华,因为他是掌管天地的神,必然赐福给敬拜他的人。

今天,我们将从这三节经文出发,探讨如何在黑夜中仍然敬拜神,并经历神的赐福。

\subsection*{一、黑夜中的侍奉(1节)}
“看哪,耶和华的仆人,夜间站在耶和华殿中侍奉的,你们当称颂耶和华!”

\subsubsection*{1. 旧约背景:圣殿中的夜间侍奉}
诗人特别提到“夜间站在耶和华殿中侍奉的”,指的是圣殿中的利未人,他们轮流值班,负责圣殿的守卫、焚香、整理灯台等工作(历代志上9:33)。
他们的工作是不被看见的,但却是神所悦纳的。
\subsubsection*{2. 现实生活中的应用:你是否愿意在“黑夜”中仍然侍奉神?}

\hspace{0.6cm}黑夜象征困境:当我们处在苦难、迷茫、等待的时期,是否仍愿意忠心侍奉神?

黑夜象征隐藏:许多人在台前服事容易,但在无人看见的地方,仍然忠心服事的人才是真正敬畏神的人。

黑夜象征信心的考验:当我们看不到神的作为时,我们是否仍然相信他?
\paragraph*{实际行动:}

在人生的低谷,不放弃敬拜和服事神。
即使没有人看见,仍然忠心地做好神托付的事情。
在面对挑战时,不让环境左右我们的信仰,而是坚持相信神的信实。
\paragraph*{反思问题:}

我是否只在顺境中敬拜神,而在黑夜中远离他?
当我感到孤单时,是否仍然愿意忠心地侍奉神?
\subsection*{二、举手称颂神(2节)}
“你们当向圣所举手,称颂耶和华!”

\subsubsection*{1. “举手”的象征}
在圣经中,“举手”是一种敬拜和顺服的象征(提摩太前书2:8)。

这代表降服:承认神的权柄,愿意顺服他的旨意。

这代表信靠:即使在黑夜,我们仍然相信神的信实。

这代表祈求:举手向神,象征我们愿意领受他的恩典和帮助。

\subsubsection*{2. 现实生活中的应用:如何在黑夜中称颂神?}
\hspace{0.6cm}在痛苦中,仍然感谢神:“凡事谢恩,因为这是神在基督耶稣里向你们所定的旨意。”(帖撒罗尼迦前书5:18)

在等待中,仍然信靠神:“你当默然倚靠耶和华,耐性等候他。”(诗篇37:7)

在迷茫中,仍然寻求神:“你们寻求我,若专心寻求我,就必寻见。”(耶利米书29:13)

\paragraph*{实际行动:}

养成每日敬拜和感恩的习惯,即使在困难中也不停止。
在低谷时,不埋怨神,而是主动寻找他的应许。
以举手祷告的方式,象征我们愿意降服于神的带领。
\paragraph*{反思问题:}

我是否在困境中仍愿意敬拜神,还是只在顺境中感恩?
我是否愿意全然顺服神的旨意,而不是只求自己的想法成就?
\subsection*{三、神的赐福(3节)}
“愿造天地的耶和华,从锡安赐福给你们!”

\subsubsection*{1. 神是掌管天地的主}
诗人用“造天地的耶和华”来强调神的全能,他既然创造了天地,就必能供应我们的需要。
神的祝福不是短暂的,而是有永恒价值的。
从锡安赐福:锡安象征神的同在,表明真正的祝福不是来自世界,而是来自神自己。
\subsubsection*{2. 现实生活中的应用:如何领受神的祝福?}
\hspace{0.6cm}敬拜神,就能领受神的赐福:“敬畏耶和华,专心寻求他的人便为有福。”(诗篇128:1)

信靠神,就能经历神的供应:“先求他的国和他的义,这些东西都要加给你们了。”(马太福音6:33)

遵行神的旨意,就能活在他的祝福中:“你们若遵守我的命令,就常在我的爱里。”(约翰福音15:10)

\paragraph*{实际行动:}

在困境中,仍然寻求神的国和他的义,相信他必供应。
不依靠世界的方式,而是倚靠神的能力和祝福。
以感恩的心生活,不论环境如何,都相信神在掌权。
\paragraph*{反思问题:}

我是否真正相信,神的祝福远超过世界所能给予的?
我的生活是否建立在对神的信靠,而不是对环境的依赖?
\subsection*{结论:黑夜中的敬拜带来祝福}
诗篇134篇提醒我们:

无论身处何种光景,我们都要敬拜神,即使是在黑夜中。

举手敬拜象征我们的顺服、信靠和祈求,愿意让神掌管我们的生命。

神是造天地的主,他的祝福是永恒的,超越我们眼前的困境。

愿我们都学习在黑夜中仍然敬拜,经历神所命定的祝福!

\subsection*{祷告}
\textbf{亲爱的天父,}

我们感谢你,因为你是造天地的主,掌管我们的生命。求你帮助我们,无论是在顺境还是逆境,都能忠心地敬拜你。在黑夜中,我们仍然举手称颂你,愿意顺服你的旨意。主啊,求你赐福给我们,让我们在等候中经历你的信实,在困境中经历你的恩典,愿你的名在我们的生命中被高举!

奉耶稣基督的名祷告,阿们!
%-----------------------------------------------------------------------------
\newpage
\section{诗篇第135篇:称颂耶和华的名}
% 《》
% ——诗篇135篇讲章

% 经文:诗篇135篇

“你们要赞美耶和华!你们要赞美耶和华的名!耶和华的仆人,站在耶和华殿中,站在我们神殿院中的,你们要赞美他!耶和华本为善,歌颂他的名是美好的。”(诗篇135:1-3)

\subsection*{引言:为何要赞美神?}
\hspace{0.6cm}诗篇135篇是一首呼召以色列人敬拜耶和华的诗篇,它不仅强调了神的伟大,也特别指出他的作为、信实和审判。

在现实生活中,我们时常会被环境左右,当顺利时,我们容易赞美神;但当遭遇困境,我们的赞美就减少了。诗篇135篇提醒我们,无论环境如何,我们都有充分的理由来称颂神,因为他的本性是善的,他的作为是奇妙的,他的公义是永存的。

今天,我们要从这篇诗篇中学习为何要赞美神,以及如何在实际生活中活出赞美的生命。

\subsection*{一、因神的美善赞美他(1-4节)}
“你们要赞美耶和华!你们要赞美耶和华的名!……耶和华本为善,歌颂他的名是美好的。”(诗篇135:1-3)

\subsubsection*{1. 神的美善本性}
诗篇一开始,就以\textbf{“要赞美耶和华”(希伯来文“Hallelujah”)作为呼召。这是一种带着命令的邀请},提醒我们:神的美善是我们赞美的根基。
神的美善表现在:

他创造我们,并赐我们生命(诗篇100:3)。

他供应我们的需要(马太福音6:26)。

他拯救我们,赐我们永生(约翰福音3:16)。

\subsubsection*{2. 现实生活中的应用:如何因神的美善而赞美他?}
即使在困难中,仍然相信神的美善:“凡事谢恩”(帖撒罗尼迦前书5:18)。
数算神的恩典,培养感恩的心:每天列出三件可以感恩的事。
在敬拜和祷告中常常称颂神,即使心情不好,也可以选择歌颂他。
\paragraph*{反思问题:}

我是否在一切环境中都愿意赞美神?
我是否只在顺境时才感受到神的美善?
\subsection*{二、因神的伟大作为赞美他(5-12节)}
“我知道耶和华为大,也知道我们的主超乎万神之上。”(诗篇135:5)

\subsubsection*{1. 神的权能无可比拟}
他掌管天地:“耶和华在天上,在地下,在海中,在一切的深处,都随自己的意旨而行。”(诗篇135:6)
他行过大事:以色列人曾亲身经历神在埃及降十灾、红海分开、旷野供应、战胜迦南诸王。
\subsubsection*{2. 现实生活中的应用:如何因神的作为而赞美他?}
回顾神在自己生命中的带领:回想过去神如何供应、拯救、医治你。
在试炼中仍然信靠神的作为:“我们行事为人是凭着信心,不是凭着眼见。”(哥林多后书5:7)
向人见证神的伟大:当你经历神的作为,不要隐藏,而是见证他。
\paragraph*{反思问题:}

我是否在等候神作为的时候,仍然坚定信靠?
我是否愿意把神的作为告诉别人,见证他的信实?
\subsection*{三、因神的公义审判赞美他(13-18节)}
“耶和华啊,你的名存到永远!耶和华啊,你可记念的名存到万代!”(诗篇135:13)

\subsubsection*{1. 偶像无能,唯独神掌权}
诗人指出,外邦人的偶像是虚空的(诗篇135:15-18):它们“有口不能言,有眼不能看”。
但耶和华是活的神,他的公义必然施行。
\subsubsection*{2. 现实生活中的应用:如何因神的公义而赞美他?}
信靠神的公义,不为恶人得胜而焦虑(诗篇37:1-2)。
不随从世界的偶像(物质主义、权力、名声),单单敬拜神。
以敬畏神的心,活出公义的生活(弥迦书6:8)。
\paragraph*{反思问题:}

我是否信靠神的公义,还是被世界的现象影响?
我是否在日常生活中避免“现代偶像崇拜”(如金钱、名誉、地位)?
\subsection*{四、因神的百姓蒙福赞美他(19-21节)}
“你们当称颂耶和华!”(诗篇135:19)

\subsubsection*{1. 神的百姓蒙福}
诗人呼召以色列家、亚伦家、利未家以及所有敬畏耶和华的人来称颂他。
今天,所有相信基督的人,都是属神的百姓,也当欢喜称颂神。
\subsubsection*{2. 现实生活中的应用:如何因神的恩典蒙福?}
以敬拜为生活方式,不仅仅在主日敬拜,而是每天亲近神。
建立属灵群体,一同敬拜和分享神的作为。
在福音使命上忠心,成为神祝福他人的管道。
\paragraph*{反思问题:}

我是否活出神子民的身份,每天赞美他?
我是否在群体中鼓励别人也来敬拜神?
\subsection*{结论:活出赞美的生命}
诗篇135篇提醒我们,我们要因神的本性、作为、公义和恩典来赞美他,并且这种赞美不该只是偶尔,而应该成为我们生命的一部分。
愿我们都能在生活的每一个处境中选择敬拜神、见证神、信靠神,并活出神的荣耀!

\subsection*{祷告}
\textbf{亲爱的天父,}

感谢你!你是配得称颂的神,你的美善、伟大、公义和恩典都充满我们的生命。求你帮助我们,在顺境和逆境中都能称颂你的名,不凭感觉,而凭信心敬拜你。愿我们的生命成为你荣耀的见证,愿我们的口不断发出感恩和赞美!

奉耶稣基督的名祷告,阿们!
%-----------------------------------------------------------------------------
\newpage
\section{诗篇第136篇:因耶和华的慈爱永远长存}
% 《》
% ——诗篇136篇讲章

% 经文:诗篇136篇

“你们要称谢耶和华,因他本为善;他的慈爱永远长存!”(诗篇136:1)

\subsection*{引言:赞美神的慈爱}
\hspace{0.6cm}诗篇136篇是一篇充满感恩和敬拜的诗篇,整篇诗反复强调“他的慈爱永远长存”这一真理。这句话在整篇诗篇中出现了26次,不断提醒我们:无论环境如何、无论人生经历什么变化,神的慈爱始终不变。

在现实生活中,我们常常因环境的改变而对神的爱产生怀疑。当顺利时,我们说神是慈爱的;但当遭遇苦难,我们就容易怀疑神的美意。今天,我们要透过诗篇136篇,深入思想神的慈爱,并学习如何在日常生活中活出感恩和信靠。

\subsection*{一、因神的本性称谢他(1-3节)}
“你们要称谢耶和华,因他本为善;他的慈爱永远长存。”(诗篇136:1)

\subsubsection*{1. 神的本性是善的}
诗篇一开始就提醒我们,神的善良和慈爱是我们称谢他的根本原因。神的本性不会改变,无论我们遇见何种处境,他依然是那位良善的神。

\subsubsection*{2. 现实生活中的应用}
学会在任何环境下信靠神的美善,即使看不到答案,也要相信神的美意(罗马书8:28)。
操练感恩的生活方式,每天记录神的恩典,即使是微小的祝福也不忽略。
\paragraph*{反思问题:}

我是否相信神的美善,而不是只凭环境来决定自己的信心?
我是否每天都带着感恩的心来回应神的爱?
\subsection*{二、因神的创造称谢他(4-9节)}
“称谢那独行大奇事的,因他的慈爱永远长存。”(诗篇136:4)

\subsubsection*{1. 神的创造彰显他的慈爱}
诗人接着提醒我们,神创造天地万物,是他慈爱的见证。从浩瀚的星空到微小的花草,一切受造之物都反映了神的智慧和爱(诗篇19:1)。

\subsubsection*{2. 现实生活中的应用}
培养敬畏神的心,在自然界中看见神的荣耀(罗马书1:20)。
珍惜神所赐的生命,善待自己的身体、环境和身边的人。
\paragraph*{反思问题:}

我是否留意到神在创造中的作为,并为此感恩?
我是否以感恩的心善待神所赐的一切,包括我的身体、家庭和自然环境?
\subsection*{三、因神的拯救称谢他(10-24节)}
“称谢那击杀埃及人之长子的,因他的慈爱永远长存。”(诗篇136:10)

\subsubsection*{1. 神在历史中的拯救}
诗篇136篇回顾了以色列人出埃及的历史,强调神如何带领他们脱离为奴之地,分开红海,引导他们进入应许之地。
这一段经文提醒我们:神不仅是创造主,更是拯救主。他不仅在历史上拯救了以色列人,今天他也在我们生命中施行拯救。

\subsubsection*{2. 现实生活中的应用}
相信神仍然在掌权,即使我们正在困境中(以赛亚书43:2)。
回顾自己生命中的属灵历程,记住神如何带领自己度过难关。
\paragraph*{反思问题:}

我是否在困难中仍然相信神的拯救?
我是否常常回顾并感恩神在我生命中的作为?
\subsection*{四、因神的供应称谢他(25-26节)}
“称谢那赐粮食给凡有血气之物的,因他的慈爱永远长存。”(诗篇136:25)

\subsubsection*{1. 神供应我们的一切}
诗篇最后提醒我们,神不仅创造天地、施行拯救,他也每天供应我们所需的粮食和生活必需品。耶稣在主祷文中教导我们:“我们日用的饮食,今日赐给我们”(马太福音6:11),这正是神慈爱的体现。

\subsubsection*{2. 现实生活中的应用}
不为明天忧虑,而是信靠神的供应(马太福音6:31-33)。
学习分享,将神的祝福传递给有需要的人(箴言19:17)。
\paragraph*{反思问题:}

我是否因神的供应而知足,而不是总是抱怨?
我是否愿意成为神祝福他人的管道?
\subsection*{结论:如何回应神的慈爱?}
诗篇136篇向我们展示了神慈爱的四个方面:

因神的本性称谢他——神是善的。

因神的创造称谢他——神掌管天地。

因神的拯救称谢他——神带领我们脱离困境。

因神的供应称谢他——神每天赐我们所需。

这篇诗篇反复强调“他的慈爱永远长存”,提醒我们:无论环境如何,神的爱永不改变。今天,我们是否愿意用一生来回应他的爱,并活出感恩的生命?

\subsection*{祷告}
\textbf{慈爱的天父,}

感谢你!你是创造天地的神,你的慈爱永远长存。你不仅供应我们的需要,更在困境中拯救我们。求你帮助我们在任何环境中都学会感恩,不凭感觉,而凭信心赞美你。愿我们的生命成为你荣耀的见证!

奉主耶稣基督的名祷告,阿们!
%-----------------------------------------------------------------------------
\newpage
\section{诗篇第137篇:在异乡的哀歌:信仰的坚守与盼望}
% 《》
% ——诗篇137篇讲章

% 经文:诗篇137篇

“我们曾在巴比伦的河边坐下,一追想锡安就哭了。”(诗篇137:1)

\subsection*{引言:流亡中的信仰挑战}
诗篇137篇是一首充满哀伤与信仰挣扎的诗篇,它记录了以色列人在被掳到巴比伦时的痛苦经历。他们远离家乡,被逼迫、被嘲讽,甚至被要求唱耶和华的歌来娱乐敌人(诗篇137:3-4)。然而,在这悲伤中,我们可以看到:

信仰的坚守——即使身处困境,他们仍思念耶路撒冷,不愿忘记神的圣城。

公义的呼求——诗人向神倾诉冤屈,祈求神施行公义。

盼望的坚持——尽管现实残酷,他们仍然相信神会复兴以色列。

今天,我们虽未经历被掳之痛,但在这世上,我们也如寄居者,面对各种试炼、信仰挑战,甚至逼迫。这篇诗篇给了我们重要的启示,让我们学习如何在困境中坚守信仰,并仰望神的公义与救赎。

\subsubsection*{一、流亡中的眼泪:面对苦难的信仰态度(1-4节)}
“我们曾在巴比伦的河边坐下,一追想锡安就哭了。”(诗篇137:1)

\subsubsection*{1. 苦难使人哀哭,但不要绝望}
以色列人被掳到巴比伦,远离故土,面对残酷的现实,他们的第一反应是\textbf{“哭泣”}。他们的哭泣不是绝望,而是对神的思念,是因失去圣殿、失去敬拜的中心而感到痛苦。
今天,基督徒也会遇到艰难:

当信仰受到挑战时,如面对不信的家人、职场的歧视、社会的压力,我们是否仍然渴慕神?

当生活充满困境时,如经济问题、疾病、家庭破碎,我们是否仍然相信神掌权?

\subsubsection*{2. 现实生活中的应用}
在困难中,不要埋怨,而是向神倾诉(诗篇62:8)。
不要让环境影响我们对神的信心,反而更要追求他(哈巴谷书3:17-18)。
\paragraph*{反思问题:}

我是否在困难中仍然寻求神,而不是远离他?
我的信仰是否因环境改变而冷淡?
\subsection*{二、信仰的坚持:不向世界妥协(5-6节)}
“耶路撒冷啊,我若忘记你,情愿我的右手忘记技巧!”(诗篇137:5)

\subsubsection*{1. 以色列人拒绝被同化}
他们虽然身处巴比伦,但心仍然归向耶路撒冷。他们不愿意唱锡安的歌给敌人听,因为他们的敬拜是神圣的,不是取悦世界的。

\subsubsection*{2. 现实生活中的应用}
基督徒不该被世界同化,而要坚守信仰(罗马书12:2)。
面对压力时,我们仍要忠于神,而不是妥协(但以理书3:16-18)。
\paragraph*{反思问题:}

我是否因环境压力而妥协信仰?
我的生活是否仍然见证神的荣耀?
\subsection*{三、公义的呼求:神必施行审判(7-9节)}
“耶和华啊,求你记念以东人在耶路撒冷遭难的日子所说的话;他们说:‘拆毁,拆毁,直拆到根基!’”(诗篇137:7)

\subsubsection*{1. 诗人向神呼求公义}
诗篇最后几节充满愤怒,诗人求神审判巴比伦,并记念那些在以色列遭难时落井下石的人。这种呼求并非报复之心,而是对神公义的信靠,相信神必审判邪恶。

\subsubsection*{2. 现实生活中的应用}
当遭受不公义时,我们要把愤怒交给神,而不是自己复仇(罗马书12:19)。
相信神的公义,耐心等候他的时间(诗篇37:7)。
\paragraph*{反思问题:}

当面对不公义时,我是自己伸冤,还是交托给神?
我是否相信神的公义,并愿意等候他的时间?
\subsection*{结论:如何在困境中活出信仰?}

\hspace{0.6cm}在苦难中仰望神,不被环境吞噬(1-4节)。

坚守信仰,不向世界妥协(5-6节)。

相信神的公义,忍耐等候他的审判(7-9节)。

我们活在世上,如同寄居者,面对世界的压力、挑战和逼迫。但愿我们像诗篇137篇的诗人一样,即使身处异地,也不忘属天的家乡,在苦难中仍然敬拜神,并相信他的公义。

\subsection*{祷告}
\textbf{天父,}

我们感谢你!在困境中,你仍与我们同在。求你帮助我们,即使身处“巴比伦”,也不忘记你的圣城。求你赐给我们力量,使我们不被世界同化,而是坚守信仰,持守盼望。面对不公义时,求你赐我们忍耐的心,相信你的公义必然成就。愿我们的生命成为你荣耀的见证!

奉主耶稣基督的名祷告,阿们!
%-----------------------------------------------------------------------------
\newpage
\section{诗篇第138篇:在困境中依然感恩——全心赞美的生命}
% 《》
% ——诗篇138篇讲章

% 经文:诗篇138篇

“我要一心称谢你,在诸神面前歌颂你。”(诗篇138:1)

\subsection*{引言:感恩的力量}
诗篇138篇是大卫的感恩诗,他在经历挑战、困境甚至敌人的威胁时,依然选择一心称谢神。今天,我们也会面对生活的压力、信仰的挑战、甚至来自世界的敌对。我们如何在这样的环境中仍能全心感恩、坚定信靠神呢?

\subsection*{一、因神的信实而感恩(1-3节)}
“我要一心称谢你,在诸神面前歌颂你。”(诗篇138:1)

\subsubsection*{1. 以色列人为何感谢神?}
\hspace{0.6cm}大卫在困难中仍然一心感恩,不是因为环境好,而是因为神的信实从未改变。他深知:神的名和他的话高过一切(2节)。

“我呼求的日子,你就应允我,鼓励我,使我心里有能力。”(诗篇138:3)

大卫在祷告中经历神的应允,这使他更加坚定地敬拜神。

\subsubsection*{2. 现实生活中的应用}
当环境艰难时,我们仍要感恩神的信实,因他的应许不会落空(民数记23:19)。
养成感恩的习惯——每天数算神的恩典,无论环境如何,都选择感谢。
\paragraph*{反思问题:}

我是否在顺境才感谢神,而在困难中就埋怨?
我是否相信神必按他的信实成就他的应许?
\subsection*{二、因神的主权而感恩(4-6节)}
“耶和华虽然至高,仍看顾低微的人;他却从远处看出骄傲的人。”(诗篇138:6)

\subsubsection*{1. 万王之王,眷顾卑微的人}
大卫在此强调:神的伟大并不意味着他远离人,反而他看顾卑微的人。这让我们看到神的主权不仅仅是统管万有,更是亲自关怀我们的生活。

\subsubsection*{2. 现实生活中的应用}
当我们觉得自己微不足道时,记住神看顾我们(马太福音10:29-31)。
不要因环境而骄傲或自卑,因为神的主权掌管一切,我们当谦卑信靠。
\paragraph*{反思问题:}

我是否在骄傲时忘记神,在低谷时怀疑神?
我是否愿意谦卑地信靠神的主权?
\subsection*{三、因神的拯救而感恩(7-8节)}
“我虽然行在患难中,你必将我救活。”(诗篇138:7)

\subsubsection*{1. 在困境中,神依然掌权}
大卫承认他仍然在患难中,但他的信心没有动摇。他相信神的右手必扶持他、成就他的旨意(8节)。

神并没有应许我们不会遇到困难,但他应许在困境中必与我们同在!

\subsubsection*{2. 现实生活中的应用}
当我们面临挑战时,要仰望神的保护,而不是陷入恐惧。
神会完成他在我们身上的旨意,即使现在的环境看起来不理想。(腓立比书1:6)
\paragraph*{反思问题:}

我是否愿意相信神在困境中仍然掌权?
我是否相信神必成就他对我生命的计划?
\subsection*{结论:活出感恩的生命}
诗篇138篇提醒我们:感恩不是因为环境好,而是因为神永不改变!

因神的信实而感恩——无论环境如何,神的应许不落空。

因神的主权而感恩——无论高低起伏,神仍掌权。

因神的拯救而感恩——无论身处何地,神必成就他的旨意。

愿我们学习大卫的心志,在任何环境下,都以感恩的态度回应神的恩典!

\subsection*{祷告}
\textbf{亲爱的天父,}

感谢你!你是信实的神,你的应许永不落空。无论我们身处何种环境,都愿意用全心来感谢你。求你帮助我们在顺境中不骄傲,在逆境中不埋怨,而是始终信靠你。你掌管我们的生命,你必成就你在我们身上的旨意。愿我们的生命成为你荣耀的见证!

奉主耶稣基督的名祷告,阿们!
%-----------------------------------------------------------------------------
\newpage
\section{诗篇第139篇:被神完全认识,仍被完全爱}
% 《》
% ——诗篇139篇讲章

% 经文:诗篇139篇

“耶和华啊,你已经鉴察我,认识我。”(诗篇139:1)

\subsection*{引言:被神完全认识的安慰}
\hspace{0.6cm}诗篇139篇是大卫对神全知、全在、全能的深刻默想。他在诗中表达了一个令人震撼的事实:神完全认识我们,甚至比我们自己更认识自己! 他知道我们的一切,不仅是言语和行为,甚至是我们内心深处的意念。然而,尽管神完全认识我们,他仍然完全爱我们!

在现实生活中,我们常常渴望被理解,被爱,但也害怕被人真正看透。因为人一旦真正知道我们的软弱、失败、隐藏的罪,可能会远离我们。但诗篇139篇告诉我们,神是那位最了解我们、却仍然爱我们的神! 这给了我们极大的安慰和信心。

\subsection*{一、神的全知:他完全认识我们(1-6节)}
“耶和华啊,你已经鉴察我,认识我。”(诗篇139:1)

\subsubsection*{1. 神知道我们的思想、言语和行为}
\hspace{0.6cm}他知道我们的日常生活(2节):神知道我们什么时候坐下,什么时候起来,甚至知道我们心里的意念。

他知道我们所有的言语(4节):我们还没开口,神已经完全知道我们要说什么。

他的知识超过我们的理解(6节):神的知识奇妙高深,不是人所能测度的。
\subsubsection*{2. 现实生活中的应用}
我们无法向神隐藏——不要试图掩饰自己的罪,而是勇敢地来到神面前,求他洁净(约翰一书1:9)。
我们可以完全信靠神——神知道我们的软弱,也知道我们的需要,因此我们可以放心依靠他(马太福音6:8)。
\paragraph*{反思问题:}

我是否愿意在神面前敞开心扉,承认自己的软弱?
我是否相信神知道我的一切,仍然爱我?
\subsection*{二、神的无所不在:他一直与我们同在(7-12节)}
“我往哪里去躲避你的灵?我往哪里逃、躲避你的面呢?”(诗篇139:7)

\subsubsection*{1. 无论我们在哪里,神都在那里}
无论是天上、地上、深渊,神都同在(8节)。
无论是清晨的翅膀飞到海极,神的手仍然引导我们(9-10节)。
即使黑暗遮蔽我们,在神看来仍如白昼(11-12节)。
\subsubsection*{2. 现实生活中的应用}
在顺境中,神与我们同在,给予祝福。
在逆境中,神与我们同在,给予安慰和帮助。
即使我们软弱跌倒,神仍然没有离开我们(申命记31:6)。
\paragraph*{反思问题:}

我是否在困难中怀疑神的同在?
我是否相信神即使在黑暗时刻仍然牵引我?
\subsection*{三、神的创造:他奇妙地造我们(13-18节)}
“我的肺腑是你所造的;我在母腹中,你已覆庇我。”(诗篇139:13)

\subsubsection*{1. 神按他的形象造我们}
我们不是偶然的存在,而是神精心设计的杰作(14节)。
神在我们未出生前,就已经知道我们的一生(16节)。
\subsubsection*{2. 现实生活中的应用}
不要自卑或贬低自己,因为我们是神亲手创造的。
即使我们有缺陷或软弱,神仍然有他的美意。
我们要善待自己,也要尊重他人的价值。
\paragraph*{反思问题:}

我是否相信自己是神奇妙的创造,而不是偶然的存在?
我是否尊重自己和他人的生命,因为他们都是神的杰作?
\subsection*{四、神的公义:他鉴察我们的心(19-24节)}
“神啊,求你鉴察我,知道我的心思;试炼我,知道我的意念。”(诗篇139:23)

\subsubsection*{1. 诗人的公义呼求}
诗人希望恶人被审判,因为神是公义的(19-22节)。
诗人愿意让神鉴察自己的内心,看是否有恶行(23-24节)。
\subsubsection*{2. 现实生活中的应用}
我们要对神保持敬畏,愿意被他光照。
我们要定期省察自己,愿意被神改变,而不是论断他人(马太福音7:3-5)。
信靠神的审判,不自己伸冤(罗马书12:19)。
\paragraph*{反思问题:}

我是否愿意让神光照我的内心,而不是只看别人的问题?
我是否相信神的公义,并愿意让他来审判,而不是自己伸冤?
\subsection*{结论:被神完全认识,仍被完全爱}

\hspace{0.6cm}神完全认识我们,因此我们不需要隐藏,可以坦然来到他面前。

神无所不在,因此我们无论在哪里,都可以依靠他。

神奇妙地创造我们,因此我们要珍惜自己的价值。

神是公义的,因此我们要让他鉴察我们的心,使我们行在正道上。

愿我们在生活中,常常思想神的全知、全在、全能,并且让他的话语引导我们的生命!

\subsection*{祷告}
\textbf{天父,}

感谢你!你是无所不知、无所不在、无所不能的神。你完全认识我,却仍然爱我!求你帮助我在你面前坦然无惧,不再隐藏自己。无论遇到顺境还是逆境,求你让我依靠你的同在,不再害怕。帮助我看待自己和他人时,存感恩之心,因为我们都是你奇妙的创造。更求你鉴察我的心,使我行在你的正道上。

奉主耶稣基督的名祷告,阿们!
%-----------------------------------------------------------------------------
\newpage
\section{诗篇第140篇:在恶人的压迫中信靠神的公义}
% 《》
% ——诗篇140篇讲章

% 经文:诗篇140篇

“耶和华啊,求你拯救我脱离恶人,保护我脱离强暴的人!”(诗篇140:1)

\subsection*{引言:当我们面对恶人时}
\hspace{0.6cm}生活中,我们无法避免遇到恶人——那些说谎诡诈、充满恶意、攻击正直之人。无论是职场中的不公、生活中的欺骗,还是信仰上的逼迫,我们如何面对?

诗篇140篇是大卫的祷告,他在恶人的围困中,不是靠自己的聪明、权势或武力,而是单单仰望神的公义和拯救。这篇诗提醒我们,当面对恶人的攻击时,我们应当依靠神,而不是自己伸冤。

\subsection*{一、恶人的特征(1-5节)}
\subsubsection*{1. 恶人说谎诡诈,心存恶意(2节)}
“他们心中图谋奸恶,常常聚集要争战。”

恶人常常设计伤害他人,不仅是身体上的攻击,更包括言语上的毁谤、欺骗和诡计。
现实生活中,我们可能会遭遇恶人的谎言、诽谤、排挤甚至逼迫。
\subsubsection*{2. 恶人的言语如毒蛇(3节)}
“他们使舌头尖利如蛇,嘴里有虺蛇的毒气。”

言语可以成为可怕的武器,伤害人的心灵。
在学校、工作、社交圈,恶人可能用言语中伤我们,甚至故意散布谣言。
\subsubsection*{3. 他们设下网罗,要陷害义人(5节)}
“骄傲人为我暗设网罗和绳索,他们在路旁铺下网,设下圈套。”

恶人会设下陷阱,希望让正直人跌倒。
在职场、社会,甚至信仰环境中,我们可能遭遇不公平的对待,但神察看一切。
\paragraph*{现实生活应用}
当面对诽谤和陷害时,我们不要用同样的手段反击,而是要信靠神的公义。(罗马书12:19)
求神赐我们智慧和忍耐,不被恶人的言语和行为影响,而是坚守正道。
\subsection*{二、向神呼求拯救(6-8节)}
“耶和华啊,我曾对你说:‘你是我的神。’耶和华啊,求你留心听我恳求的声音!”(诗篇140:6)

\subsubsection*{1. 承认神是我们的倚靠(6节)}
大卫没有寻求自己的力量来对抗恶人,而是完全倚靠神。
面对攻击,我们的第一反应不该是愤怒或报复,而是转向神,求他施行公义。
\subsubsection*{2. 依靠神的保护(7节)}
“主耶和华啊,我救恩的力量啊,在争战的日子,你遮蔽了我的头。”

神是我们的盾牌,能保护我们不受恶人伤害。
我们要学习把重担交给神,而不是陷入愤怒或忧虑。(腓立比书4:6-7)
\paragraph*{现实生活应用}
当受到攻击时,我们要学会第一时间祷告,而不是第一时间抱怨或回击。
神是我们的避难所,他会按照他的时间和方式来拯救我们。
\subsection*{三、神的公义审判(9-11节)}
“愿那些围绕我的恶人,自己嘴唇的奸恶掩盖他们!”(诗篇140:9)

\subsubsection*{1. 恶人终将自食其果(9节)}
恶人的恶行不会长久,神必使他们落入自己设下的网罗中。
不要羡慕恶人短暂的得胜,他们终将面临神的审判。(诗篇37:1-2)
\subsubsection*{2. 神要使恶人灭亡(10-11节)}
“恶人必不在地上立定,强暴人必被祸患追赶。”

神是公义的,恶人无法永远得逞。
我们不需要自己伸冤,而是交托给神,他必要按公义行事。(罗马书12:19)
\paragraph*{现实生活应用}
当看到恶人猖狂时,不要心怀不平,而要安静等候神。
神的审判可能不会立刻来到,但他的公义永远不会落空。
\subsection*{四、义人的盼望(12-13节)}
“我知道耶和华必为困苦人伸冤,必为穷乏人辨屈。”(诗篇140:12)

\subsubsection*{1. 神必为义人伸冤(12节)}
神知道困苦人的痛苦,他必不丢弃他们。
即使世界不公,神仍是我们的申冤者和保护者。
\subsubsection*{2. 义人要永远称颂神(13节)}
“义人必要称谢你的名,正直人必住在你面前。”

最终,义人将得胜,并在神面前享受永远的平安。
我们的信仰和忍耐不会白费,神必赏赐那些忠心信靠他的人。
\paragraph*{现实生活应用}
坚持行善,不因恶人的得势而灰心。(加拉太书6:9)
即使在逼迫中,我们仍然可以喜乐,因为神掌权。
\subsection*{结论:信靠神的公义,而非自己伸冤}
诗篇140篇提醒我们:

恶人虽然猖狂,但他们的结局已定,神必审判。

当我们受逼迫时,不要报复,而是要向神呼求。

神是我们的保护者,他的公义必然得胜。

义人最终要在神的同在中喜乐,永远称颂他。

无论你现在是否正经历恶人的攻击,愿你能像大卫一样,把重担交托给神,信靠他的公义和拯救!

\subsection*{祷告}
\textbf{慈爱的天父,}

我们感谢你!当我们面对恶人的欺压时,你是我们的避难所。当我们遭遇不公,你是公义的申冤者。求你帮助我们,不被愤怒和仇恨吞噬,而是学习信靠你的公义和救恩。愿我们在困境中仍然坚守信仰,相信你掌权,最终义人必在你面前喜乐。

奉主耶稣基督的名祷告,阿们!
%-----------------------------------------------------------------------------
\newpage
\section{诗篇第141篇:求主保守我们的口、心与脚}
% 讲章:——诗篇141篇
\subsection*{引言:面对试探的挑战}
弟兄姊妹,你是否曾经因为一句话而后悔?
你是否曾经面对试探,心里挣扎,不知道如何选择?
你是否渴望活出圣洁的生活,却发现自己的软弱?
这首诗篇不仅仅是大卫的祷告,也应当成为我们每天的祷告。今天,我们就从这三方面来思考如何在试探和挑战中活出圣洁的生命。

\subsection*{一、求主保守我们的口舌(言语)}
“耶和华啊,我曾求祢快快临到我这里!我向祢呼求的时候,求祢留心听我的声音。”(诗篇141:1)
“耶和华啊,求祢禁止我的口,把守我的嘴唇。”(诗篇141:3)

\subsubsection*{1. 话语的力量}
圣经告诉我们,“生死在舌头的权下。”(箴言18:21)

我们的一句话可以安慰人,也可以伤害人。
一句充满爱心的话,可以鼓励一个人走出低谷;但一句恶言恶语,也可以让人痛苦不堪。
有时候,我们的怒言、批评论断、不负责任的话语,都会带来深远的影响。
\subsubsection*{2. 如何保守我们的言语?}
在说话之前,先祷告:“主啊,这话是造就人的,还是伤害人的?”
凡事求圣灵提醒,让我们的言语充满爱和智慧。
避免闲言碎语、论断和埋怨,而是用言语成为别人的祝福。
\paragraph*{应用问题:}

你最近有没有说过让自己后悔的话?
你是否愿意求神帮助,让你的言语成为别人的祝福?
\subsection*{二、求主保守我们的心灵(内心)}
“求祢不叫我的心偏向邪恶的事,不和作孽的人同行,以致我作恶。”(诗篇141:4)

\subsubsection*{1. 罪恶的试探始于内心}
试探最初不是外在的,而是从我们的心里开始的。
耶稣说:“从心里发出的,有恶念、凶杀、奸淫、苟合、偷盗、妄证、谤讟。”(马太福音15:19)
魔鬼的工作就是让我们心里产生偏离神的思想,最终引导我们走向罪恶的道路。
\subsubsection*{2. 如何保守我们的心?}
默想神的话语,抵挡罪恶的思想:“少年人用什么洁净他的行为呢?是要遵行祢的话。”(诗篇119:9)
远离恶人的影响:“不要效法这个世界,只要心意更新而变化。”(罗马书12:2)
用敬虔的朋友代替属世的影响(诗篇141:5)——愿意接受敬虔之人的劝诫,远离恶人的引诱。
\paragraph*{应用问题:}

你最近的思想是否常常被试探所充满?
你是否愿意让神的话语洁净你的内心?
\subsection*{三、求主保守我们的脚步(行为)}
“求祢救我脱离恶人的网罗。”(诗篇141:9)
“恶人要落在自己的网里,我却要逃脱。”(诗篇141:10)

\subsubsection*{1. 罪的陷阱无处不在}
这个世界充满各种各样的\textbf{“网罗”},让我们偏离神的道路。
有些陷阱看似无害,比如过度的娱乐、懒惰、自私,但最终会让我们远离神。
罪不会一下子让人跌倒,而是逐步诱惑人走向深渊。
\subsubsection*{2. 如何保守我们的脚步?}
时常自省:“我所走的道路是否合神心意?”
依靠神的引导:“你的话是我脚前的灯,是我路上的光。”(诗篇119:105)
逃离罪的环境,不给魔鬼留地步(以弗所书4:27)。
\paragraph*{应用问题:}

你是否发现自己正在一步步走向试探的深处?
你是否愿意求神帮助,远离罪的网罗?
\subsection*{结论:每天祷告求神保守}
诗篇141篇是一个求主保守的祷告,提醒我们:

求神保守我们的口舌,让我们的言语成为祝福,而不是伤害。

求神保守我们的心灵,不让邪恶的思想占据我们的内心。

求神保守我们的脚步,使我们远离罪的网罗,走在神的道路上。

弟兄姊妹,今天就让我们一起祷告,求主帮助我们在这个充满诱惑的世界中,持守圣洁,紧紧跟随他!

\subsection*{结束祷告}
\textbf{亲爱的天父,}

我们感谢祢,今天借着诗篇141篇,让我们学会如何在祢面前保持圣洁。

主啊,我们承认,我们的口舌常常说出不该说的话,求祢洁净我们的嘴唇,让我们说出造就人的话语,而不是伤害人的言语。

主啊,我们的心常常被这个世界的诱惑吸引,求祢帮助我们,使我们的心不偏向邪恶,而是专心爱祢,遵行祢的旨意。

主啊,我们的脚步常常落入试探之中,求祢引导我们,使我们远离恶人的网罗,走在祢的道路上。

主啊,我们不能靠自己胜过试探,但在祢里面,我们可以得胜!求祢每天保守我们的口、心、脚,使我们活出圣洁的生命!

奉主耶稣基督的名祷告,阿们!
%-----------------------------------------------------------------------------
\newpage
\section{诗篇第142篇:在困境中向神呼求}
% 讲章:——诗篇142篇
\subsection*{引言:当你孤单无助时,你向谁呼求?}
弟兄姊妹,你是否曾经感到被人遗忘?
你是否曾经历过心灵的低谷,感觉无处可逃?
你是否曾向人诉苦,却发现没有人真正理解你?
诗篇142篇是一首大卫在洞穴中的祷告诗,当时他被扫罗追杀,四面受敌,心灵极度痛苦。他无助、孤单、恐惧,但他却选择向神呼求。
今天,我们也会经历人生的“洞穴时刻”——失业、家庭问题、健康危机、朋友的背叛……在这些时刻,我们如何面对?
\subsection*{一、在困境中,我们可以向神倾诉一切}
“我发声哀告耶和华,发声恳求耶和华。”(诗篇142:1)
“我在他面前吐露我的苦情,陈说我的患难。”(诗篇142:2)

\subsubsection*{1. 人的本能反应:把痛苦藏在心里}
我们常常害怕向别人诉苦,担心别人不理解,甚至嘲笑我们。
许多人在痛苦中选择沉默、逃避、伪装坚强,但这样只会让我们越来越孤单。
\subsubsection*{2. 神邀请我们向他倾诉}
大卫没有选择沉默,而是大声呼求神。
神不是冷漠的神,他愿意听我们的呼求。
耶稣曾说:“凡劳苦担重担的人,可以到我这里来,我就使你们得安息。”(马太福音11:28)
\paragraph*{应用问题:}
你是否曾把痛苦藏在心里,而不是带到神面前?
你愿意开始操练向神倾诉一切吗?
\subsection*{二、在孤独中,我们可以信靠神的看顾}
“求你看顾我的右边,因我无人与我认识;我无处避难,也无人眷顾我。”(诗篇142:4)

\subsubsection*{1. 人生的低谷常常让我们感到孤单}
有时候,连最亲近的人都无法理解我们的痛苦。
大卫当时正被扫罗追杀,甚至连朋友都不敢帮助他。
\subsubsection*{2. 即使无人帮助,神仍然看顾我们}
人可能会遗忘你,但神永远不会遗忘你!
“妇人焉能忘记她吃奶的婴孩?…即或她们忘记,我却不忘记你。”(以赛亚书49:15)
当我们感到被遗弃时,神比任何人都更了解我们的感受。
\paragraph*{应用问题:}
你是否曾有“无人眷顾”的感觉?
你愿意相信神永远不会忘记你吗?
\subsection*{三、在软弱中,我们可以经历神的拯救}
“耶和华啊,我曾向祢呼求,我说:‘祢是我的避难所,在活人之地祢是我的福分。’”(诗篇142:5)
“求祢救我脱离逼迫我的人,因为他们比我强盛。”(诗篇142:6)

\subsubsection*{1. 我们靠自己无法胜过困境}
大卫知道,自己的力量有限,敌人太强大,他必须依靠神。
很多时候,我们越努力挣扎,反而越感到无助。
\subsubsection*{2. 神的拯救是我们唯一的出路}
“神是我们的避难所,是我们的力量,是我们在患难中随时的帮助。”(诗篇46:1)
大卫最终从洞穴里出来,成为以色列的王——神拯救了他!
我们今天的困境,也终将成为见证神的荣耀之处!
\paragraph*{应用问题:}
你是否在依靠自己,而不是完全信靠神?
你愿意把困境交托给神,等待他的拯救吗?
\subsection*{结论:无论何时,我们都可以向神呼求}
诗篇142篇提醒我们:

当我们痛苦时,可以向神倾诉,因为他愿意聆听。

当我们孤单时,可以信靠神的看顾,因为他永远不离开我们。

当我们软弱时,可以经历神的拯救,因为他是我们的避难所。

无论你的处境如何,今天就向神祷告吧!他必听你的呼求,拯救你出离困境!

\subsection*{结束祷告}
\textbf{亲爱的天父,}

我们感谢祢,因为祢是我们随时的帮助和拯救。

主啊,当我们陷入困境时,求祢帮助我们,不再藏匿自己的痛苦,而是勇敢向祢倾诉。

主啊,当我们感到孤单、无助时,求祢提醒我们,祢一直在看顾我们,祢从未离开我们。

主啊,当我们软弱、被困住时,求祢伸手拯救我们,让我们经历祢的恩典和大能!

我们愿意把所有的重担交托给祢,相信祢是我们的避难所,是我们在活人之地最美好的福分!

奉主耶稣基督的名祷告,阿们!
%-----------------------------------------------------------------------------
\newpage
\section{诗篇第143篇:黑暗中的盼望}
% 讲章:——诗篇143篇
\subsection*{引言:当你陷入黑暗时,你如何寻求出路?}
弟兄姊妹,你是否曾经历过人生的低谷?

也许你遭遇了重大失败,内心充满自责。
也许你正在面对人际关系的破裂,感到孤独无助。
也许你陷入属灵的低潮,感觉神离你很远。

大卫在诗篇143篇中表达了自己内心的痛苦,他被敌人追赶,心灵被黑暗笼罩,甚至觉得生命快要枯竭。然而,他没有放弃,而是转向神,向神发出迫切的呼求。

无论你现在是否处于困境,都可以透过这篇诗篇找到黑暗中的盼望!

\subsection*{一、向神倾诉困境,不要自己承受}
“耶和华啊,求祢听我的祷告,留心听我的恳求,凭祢的信实和公义应允我。”(诗篇143:1)
“仇敌逼迫我,将我打倒在地,使我住在幽暗之处,像死了许久的人一样。”(诗篇143:3)

\subsubsection*{1. 人的本能反应:自己消化痛苦}
很多人遇到问题时,会选择沉默、逃避、压抑,或者试图靠自己解决。
但这样的方式往往让我们更加焦虑、无助,甚至陷入抑郁。
\subsubsection*{2. 神邀请我们向他倾诉}
大卫没有自己硬扛,而是向神倾诉他的痛苦。
他求神凭信实和公义来回应他的祷告——因为神不会弃绝他的儿女!
耶稣也曾在客西马尼园向天父倾诉他的忧伤(马太福音26:38-39),我们更应当效法。
\paragraph*{应用问题:}
你是否常常自己默默承受痛苦,而不是带到神面前?
你愿意操练每天向神祷告,把内心的重担交给他吗?
\subsection*{二、回忆神的作为,重新点燃信心}
“我追想古时之日,思想祢的一切作为,默念祢手的工作。”(诗篇143:5)

\subsubsection*{1. 困境容易让人只看见眼前的黑暗}
当问题接踵而至,我们常常只看到当下的痛苦,而忘记神过去的恩典。
焦虑会让我们忽略神的信实,就像以色列百姓在旷野里常常忘记神曾经拯救他们出埃及。
\subsubsection*{2. 记念神的作为,帮助我们恢复盼望}
回顾神的恩典,让我们知道:神过去如何帮助我们,现在也必然会帮助我们!
大卫的秘诀:回想神的作为,让他的信心重新被点燃!
诗篇77:11-12 说:“我要提说耶和华所行的,我要记念祢古时的奇事。”
\paragraph*{应用问题:}
你是否常常因眼前的问题而失去盼望?
你可以回想神在你过去的生活中有哪些作为,并为此感恩吗?
\subsection*{三、寻求神的引导,走出黑暗}
“求祢使我清晨得听祢慈爱之言,因我倚靠祢;求祢使我知道当行的路,因我的心仰望祢。”(诗篇143:8)
“求祢指教我遵行祢的旨意,因祢是我的神;祢的灵本为善,求祢引我到平坦之地。”(诗篇143:10)

\subsubsection*{1. 许多时候,我们不知该如何走下去}
面对人生的困境,我们常常迷失方向,不知道该怎么办。
我们可能会凭自己的感觉和经验行事,但这样往往会让情况变得更糟。
\subsubsection*{2. 神应许引导寻求他的人}
神的道是我们脚前的灯,路上的光(诗篇119:105)。
当我们向神寻求指引,他不会让我们继续在黑暗中徘徊。
我们需要的是等候神的时间,而不是急于用自己的方法解决问题。
\paragraph*{应用问题:}
你是否曾因焦急而做出错误的决定?
你是否愿意在困境中耐心等候神的带领,而不是靠自己乱闯?
\subsection*{结论:黑暗中的盼望}
诗篇143篇教导我们,在困境中如何寻求神:

向神倾诉,不要自己承受——神愿意听我们的呼求。

回忆神的作为,点燃信心——神过去如何帮助我们,现在也必然帮助我们。

寻求神的引导,走出黑暗——神的灵会带领我们进入平安之地。

当你感到痛苦、孤独、迷失时,不要自己硬扛,不要只关注当下的黑暗,而是选择仰望神!

神不会让你永远停留在黑暗中,他的恩典足够支撑你走出来!

\subsection*{结束祷告}
\textbf{亲爱的天父,}

我们感谢祢,因祢是信实的神。

主啊,我们承认,在困难中,我们常常试图靠自己,而忘记向祢倾诉。
求祢帮助我们,让我们学会把所有的忧伤和重担交给祢。

主啊,我们也常常被眼前的问题蒙蔽,忘记了祢曾经如何帮助我们。
求祢让我们回想祢过去的恩典,使我们的信心再次被点燃。

主啊,我们在黑暗中时,常常迷失方向,不知道该往哪里走。
求祢引导我们,使我们行在祢的光中,走向祢所预备的道路!

奉主耶稣基督的名祷告,阿们!
%-----------------------------------------------------------------------------
\newpage
\section{诗篇第144篇:耶和华是我们的保障与力量}
% 讲章:——诗篇144篇
\subsection*{引言:你依靠什么面对人生的挑战?}
弟兄姊妹,我们每个人都会遇到挑战和争战:

事业上的竞争和压力

家庭中的矛盾和责任

属灵生活中的试探和挣扎

面对这些,我们常常想依靠自己的能力、资源、人脉,甚至经验来应对。但诗篇144篇告诉我们,真正的保障和力量不是来自世界,而是来自耶和华!

\subsection*{一、耶和华是我们争战的保障(1-4节)}
“耶和华我的磐石,是应当称颂的!他教导我的手争战,教导我的指头打仗。”(诗篇144:1)

\subsubsection*{1. 人生充满争战}
大卫是战士,他面对真正的战争,但我们每个人也有自己的“战场”。
可能是面对困难的决定,经济的压力,或者信仰上的挣扎。
\subsubsection*{2. 神装备我们面对争战}
神不仅拯救我们,也教导我们如何争战。
“教导我的手争战,教导我的指头打仗”意味着神赋予我们智慧、勇气和能力去面对挑战,而不是逃避它们。
\subsubsection*{3. 但人的生命如影儿(3-4节)}
“人算什么,祢竟认识他?世人算什么,祢竟顾念他?”(144:3)
大卫提醒我们,人是有限的,我们的依靠必须是神,而不是自己。
\paragraph*{应用问题:}
你现在正面对什么样的争战?
你是否愿意依靠神,而不是单靠自己的力量?
\subsection*{二、耶和华是我们呼求的拯救者(5-11节)}
“耶和华啊,求祢使天下垂,亲自降临!”(诗篇144:5)

\subsubsection*{1. 大卫的急切呼求}
他求神亲自介入,显出神的大能,粉碎仇敌的攻击。
在危难中,我们也可以大声向神呼求,他会听见!
\subsubsection*{2. 神的拯救超越我们的想象}
“求祢救我、搭救我脱离大水,脱离外邦人的手。”(144:7)
这里的“大水”代表巨大的困境和试探,但神能伸手拯救我们。
\subsubsection*{3. 我们当以感恩回应}
“神啊,我要向祢唱新歌!”(144:9)
信靠神的拯救不仅带来平安,也带来喜乐和感恩。
\paragraph*{应用问题:}
你是否愿意在困境中依靠神,而不是焦虑和抱怨?
你有因神的拯救而向他献上感恩吗?
\subsection*{三、耶和华是赐福的源头(12-15节)}
“我们的儿子从幼年好像树栽子长大;我们的女儿如同殿角石。”(诗篇144:12)

\subsubsection*{1. 家庭的兴盛是神的赐福}
神所赐的福,不仅是个人的成功,也包括家庭的兴旺。
儿女健康成长,如同大树坚固,如同殿角石稳定。
\subsubsection*{2. 物质的丰盛也是神的恩典}
“我们的仓盈满,能出各样的粮食;我们的羊在田间增添千千万万。”(144:13)
这提醒我们,一切的供应都来自神,而不是我们的努力。
\subsubsection*{3. 真正的福气在于神同在}
“以耶和华为神的,那百姓便为有福!”(144:15)
真正的福分不是财富,而是有神的同在,带来平安和喜乐。
\paragraph*{应用问题:}
你是否愿意把你的家庭和未来交托给神?
你是否愿意让神成为你生命中真正的祝福?
\subsection*{结论:神是我们生命的磐石}
诗篇144篇提醒我们:

人生是争战,但耶和华是我们的保障。

我们需要向神呼求,他是我们的拯救。

真正的祝福来自神,而不是世界的财富。

无论你今天面临什么挑战,请相信:耶和华是我们的磐石,我们的盾牌,我们的保障!

\subsection*{结束祷告}
\textbf{亲爱的天父,}

感谢祢,因祢是我们的磐石和盾牌。

当我们面对人生的争战时,求祢教导我们的手争战,赐我们智慧和勇气。
当我们陷入困境时,求祢伸手拯救我们,使我们经历祢的大能。
当我们追求祝福时,求祢让我们知道,真正的福分是在于祢的同在。

主啊,我们愿意依靠祢,而不是依靠自己的力量;愿意等候祢,而不是凭自己行事。
求祢赐福我们的家庭、事业、属灵生命,使我们成为以祢为神的蒙福百姓!

奉主耶稣基督的名祷告,阿们!
%-----------------------------------------------------------------------------
\newpage
\section{诗篇第145篇:颂赞伟大的神}
% 讲章:——诗篇145篇
\subsection*{引言:我们是否常常颂赞神?}
在日常生活中,我们是否常常因神的作为而发出赞美?有时,我们太忙、太焦虑,甚至觉得神的恩典是理所当然的,而忘记了赞美他。
诗篇145篇是一篇充满敬拜和赞美的诗,是大卫对神的伟大、公义、慈爱和信实的颂扬。

\subsection*{一、神的伟大无可测度(1-7节)}
“我要尊崇祢,我的神,我的王!我要永永远远称颂祢的名!”(诗篇145:1)

\subsubsection*{1. 个人的赞美(1-2节)}
赞美神不是一种宗教仪式,而是个人与神之间的真实关系。
大卫的心中充满感恩,他立志\textbf{“天天称颂祢,永永远远赞美祢的名。”(2节)}
我们是否每天愿意花时间称颂神,而不仅是在顺境中才赞美他?
\subsubsection*{2. 世世代代的传承(3-4节)}
“这代要对那代颂赞祢的作为,也要传扬祢的大能。”(4节)
信仰不仅是个人的经历,更需要世世代代传承!
家庭、教会、社会都需要认识神的伟大,我们要成为神荣耀的见证人!
\subsubsection*{3. 述说神的作为(5-7节)}
神的荣耀、奇事、能力是值得述说的(5-6节)。
我们是否常常思想神的恩典,并向别人分享?
我们可以通过祷告、诗歌、见证分享神的奇妙作为!
\paragraph*{应用问题:}
你是否愿意每天花时间赞美神?
你是否愿意向下一代传递信仰?
你是否有述说神恩典的习惯?
\subsection*{二、神的恩典与怜悯长存(8-13节)}
“耶和华有恩惠,有怜悯,不轻易发怒,大有慈爱。”(诗篇145:8)

\subsubsection*{1. 神的性情——恩惠、怜悯、慈爱(8-9节)}
神不像人一样易怒或冷漠,他“不轻易发怒,大有慈爱”(8节)。
他“善待万民”(9节),包括信他的人和不信的人。
我们的生命中有没有活出神的恩慈和怜悯?
\subsubsection*{2. 神的国度存到永远(10-13节)}
神的国是永恒的,而世界的国度都会过去。
我们是否把盼望建立在神的国度上,而不是世界的短暂成就?
“你的国是永远的国,祢执掌的权柄存到万代。”(13节)——我们的生命是否以神的国度为中心?
\paragraph*{应用问题:}
你是否常常思想神的恩典,并向别人分享?
你是否愿意活出神的慈爱,对待身边的人?
你是否把盼望建立在神的国度,而不是世界的成功?
\subsection*{三、神的供应和信实可靠(14-21节)}
“凡跌倒的,耶和华将他们扶持;凡被压下的,将他们扶起。”(诗篇145:14)

\subsubsection*{1. 神扶持软弱者(14节)}
人生总有低谷,但神是我们的扶持者!
你是否愿意依靠神,而不是靠自己?
\subsubsection*{2. 神供应一切需要(15-16节)}
“万民都举目仰望祢,祢随时给他们食物。”(15节)
神是信实的供应者,我们可以放心交托!
\subsubsection*{3. 神垂听祷告(18-20节)}
“凡求告耶和华的,就是诚心求告他的,耶和华便与他们相近。”(18节)
他听我们的祷告,拯救我们,保护我们。
我们是否在困难中选择祷告,而不是忧虑?
\subsubsection*{4. 结论——我们当永远赞美神(21节)}
“我的口要说出耶和华的赞美。”(21节)
这篇诗篇开始于赞美,也结束于赞美,提醒我们:不论顺境或逆境,我们都要赞美神!
\paragraph*{应用问题:}
你是否愿意在软弱时依靠神,而不是靠自己?
你是否信靠神的供应,而不是忧虑明天?
你是否在祷告中寻求神,而不是自己想办法解决?
\subsection*{结论:让我们的生命充满颂赞!}
诗篇145篇提醒我们:

神的伟大无可测度,我们当世世代代述说他的奇妙作为。
神的恩典和怜悯长存,我们要活出他的慈爱,并盼望他的国度。
神的供应和信实可靠,我们要信靠他,并常常祷告、赞美他。
让我们每天都用口、用心、用行动来赞美神,让我们的生命成为他荣耀的见证!

\subsection*{结束祷告}
\textbf{亲爱的天父,}

我们感谢祢,因为祢是伟大、慈爱、信实的神!

求祢教导我们每天敬拜祢,述说祢的奇妙作为。
求祢帮助我们活出祢的恩典,对待他人也满有怜悯和慈爱。
求祢赐我们信心,让我们在困境中信靠祢的供应,不再忧虑。
愿我们的生命成为赞美祢的见证,使更多人认识祢的美善!

奉主耶稣基督的名祷告,阿们!
%-----------------------------------------------------------------------------
\newpage
\section{诗篇第146篇:信靠耶和华,得享真正的福乐}
% 讲章:——诗篇146篇
\subsection*{引言:你信靠谁?}
在这个充满不确定性的世界里,人们总是寻找依靠。有些人信靠财富,有些人依赖权力,有些人把希望寄托在人的承诺上。然而,这一切都是短暂的,随时可能破灭。
诗篇146篇教导我们:真正的福乐来自于信靠耶和华! 这篇诗篇是一篇充满信心和喜乐的赞美诗,
让我们一同来学习,如何把信心建立在那位永不改变、永远掌权的神身上!

\subsection*{一、单单信靠神,不倚靠世人(1-4节)}
“你们不要倚靠君王,不要倚靠世人,他们一点不能帮助。”(诗篇146:3)

\subsubsection*{1. 诗人的决心:一生赞美神(1-2节)}
诗篇146篇从\textbf{“你要赞美耶和华”(1节)开始,提醒我们生命的首要目标就是敬拜神!}
赞美神不仅仅是口头的敬拜,更是生活的态度。
诗人说:“我一生要赞美耶和华”(2节),这意味着无论环境如何,我们都要敬拜他!
\subsubsection*{2. 世人不能成为真正的倚靠(3-4节)}
人常常倚靠权力、金钱、聪明才智,但诗人提醒我们:

君王不能拯救我们(3节)。

世人的计划终必归于无有(4节)。

人有限,会死亡,而神是永恒的。
我们是否过度依赖人的帮助,而忽略了神?
\paragraph*{应用问题:}
你是否愿意像诗人一样,一生都赞美神?
你是否倚靠神,还是倚靠人或自己的聪明?
你是否在危难中首先寻求神,而不是只依赖世人的方法?
\subsection*{二、神是信实的帮助者,眷顾困苦人(5-9节)}
“以雅各的神为帮助,仰望耶和华他神的,这人便为有福!”(诗篇146:5)

\subsubsection*{1. 以神为帮助的人是有福的(5节)}
倚靠神的人不会失望,反而是有福的!
这个福不是短暂的物质祝福,而是内心的满足和永恒的保障。
\subsubsection*{2. 神的权能与信实(6节)}
耶和华造天地海(6节),他掌管一切,他的信实存到永远!
这意味着,他的应许永不改变,他的帮助永远可靠!
\subsubsection*{3. 神眷顾困苦人(7-9节)}
诗人列举了神的作为,他是:

为受欺压的人伸冤(7节)。

赐粮食给饥饿的人(7节)。

释放被囚的人(7节)。

医治瞎眼的人(8节)。

扶起被压下的人(8节)。

保护寄居的、扶持孤儿寡妇(9节)。

这正是耶稣在世上的服事! 他医治瞎子、释放被捆绑的人,向贫穷人传福音(路加福音4:18)。

\paragraph*{应用问题:}
你是否愿意单单仰望神,经历他的信实?
你是否相信神关心你的困境?
你是否愿意效法神,去帮助困苦的人?
\subsection*{三、神的国度永远掌权(10节)}
“耶和华要作王,直到永远!”(诗篇146:10)

\subsubsection*{1. 世上的政权会改变,神的国度却永存}
历史上的国度都曾强盛,但最终都会衰败。
只有神的国度存到永远,他的公义不会改变。
这提醒我们,不要把盼望放在地上的事物,而要追求天国的价值!
\subsubsection*{2. 以神的国为中心的人是蒙福的}
当我们把神的国度放在首位,我们的生活就会充满真正的平安和满足。
耶稣教导我们:“先求神的国和神的义,这些东西都要加给你们了。”(马太福音6:33)
我们是否真的把神的国度放在首位?还是仍然把地上的成就看得比神的旨意更重要?
\paragraph*{应用问题:}
你是否真正相信神是掌权的?
你是否愿意寻求神的国,而不是被世上的追求捆绑?
你是否愿意为神的国度摆上自己,成为神的见证人?
\subsection*{结论:你愿意信靠耶和华吗?}
诗篇146篇提醒我们:

不要倚靠世人,而要一生敬拜神。

神是信实的帮助者,他眷顾困苦人,我们当信靠他。

神的国度永存,我们要把盼望建立在永恒的国度里。

愿我们都成为真正信靠耶和华的人,得享他所赐的福乐!

\subsection*{结束祷告}
\textbf{亲爱的天父,}

我们感谢祢,因为祢是信实的神,祢掌管万有!

求祢帮助我们,不倚靠短暂的事物,而是单单信靠祢!
求祢赐我们信心,让我们在困境中仍然仰望祢,知道祢是我们的帮助。
求祢改变我们的生命,使我们愿意行在祢的旨意中,为祢的国度而活!
愿我们的生命成为祢荣耀的见证,愿我们的心常常敬拜祢!

奉主耶稣基督的名祷告,阿们!
%-----------------------------------------------------------------------------
\newpage
\section{诗篇第147篇:赞美伟大的神}
% 讲章:——诗篇147篇
\subsection*{引言:你是否常常赞美神?}
在忙碌的生活中,我们常常容易抱怨,或者被忧虑充满,而忽略了最重要的事——赞美神。诗篇147篇提醒我们:无论环境如何,我们都当颂赞耶和华,因为他是伟大的神!
本篇诗篇从多个角度展示了神的伟大,包括他的恩典、创造和护理。让我们今天一同学习如何透过诗篇147篇,认识这位配得赞美的神,并将赞美变成我们生活的态度。

\subsection*{一、因神的恩典而赞美他(1-6节)}
“你们要赞美耶和华!因歌颂我们的神为善为美,赞美的话是合宜的。”(诗篇147:1)

诗篇147篇一开始就发出了赞美的呼召,并指出赞美神是合宜的、是美好的!

\subsubsection*{1. 神是医治者(2-3节)}
\hspace{0.4cm}“耶和华建造耶路撒冷,聚集以色列中被赶散的人。”(2节)

这节经文指向以色列被掳之后,神如何使他们归回,重新建立圣城。
今天,神仍在医治和恢复破碎的生命。 你是否经历过神在你人生低谷中的安慰?

“他医好伤心的人,裹好他们的伤处。”(3节)

无论是情感的伤害、生活的挫折,还是心灵的痛苦,神都能医治。
你是否愿意把你的忧伤交给神,让他来安慰你?
\subsubsection*{2. 神是大能者(4-6节)}
\hspace{0.6cm}神能数算星宿,并一一称它的名(4节)

这表明神的无所不知,连浩瀚宇宙的星辰都在他的掌管之下。
你是否愿意相信,他也完全掌管你的人生?

“耶和华扶持谦卑人,将恶人倾覆于地。”(6节)

神眷顾谦卑的人,而不是骄傲的人。
在你的生命中,你是否愿意谦卑下来,顺服神的带领?
\paragraph*{应用问题:}
你是否把自己的伤痛交托给神,让他来医治?
你是否相信神掌管你的生命,如同掌管星辰?
你是否愿意放下骄傲,以谦卑的心来寻求神?
\subsection*{二、因神的供应而赞美他(7-11节)}
“他以云遮天,为地降雨,使草生长在山上。”(诗篇147:8)

这几节经文让我们看到神如何供应世界万物,并照顾他的子民。

\subsubsection*{1. 神供应万物}
神降雨、使草生长,供养田地的走兽和乌鸦的雏鸟(8-9节)
这提醒我们:神是供应者,一切的需要都由他而来。
我们是否真的相信神的供应,而不是焦虑和依赖自己的努力?
\subsubsection*{2. 神喜悦敬畏他的人}
“他不喜悦马的力大,不喜爱人的腿快。耶和华喜爱敬畏他和盼望他慈爱的人。”(10-11节)
神不是看人的能力、才干,而是看人的心。
你是否愿意成为一个真正敬畏神、信靠神的人?
\paragraph*{应用问题:}
你是否在经济或生活的困难中仍然信靠神的供应?
你是否以为靠自己的能力就能掌控人生,而忽略了神?
你是否用信心来等候神的慈爱,而不是焦虑和自我奋斗?
\subsection*{三、因神的保护而赞美他(12-20节)}
“耶路撒冷啊,你要颂赞耶和华!锡安哪,你要赞美你的神!”(诗篇147:12)

这段经文特别强调神的保护和话语的权能。

\subsubsection*{1. 神保护他的子民(13-14节)}
神坚固城门、赐平安,使他的子民得享丰富的粮食。
今天,神仍然保护属他的人。 你是否愿意在不安的环境中信靠神的保守?
\subsubsection*{2. 神的话语大有能力(15-18节)}
神发命令,雪就降下,霜就铺满大地(16-17节)。
神一出命令,冰雪消融,风就吹起(18节)。
这提醒我们:神的话语充满能力,能够改变环境,更新生命!
你是否愿意更多地亲近神的话语,经历他的大能?
\subsubsection*{3. 神将他的话赐给他的百姓(19-20节)}
神将律法赐给以色列,这是他特别的恩典。
今天,我们有完整的圣经,神的话语是我们生命的指南!
你是否珍惜神的话语,每天研读并遵行?
\paragraph*{应用问题:}
你是否在不安中仍然相信神的保护?
你是否愿意更多亲近神的话语,让他的话语指引你的人生?
你是否常常用神的话语来祷告,并在生活中活出他的真理?
\subsection*{结论:让我们用一生来赞美神!}
诗篇147篇提醒我们:

因神的恩典而赞美他——他医治伤心的人,扶持谦卑人!

因神的供应而赞美他——他供应万物,喜悦敬畏他的人!

因神的保护而赞美他——他赐平安,并用他的话语指引我们!

愿我们都成为敬畏神、信靠神、亲近神话语的人,并且用一生来颂赞他!

\subsection*{结束祷告}

\textbf{慈爱的天父,}

我们感谢祢,因祢是伟大、信实的神!
祢医治伤心的人,祢供应我们的需要,祢保护我们,使我们得享平安。
求祢赐给我们一颗敬畏祢的心,让我们单单信靠祢,不再倚靠自己。
求祢的话语光照我们,使我们的生命被更新,活出祢的真理!
愿我们的心常常充满赞美,无论顺境逆境,都来颂赞祢的圣名!

奉主耶稣基督的名祷告,阿们!
%-----------------------------------------------------------------------------
\newpage
\section{诗篇第148篇:万物当赞美耶和华}
% 讲章:——诗篇148篇
\subsection*{引言:我们为什么要赞美神?}
当你看见日出东方、群星闪耀,或者听见鸟儿欢唱、溪流潺潺时,你是否曾经停下来,为这一切美好而向神发出赞美? 诗篇148篇是一首全地赞美诗,呼召天上地下的一切受造之物都来颂赞神。
今天,我们不仅要明白这首诗篇的含义,更要学习如何在生活中真实地活出赞美的生命!

\subsection*{一、天上的创造物当赞美神(1-6节)}
“你们要赞美耶和华!从天上赞美耶和华,在高处赞美他!”(诗篇148:1)

诗篇148篇首先呼召天上的受造物——天使、日月星辰、天上的水——都要来赞美神。

\subsubsection*{1. 天使赞美神(2节)}
“他的众使者都要赞美他,他的诸军都要赞美他。”
天使是神的仆役,昼夜不息地赞美他。
今天,我们作为神的儿女,更应该用生命来赞美神!
\subsubsection*{2. 日月星辰赞美神(3节)}
“日头月亮,你们要赞美他!放光的星宿,你们都要赞美他!”
大自然本身见证了神的荣耀! 我们每天看到的太阳、夜晚的星辰,都是神所创造的,它们的存在就是对神的赞美。
你是否愿意像它们一样,在自己的岗位上,用实际行动见证神的荣耀?
\subsubsection*{3. 神创造万物,并命定它们的秩序(4-6节)}
神设立天的穹苍,并且命定它们永远存立(5-6节)。
这提醒我们:神掌管历史,他的命令永不改变!
你是否愿意相信神的主权,并顺服他的带领?
\paragraph*{实际应用:}
你是否像天使一样,不断地敬拜神?
你是否像太阳星辰一样,在自己的岗位上荣耀神?
你是否愿意顺服神的命令,相信他掌管一切?
\subsection*{二、地上的受造物当赞美神(7-12节)}
“所有在地上的大鱼和一切深洋,火与冰雹,雪和雾气,成就他命的狂风,都要赞美耶和华!”(诗篇148:7-8)

\subsubsection*{1. 大自然的奇妙见证神的荣耀(7-10节)}

这些自然现象看似无生命,却在彰显神的荣耀!
风、雨、火、山岳、动物……一切都在见证神的伟大!
你是否曾因风暴或挑战而害怕?但请记住:神掌管一切,自然界都顺服他,我们又怎能不信靠他呢?
\subsubsection*{2. 世人也当赞美神(11-12节)}

地上的君王、百姓、少年人、老年人——都当赞美耶和华!
无论身份高低、年龄大小,每个人都蒙召来赞美神!
你是否愿意用你的生命,来回应神的伟大?
\paragraph*{实际应用:}
你是否在困难中仍然信靠神,而不是惧怕?
你是否愿意用你的身份、职业来荣耀神?
你是否让你的家庭和教会都充满对神的赞美?
\subsection*{三、神的子民当特别赞美神(13-14节)}
“愿这些都赞美耶和华的名!因独有他的名被尊崇;他的荣耀在天地之上。”(诗篇148:13)

\subsubsection*{1. 赞美神的名,因为他独一无二(13节)}
只有神的名被尊崇,他的荣耀超过天地!
世上的权势、财富都会过去,唯有神的名永存!
你是否愿意在生活中高举神的名,而不是自己的名?
\subsubsection*{2. 神的子民是蒙恩的(14节)}
神特别拣选了以色列人,也拣选了我们!
我们得着救赎、恩典,这更是我们要赞美神的原因!
你是否感恩自己是神的儿女?你是否愿意活出与蒙恩身份相称的生命?
\paragraph*{实际应用:}
你是否真正敬畏神,单单高举他的名?
你是否感恩自己是神的子民,并愿意荣耀他?
\subsection*{结论:让赞美成为我们一生的态度!}
诗篇148篇呼召天上、地上、所有受造之物都要来赞美神。

\subsubsection*{为什么?}

\hspace{0.6cm}因为他是创造主,一切都出于他。

因为他掌管万有,一切都顺服他。

因为他特别恩待我们,使我们成为他的子民。

愿我们不只是口头赞美神,更要用生命来赞美他!

\subsection*{结束祷告}
\textbf{慈爱的天父,}

祢是创造天地的主,我们要向祢献上赞美!
祢的名超乎万名,祢的荣耀充满天地,愿一切受造之物都来尊崇祢!
求祢帮助我们,不仅用口赞美,更用生命来荣耀祢!
无论在高山或低谷,顺境或逆境,都让我们坚定信靠祢,并让祢的名在我们生命中被高举!
愿我们的家庭、教会、国家,都满有赞美祢的声音!

奉主耶稣基督的名祷告,阿们!
%-----------------------------------------------------------------------------
\newpage
\section{诗篇第149篇:以赞美迎接得胜}
% 讲章:——诗篇149篇
\subsection*{引言:赞美不仅是敬拜,更是属灵争战!}
诗篇149篇是一首充满喜乐和得胜的赞美诗,它提醒我们:赞美不仅是敬拜的表达,更是属灵争战的武器! 许多时候,我们把敬拜和生活分开,但圣经告诉我们,敬拜和得胜密不可分。
今天,让我们深入学习这篇诗篇,并思考:如何在生活中以赞美迎接神的得胜?

\subsection*{一、用新歌向神发出赞美(1-3节)}
“你们要向耶和华唱新歌,在圣民的会中赞美他!”(诗篇149:1)

诗篇149篇从呼召敬拜开始,提醒我们:赞美神应该是不断更新的!

\subsubsection*{1. 赞美是“新歌”,意味着持续经历神}
“新歌”代表新鲜的经历——如果我们每天都在经历神的恩典,就不会只有过去的见证,而是不断有新的赞美!
你的赞美是陈旧的吗? 还是你每天都因神的作为而有新的感恩?
\subsubsection*{2. 赞美要在圣民的会中}
诗篇强调:赞美不仅是个人的,更要在群体中进行(教会、团契、敬拜聚会)。
你的敬拜是否仅限于个人? 你是否也在教会中,与弟兄姐妹一同敬拜?
\subsubsection*{3. 用跳舞、乐器敬拜神(2-3节)}
“愿以色列因造他的主欢喜,愿锡安的民因他们的王快乐!”(2节)
真正的赞美应该充满喜乐! 不是冷漠地唱歌,而是发自内心的敬拜。
你的敬拜是否带着喜乐? 还是机械地唱诗?
\paragraph*{实际应用:}
每天在生活中寻找神的新作为,并用新歌赞美他!
在教会中与众圣徒一起敬拜,不要独自敬拜!
让敬拜充满喜乐和热情,不要冷漠!
\subsection*{二、赞美是属灵争战的武器(4-6节)}
“因为耶和华喜爱他的百姓,他要用救恩当作谦卑人的妆饰。”(诗篇149:4)

很多人以为赞美只是表达感恩,但诗篇149篇告诉我们:赞美也是属灵争战的武器!

\subsubsection*{1. 神喜悦他的百姓(4节)}
神爱我们,不是因为我们完美,而是因着他的恩典。
当我们谦卑自己,神就以救恩装饰我们,让我们活出得胜的生命!
你是否知道自己是蒙神喜悦的? 还是活在自卑和控告之中?
\subsubsection*{2. 赞美能带来属灵的得胜(5-6节)}
\hspace{0.4cm}“愿圣民因所得的荣耀高兴,愿他们在床上欢呼。”(5节)

这里的“床上”可以指夜晚的安息,也可以象征在困境中仍然欢喜。
当我们在困难中仍然赞美神,就会经历得胜!

“愿他们口中称赞神为高,手里有两刃的刀。”(6节)

“称赞神”是口中的武器,
“两刃的刀”代表神的话语(希伯来书4:12),
赞美和神的话语是基督徒争战的武器!

\paragraph*{实际应用:}
当你感到软弱时,不要抱怨,而是开始赞美神!
在属灵争战中,用神的话语和敬拜来得胜!
即使在夜晚(困难、疲惫时),仍要欢呼赞美神!
\subsection*{三、赞美带来神的审判与公义(7-9节)}
“为要报复列邦,刑罚万民。”(诗篇149:7)

这几节经文看似充满审判,其实是在表达:神的百姓最终会与神一同治理世界,带来公义!

\subsubsection*{1. 神的百姓将要得胜}
神的百姓并不是被动的,而是要与神同工,执行他的审判(启示录20:6)!
这提醒我们:神的公义终将得胜,恶人不会永远得势!
\subsubsection*{2. 赞美与公义密不可分}
真正的敬拜者,必然追求公义!
当我们赞美神,也应该活出公义,见证神的荣耀!
你是否只是敬拜神,却在生活中妥协、不追求圣洁?
\paragraph*{实际应用:}
相信神的公义最终会显明,不要灰心!
让敬拜成为你生活的一部分,活出公义!
\subsection*{结论:让赞美成为我们得胜的方式!}
诗篇149篇告诉我们:赞美不仅是敬拜的表达,更是得胜的钥匙!

用新歌向神发出赞美——让敬拜成为生活的一部分!

在争战中赞美——让敬拜成为属灵得胜的武器!

在公义中赞美——让敬拜带出圣洁和见证!

愿我们不只是唱诗敬拜,更要活出敬拜的生命,并在每一次挑战中,都用赞美迎接神的得胜!

\subsection*{结束祷告}
\textbf{慈爱的天父,}

我们感谢祢,因祢是我们的王,我们的得胜者!
求祢帮助我们,不仅用口赞美,更用生命来敬拜祢!
无论在顺境或逆境,都让我们凭信心高举祢的名!
求祢赐下勇气,让我们在属灵争战中,用赞美和祢的话语得胜!
愿我们的一生,都成为向祢献上的“新歌”,荣耀祢的名!

奉主耶稣基督的名祷告,阿们!
%-----------------------------------------------------------------------------
\newpage
\section{诗篇第150篇:全心全意赞美耶和华}
% 讲章:——诗篇150篇
\subsection*{引言:赞美是我们生命的目的}
\hspace{0.6cm}诗篇150篇是《诗篇》的最后一篇,也是对全书的总结。它用最直接、最热烈的方式呼召所有受造物赞美神。整本诗篇的高潮就在这里:“你们要赞美耶和华!”(诗150:1)

在我们的信仰生活中,赞美常常被误解,有些人认为赞美只是唱诗,有些人认为只有在顺境中才值得赞美。但诗篇150篇告诉我们,无论在何处、用何种方式,我们都当全心全意地赞美神!

今天,我们要深入剖析这篇诗篇,并思考:如何让赞美成为我们生命的中心?

\subsection*{一、赞美的对象——耶和华(1节)}
“你们要赞美耶和华!在神的圣所赞美他,在他显能力的穹苍赞美他!”(诗150:1)

赞美的首要问题是:我们为什么赞美? 诗篇150篇从一开始就给出答案:因为神是配得赞美的!

\subsubsection*{1. 在“圣所”赞美他——神在他子民中间}
“圣所”代表神的居所,在旧约是圣殿,在新约,神的圣所是教会(信徒的聚集)和我们的心(林前3:16)。
这意味着:我们的聚会和个人灵修,都要以赞美为核心!
你是否常常在教会中敬拜神? 还是把赞美当成可有可无的部分?
\subsubsection*{2. 在“穹苍”赞美他——神的荣耀遍满宇宙}
神的荣耀不受地点限制,他的创造本身就是敬拜的场所!
无论是在教会,还是在生活的每一个角落,我们都当赞美神!
你是否把敬拜局限在教堂,而忽略了在生活中敬拜神?
\paragraph*{实际应用:}
在教会里积极参与敬拜,不要做旁观者!
在生活中,无论在哪里,都以敬拜的心态面对工作、学习和人际关系!
\subsection*{二、赞美的原因——神的作为与伟大(2节)}
“要因他大能的作为赞美他,按着他极美的大德赞美他!”(诗150:2)

\subsubsection*{1. 赞美神的大能作为}
神创造天地,掌管历史,施行救恩,这是我们赞美的基础!
你是否常常数算神的恩典,还是只在困难时才想到神?
\subsubsection*{2. 赞美神的伟大品格}
神的公义、慈爱、信实、怜悯,都值得我们赞美!
如果我们的敬拜只是基于感受,而不是基于神的属性,那么我们的信仰就会变得肤浅。
你是否在困难时仍然因神的美善赞美他?
\paragraph*{实际应用:}
写下你曾经历过神的作为,并用这些经历来赞美他!
不论环境如何,学会因神的属性本身而赞美,而不是只在好事发生时才赞美!
\subsection*{三、赞美的方式——用全人全心敬拜(3-5节)}
“要吹角赞美他,鼓瑟弹琴赞美他!击鼓跳舞赞美他,用丝弦的乐器和箫的声音赞美他!用大响的钹赞美他!用高声的钹赞美他!”(诗150:3-5)

诗篇150篇列出了多种乐器,这表明:神配得我们用一切可以使用的方式来敬拜他!

\subsubsection*{1. 赞美可以是大声的、喜乐的}
圣经中常常提到击鼓、跳舞、吹角,这些都是充满热情的敬拜方式!
你是否把敬拜当作一种“仪式”,而不是发自内心的热爱?
\subsubsection*{2. 赞美也可以是深沉的、默想的}
虽然这里强调热烈的敬拜,但诗篇中也有许多安静的敬拜(诗46:10,“你们要休息,要知道我是神”)。
这提醒我们:敬拜不仅是外在的形式,更是内心的真实回应!
你是否仅仅追求敬拜的情绪,而忽略了敬拜的对象——神?
\paragraph*{实际应用:}
不要害怕用肢体语言表达对神的敬爱(如举手、跪下祷告)。
让敬拜成为你全人的表达,不要只是嘴唇的歌唱,而是心灵和诚实的敬拜!
\subsection*{四、赞美的呼召——一切受造之物都要赞美神(6节)}
“凡有气息的,都要赞美耶和华!你们要赞美耶和华!”(诗150:6)

\subsubsection*{1. 赞美是每个人的责任}
诗篇最后的呼召是普世性的,凡有气息的——所有受造物都要赞美神!
无论你是谁,无论你在哪个阶段,你都当赞美神!
你是否认为赞美只是“敬拜团队”的事情,而忽略了自己作为个体也要赞美神?
\subsubsection*{2. 赞美是我们永恒的使命}
地上的敬拜是天上敬拜的预演(启示录5:13)。
如果你现在不习惯敬拜,将来在天上如何适应那永恒的敬拜?
你是否真正享受在神面前的敬拜?还是觉得敬拜只是“做给别人看”的?
\paragraph*{实际应用:}
每天都要操练赞美,不只是主日才敬拜神!
让你的生命成为敬拜,让你的言行都反映出对神的尊崇!
\subsection*{结论:让敬拜成为你的生命方式!}
诗篇150篇告诉我们:敬拜不是一项任务,而是我们存在的目的!

敬拜的对象是神,不是人,也不是我们的感受!

敬拜的原因是神的作为与属性,不是我们的环境!

敬拜的方式是全人、全心、全力的投入,不是被动的形式!

敬拜的呼召是普世性的,每一个受造物都当赞美神!

愿我们每天都用生命见证神的荣耀,并在一切环境中赞美他!

\subsection*{结束祷告}
\textbf{慈爱的天父,}

我们感谢祢!
因为祢是配得一切荣耀、尊贵和赞美的神!
求祢教导我们,让我们的敬拜不仅仅停留在口头,而是成为我们生命的方式!
无论顺境逆境,都让我们发自内心地赞美祢!
求祢让我们的生命成为祢荣耀的见证,凡有气息的都要赞美祢!

奉主耶稣基督的名祷告,阿们!

%-----------------------------------------------------------------------------
\end{document}
